\documentclass[a4paper,12pt]{article}
\usepackage[T2A]{fontenc}
\usepackage[utf8]{inputenc}
\usepackage[russian]{babel}
\usepackage{enumitem}
\usepackage[letterpaper,top=2cm,bottom=2cm,left=3cm,right=3cm,marginparwidth=1.75cm]{geometry}
\usepackage{listings}
\usepackage{color}

\definecolor{dkgreen}{rgb}{0,0.6,0}
\definecolor{gray}{rgb}{0.5,0.5,0.5}
\definecolor{mauve}{rgb}{0.58,0,0.82}
% Useful packages
\usepackage{amsmath}
\usepackage{graphicx}
\usepackage[colorlinks=true, allcolors=black]{hyperref}


\title{Отчет по заданию маркировка компонент на бинарном изображении. Вариант 31.}
\author{Выполнил: Милованкин Максим, 
Группа: 3822Б1ПР1}
\begin{document}
\lstset{frame=tb,
  language=C++,
  aboveskip=3mm,
  belowskip=3mm,
  showstringspaces=false,
  columns=flexible,
  basicstyle={\small\ttfamily},
  numbers=none,
  numberstyle=\tiny\color{gray},
  keywordstyle=\color{blue},
  commentstyle=\color{dkgreen},
  stringstyle=\color{mauve},
  breaklines=true,
  breakatwhitespace=true,
  tabsize=3
}
\maketitle

\newpage
\section*{Актуальность задачи}
Обработка изображений является важной задачей в компьютерном зрении и анализе данных. Распознавание связанных компонент на бинарных изображениях используется в таких областях, как медицинская диагностика, автоматическое управление и анализ спутниковых данных.

Последовательные алгоритмы, реализующие данный процесс, могут быть вычислительно затратными, особенно при работе с изображениями большого размера. Для повышения производительности используют параллельные алгоритмы, распределяющие нагрузку между процессами или потоками.

Реализация параллельного подхода к задаче распознавания компонент позволяет значительно сократить время выполнения, особенно на многопроцессорных системах, что делает исследование и разработку таких решений актуальными.

\newpage
\section*{Постановка задачи}
Целью работы является разработка последовательной и параллельной версий алгоритма распознавания компонент на бинарных изображениях. Для достижения этой цели необходимо выполнить следующие этапы:
\begin{enumerate}
\item Реализовать последовательный алгоритм с использованием подхода последовательной маркировки пикселей.
\item Реализовать параллельную версию алгоритма с использованием технологии MPI для распределения вычислений между процессами.
\item Провести тестирование и сравнение производительности обеих версий алгоритмов на наборе изображений разного размера.
\end{enumerate}

Ожидаемый результат: последовательный и параллельный алгоритмы, корректно выполняющие маркировку компонент. Параллельный алгоритм должен демонстрировать ускорение по сравнению с последовательным на больших изображениях.

\newpage
\section*{Описание алгоритма последовательной версии}
Последовательный алгоритм распознавания компонент выполняется следующим образом:
\begin{enumerate}
\item Задается начальное изображение и выделяется память для хранения меток компонент.
\item Выполняется проход по каждому пикселю изображения. Если пиксель принадлежит объекту (значение 1), то:
\begin{itemize}
\item Определяются метки соседних пикселей (слева, сверху, слева-сверху).
\item Если все соседние пиксели фона (значение 0), то пикселю присваивается новая метка.
\item Если метка одного из соседей уже существует, то используется она.
\item Если метки соседей различны, фиксируется их эквивалентность.
\end{itemize}
\item После завершения прохода выполняется объединение эквивалентных меток.
\end{enumerate}

Этот алгоритм обеспечивает корректную маркировку всех компонент, но его вычислительная сложность увеличивается с размером изображения, что ограничивает его применение для больших данных.

\newpage
\section*{Описание алгоритма параллельной версии}
Параллельная версия алгоритма реализована с использованием MPI и включает следующие шаги:
\begin{enumerate}
\item Исходное изображение передается нулевому процессу, который распределяет его строки между процессами.
\item Каждый процесс выполняет маркировку своих строк:
\begin{itemize}
\item Пиксели маркируются локально, и фиксируются локальные эквивалентности меток.
\item Пограничные строки обмениваются между соседними процессами для согласования меток.
\end{itemize}
\item На нулевом процессе выполняется объединение меток всех процессов и устранение глобальных эквивалентностей.
\item Итоговые метки собираются на нулевом процессе и формируют выходное изображение.
\end{enumerate}

Параллельный подход позволяет эффективно использовать ресурсы многопроцессорных систем, сокращая время выполнения алгоритма для больших изображений. Однако синхронизация между процессами требует дополнительных затрат, что может снижать производительность на небольших изображениях.

\newpage
\section*{Описание эксперимента}
Для оценки эффективности разработанных алгоритмов проводилось тестирование на наборе бинарных изображений разного размера. Были измерены:
\begin{itemize}
\item Время выполнения последовательной и параллельной версий алгоритмов.
\item Корректность маркировки компонент для обоих алгоритмов.
\item Ускорение, достигаемое параллельной версией.
\end{itemize}

\subsection*{Результаты}
Эксперименты проводились на изображениях размеров от $256 \times 256$ и $1920 \times 1080$. Ниже представлены результаты тестирования:
\begin{itemize}
\item Для изображения $256 \times 256$:
\begin{itemize}
\item Последовательная версия - pipeline: 30 мс, task: 26 мс
\item Параллельная версия (4 процесса) - pipeline  18 мс, task: 12 мс
\end{itemize}
\item Для изображения $1920 \times 1080$:
\begin{itemize}
\item Последовательная версия - pipeline: 997 мс, task: 970 мс
\item Параллельная версия (4 процесса) - pipeline  431 мс, task: 344 мс
\end{itemize}
\end{itemize}
Результаты показали, что параллельная версия демонстрирует ускорение в 2-3 раза, сохраняя корректность маркировки.

\newpage
\section*{Заключение}
В ходе работы были разработаны последовательная и параллельная версии алгоритма распознавания компонент на бинарных изображениях. Параллельный алгоритм показал значительное ускорение по сравнению с последовательным на изображениях большого размера благодаря распределению вычислений между процессами.

Основной сложностью при реализации параллельного алгоритма стала синхронизация данных между процессами, что потребовало дополнительных вычислительных затрат. Однако на больших изображениях эти затраты компенсируются общим ускорением алгоритма.

Полученные результаты подтверждают эффективность использования параллельных вычислений для решения задач обработки изображений и открывают возможности для дальнейшего исследования в этой области.

\section{Приложение}
\subsection{ops\_mpi.cpp (Функция run() в параллельной реализации)}
\begin{lstlisting}
bool ComponentLabelingPar::run() {
    internal_order_test();
    
    boost::mpi::broadcast(world, rows, 0);
    boost::mpi::broadcast(world, cols, 0);
    
    int rank = world.rank();
    int world_size = world.size();
    
    std::vector<int> send_counts(world_size, 0);
    std::vector<int> offsets(world_size, 0);
    
    // Each segment starts at the beginning of a row
    int row_offset = 0;
    for (int i = 0; i < world_size; ++i) {
        int rows_for_proc = rows / world_size;
        send_counts[i] = rows_for_proc * cols;
        offsets[i] = row_offset * cols;
        row_offset += rows_for_proc;
    
        if (i == world_size - 1) send_counts[i] += (cols * rows - (rows / world_size) * cols * world_size);
    }
    
    // Determine the local size for the current process
    int local_size = send_counts[rank];
    local_image_.resize(local_size);
    
    // Scatter the data
    boost::mpi::scatterv(world, input_image_.data(), send_counts, offsets, local_image_.data(), local_size, 0);
    
    // auto linear_index = [this](std::size_t row, std::size_t col) -> std::size_t { return row * cols + col; };
    auto coord_index = [this](std::size_t index) -> std::pair<std::size_t, std::size_t> {
        return std::make_pair(index / cols, index % cols);
    };  // row; column
    
    //
    // Run algorithm for each segment
    //
    
    uint32_t next_label = 1 + offsets[rank];
    std::vector<uint32_t> local_labels_(local_size);
    
    // Assign labels and record equivalences
    std::unordered_map<uint32_t, uint32_t> local_label_equivalences;
    for (std::size_t i = 0; i < (size_t)local_size; ++i) {
        if (local_image_[i] == 0) continue;
    
        std::pair<std::size_t, std::size_t> coord_r_c = coord_index(i);
        size_t row = coord_r_c.first;
        size_t col = coord_r_c.second;
    
        // D C
        // B A
        uint32_t label_B = (col > 0) ? local_labels_[i - 1] : 0;
        uint32_t label_C = (row > 0) ? local_labels_[i - cols] : 0;
        uint32_t label_D = (row > 0 && col > 0) ? local_labels_[i - cols - 1] : 0;
    
        if (label_B == 0 && label_C == 0 && label_D == 0) {
        local_labels_[i] = next_label++;
        } else {
        uint32_t min_label = std::min({label_B, label_C, label_D}, [](uint32_t a, uint32_t b) {
            if (a == 0) {
            return false;
            }
            if (b == 0) {
            return true;
            }
            return a < b;  // min higher than 0
        });
        local_labels_[i] = min_label;
    
        for (uint32_t lbl : {label_B, label_C, label_D}) {
            if (lbl != 0 && lbl != min_label) {
            local_label_equivalences[std::max(lbl, min_label)] = std::min(lbl, min_label);
            }
        }
        }
    }
    // Update table according to equivalences
    for (auto& label : local_labels_) {
        while (local_label_equivalences.contains(label)) {
        label = local_label_equivalences[label];
        }
    }
    
    if (rank == 0) labels_.resize(rows * cols);
    boost::mpi::gatherv(world, local_labels_, labels_.data(), send_counts, offsets, 0);
    
    // Merge labels from nearby sections
    if (rank == 0) {
        std::unordered_map<uint32_t, uint32_t> label_equivalences;
    
        for (int section = 1; section < world_size; ++section) {
        int border = offsets[section];
    
        for (int lbl = border; (size_t)lbl < border + cols; ++lbl) {
            if (labels_[lbl] == 0) continue;
    
            std::pair<std::size_t, std::size_t> coord_r_c = coord_index(lbl);
            size_t row = coord_r_c.first;
            size_t col = coord_r_c.second;
    
            // D C
            // B A
            uint32_t label_B = (col > 0) ? labels_[lbl - 1] : 0;
            uint32_t label_C = (row > 0) ? labels_[lbl - cols] : 0;
            uint32_t label_D = (row > 0 && col > 0) ? labels_[lbl - cols - 1] : 0;
    
            if (label_B != 0 || label_C != 0 || label_D != 0) {
            uint32_t min_label = std::min({label_B, label_C, label_D}, [](uint32_t a, uint32_t b) {
                if (a == 0) {
                return false;
                }
                if (b == 0) {
                return true;
                }
                return a < b;  // min higher than 0
            });
            labels_[lbl] = min_label;
    
            for (uint32_t lbl_ : {label_B, label_C, label_D}) {
                if (lbl_ != 0 && lbl_ != min_label) {
                label_equivalences[std::max(lbl_, min_label)] = std::min(lbl_, min_label);
                }
            }
            }
        }
        }
        // Update table according to equivalences
        for (auto& label : labels_) {
        while (label_equivalences.contains(label)) {
            label = label_equivalences[label];
        }
        }
    }
    
    return true;
}
\end{lstlisting}
\subsection{ops\_mpi.cpp (Функция run() в последовательной реализации)}
\begin{lstlisting}
bool ComponentLabelingSeq::run() {
    internal_order_test();
    
    std::unordered_map<uint32_t, uint32_t> label_equivalences;
    uint32_t next_label = 1;
    
    labels_.resize(rows * cols, 0);
    
    auto linear_index = [&cols = this->cols](std::size_t row, std::size_t col) -> std::size_t {
        return row * cols + col;
    };
    
    // Assign labels and record equivalences
    for (std::size_t row = 0; row < rows; ++row) {
        for (std::size_t col = 0; col < cols; ++col) {
        if (input_image_[linear_index(row, col)] == 0) {
            continue;
        }
    
        uint32_t label_B = (col > 0) ? labels_[linear_index(row, col - 1)] : 0;
        uint32_t label_C = (row > 0) ? labels_[linear_index(row - 1, col)] : 0;
        uint32_t label_D = (row > 0 && col > 0) ? labels_[linear_index(row - 1, col - 1)] : 0;
    
        if (label_B == 0 && label_C == 0 && label_D == 0) {
            labels_[linear_index(row, col)] = next_label++;
        } else {
            uint32_t min_label = std::min({label_B, label_C, label_D}, [](uint32_t a, uint32_t b) {
            if (a == 0) {
                return false;
            }
            if (b == 0) {
                return true;
            }
            return a < b;  // min higher than 0
            });
    
            labels_[linear_index(row, col)] = min_label;
    
            for (uint32_t lbl : {label_B, label_C, label_D}) {
            if (lbl != 0 && lbl != min_label) {
                label_equivalences[std::max(lbl, min_label)] = std::min(lbl, min_label);
            }
            }
        }
        }
    }
    
    // Resolve label equivalences
    for (auto& label : labels_) {
        while (label_equivalences.contains(label)) {
        label = label_equivalences[label];
        }
    }
    
    return true;
}
\end{lstlisting}
\end{document}