\documentclass[a4paper,12pt]{article}

% Подключение пакетов
\usepackage[utf8]{inputenc}
\usepackage[russian]{babel}
\usepackage{amsmath}
\usepackage{listings}
\usepackage{xcolor}
\usepackage{graphicx}
\usepackage{hyperref}
\usepackage{geometry}

% Настройки для кода
\geometry{
    a4paper,
    top=20mm,
    right=15mm,
    bottom=20mm,
    left=25mm,
}

\setlength{\parindent}{12.5mm} 

% Оформление для кода
\lstset{
    language=C++,
    basicstyle=\ttfamily\small,
    keywordstyle=\color{blue}\bfseries,
    stringstyle=\color{red},
    commentstyle=\color{green!60!black},
    numbers=left,
    numberstyle=\tiny, 
    stepnumber=1,
    numbersep=5pt,
    backgroundcolor=\color{gray!10},
    frame=single,
    tabsize=2,
    captionpos=b,
    breaklines=true,
    breakatwhitespace=false,
    showspaces=false,
    showtabs=false,
    columns=fullflexible,
    keepspaces=true
}

\begin{document}

\begin{centering}
\thispagestyle{empty}
МИНИСТЕРСТВО НАУКИ И ВЫСШЕГО ОБРАЗОВАНИЯ РОССИЙСКОЙ ФЕДЕРАЦИИ\\
Федеральное государственное автономное образовательное учреждение высшего образования\\
    {\bfseries\large «Национальный исследовательский Нижегородский государственный университет им. Н.И. Лобачевского»}\\
    {\bfseries\large Институт информационных технологий, математики и механики}\\
Направление подготовки: 09.03.04 Программная инженерия\\
Направленность (профиль) программы бакалавриата: Разработка программно-информационных систем\\

    \vspace{6cm} 
    {\LARGE\bfseries Отчёт по задаче}\\
    {\large Сортировка Шелла с чётно-нечётным слиянием Бэтчера}\\
    \vspace{6cm} 

    {\hfill\parbox{0.5\textwidth}{%
        Выполнил:\\
        Петров Артём Андреевич\\
        группа 3822Б1ПР2\\

        \vspace{0.1cm}

        Проверил:\\
        доцент, кандидат технических наук\\
        Сысоев Александр Владимирович}
    }
    \vspace{2cm} 

    Нижний Новгород, 2024 г.

\end{centering}

\tableofcontents
\newpage

\section{Введение}
Сортировка является одной из ключевых задач в программировании. Алгоритм сортировки Shell Sort с чётно-нечётным слиянием Бэтчера выделяется своей эффективностью при работе с большими массивами данных. В данном проекте была выполнена параллельная реализация этого алгоритма с использованием MPI (Message Passing Interface), что позволило повысить производительность путём распределения вычислений между несколькими процессами.

\section{Постановка задачи}
Целью проекта является реализация параллельной версии алгоритма Shell Sort с чётно-нечётным слиянием Бэтчера с использованием MPI. Основные задачи:
\begin{itemize}
    \item Разработать алгоритм распараллеливания для Shell Sort с чётно-нечётным слиянием Бэтчера.
    \item Реализовать код с использованием MPI.
    \item Провести тестирование и оценить производительность программы.
    \item Сравнить результаты с последовательной версией.
\end{itemize}

\section{Описание алгоритма}
Алгоритм Shell Sort с чётно-нечётным слиянием Бэтчера состоит из двух основных частей: сортировка Шелла и этап чётно-нечётного слияния. Ниже приведено подробное описание этапов алгоритма.

\subsection{Этап сортировки Шелла}
Сортировка Шелла улучшает сортировку вставками за счёт уменьшения количества перемещений элементов. Она работает следующим образом:
\begin{enumerate}
    \item Делит массив на подгруппы, используя шаг (gap).
    \item Выполняет сортировку вставками для каждой подгруппы.
    \item Уменьшает шаг и повторяет процесс до тех пор, пока шаг не станет равным 1.
\end{enumerate}

Пример кода для этапа сортировки Шелла:
\begin{lstlisting}
for (int gap = size / 2; gap > 0; gap /= 2) {
    for (int i = gap; i < size; ++i) {
        int temp = data[i];
        int j;
        for (j = i; j >= gap && data[j - gap] > temp; j -= gap) {
            data[j] = data[j - gap];
        }
        data[j] = temp;
    }
}
\end{lstlisting}

\subsection{Этап чётно-нечётного слияния Бэтчера}
Чётно-нечётное слияние используется для объединения отсортированных подгрупп массива. Это делается путём попарного сравнения и обмена элементов для создания общего порядка.

Пример кода для чётно-нечётного слияния:
\begin{lstlisting}
for (int step = 1; step < size; step *= 2) {
    for (int i = 0; i < size; i += 2 * step) {
        for (int j = i; j < i + step && j + step < size; ++j) {
            if (data[j] > data[j + step]) {
                std::swap(data[j], data[j + step]);
            }
        }
    }
}
\end{lstlisting}

\section{Описание схемы распараллеливания}
Параллельная версия алгоритма была реализована следующим образом:
\begin{itemize}
    \item Исходный массив распределяется между процессами с использованием функции \texttt{MPI\_Scatter}.
    \item Каждый процесс выполняет сортировку своей части массива с помощью локального алгоритма Shell Sort.
    \item После локальной сортировки данные собираются обратно на главный процесс с использованием функции \texttt{MPI\_Gather}.
    \item Главный процесс выполняет чётно-нечётное слияние для объединения результатов.
\end{itemize}

Пример вызова MPI функций:
\begin{lstlisting}
MPI_Scatter(data.data(), local_size, MPI_INT, 
            local_data.data(), local_size, MPI_INT, 0, MPI_COMM_WORLD);

shellSort(local_data, local_size);

MPI_Gather(local_data.data(), local_size, MPI_INT, 
           data.data(), local_size, MPI_INT, 0, MPI_COMM_WORLD);
\end{lstlisting}

\section{Описание программной реализации}
Код был реализован с использованием MPI. Основные этапы:
\begin{itemize}
    \item \textbf{Предобработка:} проверка входных данных и распределение массива между процессами.
    \item \textbf{Основной этап:} локальная сортировка и взаимодействие между процессами.
    \item \textbf{Постобработка:} сборка результатов, выполнение чётно-нечётного слияния и их валидация.
\end{itemize}

Пример использования функций MPI:
\begin{lstlisting}
MPI_Init(&argc, &argv);
MPI_Comm_size(MPI_COMM_WORLD, &num_processes);
MPI_Comm_rank(MPI_COMM_WORLD, &rank);

MPI_Finalize();
\end{lstlisting}

\section{Эксперименты}
Эксперименты проводились для оценки времени работы алгоритма и проверки его корректности. Были выполнены следующие тесты:
\begin{itemize}
    \item Тесты на корректность с различными входными данными.
    \item Измерение времени работы для массивов разного размера.
    \item Сравнение производительности последовательной и параллельной версий.
\end{itemize}

Результаты показывают, что параллельная версия выигрывает по времени при увеличении размера массива.

\section{Выводы из результатов}
На основе полученных данных можно сделать следующие выводы:
\begin{itemize}
    \item Алгоритм Shell Sort с чётно-нечётным слиянием Бэтчера хорошо поддаётся распараллеливанию.
    \item MPI обеспечивает значительное ускорение для больших массивов.
    \item Для небольших массивов параллельная версия может уступать последовательной из-за накладных расходов на коммуникацию.
\end{itemize}

\section{Заключение}
В данном проекте была реализована и протестирована параллельная версия алгоритма Shell Sort с чётно-нечётным слиянием Бэтчера с использованием MPI. Эксперименты подтвердили корректность реализации и продемонстрировали её преимущества по производительности.

\section{Литература}
\begin{enumerate}
    \item Основы MPI 3 статьи \url{https://habr.com/ru/articles/548266/}
    \item Сортировка Шелла \url{https://ru.wikipedia.org/wiki/%D0%A1%D0%BE%D1%80%D1%82%D0%B8%D1%80%D0%BE%D0%B2%D0%BA%D0%B0_%D0%A8%D0%B5%D0%BB%D0%BB%D0%B0}
    \item Документация по библиотеке Boost MPI.
\end{enumerate}

\section{Приложение}
\begin{lstlisting}
namespace petrov_a_Shell_sort_mpi {

bool TestTaskMPI::pre_processing() {
  int world_rank;
  MPI_Comm_rank(MPI_COMM_WORLD, &world_rank);
  if (world_rank == 0) {
    size_t input_size = taskData->inputs_count[0];
    const auto* raw_data = reinterpret_cast<const unsigned char*>(taskData->inputs[0]);

    data_.resize(input_size);
    memcpy(data_.data(), raw_data, input_size * sizeof(int));
  }

  return true;
}

bool TestTaskMPI::validation() {
  int world_rank;
  MPI_Comm_rank(MPI_COMM_WORLD, &world_rank);
  if (world_rank == 0) {
    if (taskData->inputs.empty() || taskData->inputs_count.empty()) {
      return false;
    }

    if (taskData->outputs.empty() || taskData->outputs_count.empty()) {
      return false;
    }

    if (!taskData->outputs.empty() && taskData->outputs_count[0] == 0) {
      return false;
    }
  }
  return true;
}

bool TestTaskMPI::run() {
  int world_size;
  int world_rank;
  MPI_Comm_size(MPI_COMM_WORLD, &world_size);
  MPI_Comm_rank(MPI_COMM_WORLD, &world_rank);
  int n;

  if (world_rank == 0) {
    n = data_.size();
  }

  MPI_Bcast(&n, 1, MPI_INT, 0, MPI_COMM_WORLD);

  int local_size = n / world_size;
  std::vector<int> local_data(local_size);

  MPI_Scatter(data_.data(), local_size, MPI_INT, local_data.data(), local_size, MPI_INT, 0, MPI_COMM_WORLD);

  for (int gap = local_size / 2; gap > 0; gap /= 2) {
    for (int i = gap; i < local_size; ++i) {
      int temp = local_data[i];
      int j;
      for (j = i; j >= gap && local_data[j - gap] > temp; j -= gap) {
        local_data[j] = local_data[j - gap];
      }
      local_data[j] = temp;
    }
  }

  MPI_Gather(local_data.data(), local_size, MPI_INT, data_.data(), local_size, MPI_INT, 0, MPI_COMM_WORLD);

  if (world_rank == 0) {
    std::sort(data_.begin(), data_.end());
  }

  return true;
}

bool TestTaskMPI::post_processing() {
  int world_rank;
  MPI_Comm_rank(MPI_COMM_WORLD, &world_rank);

  if (world_rank == 0) {
    size_t output_size = taskData->outputs_count[0];
    auto* raw_output_data = reinterpret_cast<unsigned char*>(taskData->outputs[0]);
    memcpy(raw_output_data, data_.data(), output_size * sizeof(int));
  }

  return true;
}

}  // namespace petrov_a_Shell_sort_mpi
\end{lstlisting}

\end{document}