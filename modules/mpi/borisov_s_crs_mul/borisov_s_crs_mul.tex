\documentclass[12pt]{article}

% Кодировка и русский язык
\usepackage[T2A]{fontenc}
\usepackage[utf8]{inputenc}
\usepackage[russian]{babel}

% Дополнительные пакеты
\usepackage{amsmath,amssymb}
\usepackage{geometry}
\usepackage{graphicx}
\usepackage{listings}
\usepackage{hyperref}
\geometry{left=2cm,right=2cm,top=2cm,bottom=2cm}

% Настройка отображения кода в листингах (пример)
\lstset{
  basicstyle=\small\ttfamily,
  breaklines=true,
  tabsize=2,
  language=C++,
  numbers=left,
  numberstyle=\tiny
}

\begin{document}

%---------------------- Титульный лист ----------------------------

\begin{titlepage}
    \begin{center}
        \large 
        \textbf{ННГУ им. Лобачевского / ИИТММ / ПМИ}\\[0.5cm]
        
        \vspace{4cm}
        \textbf{\Large Отчёт по выполнению задания}\\
        \textbf{\large «Умножение разреженных матриц в формате CRS с использованием MPI»}\\[3cm]
        
        \vspace{3cm}
        \textbf{Выполнил:}\\
        студент группы 3822Б1ПМоп3 \\
        \textit{Борисов Сергей Михайлович}\\[1cm]

        \textbf{Преподаватель:}\\
        \textit{Нестеров Александр Юрьевич, аспирант}\\[2cm]
        
        \vfill
        \textbf{Нижний Новгород, 2024 г.}
    \end{center}
\end{titlepage}

%--------------------------- Оглавление ----------------------------
\tableofcontents
\newpage

%---------------------- Введение -----------------------------------

\section*{Введение}
\addcontentsline{toc}{section}{Введение}

В данном отчёте рассматривается задача умножения разреженных матриц, элементы которых имеют тип \texttt{double}, с использованием модели параллельных вычислений MPI). 
Основная цель работы - реализовать эффективный алгоритм умножения двух разреженных матриц в формате CRS (Compressed Row Storage) и исследовать его производительность по сравнению с последовательным вариантом.

%---------------------- Постановка задачи --------------------------

\section{Постановка задачи}
В рамках данной работы необходимо:
\begin{enumerate}
    \item Разработать алгоритм умножения двух разреженных матриц $A$ и $B$ (размером $m\times n$ и $n\times k$ соответственно), хранящихся в формате CRS.
    \item Выполнить распараллеливание алгоритма с помощью MPI: распределить строки первой матрицы между несколькими процессами.
    \item Провести валидацию корректности результата (сравнение с последовательной реализацией).
    \item Проанализировать и сравнить производительность:
        \begin{itemize}
            \item последовательного варианта;
            \item параллельных вариантов (2, 4 и более процессов).
        \end{itemize}
\end{enumerate}

Формат CRS (Compressed Row Storage) предполагает хранение в трёх массивах:
\begin{itemize}
    \item \texttt{values}: значения ненулевых элементов;
    \item \texttt{col\_index}: индексы столбцов соответствующих ненулевых элементов;
    \item \texttt{row\_ptr}: префиксные суммы (указывают, где в \texttt{values} и \texttt{col\_index} начинается строка).
\end{itemize}

%---------------------- Описание алгоритма -------------------------

\section{Описание алгоритма умножения}

Для умножения матриц \(A\) (размер \(m \times n\)) и \(B\) (размер \(n \times k\)) в формате CRS (Compressed Row Storage) реализуется следующий общий план:

\begin{enumerate}
    \item \textbf{Чтение и подготовка данных.}\\
    \(\texttt{A\_values}, \texttt{A\_col\_index}, \texttt{A\_row\_ptr}\) описывают ненулевые элементы и структуру строк матрицы \(A\). Аналогичные три вектора заданы для \(B\). Размеры \(A\) и \(B\) (число строк и столбцов) вычисляются по \(\texttt{row\_ptr}\) и \(\max(\texttt{col\_index}) + 1\). Для результирующей матрицы \(C\) (размер \(m \times k\)) инициализируются пустые вектора \(\texttt{C\_values}\), \(\texttt{C\_col\_index}\) и массив \(\texttt{C\_row\_ptr}\) длиной \(m+1\).

    \item \textbf{Основной цикл по строкам \(A\).}\\
    Для каждой строки \(i \in [0..m-1]\):
    \begin{enumerate}
        \item Выделяется временный буфер \(\texttt{temp}\) размером \(k\), обнуляется.
        \item Считывается диапазон индексов \([\texttt{A\_row\_ptr}[i], \texttt{A\_row\_ptr}[i+1]])\) ненулевых элементов строки \(i\).
        \item Для каждого ненулевого элемента \(A[i,a\_col]\):
        \begin{itemize}
            \item Определяется диапазон ненулевых элементов строки \(a\_col\) в \(B\): \([\texttt{B\_row\_ptr}[a\_col], \texttt{B\_row\_ptr}[a\_col + 1]])\).
            \item Для каждого \(\texttt{b\_col}\) из этого диапазона обновляется 
            \[
            \texttt{temp}[b\_col] \;{+}{=}\; A[i,a\_col] \times B[a\_col,b\_col].
            \]
        \end{itemize}
        \item Собираются все ненулевые значения из \(\texttt{temp}\) и добавляются в \(\texttt{C\_values}\) и \(\texttt{C\_col\_index}\) (индекс столбца и значение).
        \item Текущее положение в \(\texttt{C\_values}\) фиксируется в \(\texttt{C\_row\_ptr}[i]\).
    \end{enumerate}

    \item \textbf{Завершение формирования \(C\).}\\
    Элемент \(\texttt{C\_row\_ptr}[m]\) указывает на итоговое число ненулевых элементов в \(C\). Таким образом, результирующая матрица \(C\) хранится в CRS-формате через векторы \(\texttt{C\_values}\), \(\texttt{C\_col\_index}\) и \(\texttt{C\_row\_ptr}\).
\end{enumerate}

В итоге мы получаем матрицу 
\[
C = A \times B,
\]
где работа идёт только по ненулевым элементам матриц \(A\) и \(B\), что существенно экономит время при умножении разреженных структур.

%---------------------- Схема распараллеливания --------------------

\section{Схема распараллеливания}

В моей реализации для умножения матриц \(A\) и \(B\) в формате CRS используется общий подход к параллелизации:

\begin{enumerate}
    \item \textbf{Распределение данных}. 
    Нулевой (корневой) процесс инициализирует данные о матрицах и передаёт их всем остальным процессам.
    
    \item \textbf{Разделение строк матрицы \(A\)}. 
    Все строки \(A\) равномерно (по возможности) распределяются между процессами. Каждый процесс получает свой поддиапазон строк, который ему предстоит обработать.
    
    \item \textbf{Локальное умножение}. 
    На каждом процессе выполняется умножение «своих» строк \(A\) на общую матрицу \(B\). Результат аккумулируется в локальные структуры, соответствующие части результирующей матрицы \(C\).
    
    \item \textbf{Сбор результатов}. 
    По завершении умножения на каждом процессе фрагменты итоговой матрицы \(C\) отправляются корневому процессу, где все локальные результаты объединяются в итоговую матрицу \(C\) в формате CRS.
\end{enumerate}

Таким образом, основная идея --- \textbf{разделить строки \(A\) между процессами} и передать каждому полную информацию о матрице \(B\). Затем все частичные результаты собираются в финальную разреженную матрицу \(C\). 


%---------------------- Описание MPI-версии ------------------------

\section{Описание MPI-реализации}

В данном проекте для параллельного умножения разреженных матриц в формате CRS используется класс 
\texttt{CrsMatrixMulTaskMPI}.

\subsection{Основная структура класса}

Файл \texttt{ops\_mpi.hpp} содержит объявление класса 
\texttt{CrsMatrixMulTaskMPI : public ppc::core::Task}, где хранятся следующие ключевые поля:
\begin{itemize}
    \item \texttt{A\_values\_}, \texttt{A\_col\_index\_}, \texttt{A\_row\_ptr\_} --- разреженное описание матрицы \(A\).
    \item \texttt{B\_values\_}, \texttt{B\_col\_index\_}, \texttt{B\_row\_ptr\_} --- аналогичные вектора для матрицы \(B\).
    \item \texttt{C\_values\_}, \texttt{C\_col\_index\_}, \texttt{C\_row\_ptr\_} --- вектора для будущей результирующей матрицы \(C\).
    \item \texttt{A\_nrows\_}, \texttt{A\_ncols\_}, \texttt{B\_nrows\_}, \texttt{B\_ncols\_}, \texttt{C\_nrows\_}, \texttt{C\_ncols\_} --- размеры матриц.
    \item \texttt{world} --- \texttt{boost::mpi::communicator}, позволяющий взаимодействовать с остальными процессами.
\end{itemize}

Класс \texttt{CrsMatrixMulTaskMPI} переопределяет виртуальные методы:
\begin{itemize}
    \item \texttt{validation()},
    \item \texttt{pre\_processing()},
    \item \texttt{run()},
    \item \texttt{post\_processing()}.
\end{itemize}
Каждый из этих методов отвечает за определённый этап работы параллельного умножения.

\subsection{Этапы выполнения в \texttt{ops\_mpi.cpp}}

\subsubsection{\texttt{validation()}}
Здесь проверяется корректность входных данных:
\begin{itemize}
    \item Убеждаемся, что в \texttt{taskData} есть шесть входных векторов (три для матрицы \(A\) и три для \(B\)) и три выходных (для \(C\)).
    \item Проверяем совпадение размеров: число столбцов \(A\) (рассчитанное через \(\max(\texttt{A\_col\_index}) + 1\)) должно совпадать с числом строк \(B\).
\end{itemize}
Если данные не согласованы, метод возвращает \texttt{false}, и задача не выполняется.

\subsubsection{\texttt{pre\_processing()}}
Здесь нулевой процесс (\texttt{rank = 0}) считывает всю информацию о матрицах \(A\) и \(B\) из \texttt{taskData->inputs}, чтобы впоследствии передать её остальным процессам. В частности:
\begin{itemize}
    \item \(\texttt{A\_values\_}\), \(\texttt{A\_col\_index\_}\), \(\texttt{A\_row\_ptr\_}\) извлекаются из \texttt{inputs[0..2]}.
    \item Аналогичные три вектора для \(B\) --- из \texttt{inputs[3..5]}.
    \item Определяются \(A\_nrows\_, A\_ncols\_, B\_nrows\_, B\_ncols\_\).
    \item Для будущей матрицы \(C\) задаются \(C\_nrows\_ = A\_nrows\_\) и \(C\_ncols\_ = B\_ncols\_\), а также инициализируется \(\texttt{C\_row\_ptr\_}\) нулями.
\end{itemize}
На этом этапе данные ещё не рассылаются - это произойдёт внутри \texttt{run()}.

\subsubsection{\texttt{run()}}
Это ключевой метод, где осуществляется сама параллелизация. Логика следующая:
\begin{enumerate}
    \item \textbf{Рассылка} (\texttt{broadcast}):
    \begin{itemize}
        \item Все вектора \(\texttt{A\_values\_}\), \(\texttt{A\_col\_index\_}\), \(\texttt{A\_row\_ptr\_}\) и аналогичные для \(B\), а также \(C\_nrows\_\) и \(C\_ncols\_\) рассылаются от корневого процесса (rank = 0) всем остальным с помощью \texttt{boost::mpi::broadcast}.
    \end{itemize}
    \item \textbf{Разделение строк матрицы \(A\)}:
    \begin{itemize}
        \item Вычисляется, сколько строк приходится на процесс: 
        \[
          \texttt{rows\_per\_process} = \lfloor \frac{C\_nrows\_}{\texttt{world.size()}} \rfloor.
        \]
        \item Для текущего процесса \(\texttt{rank}\) назначается диапазон \([\texttt{start\_row}, \texttt{end\_row})\). 
        \item Исключением является последний процесс, который может взять «остаток» строк (если \(C\_nrows\_\) не делится на \(\texttt{world.size()}\)).
    \end{itemize}
    \item \textbf{Локальное умножение}:
    \begin{itemize}
        \item Каждый процесс инициализирует временный буфер \(\texttt{temp}\) размера \(\texttt{C\_ncols\_}\) = \(B\_ncols\_\).
        \item Проходит по «своим» строкам \(A\) (с индексами \([\texttt{start\_row}, \texttt{end\_row})\)) и умножает их на все ненулевые элементы соответствующих строк \(B\).
        \item Ненулевые результаты складываются в \(\texttt{local\_values}\) и \(\texttt{local\_col\_index}\), а \(\texttt{local\_row\_ptr}\) хранит, где начинается строка \(i\) в локальном буфере.
    \end{itemize}
    \item \textbf{Сбор результатов} (\texttt{gather}):
    \begin{itemize}
        \item На нулевой процесс (\texttt{rank = 0}) собираются все локальные куски \(\texttt{local\_values}\), \(\texttt{local\_col\_index}\) и \(\texttt{local\_row\_ptr}\).
        \item Затем происходит «сшивание» этих кусков в единые вектора \(\texttt{C\_values\_}\), \(\texttt{C\_col\_index\_}\) и \(\texttt{C\_row\_ptr\_}\). 
          При этом сдвигаются индексы, чтобы сохранить целостность структуры \(\texttt{C\_row\_ptr\_}\) (для второго, третьего и т. д. процесса).
    \end{itemize}
\end{enumerate}

Таким образом, \textbf{run()} одновременно:
\begin{itemize}
    \item Рассылает всю необходимую информацию о \(A\) и \(B\) всем процессам,
    \item Выполняет локальное умножение на каждом процессе,
    \item Собирательно формирует результирующую матрицу \(C\) на корневом процессе.
\end{itemize}

\subsubsection{\texttt{post\_processing()}}
Наконец, если текущий процесс --- корневой (rank = 0), результаты умножения (вектора \texttt{C\_values\_}, \texttt{C\_col\_index\_}, \texttt{C\_row\_ptr\_}) укладываются обратно в \(\texttt{taskData->outputs}\). Так в итоговом формате CRS матрицу \(C\) можно использовать дальше (либо сравнить с последовательным результатом).

\subsection{Итоговый процесс параллелизации}

В итоге, реализация \texttt{CrsMatrixMulTaskMPI} позволяет:
\begin{itemize}
    \item \textbf{Разделить строки} матрицы \(A\) между всеми процессами,
    \item \textbf{Хранить матрицу \(B\) полностью} на каждом процессе (после \texttt{broadcast}),
    \item \textbf{Собрать части результата} в структуру \(\texttt{C\_values\_}, \texttt{C\_col\_index\_}, \texttt{C\_row\_ptr\_}\) лишь на корневом процессе,
    \item \textbf{Минимизировать расходы на передачу данных}, так как параллельно передаются лишь исходные векторы (один раз) и локальные части результата (на этапе \texttt{gather}).
\end{itemize}

Код, описывающий данную логику, расположен в:
\begin{itemize}
    \item \texttt{ops\_mpi.hpp} - заголовок с описанием класса \texttt{CrsMatrixMulTaskMPI} и его полей,
    \item \texttt{ops\_mpi.cpp} - реализация методов \texttt{validation()}, \texttt{pre\_processing()}, \texttt{run()} и \texttt{post\_processing()}.
\end{itemize}

Таким образом, при запуске на \(p\) процессах мы распараллеливаем вычислительную нагрузку по строкам \(A\) и в завершении аккуратно сливаем все локальные результаты в итоговую разреженную матрицу \(C\). 


%---------------------- Результаты экспериментов -------------------

\section{Результаты экспериментов}

\subsection{Условия проведения экспериментов}

\begin{itemize}
    \item \textbf{Аппаратная конфигурация}: 
    \begin{itemize}
        \item CPU: Intel Core i5
        \item RAM: 32 GB
        \item ОС: Windows 11
    \end{itemize}
    \item \textbf{Тестовые размеры}: 
    \begin{itemize}
        \item Основные тесты: 4000$\times$4000 (плотность 1\%).
        \item Проверочные тесты: 3$\times$3, 1000$\times$1000 и т.д.
    \end{itemize}
\end{itemize}

\subsection{Пример замера времени}
Для сравнения взяты:
\begin{itemize}
    \item \textbf{Последовательная (seq) версия}
    \item \textbf{MPI-версия}: 2 процесса
    \item \textbf{MPI-версия}: 4 процесса
\end{itemize}

Ниже приведены выдержки из протокола тестов:

\begin{itemize}
    \item \textbf{SEQ (последовательная версия)}: 
    \begin{verbatim}
    [ RUN      ] borisov_s_crs_seq_test.Test_Pipeline_Run
    pc-2024-autumn\tasks\seq\borisov_s_crs_mul:pipeline:0.8324172000
    [       OK ] ... (1255 ms)
    ...
    [ RUN      ] borisov_s_crs_seq_test.Test_Task_Run
    pc-2024-autumn\tasks\seq\borisov_s_crs_mul:task_run:1.2700391000
    [       OK ] ... (1754 ms)
    Итого: ~3011 ms
    \end{verbatim}

    \item \textbf{MPI, 2 процесса}:
    \begin{verbatim}
    [ RUN      ] borisov_s_crs_mpi_test.Test_Pipeline_Run
    pc-2024-autumn\tasks\mpi\borisov_s_crs_mul:pipeline:1.2260979000
    [       OK ] ... (1707 ms)
    ...
    [ RUN      ] borisov_s_crs_mpi_test.Test_Task_Run
    pc-2024-autumn\tasks\mpi\borisov_s_crs_mul:task_run:1.4502339000
    [       OK ] ... (2065 ms)
    Итого: ~3773 ms
    \end{verbatim}

    \item \textbf{MPI, 4 процесса}:
    \begin{verbatim}
    [ RUN      ] borisov_s_crs_mpi_test.Test_Pipeline_Run
    pc-2024-autumn\tasks\mpi\borisov_s_crs_mul:pipeline:0.8284928000
    [       OK ] ... (1213 ms)
    ...
    [ RUN      ] borisov_s_crs_mpi_test.Test_Task_Run
    pc-2024-autumn\tasks\mpi\borisov_s_crs_mul:task_run:1.2442223000
    [       OK ] ... (1848 ms)
    Итого: ~3261 ms
    \end{verbatim}
\end{itemize}

\subsection{Вывод по результатам}
\begin{enumerate}
    \item При размере 4000$\times$4000 (плотность $\approx 1\%$) на данном оборудовании параллельная реализация всё ещё уступает или почти не выигрывает у последовательной из-за накладных расходов.
    \item На 2 процессах суммарное время оказалось ~3773 ms, на 4 процессах ~3261 ms, тогда как последовательная версия показала ~3011 ms.
    \item Возможная причина: при относительно «небольших» матрицах (с учётом их плотности) коммуникационные затраты превышают выигрыш от распараллеливания.
    \item На более крупных задачах или с оптимизацией обмена данных MPI-вариант может показать лучшую масштабируемость.
\end{enumerate}

%---------------------- Выводы -------------------------------------

\section{Выводы}
В ходе работы:
\begin{itemize}
    \item Реализовано умножение разреженных матриц (double) в формате CRS.
    \item Осуществлены последовательная и MPI-версии, прошедшие функциональное тестирование (корректность).
    \item Изучена производительность (2 и 4 процесса). На тестовом примере (4000$\times$4000, плотность 1\%) не получено ускорения относительно последовательной реализации.
\end{itemize}

Основные направления улучшения:
\begin{itemize}
    \item Оптимизировать порядок коммуникации и передачу данных.
    \item Использовать другие схемы разбиения (например, блочную для обоих измерений).
    \item Пробовать другие форматы разреженных матриц (например, ELLPACK, CSR5) или гибридные подходы.
\end{itemize}

%---------------------- Список литературы --------------------------

\section{Список литературы}
\begin{enumerate}
    \item Документация по Boost.MPI: \url{https://www.boost.org/doc/libs/master/doc/html/mpi/tutorial.html}
    \item Лекции: \url{https://disk.yandex.ru/d/LxDwnOlUo3Eqfg}
\end{enumerate}

%---------------------- Приложения ---------------------------------
\appendix
\section*{Приложение}
\addcontentsline{toc}{section}{Приложение}

Для полноты приводятся тексты файлов с реализацией последовательной и MPI-версий умножения разреженных матриц.

%-------------------------------------------------------------------
\subsection*{Файл \texttt{ops\_seq.hpp}}

\begin{lstlisting}
#pragma once

#include <string>
#include <vector>

#include "core/task/include/task.hpp"

namespace borisov_s_crs_mul {

class CrsMatrixMulTask : public ppc::core::Task {
 public:
  explicit CrsMatrixMulTask(std::shared_ptr<ppc::core::TaskData> taskData_) : Task(std::move(taskData_)) {}

  bool pre_processing() override;
  bool validation() override;
  bool run() override;
  bool post_processing() override;

  void initialize(const std::vector<double>& A_values, const std::vector<int>& A_col_index,
                  const std::vector<int>& A_row_ptr, const std::vector<double>& B_values,
                  const std::vector<int>& B_col_index, const std::vector<int>& B_row_ptr, int A_nrows, int A_ncols,
                  int B_nrows, int B_ncols) {
    A_values_ = A_values;
    A_col_index_ = A_col_index;
    A_row_ptr_ = A_row_ptr;

    B_values_ = B_values;
    B_col_index_ = B_col_index;
    B_row_ptr_ = B_row_ptr;

    A_nrows_ = A_nrows;
    A_ncols_ = A_ncols;
    B_nrows_ = B_nrows;
    B_ncols_ = B_ncols;

    C_nrows_ = A_nrows;
    C_ncols_ = B_ncols;
  }

 private:
  std::vector<double> A_values_;
  std::vector<int> A_col_index_;
  std::vector<int> A_row_ptr_;

  std::vector<double> B_values_;
  std::vector<int> B_col_index_;
  std::vector<int> B_row_ptr_;

  std::vector<double> C_values_;
  std::vector<int> C_col_index_;
  std::vector<int> C_row_ptr_;

  int A_nrows_, A_ncols_, A_nnz_;
  int B_nrows_, B_ncols_, B_nnz_;
  int C_nrows_, C_ncols_;
  int C_nnz_;
};

}  // namespace borisov_s_crs_mul
\end{lstlisting}

%-------------------------------------------------------------------
\subsection*{Файл \texttt{ops\_seq.cpp}}

\begin{lstlisting}
#include "seq/borisov_s_crs_mul/include/ops_seq.hpp"

#include <algorithm>

using namespace std::chrono_literals;

namespace borisov_s_crs_mul {

bool CrsMatrixMulTask::validation() {
  internal_order_test();

  if (taskData->inputs.size() != 6 || taskData->outputs.size() != 3) {
    return false;
  }

  if (taskData->inputs_count.size() < 6 || taskData->outputs_count.size() < 3) {
    return false;
  }

  int A_nrows = static_cast<int>(taskData->inputs_count[2] - 1);
  int B_nrows = static_cast<int>(taskData->inputs_count[5] - 1);

  return A_nrows > 0 && B_nrows > 0;
}

bool CrsMatrixMulTask::pre_processing() {
  internal_order_test();

  A_nnz_ = static_cast<int>(taskData->inputs_count[0]);
  A_values_.assign(reinterpret_cast<const double*>(taskData->inputs[0]),
                   reinterpret_cast<const double*>(taskData->inputs[0]) + A_nnz_);
  A_col_index_.assign(reinterpret_cast<const int*>(taskData->inputs[1]),
                      reinterpret_cast<const int*>(taskData->inputs[1]) + A_nnz_);
  A_row_ptr_.assign(reinterpret_cast<const int*>(taskData->inputs[2]),
                    reinterpret_cast<const int*>(taskData->inputs[2]) + taskData->inputs_count[2]);
  A_nrows_ = static_cast<int>(taskData->inputs_count[2] - 1);
  A_ncols_ = *std::max_element(A_col_index_.begin(), A_col_index_.end()) + 1;

  B_nnz_ = static_cast<int>(taskData->inputs_count[3]);
  B_values_.assign(reinterpret_cast<const double*>(taskData->inputs[3]),
                   reinterpret_cast<const double*>(taskData->inputs[3]) + B_nnz_);
  B_col_index_.assign(reinterpret_cast<const int*>(taskData->inputs[4]),
                      reinterpret_cast<const int*>(taskData->inputs[4]) + B_nnz_);
  B_row_ptr_.assign(reinterpret_cast<const int*>(taskData->inputs[5]),
                    reinterpret_cast<const int*>(taskData->inputs[5]) + taskData->inputs_count[5]);
  B_nrows_ = static_cast<int>(taskData->inputs_count[5] - 1);
  B_ncols_ = *std::max_element(B_col_index_.begin(), B_col_index_.end()) + 1;

  C_nrows_ = A_nrows_;
  C_ncols_ = B_ncols_;
  C_row_ptr_.assign(C_nrows_ + 1, 0);
  C_values_.clear();
  C_col_index_.clear();

  return true;
}

bool CrsMatrixMulTask::run() {
  internal_order_test();

  std::vector<double> temp(C_ncols_, 0.0);

  for (int i = 0; i < C_nrows_; ++i) {
    std::fill(temp.begin(), temp.end(), 0.0);

    int startA = A_row_ptr_[i];
    int endA = A_row_ptr_[i + 1];

    for (int posA = startA; posA < endA; ++posA) {
      double a_val = A_values_[posA];
      int a_col = A_col_index_[posA];

      int startB = B_row_ptr_[a_col];
      int endB = B_row_ptr_[a_col + 1];

      for (int posB = startB; posB < endB; ++posB) {
        int b_col = B_col_index_[posB];
        double b_val = B_values_[posB];
        temp[b_col] += a_val * b_val;
      }
    }

    C_row_ptr_[i] = static_cast<int>(C_values_.size());
    for (int col = 0; col < C_ncols_; ++col) {
      if (temp[col] != 0.0) {
        C_values_.push_back(temp[col]);
        C_col_index_.push_back(col);
      }
    }
  }

  C_row_ptr_[C_nrows_] = static_cast<int>(C_values_.size());
  C_nnz_ = static_cast<int>(C_values_.size());

  return true;
}

bool CrsMatrixMulTask::post_processing() {
  internal_order_test();

  taskData->outputs_count[0] = static_cast<unsigned int>(C_values_.size());
  taskData->outputs_count[1] = static_cast<unsigned int>(C_col_index_.size());
  taskData->outputs_count[2] = static_cast<unsigned int>(C_row_ptr_.size());

  taskData->outputs[0] = reinterpret_cast<uint8_t*>(C_values_.data());
  taskData->outputs[1] = reinterpret_cast<uint8_t*>(C_col_index_.data());
  taskData->outputs[2] = reinterpret_cast<uint8_t*>(C_row_ptr_.data());

  return true;
}

}  // namespace borisov_s_crs_mul
\end{lstlisting}

%-------------------------------------------------------------------
\subsection*{Файл \texttt{ops\_mpi.hpp}}

\begin{lstlisting}
#pragma once

#include <gtest/gtest.h>

#include <boost/mpi/collectives.hpp>
#include <boost/mpi/communicator.hpp>
#include <memory>
#include <vector>

#include "core/task/include/task.hpp"

namespace borisov_s_crs_mul_mpi {

class CrsMatrixMulTaskMPI : public ppc::core::Task {
 public:
  explicit CrsMatrixMulTaskMPI(std::shared_ptr<ppc::core::TaskData> taskData_) : Task(std::move(taskData_)) {}

  bool pre_processing() override;
  bool validation() override;
  bool run() override;
  bool post_processing() override;

 private:
  std::vector<double> A_values_;
  std::vector<int> A_col_index_;
  std::vector<int> A_row_ptr_;

  std::vector<double> B_values_;
  std::vector<int> B_col_index_;
  std::vector<int> B_row_ptr_;

  std::vector<double> C_values_;
  std::vector<int> C_col_index_;
  std::vector<int> C_row_ptr_;

  int A_nrows_, A_ncols_;
  int B_nrows_, B_ncols_;
  int C_nrows_, C_ncols_;

  boost::mpi::communicator world;
};

}  // namespace borisov_s_crs_mul_mpi
\end{lstlisting}

%-------------------------------------------------------------------
\subsection*{Файл \texttt{ops\_mpi.cpp}}

\begin{lstlisting}
#include "mpi/borisov_s_crs_mul/include/ops_mpi.hpp"

#include <algorithm>
#include <boost/serialization/vector.hpp>
#include <functional>
#include <vector>

using namespace std::chrono_literals;

namespace borisov_s_crs_mul_mpi {

bool CrsMatrixMulTaskMPI::pre_processing() {
  internal_order_test();

  if (world.rank() == 0) {
    A_values_.assign(reinterpret_cast<const double*>(taskData->inputs[0]),
                     reinterpret_cast<const double*>(taskData->inputs[0]) + taskData->inputs_count[0]);
    A_col_index_.assign(reinterpret_cast<const int*>(taskData->inputs[1]),
                        reinterpret_cast<const int*>(taskData->inputs[1]) + taskData->inputs_count[1]);
    A_row_ptr_.assign(reinterpret_cast<const int*>(taskData->inputs[2]),
                      reinterpret_cast<const int*>(taskData->inputs[2]) + taskData->inputs_count[2]);

    B_values_.assign(reinterpret_cast<const double*>(taskData->inputs[3]),
                     reinterpret_cast<const double*>(taskData->inputs[3]) + taskData->inputs_count[3]);
    B_col_index_.assign(reinterpret_cast<const int*>(taskData->inputs[4]),
                        reinterpret_cast<const int*>(taskData->inputs[4]) + taskData->inputs_count[4]);
    B_row_ptr_.assign(reinterpret_cast<const int*>(taskData->inputs[5]),
                      reinterpret_cast<const int*>(taskData->inputs[5]) + taskData->inputs_count[5]);

    A_nrows_ = static_cast<int>(taskData->inputs_count[2] - 1);
    B_nrows_ = static_cast<int>(taskData->inputs_count[5] - 1);
    A_ncols_ = *std::max_element(A_col_index_.begin(), A_col_index_.end()) + 1;
    B_ncols_ = *std::max_element(B_col_index_.begin(), B_col_index_.end()) + 1;

    C_nrows_ = A_nrows_;
    C_ncols_ = B_ncols_;
    C_values_.clear();
    C_col_index_.clear();
    C_row_ptr_.assign(C_nrows_ + 1, 0);
  }

  return true;
}

bool CrsMatrixMulTaskMPI::validation() {
  internal_order_test();

  if (world.rank() == 0) {
    if (taskData->inputs.size() != 6 || taskData->outputs.size() != 3) {
      return false;
    }

    if (taskData->inputs_count.size() < 6 || taskData->outputs_count.size() < 3) {
      return false;
    }

    int A_ncols = *std::max_element(reinterpret_cast<const int*>(taskData->inputs[1]),
                                    reinterpret_cast<const int*>(taskData->inputs[1]) + taskData->inputs_count[1]) +
                  1;

    int B_nrows = static_cast<int>(taskData->inputs_count[5] - 1);

    if (A_ncols != B_nrows) {
      return false;
    }
  }

  return true;
}

bool CrsMatrixMulTaskMPI::run() {
  internal_order_test();

  boost::mpi::broadcast(world, A_values_, 0);
  boost::mpi::broadcast(world, A_col_index_, 0);
  boost::mpi::broadcast(world, A_row_ptr_, 0);
  boost::mpi::broadcast(world, B_values_, 0);
  boost::mpi::broadcast(world, B_col_index_, 0);
  boost::mpi::broadcast(world, B_row_ptr_, 0);
  boost::mpi::broadcast(world, C_nrows_, 0);
  boost::mpi::broadcast(world, C_ncols_, 0);

  int rows_per_process = C_nrows_ / world.size();
  int start_row = world.rank() * rows_per_process;
  int end_row = (world.rank() == world.size() - 1) ? C_nrows_ : start_row + rows_per_process;

  std::vector<double> temp(C_ncols_, 0.0);
  std::vector<double> local_values;
  std::vector<int> local_col_index;
  std::vector<int> local_row_ptr(end_row - start_row + 1, 0);

  int local_nnz_count = 0;

  for (int i = start_row; i < end_row; ++i) {
    std::fill(temp.begin(), temp.end(), 0.0);

    for (int posA = A_row_ptr_[i]; posA < A_row_ptr_[i + 1]; ++posA) {
      double a_val = A_values_[posA];
      int a_col = A_col_index_[posA];
      for (int posB = B_row_ptr_[a_col]; posB < B_row_ptr_[a_col + 1]; ++posB) {
        int b_col = B_col_index_[posB];
        double b_val = B_values_[posB];
        temp[b_col] += a_val * b_val;
      }
    }

    local_row_ptr[i - start_row] = local_nnz_count;
    for (int col = 0; col < C_ncols_; ++col) {
      if (temp[col] != 0.0) {
        local_values.push_back(temp[col]);
        local_col_index.push_back(col);
        ++local_nnz_count;
      }
    }
  }

  local_row_ptr[end_row - start_row] = local_nnz_count;

  if (world.rank() == 0) {
    std::vector<std::vector<double>> all_values(world.size());
    std::vector<std::vector<int>> all_col_indices(world.size());
    std::vector<std::vector<int>> all_row_ptrs(world.size());

    boost::mpi::gather(world, local_values, all_values, 0);
    boost::mpi::gather(world, local_col_index, all_col_indices, 0);
    boost::mpi::gather(world, local_row_ptr, all_row_ptrs, 0);

    for (int i = 0; i < world.size(); ++i) {
      C_values_.insert(C_values_.end(), all_values[i].begin(), all_values[i].end());
      C_col_index_.insert(C_col_index_.end(), all_col_indices[i].begin(), all_col_indices[i].end());

      if (i == 0) {
        C_row_ptr_ = all_row_ptrs[i];
      } else {
        int offset = C_row_ptr_.back();
        for (size_t j = 1; j < all_row_ptrs[i].size(); ++j) {
          C_row_ptr_.push_back(all_row_ptrs[i][j] + offset);
        }
      }
    }
  } else {
    boost::mpi::gather(world, local_values, 0);
    boost::mpi::gather(world, local_col_index, 0);
    boost::mpi::gather(world, local_row_ptr, 0);
  }

  return true;
}

bool CrsMatrixMulTaskMPI::post_processing() {
  internal_order_test();

  if (world.rank() == 0) {
    taskData->outputs_count[0] = C_values_.size();
    taskData->outputs_count[1] = C_col_index_.size();
    taskData->outputs_count[2] = C_row_ptr_.size();

    taskData->outputs[0] = reinterpret_cast<uint8_t*>(C_values_.data());
    taskData->outputs[1] = reinterpret_cast<uint8_t*>(C_col_index_.data());
    taskData->outputs[2] = reinterpret_cast<uint8_t*>(C_row_ptr_.data());
  }

  return true;
}

}  // namespace borisov_s_crs_mul_mpi
\end{lstlisting}

\end{document}
