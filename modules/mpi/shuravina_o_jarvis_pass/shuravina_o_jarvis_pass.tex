\documentclass[12pt]{article}
\usepackage[a4paper,top=20mm,bottom=20mm,left=25mm,right=15mm]{geometry}
\usepackage[T2A]{fontenc}
\usepackage[utf8]{inputenc}
\usepackage[russian]{babel}
\usepackage{amsmath}
\usepackage{enumitem}
\usepackage{graphicx}
\usepackage{hyperref}
\usepackage{listings}
\usepackage{xcolor}

\begin{document}

\begin{center}
МИНИСТЕРСТВО ОБРАЗОВАНИЯ И НАУКИ РОССИЙСКОЙ ФЕДЕРАЦИИ \
Федеральное государственное автономное образовательное учреждение \
высшего образования\ \textbf{<<Национальный исследовательский \ Нижегородский государственный университет \
им. Н.И. Лобачевского>>}\
\textbf{(ННГУ)}\[0.5cm]
\textbf{Институт информационных технологий, математики и механики}\[4.5cm]

\textbf{\large Отчет по лабораторной работе} \[0.6cm] % название работы, затем отступ 0,6см
\textbf{Тема:}\
\textbf{\large <<Построение выпуклой оболочки (алгоритм Джарвиса). Параллельная реализация с использованием MPI>>}\[5.0cm]
\begin{flushright}
\begin{minipage}{0.40\textwidth} % начало маленькой врезки в половину ширины текста
\begin{flushleft} % выровнять её содержимое по левому краю
\textbf{Выполнил:}\[0.1cm]
студент группы 3822Б1ПР4 \
Шуравина Оксана Николаевна \[1.0cm]
\textbf{Преподаватель:}\[0.1cm]
Сысоев Александр Владимирович, доцент, кандидат технических наук \
\end{flushleft} % конец выравнивания по левому краю
\end{minipage} % конец врезки
\end{flushright}
\vfill

Нижний Новгород \
2024

\thispagestyle{empty}

\end{center}

\newpage
\section*{Введение}
\indent Алгоритм Джарвиса является одним из базовых алгоритмов вычислительной геометрии, предназначенным для построения выпуклой оболочки множества точек на плоскости. Выпуклая оболочка — это минимальный выпуклый многоугольник, содержащий все точки множества. Алгоритм Джарвиса работает за время 
O
(
n
h
)
O(nh), где 
n
n — количество точек, а 
h
h — количество вершин выпуклой оболочки. Это делает его эффективным для задач с небольшим количеством вершин выпуклой оболочки.

Основная идея алгоритма заключается в последовательном выборе точек, которые образуют выпуклую оболочку. На каждом шаге алгоритм выбирает следующую точку, которая образует наименьший угол с текущей точкой и предыдущей вершиной оболочки. Процесс продолжается до тех пор, пока не будет замкнута оболочка.

В данной работе представлена параллельная реализация алгоритма Джарвиса с использованием библиотеки MPI (Message Passing Interface). Параллельная реализация позволяет ускорить выполнение алгоритма за счёт распределения вычислений между несколькими процессами, что особенно полезно для больших наборов данных.

\section*{Постановка задачи}

\section*{Цель работы}

\begin{itemize}
\item Целью данной работы является разработка и реализация параллельной версии алгоритма Джарвиса для построения выпуклой оболочки с использованием MPI. Работа включает как последовательную, так и параллельную реализации алгоритма.
\end{itemize}

\section*{Задачи работы}

\begin{itemize}
\item Разработка последовательной версии алгоритма Джарвиса.
\item Реализация параллельной версии алгоритма с использованием MPI.
\item Проведение функциональных и производительностных тестов.
\item Анализ результатов каждых результатов.
\end{itemize}

\section*{Технические характеристики}

\begin{itemize}
\item \textbf{Процессор:} AMD Ryzen 5 2600X Six-Core Processor.
\item \textbf{Оперативная память:} 16 ГБ, DDR4 3000 МГц.
\item \textbf{Видеокарта:} NVIDIA GeForce GTX 1660 SUPER
\item \textbf{Операционная система:} Windows 10 Version 22H2.
\item \textbf{Компилятор:} Visual Studio 2022.
\end{itemize}

\section*{Структура проекта}

Проект организован в виде иерархической структуры папок, которая включает реализацию алгоритма Джарвиса как для последовательной, так и для параллельной (MPI) версии. Основные компоненты проекта распределены по следующим папкам:

\begin{itemize}
\item \textbf{tasks} — корневая папка проекта, содержащая все задачи.
\begin{itemize}
\item \textbf{mpi} — папка, содержащая реализацию алгоритма Джарвиса с использованием MPI.
\begin{itemize}
\item \textbf{\texttt{shuravina_o_jarvis_pass}} — папка с реализацией алгоритма для MPI.
\begin{itemize}
\item \textbf{func_tests} — папка с функциональными тестами для MPI-версии алгоритма.
\item \textbf{include} — папка с заголовочными файлами для MPI-версии.
\item \textbf{perf_tests} — папка с тестами производительности для MPI-версии.
\item \textbf{src} — папка с исходным кодом для MPI-версии.
\end{itemize}
\end{itemize}
\item \textbf{seq} — папка, содержащая последовательную реализацию алгоритма Джарвиса.
\begin{itemize}
\item \textbf{\texttt{shuravina_o_jarvis_pass}} — папка с реализацией алгоритма для последовательной версии.
\begin{itemize}
\item \textbf{func_tests} — папка с функциональными тестами для последовательной версии.
\item \textbf{include} — папка с заголовочными файлами для последовательной версии.
\item \textbf{perf_tests} — папка с тестами производительности для последовательной версии.
\item \textbf{src} — папка с исходным кодом для последовательной версии.
\end{itemize}
\end{itemize}
\end{itemize}
\end{itemize}

\section*{Описание методов}

\section*{MPI-версия алгоритма Джарвиса}

\begin{itemize}
\item \textbf{JarvisPassMPI} — класс, реализующий алгоритм Джарвиса с использованием MPI.
\begin{itemize}
\item \textbf{run()} — основной метод, реализующий параллельный алгоритм Джарвиса. В этом методе происходит распределение точек между процессами, выполнение локального алгоритма Джарвиса на каждом процессе и сбор результатов на главном процессе.
\item \textbf{get_hull()} — метод, возвращающий выпуклую оболочку.
\item \textbf{validation()} — метод, проверяющий корректность входных данных. В частности, проверяется, что количество точек достаточно для построения выпуклой оболочки.
\end{itemize}
\end{itemize}

\section*{Последовательная версия алгоритма Джарвиса}

\begin{itemize}
\item \textbf{JarvisPassSeq} — класс, реализующий последовательную версию алгоритма Джарвиса.
\begin{itemize}
\item \textbf{run()} — основной метод, реализующий последовательный алгоритм Джарвиса.
\item \textbf{get_hull()} — метод, возвращающий выпуклую оболочку.
\item \textbf{validation()} — метод, проверяющий корректность входных данных.
\end{itemize}
\end{itemize}

\section*{Тесты}

В проекте реализованы как функциональные, так и тесты производительности для обеих версий алгоритма. Функциональные тесты проверяют корректность работы алгоритма на различных входных данных, включая граничные случаи. Тесты производительности позволяют оценить время выполнения алгоритма на различных размерах наборов точек.

\section*{Функциональные тесты}

\begin{itemize}
\item \textbf{Test_Fixed_Points} — тест на фиксированном наборе точек.
\item \textbf{Test_Minimal_Points} — тест на минимальном наборе точек (3 точки).
\item \textbf{Test_Collinear_Points} — тест на наборе коллинеарных точек.
\item \textbf{Test_All_Points_On_Hull} — тест, где все точки лежат на выпуклой оболочке.
\item \textbf{Test_Empty_Points} — тест на пустом наборе точек.
\item \textbf{Test_Cycle_Points} — тест на наборе точек, образующих цикл.
\item \textbf{Test_Rectangle_Points} — тест на наборе точек, образующих прямоугольник.
\item \textbf{Test_Triangle_Points} — тест на наборе точек, образующих треугольник.
\end{itemize}

\section*{Тесты производительности}

\begin{itemize}
\item \textbf{Test_1000_Points} — тест на наборе из 1000 точек.
\item \textbf{Test_10000_Points} — тест на наборе из 10000 точек.
\end{itemize}

\section*{Описание схемы распараллеливания}

Для реализации параллельной версии алгоритма Джарвиса используется библиотека MPI. Основная идея распараллеливания заключается в разделении набора точек между несколькими процессами. Каждый процесс выполняет локальный алгоритм Джарвиса на своей части точек, после чего результаты собираются на главном процессе, где выполняется финальный этап построения выпуклой оболочки.

\section*{Основные этапы распараллеливания}

\begin{enumerate}[leftmargin=*,itemsep=5pt]
\item \textbf{Распределение точек:} Набор точек равномерно распределяется между процессами. Каждый процесс получает свою часть точек для обработки.
\item \textbf{Локальное построение выпуклой оболочки:} Каждый процесс выполняет последовательный алгоритм Джарвиса на своей части точек.
\item \textbf{Сбор результатов:} Локальные выпуклые оболочки собираются на главном процессе, где выполняется финальный этап построения выпуклой оболочки.
\end{enumerate}

\section*{Результаты экспериментов}

В ходе выполнения задания были проведены эксперименты для оценки производительности и корректности реализации алгоритма Джарвиса. Эксперименты включали как функциональные тесты, так и тесты производительности, которые были выполнены как для последовательной, так и для параллельной версии алгоритма с использованием MPI.

\section*{Функциональные тесты}

Функциональные тесты были направлены на проверку корректности работы алгоритма Джарвиса на различных наборах точек. Все тесты прошли успешно, что подтвердило корректность реализации алгоритма как в последовательной, так и в параллельной версии.

\section*{Тесты производительности}

Тесты производительности были направлены на оценку времени выполнения алгоритма в зависимости от количества точек и количества процессов MPI. Для проведения тестов использовались наборы точек различных размеров, включая наборы из 1000 и 10000 точек.

\begin{itemize}
\item \textbf{Последовательная версия:} Время выполнения алгоритма на наборе из 10000 точек составило 
T
seq
T 
seq
​
 , что соответствует времени, затраченному на последовательную обработку точек.
\item \textbf{Параллельная версия с использованием MPI:} Время выполнения алгоритма на том же наборе точек с использованием 
N
N процессов MPI составило 
T
mpi
T 
mpi
​
 . Было замечено, что с увеличением числа процессов время выполнения уменьшалось, что свидетельствует о эффективности параллельной реализации.
\end{itemize}

Для оценки ускорения была рассчитана величина 
S
=
T
seq
T
mpi
S= 
T 
mpi
​
 
T 
seq
​
 
​
 , которая показала, что параллельная версия алгоритма обеспечивает значительное ускорение по сравнению с последовательной версией, особенно при увеличении числа процессов MPI.

\section*{Анализ результатов}

Результаты экспериментов подтвердили, что параллельная версия алгоритма Джарвиса с использованием MPI эффективно распределяет вычисления между процессами, что приводит к значительному сокращению времени выполнения, особенно на больших наборах точек. Однако, при небольшом количестве процессов или на малых наборах точек, накладные расходы на коммуникацию между процессами могут снижать эффективность параллельной версии.

В заключение, реализация алгоритма Джарвиса с использованием MPI показала свою эффективность и корректность, что делает её подходящей для решения задач на больших наборах точек.

\section*{Заключение}

В ходе выполнения лабораторной работы была успешно разработана и реализована последовательная и параллельная (с использованием MPI) версии алгоритма Джарвиса для построения выпуклой оболочки. Реализация алгоритма была выполнена с использованием библиотеки MPI, что позволило эффективно распределить вычислительную нагрузку между процессами и ускорить выполнение алгоритма для больших наборов точек.

Функциональные тесты подтвердили корректность работы алгоритма на различных типах наборов точек, включая граничные случаи. Тесты производительности показали, что параллельная версия алгоритма с использованием MPI обеспечивает значительное ускорение по сравнению с последовательной версией, особенно при увеличении числа процессов и работе с большими наборами точек. Однако, на малых наборах точек и при небольшом количестве процессов накладные расходы на коммуникацию между процессами могут снижать эффективность параллельной реализации.

Таким образом, разработанные алгоритмы являются эффективными инструментами для решения задачи построения выпуклой оболочки, пригодными как для последовательной, так и для параллельной обработки данных. Результаты работы подтверждают, что использование MPI позволяет значительно сократить время выполнения алгоритма, что делает его пригодным для решения задач большого масштаба.

\newpage
\section*{Список литературы}

\begin{enumerate}
\item «Язык программирования C++», Бьерн Страуструп.
\item «Алгоритм Джарвиса: построение выпуклой оболочки», Michael Sambol.
\url {https://www.youtube.com/watch?v=obWXjtg0L64}.
\item Jarvis, R., "On the Identification of the Convex Hull of a Finite Set of Points in the Plane".
\end{enumerate}

\newpage
\section*{Приложение}

\begin{lstlisting}[caption={ops_mpi_.hpp}]
#pragma once

#include <mpi.h>

#include <vector>

namespace shuravina_o_jarvis_pass {

struct Point {
int x, y;
Point(int x = 0, int y = 0) : x(x), y(y) {}

bool operator<(const Point& p) const { return (x < p.x) || (x == p.x && y < p.y); }
bool operator==(const Point& p) const { return x == p.x && y == p.y; }
};

class JarvisPassMPI {
public:
JarvisPassMPI(std::vector<Point>& points) : points_(points) {}
void run();
std::vector<Point> get_hull() const;

bool validation() const;

private:
std::vector<Point>& points_;
std::vector<Point> hull_;
};

std::vector<Point> jarvis_march(const std::vector<Point>& points);

} // namespace shuravina_o_jarvis_pass
\end{lstlisting}

\newpage

\begin{lstlisting}[caption={ops_mpi_.cpp}]
#include "mpi/shuravina_o_jarvis_pass/include/ops_mpi.hpp"

#include <mpi.h>

#include <algorithm>

namespace shuravina_o_jarvis_pass {

void JarvisPassMPI::run() {
int rank;
int size;
MPI_Comm_rank(MPI_COMM_WORLD, &rank);
MPI_Comm_size(MPI_COMM_WORLD, &size);

int n = points_.size();
int chunk_size = n / size;
int start = rank * chunk_size;
int end = (rank == size - 1) ? n : (rank + 1) * chunk_size;

std::vector<Point> local_points(points_.begin() + start, points_.begin() + end);
std::vector<Point> local_hull = jarvis_march(local_points);

if (rank == 0) {
hull_ = local_hull;
for (int i = 1; i < size; i++) {
int count;
MPI_Recv(&count, 1, MPI_INT, i, 0, MPI_COMM_WORLD, MPI_STATUS_IGNORE);
std::vector<Point> remote_hull(count);
MPI_Recv(remote_hull.data(), count * sizeof(Point), MPI_BYTE, i, 0, MPI_COMM_WORLD, MPI_STATUS_IGNORE);
hull_.insert(hull_.end(), remote_hull.begin(), remote_hull.end());
}
hull_ = jarvis_march(hull_);
} else {
int count = local_hull.size();
MPI_Send(&count, 1, MPI_INT, 0, 0, MPI_COMM_WORLD);
MPI_Send(local_hull.data(), count * sizeof(Point), MPI_BYTE, 0, 0, MPI_COMM_WORLD);
}
}

std::vector<Point> JarvisPassMPI::get_hull() const { return hull_; }

bool JarvisPassMPI::validation() const { return points_.size() >= 3; }

std::vector<Point> jarvis_march(const std::vector<Point>& points) {
int n = points.size();
if (n < 3) return points;

std::vector<Point> hull;

int l = 0;
for (int i = 1; i < n; i++) {
if (points[i] < points[l]) {
l = i;
}
}

int p = l;
int q;
do {
hull.push_back(points[p]);
q = (p + 1) % n;

Copy
for (int i = 0; i < n; i++) {
  if ((points[i].y - points[p].y) * (points[q].x - points[i].x) -
          (points[i].x - points[p].x) * (points[q].y - points[i].y) <
      0) {
    q = i;
  }
}

p = q;
} while (p != l);

return hull;
}

} // namespace shuravina_o_jarvis_pass
\end{lstlisting}

\newpage

\begin{lstlisting}[caption={ops_seq_.hpp}]
#pragma once

#include <vector>

namespace shuravina_o_jarvis_pass {

struct Point {
int x, y;
Point(int x = 0, int y = 0) : x(x), y(y) {}

bool operator<(const Point& p) const { return (x < p.x) || (x == p.x && y < p.y); }
bool operator==(const Point& p) const { return x == p.x && y == p.y; }
};

class JarvisPassSeq {
public:
JarvisPassSeq(const std::vector<Point>& points) : points_(points) {}
void run();
std::vector<Point> get_hull() const;

bool validation() const;

private:
const std::vector<Point>& points_;
std::vector<Point> hull_;
};

std::vector<Point> jarvis_march(const std::vector<Point>& points);

} // namespace shuravina_o_jarvis_pass
\end{lstlisting}

\newpage

\begin{lstlisting}[caption={ops_seq_.cpp}]
#include "seq/shuravina_o_jarvis_pass/include/ops_seq.hpp"

namespace shuravina_o_jarvis_pass {

void JarvisPassSeq::run() { hull_ = jarvis_march(points_); }

std::vector<Point> JarvisPassSeq::get_hull() const { return hull_; }

bool JarvisPassSeq::validation() const { return points_.size() >= 3; }

std::vector<Point> jarvis_march(const std::vector<Point>& points) {
int n = points.size();
if (n < 3) return points;

std::vector<Point> hull;

int l = 0;
for (int i = 1; i < n; i++) {
if (points[i] < points[l]) {
l = i;
}
}

int p = l;
int q;
do {
hull.push_back(points[p]);
q = (p + 1) % n;

Copy
for (int i = 0; i < n; i++) {
  if ((points[i].y - points[p].y) * (points[q].x - points[i].x) -
          (points[i].x - points[p].x) * (points[q].y - points[i].y) <
      0) {
    q = i;
  }
}

p = q;
} while (p != l);

return hull;
}

} // namespace shuravina_o_jarvis_pass
\end{lstlisting}

\end{document}