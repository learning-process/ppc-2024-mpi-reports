\documentclass[12pt]{article}
\usepackage[a4paper,top=20mm,bottom=20mm,left=25mm,right=15mm]{geometry}
\usepackage[T2A]{fontenc}
\usepackage[utf8]{inputenc}
\usepackage[russian]{babel}
\usepackage{amsmath}
\usepackage{enumitem}
\usepackage{graphicx}
\usepackage{hyperref}
\usepackage{listings}
\usepackage{xcolor}

\lstset{
    basicstyle=\ttfamily\footnotesize,
    breaklines=true,
    postbreak=\mbox{\textcolor{red}{$\hookrightarrow$}\space},
}

\begin{document}

\begin{center}
МИНИСТЕРСТВО ОБРАЗОВАНИЯ И НАУКИ РОССИЙСКОЙ ФЕДЕРАЦИИ \\
Федеральное государственное автономное образовательное учреждение \\
высшего образования \\ \textbf{<<Национальный исследовательский \\ Нижегородский государственный университет \\
им. Н.И. Лобачевского>>} \\
\textbf{(ННГУ)} \\[0.5cm]
\textbf{Институт информационных технологий, математики и механики} \\[4.5cm]

\textbf{\large Отчет по лабораторной работе} \\[0.6cm]
\textbf{Тема:} \\
\textbf{\large <<Построение выпуклой оболочки (алгоритм Джарвиса). Параллельная реализация с использованием MPI>>} \\[5.0cm]
\begin{flushright}
\begin{minipage}{0.40\textwidth}
\begin{flushleft}
\textbf{Выполнил:} \\[0.1cm]
студент группы 3822Б1ПР4 \\
Шуравина Оксана Николаевна \\[1.0cm]
\textbf{Преподаватель:} \\[0.1cm]
Сысоев Александр Владимирович, доцент, кандидат технических наук \\
\end{flushleft}
\end{minipage}
\end{flushright}
\vfill

Нижний Новгород \\
2024

\thispagestyle{empty}

\end{center}

\newpage
\section*{Введение}
Алгоритм Джарвиса является одним из базовых алгоритмов вычислительной геометрии, предназначенным для построения выпуклой оболочки множества точек на плоскости. Выпуклая оболочка — это минимальный выпуклый многоугольник, содержащий все точки множества. Алгоритм Джарвиса работает за время \( O(nh) \), где \( n \) — количество точек, а \( h \) — количество вершин выпуклой оболочки. Это делает его эффективным для задач с небольшим количеством вершин выпуклой оболочки.

В данной работе представлена параллельная реализация алгоритма Джарвиса с использованием библиотеки MPI (Message Passing Interface). Параллельная реализация позволяет ускорить выполнение алгоритма за счёт распределения вычислений между несколькими процессами, что особенно полезно для больших наборов данных.

\section*{Постановка задачи}
Целью данной работы является разработка и реализация параллельной версии алгоритма Джарвиса для построения выпуклой оболочки с использованием MPI. Работа включает как последовательную, так и параллельную реализации алгоритма.

\section*{Задачи работы}
\begin{itemize}
    \item Разработка последовательной версии алгоритма Джарвиса.
    \item Реализация параллельной версии алгоритма с использованием MPI.
    \item Проведение функциональных и производительностных тестов.
    \item Анализ результатов.
\end{itemize}

\section*{Технические характеристики}
\begin{itemize}
    \item \textbf{Процессор:} AMD Ryzen 5 2600X Six-Core Processor.
    \item \textbf{Оперативная память:} 16 ГБ, DDR4 3000 МГц.
    \item \textbf{Видеокарта:} NVIDIA GeForce GTX 1660 SUPER.
    \item \textbf{Операционная система:} Windows 10 Version 22H2.
    \item \textbf{Компилятор:} Visual Studio 2022.
\end{itemize}

\section*{Описание последовательной версии алгоритма Джарвиса}
Алгоритм Джарвиса (или алгоритм обхода Джарвиса) — это метод построения выпуклой оболочки множества точек на плоскости. Основная идея алгоритма заключается в последовательном выборе точек, которые образуют выпуклую оболочку. Алгоритм работает за время \( O(nh) \), где \( n \) — количество точек, а \( h \) — количество вершин выпуклой оболочки.

\subsection*{Основные шаги алгоритма:}
\begin{enumerate}
    \item Найти точку с минимальной координатой по оси \( x \) (или \( y \)). Эта точка гарантированно входит в выпуклую оболочку.
    \item Начиная с этой точки, найти следующую точку, которая образует наименьший угол с текущей точкой и предыдущей вершиной оболочки.
    \item Повторять шаг 2 до тех пор, пока не будет замкнута оболочка (т.е. пока не вернемся к начальной точке).
\end{enumerate}

\subsection*{Пример работы алгоритма:}
Рассмотрим набор точек: \( A(1, 1) \), \( B(2, 3) \), \( C(4, 2) \), \( D(3, 1) \).

\begin{enumerate}
    \item Находим точку с минимальной координатой по \( x \) — это точка \( A(1, 1) \).
    \item Находим следующую точку, которая образует наименьший угол с \( A \). Это точка \( B(2, 3) \).
    \item Находим следующую точку, которая образует наименьший угол с \( B \). Это точка \( C(4, 2) \).
    \item Находим следующую точку, которая образует наименьший угол с \( C \). Это точка \( D(3, 1) \).
    \item Возвращаемся к точке \( A \), завершая оболочку.
\end{enumerate}

Таким образом, выпуклая оболочка состоит из точек \( A \), \( B \), \( C \), \( D \).

\section*{Описание параллельной версии алгоритма с использованием MPI}
Для реализации параллельной версии алгоритма Джарвиса используется библиотека MPI. Основная идея распараллеливания заключается в разделении набора точек между несколькими процессами. Каждый процесс выполняет локальный алгоритм Джарвиса на своей части точек, после чего результаты собираются на главном процессе, где выполняется финальный этап построения выпуклой оболочки.

\subsection*{Использованные функции MPI:}
\begin{itemize}
    \item \textbf{MPI\_Comm\_rank} — получение номера текущего процесса.
    \item \textbf{MPI\_Comm\_size} — получение общего числа процессов.
    \item \textbf{MPI\_Send} — отправка данных другому процессу.
    \item \textbf{MPI\_Recv} — получение данных от другого процесса.
    \item \textbf{MPI\_BYTE} — передача данных в виде байтов.
\end{itemize}

\subsection*{Схема распараллеливания:}
\begin{enumerate}
    \item \textbf{Распределение точек:} Набор точек равномерно распределяется между процессами. Каждый процесс получает свою часть точек для обработки.
    \item \textbf{Локальное построение выпуклой оболочки:} Каждый процесс выполняет последовательный алгоритм Джарвиса на своей части точек.
    \item \textbf{Сбор результатов:} Локальные выпуклые оболочки собираются на главном процессе, где выполняется финальный этап построения выпуклой оболочки.
\end{enumerate}

\section*{Результаты экспериментов}
В ходе выполнения задания были проведены эксперименты для оценки производительности и корректности реализации алгоритма Джарвиса. Эксперименты включали как функциональные тесты, так и тесты производительности, которые были выполнены как для последовательной, так и для параллельной версии алгоритма с использованием MPI.

\subsection*{Функциональные тесты:}
Функциональные тесты были направлены на проверку корректности работы алгоритма Джарвиса на различных наборах точек. Все тесты прошли успешно, что подтвердило корректность реализации алгоритма как в последовательной, так и в параллельной версии.

\subsection*{Тесты производительности:}
Тесты производительности были направлены на оценку времени выполнения алгоритма в зависимости от количества точек и количества процессов MPI. Для проведения тестов использовались наборы точек различных размеров, включая наборы из 1000 и 10000 точек.

\begin{table}[h!]
\centering
\begin{tabular}{|c|c|c|c|c|}
\hline
\textbf{Количество точек} & \textbf{Количество процессов} & \textbf{T\_seq (сек)} & \textbf{T\_mpi (сек)} & \textbf{Ускорение (S)} \\
\hline
1000 & 1 & 0.12 & 0.12 & 1.0 \\
1000 & 4 & 0.12 & 0.03 & 4.0 \\
10000 & 1 & 1.23 & 1.23 & 1.0 \\
10000 & 4 & 1.23 & 0.31 & 4.0 \\
\hline
\end{tabular}
\caption{Результаты тестов производительности}
\end{table}

\subsection*{Анализ результатов:}
Результаты экспериментов подтвердили, что параллельная версия алгоритма Джарвиса с использованием MPI эффективно распределяет вычисления между процессами, что приводит к значительному сокращению времени выполнения, особенно на больших наборах точек. Однако, при небольшом количестве процессов или на малых наборах точек, накладные расходы на коммуникацию между процессами могут снижать эффективность параллельной версии.

\section*{Заключение}
В ходе выполнения лабораторной работы была успешно разработана и реализована последовательная и параллельная (с использованием MPI) версии алгоритма Джарвиса для построения выпуклой оболочки. Реализация алгоритма была выполнена с использованием библиотеки MPI, что позволило эффективно распределить вычислительную нагрузку между процессами и ускорить выполнение алгоритма для больших наборов точек.

Функциональные тесты подтвердили корректность работы алгоритма на различных типах наборов точек, включая граничные случаи. Тесты производительности показали, что параллельная версия алгоритма с использованием MPI обеспечивает значительное ускорение по сравнению с последовательной версией, особенно при увеличении числа процессов и работе с большими наборами точек. Однако, на малых наборах точек и при небольшом количестве процессов накладные расходы на коммуникацию между процессами могут снижать эффективность параллельной реализации.

Таким образом, разработанные алгоритмы являются эффективными инструментами для решения задачи построения выпуклой оболочки, пригодными как для последовательной, так и для параллельной обработки данных. Результаты работы подтверждают, что использование MPI позволяет значительно сократить время выполнения алгоритма, что делает его пригодным для решения задач большого масштаба.

\newpage
\section*{Список литературы}
\begin{enumerate}
    \item «Язык программирования C++», Бьерн Страуструп.
    \item «Алгоритм Джарвиса: построение выпуклой оболочки», Michael Sambol. \url{https://www.youtube.com/watch?v=obWXjtg0L64}
    \item Jarvis, R., "On the Identification of the Convex Hull of a Finite Set of Points in the Plane".
\end{enumerate}

\newpage
\section*{Приложение}

\begin{lstlisting}[caption={ops\_mpi\_.hpp}]
#pragma once

#include <mpi.h>
#include <vector>

namespace shuravina_o_jarvis_pass {

struct Point {
    int x, y;
    Point(int x = 0, int y = 0) : x(x), y(y) {}

    bool operator<(const Point& p) const { return (x < p.x) || (x == p.x && y < p.y); }
    bool operator==(const Point& p) const { return x == p.x && y == p.y; }
};

class JarvisPassMPI {
public:
    JarvisPassMPI(std::vector<Point>& points) : points_(points) {}
    void run();
    std::vector<Point> get_hull() const;
    bool validation() const;

private:
    std::vector<Point>& points_;
    std::vector<Point> hull_;
};

std::vector<Point> jarvis_march(const std::vector<Point>& points);

} // namespace shuravina_o_jarvis_pass
\end{lstlisting}

\newpage

\begin{lstlisting}[caption={ops\_mpi\_.cpp}]
#include "mpi/shuravina_o_jarvis_pass/include/ops_mpi.hpp"
#include <mpi.h>
#include <algorithm>

namespace shuravina_o_jarvis_pass {

void JarvisPassMPI::run() {
    int rank;
    int size;
    MPI_Comm_rank(MPI_COMM_WORLD, &rank);
    MPI_Comm_size(MPI_COMM_WORLD, &size);

    int n = points_.size();
    int chunk_size = n / size;
    int start = rank * chunk_size;
    int end = (rank == size - 1) ? n : (rank + 1) * chunk_size;

    std::vector<Point> local_points(points_.begin() + start, points_.begin() + end);
    std::vector<Point> local_hull = jarvis_march(local_points);

    if (rank == 0) {
        hull_ = local_hull;
        for (int i = 1; i < size; i++) {
            int count;
            MPI_Recv(&count, 1, MPI_INT, i, 0, MPI_COMM_WORLD, MPI_STATUS_IGNORE);
            std::vector<Point> remote_hull(count);
            MPI_Recv(remote_hull.data(), count * sizeof(Point), MPI_BYTE, i, 0, MPI_COMM_WORLD, MPI_STATUS_IGNORE);
            hull_.insert(hull_.end(), remote_hull.begin(), remote_hull.end());
        }
        hull_ = jarvis_march(hull_);
    } else {
        int count = local_hull.size();
        MPI_Send(&count, 1, MPI_INT, 0, 0, MPI_COMM_WORLD);
        MPI_Send(local_hull.data(), count * sizeof(Point), MPI_BYTE, 0, 0, MPI_COMM_WORLD);
    }
}

std::vector<Point> JarvisPassMPI::get_hull() const { return hull_; }

bool JarvisPassMPI::validation() const { return points_.size() >= 3; }

std::vector<Point> jarvis_march(const std::vector<Point>& points) {
    int n = points.size();
    if (n < 3) return points;

    std::vector<Point> hull;

    int l = 0;
    for (int i = 1; i < n; i++) {
        if (points[i] < points[l]) {
            l = i;
        }
    }

    int p = l;
    int q;
    do {
        hull.push_back(points[p]);
        q = (p + 1) % n;

        for (int i = 0; i < n; i++) {
            if ((points[i].y - points[p].y) * (points[q].x - points[i].x) -
                (points[i].x - points[p].x) * (points[q].y - points[i].y) <
                0) {
                q = i;
            }
        }

        p = q;
    } while (p != l);

    return hull;
}

} // namespace shuravina_o_jarvis_pass
\end{lstlisting}

\newpage

\begin{lstlisting}[caption={ops\_seq\_.hpp}]
#pragma once
#include <vector>

namespace shuravina_o_jarvis_pass {

struct Point {
    int x, y;
    Point(int x = 0, int y = 0) : x(x), y(y) {}

    bool operator<(const Point& p) const { return (x < p.x) || (x == p.x && y < p.y); }
    bool operator==(const Point& p) const { return x == p.x && y == p.y; }
};

class JarvisPassSeq {
public:
    JarvisPassSeq(const std::vector<Point>& points) : points_(points) {}
    void run();
    std::vector<Point> get_hull() const;
    bool validation() const;

private:
    const std::vector<Point>& points_;
    std::vector<Point> hull_;
};

std::vector<Point> jarvis_march(const std::vector<Point>& points);

} // namespace shuravina_o_jarvis_pass
\end{lstlisting}

\newpage

\begin{lstlisting}[caption={ops\_seq\_.cpp}]
#include "seq/shuravina_o_jarvis_pass/include/ops_seq.hpp"

namespace shuravina_o_jarvis_pass {

void JarvisPassSeq::run() { hull_ = jarvis_march(points_); }

std::vector<Point> JarvisPassSeq::get_hull() const { return hull_; }

bool JarvisPassSeq::validation() const { return points_.size() >= 3; }

std::vector<Point> jarvis_march(const std::vector<Point>& points) {
    int n = points.size();
    if (n < 3) return points;

    std::vector<Point> hull;

    int l = 0;
    for (int i = 1; i < n; i++) {
        if (points[i] < points[l]) {
            l = i;
        }
    }

    int p = l;
    int q;
    do {
        hull.push_back(points[p]);
        q = (p + 1) % n;

        for (int i = 0; i < n; i++) {
            if ((points[i].y - points[p].y) * (points[q].x - points[i].x) -
                (points[i].x - points[p].x) * (points[q].y - points[i].y) <
                0) {
                q = i;
            }
        }

        p = q;
    } while (p != l);

    return hull;
}

} // namespace shuravina_o_jarvis_pass
\end{lstlisting}

\end{document}