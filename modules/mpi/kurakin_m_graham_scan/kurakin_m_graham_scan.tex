\documentclass[12pt]{article}
\usepackage[T2A]{fontenc}
\usepackage[utf8]{inputenc}
\usepackage[russian]{babel}
\usepackage{amsmath, amssymb}
\usepackage{geometry}
\usepackage{graphicx}
\usepackage{listings}
\usepackage{hyperref}
\usepackage{xcolor} % Для цветов в блоках кода
\geometry{left=2cm, right=2cm, top=2cm, bottom=2cm}

\lstset{
  language=C++,
  basicstyle=\ttfamily\small,
  keywordstyle=\bfseries\color{blue},
  commentstyle=\itshape\color{green!50!black},
  stringstyle=\color{red!60!black},
  numbers=left,
  numberstyle=\tiny,
  stepnumber=1,
  numbersep=8pt,
  backgroundcolor=\color{gray!10},
  showspaces=false,
  showstringspaces=false,
  breaklines=true,
  frame=single,
  tabsize=2,
  captionpos=b
}

\hypersetup{hidelinks}

\begin{document}

\begin{titlepage}
\begin{center}
\textbf{МИНИСТЕРСТВО НАУКИ И ВЫСШЕГО ОБРАЗОВАНИЯ РОССИЙСКОЙ ФЕДЕРАЦИИ} \\[0.5cm]
\textbf{Федеральное государственное автономное образовательное учреждение высшего образования} \\[0.5cm]
\textbf{«Национальный исследовательский Нижегородский государственный университет им. Н.И. Лобачевского»} \\[0.5cm]
Институт информационных технологий, математики и механики \\
\vfill
{\Large
\textbf{Отчёт по лабораторной работе на тему:} \\[0.5cm]
\textbf{Построение выпуклой оболочки – проход Грэхема.} \\
}
\vfill
\begin{flushright}
Выполнил: студент группы 3822Б1ФИ1 \\
Куракин Матвей \\
\vspace{1cm}
Преподаватель: \\
Сысоев Александр Владимирович, доцент, кандидат технических наук \\
\end{flushright}
\vfill
Нижний Новгород \\
2024
\end{center}
\end{titlepage}

\tableofcontents
\newpage

\section*{Введение}
\addcontentsline{toc}{section}{Введение}

Выпуклая оболочка является важным понятием в области вычислительной геометрии и находит широкое применение в различных областях, таких как компьютерная графика, обработка изображений, робототехника и анализ данных. Выпуклая оболочка множества точек в двумерном пространстве представляет собой наименьший выпуклый многоугольник, который окружает данное множество точек. Построение выпуклой оболочки позволяет эффективно решать задачи, связанные с нахождением границ объектов, а также упрощает анализ и визуализацию данных.

Одним из наиболее известных алгоритмов для построения выпуклой оболочки является алгоритм Грэхема, который был предложен американским математиком Рональдом Грэхемом в 1972 году. 

Цель данной лабораторной работы — изучить алгоритм построения выпуклой оболочки методом Грэхемаа также реализовать его последовательную и MPI версии.

\newpage

\section{Постановка задачи}
Дано множество точек \( P = \{ p_1, p_2, \ldots, p_n \} \) в двумерном пространстве, где каждая точка \( p_i \) представлена в виде координат \( p_i = (x_i, y_i) \). Задача алгоритма Грэхема заключается в нахождении выпуклой оболочки \( H \) данного множества точек \( P \).

Выпуклая оболочка \( H \) определяется как наименьший выпуклый многоугольник, который содержит все точки из множества \( P \). Формально, \( H \) может быть описана как:

\[
H = \{ p_i \in P \mid \forall p_j \in P, \text{ точка } p_j \text{ лежит внутри или на границе многоугольника } H \}
\]

Алгоритм принимает на вход множество точек \( P \) в двумерном пространстве, представленное в виде списка координат:
\[
    P = \{ p_1, p_2, \ldots, p_n \}, \quad \text{где } p_i = (x_i, y_i) \text{ для } i = 1, 2, \ldots, n.
\]
    Число \( n \) - количество точек в множестве \( P \).

Алгоритм возвращает выпуклую оболочку \( H \), представленную в виде списка точек \( H = \{ h_1, h_2, \ldots, h_k \} \), где каждая точка \( h_j \) также представлена в виде координат \( h_j = (x_j, y_j) \) для \( j = 1, 2, \ldots, k \). \( k \) - количество точек в выпуклой оболочке \( H \).

\section{Описание алгоритма}
Алгоритм Грэхема, известный также как алгоритм построения выпуклой оболочки, является одним из наиболее эффективных методов для нахождения выпуклой оболочки множества точек в двумерном пространстве. Этот алгоритм имеет временную сложность \(O(n \log n)\), что делает его подходящим для обработки больших наборов данных.

\subsection*{Алгоритм}

\subsubsection*{1. Выбор начальной точки}
Находим точку с наименьшей координатой \(y\) (в случае равенства выбираем точку с наименьшей координатой \(x\)). Эта точка будет служить начальной для построения выпуклой оболочки и обозначается как \(p_{\text{min}}\).

\subsubsection*{2. Сортировка точек}
После выбора начальной точки, все остальные точки сортируются по углу, который они образуют с горизонтальной осью \(x\) относительно точки \(p_{\text{min}}\). Если несколько точек имеют одинаковый угол, они сортируются по расстоянию от \(p_{\text{min}}\) — ближние точки располагаются раньше.

\subsubsection*{3. Построение выпуклой оболочки}
Начинаем с точки \(p_{\text{min}}\) и добавляем точки по порядку, определенному на предыдущем шаге. Для каждой новой точки проверяем, образует ли она выпуклую оболочку с двумя предыдущими точками с помощь. левого поворота.

\subsubsection*{4. Получение результата}
Процесс продолжается до тех пор, пока не будут обработаны все точки. В конце алгоритм возвращает список точек, которые образуют выпуклую оболочку.


\subsection*{Сложность алгоритма}

Алгоритм работает за \(O(n \log n)\), где $n$ — число точек. Самая долгая операция - сортировка, она работает за \(O(n \log n)\), операции поиска и построения выпуклой оболочки линейные, работают за \(O(n)\).

\section{Описание способа распараллеливания}

\subsection*{Алгоритм}

\subsubsection*{1. Распределение входных данных}
Главный процесс передает каждому процессу минимум три точки для формирования оболочки.

\subsubsection*{2. Вычисления на каждом процессе}
На каждом процессе запускается последовательный алгоритм, который находит выпуклую оболочку для переданных точек.

\subsubsection*{3. Сбор результатов}
В главном процессе собираются локальные выпуклые оболочки от каждого процесса, эти точки формируют общее множество точек.

\subsubsection*{4. Вычисления на главном процессе}
Главный процесс для общего множества точек формирует выпуклую оболочку, которая является окончательной.

\section{Тестирование}

\subsection*{Функциональные тесты}
В лабораторной работе были написаны функциональные тесты при помощи Google testing framework (gtest).

Алгоритмы проходят тесты, где рассматриваются обычные ситуации, крайние случае, случайные входные данные, большие и маленькие наборы данных, также есть тесты на неправильные входные данные.

И параллельный и последовательный алгоритм успешно проходят написанные тесты

\subsection*{Тесты производительности}

Тесты производительности измеряют время работы алгоритма.
Тесты проводились на разном числе процессов и разном количестве входных данных

\subsubsection*{Результаты}
\begin{tabular}{|c|c|c|c|}
    \hline
    Число процессов & Число точек      & Время выполнения (с) \\ \hline
    1               & 1000         & 0.0139                   \\ \hline
    2               & 1000         & 0.0112                   \\ \hline
    4               & 1000         & 0.00972                  \\ \hline
    8               & 1000         & 0.0113                   \\ \hline
    16              & 1000         & 0.0138                   \\ \hline
    1               & 10000        & 0.209                    \\ \hline
    2               & 10000        & 0.161                    \\ \hline
    4               & 10000        & 0.140                    \\ \hline
    8               & 10000        & 0.124                    \\ \hline
    16              & 10000        & 0.125                    \\ \hline
    1               & 100000       & 2.728                    \\ \hline
    2               & 100000       & 2.038                    \\ \hline
    4               & 100000       & 1.715                    \\ \hline
    8               & 100000       & 1.602                    \\ \hline
    16              & 100000       & 1.565                    \\ \hline
\end{tabular}

\subsubsection*{Выводы}
Результаты экспериментов показывают, что с увеличением числа процессов уменьшается время, а также, что на большом количесве точек при увеличении числа процессов ускорение больше.

\newpage

\section*{Заключение}
\addcontentsline{toc}{section}{Заключение}

В ходе проделанной работы был подробно рассмотрен алгоритм Грэхема, который является одним из наиболее известных и эффективных методов для построения выпуклой оболочки множества точек в двумерном пространстве. Были реализованы последовательный и параллельный алгоритмы, а также мы убедились, что использование параллельного программирования при помощи MPI уменьшает скорость работы задач с большими входными данными.

\newpage

\section*{Список литературы}
\addcontentsline{toc}{section}{Список литературы}

\begin{enumerate}
    \item Boost MPI Library Documentation: 
    \url{https://www.boost.org/doc/libs/release/libs/mpi/}
    \item GTest Documentation: 
    \url{https://google.github.io/googletest/}
    \item Алгоритмика: Алгоритм Грэхема 
    \url{https://ru.algorithmica.org/cs/convex-hulls/graham/}
    \item ИТМО Викиконспекты: Статические выпуклые оболочки: Джарвис, Грэхем, Эндрю, Чен, QuickHull: \url{https://neerc.ifmo.ru/wiki/index.php?title=%D0%A1%D1%82%D0%B0%D1%82%D0%B8%D1%87%D0%B5%D1%81%D0%BA%D0%B8%D0%B5_%D0%B2%D1%8B%D0%BF%D1%83%D0%BA%D0%BB%D1%8B%D0%B5_%D0%BE%D0%B1%D0%BE%D0%BB%D0%BE%D1%87%D0%BA%D0%B8:_%D0%94%D0%B6%D0%B0%D1%80%D0%B2%D0%B8%D1%81,_%D0%93%D1%80%D1%8D%D1%85%D0%B5%D0%BC,_%D0%AD%D0%BD%D0%B4%D1%80%D1%8E,_%D0%A7%D0%B5%D0%BD,_QuickHull}
    \item Алголист: Алгоритм Грэхема: 
    \url{https://algolist.ru/maths/geom/convhull/graham.php}
\end{enumerate}

\newpage

\section*{Приложение}
\addcontentsline{toc}{section}{Приложение}

\subsection*{mpi/kurakin\_m\_graham\_scan/include/kurakin\_graham\_scan\_ops\_mpi.hpp}
\begin{lstlisting}[language=C++]
#pragma once

#include <gtest/gtest.h>

#include <boost/mpi/collectives.hpp>
#include <boost/mpi/communicator.hpp>
#include <memory>
#include <numeric>
#include <utility>
#include <vector>

#include "core/task/include/task.hpp"

namespace kurakin_m_graham_scan_mpi {

bool isLeftAngle(std::vector<double>& p1, std::vector<double>& p2, std::vector<double>& p3);

int grahamScan(std::vector<std::vector<double>>& input_);

int getCountPoint(int count_point, int size, int rank);

class TestMPITaskSequential : public ppc::core::Task {
 public:
  explicit TestMPITaskSequential(std::shared_ptr<ppc::core::TaskData> taskData_) : Task(std::move(taskData_)) {}
  bool pre_processing() override;
  bool validation() override;
  bool run() override;
  bool post_processing() override;

 private:
  int count_point{};
  std::vector<std::vector<double>> input_;
  std::vector<double> res;
};

class TestMPITaskParallel : public ppc::core::Task {
 public:
  explicit TestMPITaskParallel(std::shared_ptr<ppc::core::TaskData> taskData_) : Task(std::move(taskData_)) {}
  bool pre_processing() override;
  bool validation() override;
  bool run() override;
  bool post_processing() override;

 private:
  boost::mpi::communicator world;
  int count_point{};
  std::vector<double> input_;
  int local_count_point{};
  std::vector<double> local_input_;
  std::vector<std::vector<double>> graham_input_;
};

}  // namespace kurakin_m_graham_scan_mpi
\end{lstlisting}

\subsection*{mpi/kurakin\_m\_graham\_scan/src/kurakin\_graham\_scan\_ops\_mpi.cpp}
\begin{lstlisting}[language=C++]
#include "mpi/kurakin_m_graham_scan/include/kurakin_graham_scan_ops_mpi.hpp"

#include <algorithm>
#include <cmath>
#include <functional>
#include <random>
#include <thread>
#include <vector>

bool kurakin_m_graham_scan_mpi::isLeftAngle(std::vector<double>& p1, std::vector<double>& p2, std::vector<double>& p3) {
  return ((p2[0] - p1[0]) * (p3[1] - p1[1]) - (p3[0] - p1[0]) * (p2[1] - p1[1])) < 0;
}

int kurakin_m_graham_scan_mpi::grahamScan(std::vector<std::vector<double>>& input_) {
  int count_point = input_.size();
  int ind_min_y = std::min_element(input_.begin(), input_.end(),
                                   [&](std::vector<double> a, std::vector<double> b) {
                                     return a[1] < b[1] || (a[1] == b[1] && a[0] > b[0]);
                                   }) -
                  input_.begin();
  std::swap(input_[0], input_[ind_min_y]);
  std::sort(input_.begin() + 1, input_.end(),
            [&](std::vector<double> a, std::vector<double> b) { return isLeftAngle(a, input_[0], b); });

  int k = 1;
  for (int i = 2; i < count_point; i++) {
    while (k > 0 && isLeftAngle(input_[k - 1], input_[k], input_[i])) {
      k--;
    }
    std::swap(input_[i], input_[k + 1]);
    k++;
  }
  return k + 1;
}

int kurakin_m_graham_scan_mpi::getCountPoint(int count_point, int size, int rank) {
  if (count_point / size < 3) {
    if (count_point / 3 <= rank) return 0;
    size = count_point / 3;
  }
  if (count_point % size <= rank) {
    return count_point / size;
  }
  return count_point / size + 1;
}

bool kurakin_m_graham_scan_mpi::TestMPITaskSequential::pre_processing() {
  internal_order_test();

  count_point = (int)taskData->inputs_count[0] / 2;
  input_ = std::vector<std::vector<double>>(count_point, std::vector<double>(2, 0));
  auto* tmp_ptr = reinterpret_cast<double*>(taskData->inputs[0]);
  for (int i = 0; i < count_point * 2; i += 2) {
    input_[i / 2][0] = tmp_ptr[i];
    input_[i / 2][1] = tmp_ptr[i + 1];
  }

  return true;
}

bool kurakin_m_graham_scan_mpi::TestMPITaskSequential::validation() {
  internal_order_test();
  return taskData->inputs.size() == 1 && taskData->inputs_count.size() == 1 && taskData->outputs.size() == 2 &&
         taskData->outputs_count.size() == 2 && taskData->inputs_count[0] % 2 == 0 &&
         taskData->inputs_count[0] / 2 > 2 && taskData->outputs_count[0] == 1 &&
         taskData->inputs_count[0] == taskData->outputs_count[1];
}

bool kurakin_m_graham_scan_mpi::TestMPITaskSequential::run() {
  internal_order_test();

  count_point = grahamScan(input_);

  return true;
}

bool kurakin_m_graham_scan_mpi::TestMPITaskSequential::post_processing() {
  internal_order_test();

  reinterpret_cast<int*>(taskData->outputs[0])[0] = count_point;
  for (int i = 0; i < count_point * 2; i += 2) {
    reinterpret_cast<double*>(taskData->outputs[1])[i] = input_[i / 2][0];
    reinterpret_cast<double*>(taskData->outputs[1])[i + 1] = input_[i / 2][1];
  }

  return true;
}

bool kurakin_m_graham_scan_mpi::TestMPITaskParallel::pre_processing() {
  internal_order_test();
  return true;
}

bool kurakin_m_graham_scan_mpi::TestMPITaskParallel::validation() {
  internal_order_test();

  if (world.rank() == 0) {
    return taskData->inputs.size() == 1 && taskData->inputs_count.size() == 1 && taskData->outputs.size() == 2 &&
           taskData->outputs_count.size() == 2 && taskData->inputs_count[0] % 2 == 0 &&
           taskData->inputs_count[0] / 2 > 2 && taskData->outputs_count[0] == 1 &&
           taskData->inputs_count[0] == taskData->outputs_count[1];
  }

  return true;
}

bool kurakin_m_graham_scan_mpi::TestMPITaskParallel::run() {
  internal_order_test();

  if (world.rank() == 0) {
    count_point = (int)taskData->inputs_count[0] / 2;
    input_ = std::vector<double>((int)taskData->inputs_count[0]);
    auto* tmp_ptr = reinterpret_cast<double*>(taskData->inputs[0]);
    for (int i = 0; i < count_point * 2; i += 2) {
      input_[i] = tmp_ptr[i];
      input_[i + 1] = tmp_ptr[i + 1];
    }
  }
  broadcast(world, count_point, 0);

  if (world.rank() == 0) {
    local_count_point = getCountPoint(count_point, world.size(), world.rank());
    local_input_ = std::vector<double>(input_.begin(), input_.begin() + local_count_point * 2);

    int sum_next_count_point = local_count_point;
    for (int i = 1; i < world.size(); i++) {
      int next_count_point = getCountPoint(count_point, world.size(), i);
      if (next_count_point != 0) {
        world.send(i, 0, input_.data() + sum_next_count_point * 2, next_count_point * 2);
      }
      sum_next_count_point += next_count_point;
    }
  } else {
    local_count_point = getCountPoint(count_point, world.size(), world.rank());
    if (local_count_point != 0) {
      local_input_ = std::vector<double>(local_count_point * 2);
      world.recv(0, 0, local_input_.data(), local_count_point * 2);
    }
  }

  if (local_count_point != 0) {
    graham_input_ = std::vector<std::vector<double>>(local_count_point, std::vector<double>(2, 0));
    for (int i = 0; i < local_count_point * 2; i += 2) {
      graham_input_[i / 2][0] = local_input_[i];
      graham_input_[i / 2][1] = local_input_[i + 1];
    }
    local_count_point = grahamScan(graham_input_);
    for (int i = 0; i < local_count_point * 2; i += 2) {
      local_input_[i] = graham_input_[i / 2][0];
      local_input_[i + 1] = graham_input_[i / 2][1];
    }
  }

  if (world.rank() == 0) {
    std::copy(local_input_.begin(), local_input_.begin() + local_count_point * 2, input_.begin());
    int sum_next_count_point = local_count_point;
    for (int i = 1; i < world.size(); i++) {
      int next_count_point;
      world.recv(i, 0, &next_count_point, 1);
      if (next_count_point != 0) {
        world.recv(i, 0, input_.data() + sum_next_count_point * 2, next_count_point * 2);
      }
      sum_next_count_point += next_count_point;
    }
    local_count_point = sum_next_count_point;
  } else {
    world.send(0, 0, &local_count_point, 1);
    if (local_count_point != 0) {
      world.send(0, 0, local_input_.data(), local_count_point * 2);
    }
  }

  if (world.rank() == 0) {
    graham_input_ = std::vector<std::vector<double>>(local_count_point, std::vector<double>(2, 0));
    for (int i = 0; i < local_count_point * 2; i += 2) {
      graham_input_[i / 2][0] = input_[i];
      graham_input_[i / 2][1] = input_[i + 1];
    }
    count_point = grahamScan(graham_input_);
  }

  return true;
}

bool kurakin_m_graham_scan_mpi::TestMPITaskParallel::post_processing() {
  internal_order_test();

  if (world.rank() == 0) {
    reinterpret_cast<int*>(taskData->outputs[0])[0] = count_point;
    for (int i = 0; i < count_point * 2; i += 2) {
      reinterpret_cast<double*>(taskData->outputs[1])[i] = graham_input_[i / 2][0];
      reinterpret_cast<double*>(taskData->outputs[1])[i + 1] = graham_input_[i / 2][1];
    }
  }

  return true;
}

\end{lstlisting}


\end{document}