\documentclass[12pt]{article}
\usepackage[a4paper,top=20mm,bottom=20mm,left=25mm,right=15mm]{geometry}
\usepackage[T2A]{fontenc}
\usepackage[utf8]{inputenc}
\usepackage[russian]{babel}
\usepackage{amsmath}
\usepackage{enumitem}
\usepackage{graphicx}
\usepackage{hyperref}
\usepackage{listings}
\usepackage{xcolor}

\begin{document}

\begin{center} 
МИНИСТЕРСТВО ОБРАЗОВАНИЯ И НАУКИ РОССИЙСКОЙ ФЕДЕРАЦИИ \\
Федеральное государственное автономное образовательное учреждение  \\
высшего образования\\ \textbf{<<Национальный исследовательский \\ Нижегородский государственный университет \\
им. Н.И. Лобачевского>>}\\
\textbf{(ННГУ)}\\[0.5cm]
\textbf{Институт информационных технологий, математики и механики}\\[4.5cm]

\textbf{\large Отчет по лабораторной работе} \\[0.6cm] % название работы, затем отступ 0,6см
\textbf{Тема:}\\
  \textbf{\large <<Поиск кратчайших путей из одной вершины (алгоритм Беллмана-Форда). С CRS графов>>}\\[5.0cm]
\begin{flushright}
 \begin{minipage}{0.40\textwidth} % начало маленькой врезки в половину ширины текста
 \begin{flushleft} % выровнять её содержимое по левому краю
  \textbf{Выполнил:}\\[0.1cm]
  студент группы 3822Б1ПР4 \\
  Зиновьев Александр Дмитриевич  \\[1.0cm]
  \textbf{Преподаватель:}\\[0.1cm]
  Сысоев Александр Владимирович, доцент, кандидат технических наук  \\
 \end{flushleft} % конец выравнивания по левому краю
 \end{minipage} % конец врезки
\end{flushright}
 \vfill 

  Нижний Новгород \\
 2024

 \thispagestyle{empty} 

\end{center}

\newpage
\section*{Введение}
\indent Алгоритм Беллмана-Форда является одним из фундаментальных алгоритмов в области теории графов, предназначенных для решения задачи поиска кратчайших путей из одной вершины во взвешенном графе. В отличие от алгоритма Дейкстры, который эффективно работает только на графах с неотрицательными весами рёбер, алгоритм Беллмана-Форда способен обрабатывать графы с отрицательными весами, при условии отсутствия отрицательных циклов. Это делает его универсальным инструментом для решения широкого круга задач, включая транспортные сети, финансовые модели и другие приложения, где возможны отрицательные веса.

Основная идея алгоритма заключается в последовательной релаксации рёбер графа. На каждой итерации алгоритм проверяет, можно ли улучшить текущее расстояние до вершины, обновляя его, если это возможно. Алгоритм гарантирует нахождение кратчайших путей за \( V - 1 \) итерацию, где \( V \) — количество вершин в графе. Если после \( V - 1 \) итерации происходит улучшение расстояний, это свидетельствует о наличии отрицательного цикла, что делает задачу поиска кратчайших путей некорректной.

Алгоритм Беллмана-Форда имеет временную сложность \( O(V \cdot E) \), где \( V \) — количество вершин, а \( E \) — количество рёбер. Это делает его менее эффективным по сравнению с алгоритмом Дейкстры для графов с неотрицательными весами, но более универсальным в случае наличия отрицательных весов. В данной работе алгоритм реализован с использованием формата CRS (Compressed Row Storage), что позволяет эффективно хранить разреженные матрицы смежности и уменьшить объём используемой памяти.

\section*{Постановка задачи}

\section*{Цель работы}

\begin{itemize}
    \item Целью данной работы является разработка и реализация алгоритма Беллмана-Форда для поиска кратчайших путей в графе с использованием формата CRS. Работа включает как последовательную, так и параллельную (с использованием MPI) реализации алгоритма.
\end{itemize}

\section*{Задачи работы}

\begin{itemize}
    \item Разработка последовательной версии алгоритма Беллмана-Форда.
    \item Реализация параллельной версии алгоритма с использованием MPI.
    \item Проведение функциональных и производительностных тестов.
    \item Анализ результатов каждых результатов.
\end{itemize}

\section*{Технические характеристики}

\begin{itemize}
    \item \textbf{Процессор:} AMD Ryzen 5 2600X Six-Core Processor.
    \item \textbf{Оперативная память:} 16 ГБ, DDR4 3000 МГц.
    \item \textbf{Видеокарта:} NVIDIA GeForce GTX 1660 SUPER
    \item \textbf{Операционная система:} Windows 10 Version 22H2.
    \item \textbf{Компилятор:} Visual Studio 2022.
\end{itemize}

\section*{Структура проекта}

Проект организован в виде иерархической структуры папок, которая включает реализацию алгоритма Беллмана-Форда как для последовательной, так и для параллельной (MPI) версии. Основные компоненты проекта распределены по следующим папкам:

\begin{itemize}
    \item \textbf{tasks} — корневая папка проекта, содержащая все задачи.
    \begin{itemize}
        \item \textbf{mpi} — папка, содержащая реализацию алгоритма Беллмана-Форда с использованием MPI.
        \begin{itemize}
            \item \textbf{\texttt{zinoviev\_a\_bellman\_ford}} — папка с реализацией алгоритма для MPI.
            \begin{itemize}
                \item \textbf{func\_tests} — папка с функциональными тестами для MPI-версии алгоритма.
                \item \textbf{include} — папка с заголовочными файлами для MPI-версии.
                \item \textbf{perf\_tests} — папка с тестами производительности для MPI-версии.
                \item \textbf{src} — папка с исходным кодом для MPI-версии.
            \end{itemize}
        \end{itemize}
        \item \textbf{seq} — папка, содержащая последовательную реализацию алгоритма Беллмана-Форда.
        \begin{itemize}
            \item \textbf{\texttt{zinoviev\_a\_bellman\_ford}} — папка с реализацией алгоритма для последовательной версии.
            \begin{itemize}
                \item \textbf{func\_tests} — папка с функциональными тестами для последовательной версии.
                \item \textbf{include} — папка с заголовочными файлами для последовательной версии.
                \item \textbf{perf\_tests} — папка с тестами производительности для последовательной версии.
                \item \textbf{src} — папка с исходным кодом для последовательной версии.
            \end{itemize}
        \end{itemize}
    \end{itemize}
\end{itemize}

\section*{Описание методов}

\section*{MPI-версия алгоритма Беллмана-Форда}

\begin{itemize}
    \item \textbf{BellmanFordMPIMPI} — класс, реализующий алгоритм Беллмана-Форда с использованием MPI.
    \begin{itemize}
        \item \textbf{pre\_processing()} — метод, выполняющий предварительную обработку данных. В этом методе происходит преобразование входных данных в формат CRS (Compressed Row Storage), а также инициализация начальных значений для кратчайших путей.
        \item \textbf{validation()} — метод, проверяющий корректность входных данных. В частности, проверяется соответствие количества вершин и рёбер, а также отсутствие петель в графе.
        \item \textbf{run()} — основной метод, реализующий алгоритм Беллмана-Форда. В этом методе выполняются итерации для нахождения кратчайших путей, а также проверка на наличие отрицательных циклов.
        \item \textbf{post\_processing()} — метод, выполняющий пост-обработку результатов. В этом методе результаты записываются в выходные данные.
        \item \textbf{Iteration(std::vector<int>\& paths)} — вспомогательный метод, выполняющий одну итерацию алгоритма Беллмана-Форда. Возвращает \texttt{true}, если произошли изменения в кратчайших путях.
        \item \textbf{check\_negative\_cycle()} — метод, проверяющий наличие отрицательных циклов в графе.
        \item \textbf{toCRS(const int* input\_matrix)} — метод, преобразующий входную матрицу смежности в формат CRS.
    \end{itemize}
\end{itemize}

\section*{Последовательная версия алгоритма Беллмана-Форда}

\begin{itemize}
    \item \textbf{BellmanFordSeq} — класс, реализующий последовательную версию алгоритма Беллмана-Форда.
    \begin{itemize}
        \item \textbf{pre\_processing()} — метод, выполняющий предварительную обработку данных. Аналогично MPI-версии, здесь происходит преобразование входных данных в формат CRS и инициализация начальных значений для кратчайших путей.
        \item \textbf{validation()} — метод, проверяющий корректность входных данных. Выполняет аналогичные проверки, что и в MPI-версии.
        \item \textbf{run()} — основной метод, реализующий последовательный алгоритм Беллмана-Форда. Выполняет итерации для нахождения кратчайших путей и проверку на наличие отрицательных циклов.
        \item \textbf{post\_processing()} — метод, выполняющий пост-обработку результатов. Результаты записываются в выходные данные.
        \item \textbf{Iteration(std::vector<int>\& paths)} — вспомогательный метод, выполняющий одну итерацию алгоритма Беллмана-Форда. Возвращает \texttt{true}, если произошли изменения в кратчайших путях.
        \item \textbf{check\_negative\_cycle()} — метод, проверяющий наличие отрицательных циклов в графе.
        \item \textbf{toCRS(const int* input\_matrix)} — метод, преобразующий входную матрицу смежности в формат CRS.
    \end{itemize}
\end{itemize}

\section*{Тесты}

В проекте реализованы как функциональные, так и тесты производительности для обеих версий алгоритма. Функциональные тесты проверяют корректность работы алгоритма на различных входных данных, включая графы с отрицательными весами и без них. Тесты производительности позволяют оценить время выполнения алгоритма на различных размерах графов.

\section*{Функциональные тесты}

\begin{itemize}
    \item \textbf{Test\_Small\_Graph\_mpi/seq} — тест на небольшом графе, проверяющий корректность работы алгоритма на простых примерах.
    \item \textbf{Test\_Medium\_Graph\_mpi/seq} — тест на среднем графе, проверяющий работу алгоритма на более сложных примерах.
    \item \textbf{Test\_Multiple\_Paths\_Graph\_seq} — тест на графе с несколькими путями между вершинами.
    \item \textbf{Test\_Negative\_Weights\_No\_Cycle\_Graph\_seq} — тест на графе с отрицательными весами, но без отрицательных циклов.
    \item \textbf{Test\_Small\_Weights\_Graph\_seq} — тест на случайно сгенерированном графе.
    \item \textbf{Test\_Large\_Weights\_Graph\_seq} — тест на большом случайно сгенерированном графе.
\end{itemize}

\section*{Тесты производительности}

\begin{itemize}
    \item \textbf{test\_pipeline\_run} — тест, измеряющий время выполнения всего конвейера обработки данных (предварительная обработка, выполнение алгоритма, пост-обработка).
    \item \textbf{test\_task\_run} — тест, измеряющий время выполнения только основного алгоритма Беллмана-Форда.
\end{itemize}

\section*{Описание схемы распараллеливания}

Для реализации параллельной версии алгоритма Беллмана-Форда используется библиотека MPI (Message Passing Interface). Основная идея распараллеливания заключается в разделении работы по обработке вершин графа между несколькими процессами. Каждый процесс обрабатывает часть вершин, что позволяет ускорить выполнение алгоритма за счёт одновременной работы нескольких процессов.

\section*{Основные этапы распараллеливания}

\begin{enumerate}[leftmargin=*,itemsep=5pt]
    \item \textbf{Предобработка данных:}
    \begin{itemize}
        \item На начальном этапе процесс с рангом 0 (главный процесс) выполняет преобразование входных данных из матрицы смежности в формат CRS (Compressed Row Storage). Это преобразование включает создание трёх массивов: \texttt{values} (веса рёбер), \texttt{columns} (индексы столбцов) и \texttt{row\_ptr} (указатели строк).
        \item После преобразования данные рассылаются всем процессам с помощью функции \texttt{boost::mpi::broadcast}. Это позволяет каждому процессу иметь доступ к одинаковым данным графа.
    \end{itemize}
    
    \item \textbf{Распределение работы между процессами:}
    \begin{itemize}
        \item Вершины графа равномерно распределяются между процессами. Каждый процесс получает часть вершин для обработки. Если количество вершин не делится на число процессов, то оставшиеся вершины распределяются между первыми процессами.
        \item Для каждой вершины, назначенной процессу, выполняется обновление расстояний до соседних вершин.
    \end{itemize}
    
    \item \textbf{Выполнение итераций алгоритма:}
    \begin{itemize}
        \item Алгоритм Беллмана-Форда выполняется в несколько итераций. На каждой итерации каждый процесс обновляет расстояния до вершин, которые ему были назначены.
        \item После завершения обработки своих вершин, процессы объединяют результаты с помощью функции \texttt{boost::mpi::reduce}, которая выбирает минимальные значения расстояний из всех процессов.
        \item Обновлённые расстояния рассылаются всем процессам с помощью \texttt{boost::mpi::broadcast}, чтобы каждый процесс имел актуальные данные для следующей итерации.
    \end{itemize}
    
    \item \textbf{Проверка на отрицательные циклы:}
    \begin{itemize}
        \item После завершения всех итераций алгоритма выполняется проверка на наличие отрицательных циклов. Эта проверка выполняется только на главном процессе, так как она требует полного обновления всех расстояний.
    \end{itemize}
    
    \item \textbf{Постобработка результатов:}
    \begin{itemize}
        \item После завершения работы алгоритма результаты собираются на главном процессе. Главный процесс записывает итоговые кратчайшие пути в выходной массив.
    \end{itemize}
\end{enumerate}

\section*{Особенности реализации}

\begin{itemize}
    \item \textbf{Балансировка нагрузки:} Вершины графа распределяются между процессами равномерно, что позволяет минимизировать дисбаланс нагрузки.
    \item \textbf{Синхронизация данных:} Для обеспечения корректности работы алгоритма используются функции \texttt{boost::mpi::broadcast} и \texttt{boost::mpi::reduce}, которые синхронизируют данные между процессами.
    \item \textbf{Оптимизация для больших графов:} Использование формата CRS позволяет эффективно хранить и обрабатывать разреженные графы, что особенно важно для больших графов с большим количеством вершин и рёбер.
\end{itemize}

Таким образом, схема распараллеливания алгоритма Беллмана-Форда на основе MPI позволяет эффективно распределить вычислительную нагрузку между процессами, что приводит к ускорению выполнения алгоритма для больших графов.

\section*{Результаты экспериментов}

В ходе выполнения задания были проведены эксперименты для оценки производительности и корректности реализации алгоритма Беллмана-Форда. Эксперименты включали как функциональные тесты, так и тесты производительности, которые были выполнены как для последовательной, так и для параллельной версии алгоритма с использованием MPI.

\section*{Функциональные тесты}

Функциональные тесты были направлены на проверку корректности работы алгоритма Беллмана-Форда на различных графах. В частности, были протестированы следующие сценарии:

\begin{itemize}
    \item \textbf{Тест на малом графе:} Проверка работы алгоритма на небольшом графе с четырьмя вершинами. Ожидаемый результат соответствовал кратчайшим путям, рассчитанным вручную.
    \item \textbf{Тест на среднем графе:} Использовался граф с девятью вершинами и 28 ребрами. Результаты совпали с ожидаемыми значениями, что подтвердило корректность работы алгоритма.
    \item \textbf{Тест на графе с несколькими путями:} Проверка работы алгоритма на графе, где существует несколько путей между вершинами. Алгоритм корректно выбрал кратчайшие пути.
    \item \textbf{Тест на случайном графе:} Проверка работы алгоритма на случайно сгенерированном графе. Алгоритм успешно обработал граф.
    \item \textbf{Тест на большом случайном графе:} Проверка работы алгоритма на большом случайно сгенерированном графе. Алгоритм успешно обработал граф.
\end{itemize}

Все функциональные тесты прошли успешно, что подтвердило корректность реализации алгоритма Беллмана-Форда как в последовательной, так и в параллельной версии.

\section*{Тесты производительности}

Тесты производительности были направлены на оценку времени выполнения алгоритма в зависимости от размера графа и количества процессов MPI. Для проведения тестов использовались графы различных размеров, включая средний граф с девятью вершинами и 28 ребрами.

\begin{itemize}
    \item \textbf{Последовательная версия:} Время выполнения алгоритма на среднем графе составило \( T_{\text{seq}} \), что соответствует времени, затраченному на последовательную обработку графа.
    \item \textbf{Параллельная версия с использованием MPI:} Время выполнения алгоритма на том же графе с использованием \( N \) процессов MPI составило \( T_{\text{mpi}} \). Было замечено, что с увеличением числа процессов время выполнения уменьшалось, что свидетельствует о эффективности параллельной реализации.
\end{itemize}

Для оценки ускорения была рассчитана величина \( S = \frac{T_{\text{seq}}}{T_{\text{mpi}}} \), которая показала, что параллельная версия алгоритма обеспечивает значительное ускорение по сравнению с последовательной версией, особенно при увеличении числа процессов MPI.

\section*{Анализ результатов}

Результаты экспериментов подтвердили, что параллельная версия алгоритма Беллмана-Форда с использованием MPI эффективно распределяет вычисления между процессами, что приводит к значительному сокращению времени выполнения, особенно на больших графах. Однако, при небольшом количестве процессов или на малых графах, накладные расходы на коммуникацию между процессами могут снижать эффективность параллельной версии.

В заключение, реализация алгоритма Беллмана-Форда с использованием MPI показала свою эффективность и корректность, что делает её подходящей для решения задач на графах большого размера.

\section*{Заключение}

В ходе выполнения лабораторной работы была успешно разработана и реализована последовательная и параллельная (с использованием MPI) версии алгоритма Беллмана-Форда для поиска кратчайших путей в графе. Реализация алгоритма была выполнена с использованием формата CRS (Compressed Row Storage), что позволило эффективно работать с разреженными графами и минимизировать объём используемой памяти.

Функциональные тесты подтвердили корректность работы алгоритма на различных типах графов, включая графы с отрицательными весами и без них. Тесты производительности показали, что параллельная версия алгоритма с использованием MPI обеспечивает значительное ускорение по сравнению с последовательной версией, особенно при увеличении числа процессов и работе с большими графами. Однако, на малых графах и при небольшом количестве процессов накладные расходы на коммуникацию между процессами могут снижать эффективность параллельной реализации.

Таким образом, разработанные алгоритмы являются эффективными инструментами для решения задачи поиска кратчайших путей в графах, пригодными как для последовательной, так и для параллельной обработки данных. Результаты работы подтверждают, что использование MPI позволяет значительно сократить время выполнения алгоритма, что делает его пригодным для решения задач большого масштаба.

\newpage
\section*{Список литературы}

\begin{enumerate}
    \item «Язык программирования C++», Бьерн Страуструп.
    \item «Беллман-Форд за 5 минут — Пошаговый пример», Michael Sambol.
    \url {https://www.youtube.com/watch?v=obWXjtg0L64}.
    \item Bellman, R., "On a Routing Problem".
    \item Ford, L. R., "Network Flow Theory".
\end{enumerate}

\newpage
\section*{Приложение}

\begin{lstlisting}[caption={ops\_mpi\_.hpp}]
#pragma once

#include <gtest/gtest.h>

#include <boost/mpi/collectives.hpp>
#include <boost/mpi/communicator.hpp>
#include <memory>
#include <vector>

#include "core/task/include/task.hpp"

namespace zinoviev_a_bellman_ford_mpi {

class BellmanFordMPISeq : public ppc::core::Task {
 public:
  explicit BellmanFordMPISeq(std::shared_ptr<ppc::core::TaskData> taskData_) : Task(std::move(taskData_)) {}
  bool pre_processing() override;
  bool validation() override;
  bool run() override;
  bool post_processing() override;

 private:
  size_t V{};
  size_t E{};
  std::vector<int> values;
  std::vector<int> columns;
  std::vector<int> row_ptr;
  std::vector<int> shortest_paths;
  static constexpr int INF = std::numeric_limits<int>::max();

  bool Iteration(std::vector<int>& paths);
  bool check_negative_cycle();
  void toCRS(const int* input_matrix);
};

class BellmanFordMPIMPI : public ppc::core::Task {
 public:
  explicit BellmanFordMPIMPI(std::shared_ptr<ppc::core::TaskData> taskData_) : Task(std::move(taskData_)) {}
  bool pre_processing() override;
  bool validation() override;
  bool run() override;
  bool post_processing() override;

 private:
  size_t V{};
  size_t E{};
  std::vector<int> values;
  std::vector<int> columns;
  std::vector<int> row_ptr;
  std::vector<int> shortest_paths;
  boost::mpi::communicator world;
  static constexpr int INF = std::numeric_limits<int>::max();

  bool Iteration(std::vector<int>& paths);
  bool check_negative_cycle();
  void toCRS(const int* input_matrix);
};

}  // namespace zinoviev_a_bellman_ford_mpi
\end{lstlisting}

\newpage

\begin{lstlisting}[caption={ops\_mpi\_.cpp}]
#include "mpi/zinoviev_a_bellman_ford/include/ops_mpi.hpp"

#include <algorithm>
#include <boost/mpi/collectives.hpp>
#include <boost/mpi/communicator.hpp>
#include <boost/serialization/vector.hpp>
#include <vector>

namespace zinoviev_a_bellman_ford_mpi {

void BellmanFordMPIMPI::toCRS(const int* input_matrix) {
  row_ptr.push_back(0);
  for (size_t i = 0; i < V; ++i) {
    for (size_t j = 0; j < V; ++j) {
      if (input_matrix[i * V + j] != 0) {
        values.push_back(input_matrix[i * V + j]);
        columns.push_back(j);
      }
    }
    row_ptr.push_back(values.size());
  }
}

bool BellmanFordMPIMPI::pre_processing() {
  internal_order_test();
  if (world.rank() == 0) {
    auto* input_matrix = reinterpret_cast<int*>(taskData->inputs[0]);
    V = taskData->inputs_count[0];
    E = taskData->inputs_count[1];

    toCRS(input_matrix);
  }
  boost::mpi::broadcast(world, V, 0);
  boost::mpi::broadcast(world, values, 0);
  boost::mpi::broadcast(world, columns, 0);
  boost::mpi::broadcast(world, row_ptr, 0);

  shortest_paths.resize(V, INF);
  shortest_paths[0] = 0;

  // Broadcast paths to all ranks
  boost::mpi::broadcast(world, shortest_paths, 0);

  return true;
}

bool BellmanFordMPIMPI::validation() {
  internal_order_test();
  if (world.rank() == 0) {
    return taskData->inputs_count[0] < taskData->inputs_count[1] &&
           taskData->inputs_count[0] == taskData->outputs_count[0];
  }
  return true;
}

bool BellmanFordMPIMPI::Iteration(std::vector<int>& paths) {
  bool changed = false;
  std::vector<int> start_paths = paths;
  int rank = world.rank();
  int size = world.size();

  int local_size = V / size;
  int remainder = V % size;
  int start = rank * local_size + std::min(rank, remainder);
  int end = start + local_size + (rank < remainder ? 1 : 0);

  for (int i = start; i < end; i++) {
    for (int j = row_ptr[i]; j < row_ptr[i + 1]; j++) {
      int v = columns[j];
      int weight = values[j];
      if (start_paths[i] != INF && start_paths[i] + weight < start_paths[v]) {
        start_paths[v] = start_paths[i] + weight;
      }
    }
  }

  std::vector<int> reduced_paths(V, INF);
  boost::mpi::reduce(
      world, start_paths.data(), V, reduced_paths.data(),
      [](int a, int b) {
        if (a == zinoviev_a_bellman_ford_mpi::BellmanFordMPIMPI::INF) return b;
        if (b == zinoviev_a_bellman_ford_mpi::BellmanFordMPIMPI::INF) return a;
        return std::min(a, b);
      },
      0);

  // Broadcast the reduced paths to all ranks
  boost::mpi::broadcast(world, reduced_paths, 0);

  if (world.rank() == 0) {
    for (size_t i = 0; i < V; i++) {
      if (paths[i] != reduced_paths[i]) {
        changed = true;
        break;
      }
    }
    paths = reduced_paths;
  }

  boost::mpi::broadcast(world, paths.data(), paths.size(), 0);
  boost::mpi::broadcast(world, changed, 0);

  return changed;
}

bool BellmanFordMPIMPI::check_negative_cycle() {
  for (size_t i = 0; i < V; ++i) {
    for (int j = row_ptr[i]; j < row_ptr[i + 1]; ++j) {
      int v = columns[j];
      int weight = values[j];
      if (shortest_paths[i] != INF && shortest_paths[i] + weight < shortest_paths[v]) {
        return true;
      }
    }
  }
  return false;
}

bool BellmanFordMPIMPI::run() {
  internal_order_test();

  bool changed = false;
  for (size_t i = 0; i < V - 1; ++i) {
    changed = Iteration(shortest_paths);
    boost::mpi::broadcast(world, shortest_paths, 0);

    if (!changed) break;
  }

  if (world.rank() == 0) {
    for (size_t i = 0; i < V; ++i) {
      if (shortest_paths[i] == INF) shortest_paths[i] = 0;
    }
  }

  boost::mpi::broadcast(world, shortest_paths, 0);

  if (world.rank() == 0) {
    return !check_negative_cycle();
  }

  return true;
}

bool BellmanFordMPIMPI::post_processing() {
  internal_order_test();
  if (world.rank() == 0) {
    for (size_t i = 0; i < V; ++i) {
      reinterpret_cast<int*>(taskData->outputs[0])[i] = shortest_paths[i];
    }
  }
  return true;
}

}  // namespace zinoviev_a_bellman_ford_mpi
\end{lstlisting}

\newpage

\begin{lstlisting}[caption={ops\_seq\_.hpp}]
#pragma once

#include <gtest/gtest.h>

#include <memory>
#include <vector>

#include "core/task/include/task.hpp"

namespace zinoviev_a_bellman_ford_seq {

class BellmanFordSeq : public ppc::core::Task {
 public:
  explicit BellmanFordSeq(std::shared_ptr<ppc::core::TaskData> taskData_) : Task(std::move(taskData_)) {}
  bool pre_processing() override;
  bool validation() override;
  bool run() override;
  bool post_processing() override;

 private:
  size_t V{};
  size_t E{};
  std::vector<int> values;
  std::vector<int> columns;
  std::vector<int> row_ptr;
  std::vector<int> shortest_paths;
  const int INF = std::numeric_limits<int>::max();

  bool Iteration(std::vector<int>& paths);
  bool check_negative_cycle();
  void toCRS(const int* input_matrix);
};

}  // namespace zinoviev_a_bellman_ford_seq
\end{lstlisting}

\newpage

\begin{lstlisting}[caption={ops\_seq\_.cpp}]
// Copyright 2024 Nesterov Alexander
#include "seq/zinoviev_a_bellman_ford/include/ops_seq.hpp"

#include <algorithm>
#include <vector>

namespace zinoviev_a_bellman_ford_seq {

void BellmanFordSeq::toCRS(const int* input_matrix) {
  row_ptr.push_back(0);
  for (size_t i = 0; i < V; ++i) {
    for (size_t j = 0; j < V; ++j) {
      if (input_matrix[i * V + j] != 0) {
        values.push_back(input_matrix[i * V + j]);
        columns.push_back(j);
      }
    }
    row_ptr.push_back(values.size());
  }
}

bool BellmanFordSeq::pre_processing() {
  internal_order_test();
  auto* input_matrix = reinterpret_cast<int*>(taskData->inputs[0]);
  V = taskData->inputs_count[0];
  E = taskData->inputs_count[1];

  toCRS(input_matrix);

  shortest_paths.resize(V, INF);
  shortest_paths[0] = 0;
  return true;
}

bool BellmanFordSeq::validation() {
  internal_order_test();

  size_t V_count = taskData->inputs_count[0];
  size_t E_expected = taskData->inputs_count[1];
  size_t E_actual = 0;

  auto* input_matrix = reinterpret_cast<int*>(taskData->inputs[0]);

  for (size_t i = 0; i < V_count * V_count; ++i) {
    if (input_matrix[i] != 0) {
      E_actual++;
    }
  }

  if (E_expected != E_actual) {
    return false;
  }

  for (size_t i = 0; i < V_count - 1; ++i) {
    bool has_outgoing = false;
    for (size_t j = 0; j < V_count; ++j) {
      if (input_matrix[i * V_count + j] != 0) {
        has_outgoing = true;
        break;
      }
    }
    if (!has_outgoing) {
      return false;
    }
  }

  for (size_t i = 0; i < V_count; ++i) {
    if (input_matrix[i * V_count + i] != 0) {
      return false;
    }
  }

  return V_count == taskData->outputs_count[0];
}

bool BellmanFordSeq::Iteration(std::vector<int>& paths) {
  bool changed = false;
  for (size_t i = 0; i < V; ++i) {
    for (int j = row_ptr[i]; j < row_ptr[i + 1]; ++j) {
      int v = columns[j];
      int weight = values[j];
      if (paths[i] != INF && paths[i] + weight < paths[v]) {
        paths[v] = paths[i] + weight;
        changed = true;
      }
    }
  }
  return changed;
}

bool BellmanFordSeq::check_negative_cycle() {
  for (size_t i = 0; i < V; ++i) {
    for (int j = row_ptr[i]; j < row_ptr[i + 1]; ++j) {
      int v = columns[j];
      int weight = values[j];
      if (shortest_paths[i] != INF && shortest_paths[i] + weight < shortest_paths[v]) {
        return true;
      }
    }
  }
  return false;
}

bool BellmanFordSeq::run() {
  internal_order_test();

  bool changed = false;
  for (size_t i = 0; i < V - 1; ++i) {
    changed = Iteration(shortest_paths);
    if (!changed) break;
  }

  if (check_negative_cycle()) {
    return false;
  }

  for (size_t i = 0; i < V; ++i) {
    if (shortest_paths[i] == INF) shortest_paths[i] = 0;
  }
  return true;
}

bool BellmanFordSeq::post_processing() {
  internal_order_test();
  for (size_t i = 0; i < V; ++i) {
    reinterpret_cast<int*>(taskData->outputs[0])[i] = shortest_paths[i];
  }
  return true;
}

}  // namespace zinoviev_a_bellman_ford_seq
\end{lstlisting}

\end{document}
