\documentclass[12pt]{article}
\usepackage[utf8]{inputenc}
\usepackage[russian]{babel}
\usepackage{amsmath}
\usepackage{geometry}
\usepackage{titlesec}
\usepackage{hyperref}
\usepackage{tocloft}
\usepackage{xcolor} 
\usepackage{listings}
\geometry{a4paper, left=25mm, right=15mm, top=20mm, bottom=20mm}

\hypersetup{
    colorlinks=true,
    linkcolor=black,
    urlcolor=black,
    pdftitle={Отчет по проекту},
    pdfauthor={Гусев Никита},
    pdfsubject={Поиск кратчайших путей из одной вершины, Алгоритм Дейкстры}
}

\lstset{
  language=C++,
  basicstyle=\ttfamily\small,
  keywordstyle=\bfseries\color{blue},
  commentstyle=\itshape\color{green!50!black},
  stringstyle=\color{red!60!black},
  numbers=left,
  numberstyle=\tiny,
  stepnumber=1,
  numbersep=8pt,
  backgroundcolor=\color{gray!10},
  showspaces=false,
  showstringspaces=false,
  breaklines=true,
  frame=single,
  tabsize=2,
  captionpos=b
}

\renewcommand{\cftsecaftersnum}{.}
\renewcommand{\cftsecleader}{\cftdotfill{\cftdotsep}}
\titleformat{\section}{\large\bfseries}{\thesection.}{1em}{}

\title{Отчёт по теме: \\ Поиск кратчайших путей из одной вершины \\ (Алгоритм Дейкстры с использованием CRS графов) \\ ННГУ им. Лобачевского \\ Параллельное программирование}
\author{Гусев Никита}

\begin{document}

\maketitle
\begin{abstract}
    В данном отчете рассматривается реализация алгоритма Дейкстры для поиска кратчайших путей из одной вершины в графах, представленных в формате CRS (Compressed Row Storage). Описание включает в себя структуру данных, используемую для представления графа, а также тестирование производительности и валидацию алгоритма. 
\end{abstract}
\newpage
\tableofcontents
\newpage

\section{Введение}
Поиск кратчайших путей в графах является одной из ключевых задач в области компьютерных наук и оптимизации. Графы представляют собой мощный инструмент для моделирования различных систем, таких как транспортные сети, социальные сети и компьютерные сети. Алгоритм Дейкстры, предложенный Эдсгером Дейкстрой в 1956 году, является одним из самых известных методов решения данной задачи, особенно для графов с неотрицательными весами рёбер.

Дейкстра использует жадный подход, постепенно расширяя известные кратчайшие пути от начальной вершины ко всем остальным вершинам графа. В данной работе рассматривается параллельная реализация алгоритма Дейкстры с использованием библиотеки MPI (Message Passing Interface) для эффективной обработки больших графов, представленных в формате CRS. Параллельная обработка позволяет значительно ускорить вычисления на многоядерных системах и кластерах.

\newpage
\section{Постановка задачи}
В данной работе мы рассматриваем параллельную реализацию алгоритма Дейкстры поиска кратчайшой величины, использующую библиотеку MPI (Message Passing Interface) для эффективной обработки графов в формате CRS.
\newpage
\section{Структура данных}
\subsection{SparseGraphCRS}
Граф представлен с использованием структуры данных \texttt{SparseGraphCRS}, которая включает в себя три основных массива:
\begin{itemize}
    \item \texttt{values} - массив весов рёбер,
    \item \texttt{col\_indices} - массив индексов столбцов (целевых вершин),
    \item \texttt{row\_ptr} - массив указателей на начало каждой строки.
\end{itemize}

\begin{lstlisting}[language=c++, caption={Определение структуры SparseGraphCRS}]
struct SparseGraphCRS {
    std::vector<double> values;
    std::vector<int> col_indices;
    std::vector<int> row_ptr;
    int num_vertices;

    SparseGraphCRS(int n) : num_vertices(n) { row_ptr.resize(n + 1, 0); }

    void add_edge(int u, int v, double weight) {
        values.push_back(weight);
        col_indices.push_back(v);
        for (size_t i = u + 1; i < row_ptr.size(); ++i) {
            row_ptr[i]++;
        }
    }
};
\end{lstlisting}
\subsection{Преимущества представления CRS}
Использование формата CRS имеет несколько ключевых преимуществ:
\begin{itemize}
\item \texttt{Экономия памяти}: Хранение только ненулевых значений значительно снижает потребление памяти по сравнению с полными матрицами.
\item \texttt{Быстрый доступ}: Позволяет эффективно получать доступ к рёбрам, исходящим из каждой вершины, что критично для алгоритмов поиска.
\item \texttt{Легкость в реализации}: Простота реализации операций добавления рёбер и доступа к данным упрощает разработку алгоритмов.
\end{itemize}
\newpage
\section{Реализация}
\subsection{Основные функции}
Класс \texttt{DijkstrasAlgorithmParallel} реализует алгоритм Дейкстры и включает в себя несколько ключевых методов:
\begin{itemize}
    \item \texttt{validation()} - проверяет корректность входных данных.
    \item \texttt{pre\_processing()} - выполняет предварительную обработку.
    \item \texttt{run()} - основной метод, реализующий алгоритм Дейкстры.
    \item \texttt{post\_processing()} - выполняет завершающие действия и возвращает результаты.
\end{itemize}
\subsection{Метод validation()}
Метод \texttt{validation()} проверяет данные на корректность:
\begin{lstlisting}[language=c++, caption={Сам метод}]
bool DijkstrasAlgorithmParallel::validation() {
  internal_order_test();
  if (taskData->inputs.empty() || taskData->inputs[0] == nullptr ||
      taskData->inputs.size() != taskData->inputs_count.size() || taskData->inputs_count[0] != sizeof(SparseGraphCRS)) {
    return false;
  }

  auto* graph = reinterpret_cast<SparseGraphCRS*>(taskData->inputs[0]);

  if (graph->num_vertices == 0) {
    return false;
  }

  if (taskData->outputs.empty() || taskData->outputs[0] == nullptr ||
      taskData->outputs.size() != taskData->outputs_count.size()) {
    return false;
  }

  return true;
}
\end{lstlisting}
В этом методе проверяются:\\
\begin{itemize}
    \item Входные и выходные данные (не ноль, размер)
    \item Количество вершин графа не ноль
\end{itemize}
\subsection{Метод run()}
Метод \texttt{run()} реализует основной цикл алгоритма Дейкстры, в котором происходит:
\begin{itemize}
    \item Инициализация локальных расстояний и посещённых вершин.
    \item Итеративный поиск минимальной вершины и обновление расстояний.
\end{itemize}
Основные моменты из кода:
\begin{lstlisting}
int vertices_per_proc = num_vertices / num_procs;
int start_vertex = rank * vertices_per_proc;
int end_vertex = (rank == num_procs - 1) ? num_vertices : start_vertex + vertices_per_proc;
\end{lstlisting}
Определяется, сколько вершин будет обрабатывать каждый процесс.\\
start\_vertex и end\_vertex определяют диапазон вершин, который будет обрабатываться текущим процессом.
\begin{lstlisting}
  if (rank == 0) {
    local_distances[source_vertex] = 0.0;
  }
\end{lstlisting}
Задаёт расстояние от начальной точки до самой себя = 0.
\begin{lstlisting}
local_distances.resize(num_vertices, std::numeric_limits<double>::infinity());
std::vector<bool> local_visited(num_vertices, false);
\end{lstlisting}
local\_distances инициализируется значениями бесконечности, указывая, что расстояния до всех вершин из начальной вершины еще не известны.\\
local\_visited используется для отслеживания, какие вершины были посещены.
\begin{lstlisting}[language=c++, caption={Основной цикл, где происходит поиск минимума}]
bool DijkstrasAlgorithmParallel::run() {
    // Initialize
    ...
    for (int iter = 0; iter < num_vertices - 1; ++iter) {
        MinVertex local_min{std::numeric_limits<double>::infinity(), -1};
        ...
        // Searching for a global minimum
        ...
        // Updating distances
        ...
    }
    return true;
}
\end{lstlisting}
\subsection{Метод post\_processing()}
Метод \texttt{post\_processing()} реализует копию вычисленных данных в output:
\begin{lstlisting}[language=c++, caption={Сам метод}]
bool DijkstrasAlgorithmParallel::post_processing() {
  internal_order_test();
  int rank = world.rank();

  if (rank == 0) {
    auto* output = reinterpret_cast<double*>(taskData->outputs[0]);
    std::copy(local_distances.begin(), local_distances.end(), output);
  }

  return true;
}
\end{lstlisting}
\newpage
\section{Тестирование}
\subsection{Производительность}
Для оценки производительности алгоритма были проведены тесты, где генерировались случайные графы и измерялось время выполнения алгоритма.
В этих тестах используется два вида графов - цикличные и двудольные:
\begin{lstlisting}[language=c++, caption={Пример функционального теста}]
void create_cycle_graph(
    std::shared_ptr<gusev_n_dijkstras_algorithm_mpi::DijkstrasAlgorithmParallel::SparseGraphCRS>& graph,
    int num_vertices) {
  for (int i = 0; i < num_vertices; ++i) {
    double weight = static_cast<double>(rand()) / RAND_MAX * 10.0 + 0.1;  // Random weight between 0.1 and 10.0
    graph->add_edge(i, (i + 1) % num_vertices, weight);                   // Connect in a cycle
  }
}

void create_bipartite_graph(
    std::shared_ptr<gusev_n_dijkstras_algorithm_mpi::DijkstrasAlgorithmParallel::SparseGraphCRS>& graph,
    int num_vertices) {
  int half = num_vertices / 2;
  for (int u = 0; u < half; ++u) {
    for (int v = half; v < num_vertices; ++v) {
      double weight = static_cast<double>(rand()) / RAND_MAX * 10.0 + 0.1;  // Random weight between 0.1 and 10.0
      graph->add_edge(u, v, weight);
    }
  }
}
\end{lstlisting}
\subsection{Функциональные тесты}
Функциональные тесты проверяют корректность работы алгоритма на различных типах графов, включая:
\begin{itemize}
    \item Связные графы
    \item Несвязные графы
    \item Пустые графы
\end{itemize}
Пример функционального теста, который проверяет работу алгоритма на простом графе:

\begin{lstlisting}[language=c++, caption={Пример функционального теста}]
TEST(gusev_n_dijkstras_algorithm_mpi, TestDijkstra) {
  boost::mpi::communicator world;
  auto task_data = std::make_shared<ppc::core::TaskData>();

  auto graph = std::make_shared<gusev_n_dijkstras_algorithm_mpi::DijkstrasAlgorithmParallel::SparseGraphCRS>(5);
  graph->add_edge(0, 1, 2.0);
  graph->add_edge(0, 2, 3.0);
  graph->add_edge(1, 2, 1.0);
  graph->add_edge(2, 0, 4.0);
  graph->add_edge(3, 4, 5.0);

  task_data->inputs.push_back(reinterpret_cast<uint8_t*>(graph.get()));
  task_data->inputs_count.push_back(
      sizeof(gusev_n_dijkstras_algorithm_mpi::DijkstrasAlgorithmParallel::SparseGraphCRS));

  std::vector<double> output_data(5);
  task_data->outputs.push_back(reinterpret_cast<uint8_t*>(output_data.data()));
  task_data->outputs_count.push_back(output_data.size() * sizeof(double));

  gusev_n_dijkstras_algorithm_mpi::DijkstrasAlgorithmParallel task(task_data);
  ASSERT_TRUE(task.validation());
  ASSERT_TRUE(task.pre_processing());
  ASSERT_TRUE(task.run());
  ASSERT_TRUE(task.post_processing());

  if (world.rank() == 0) {
    std::vector<double> expected_distances = {0.0, 2.0, 3.0, std::numeric_limits<double>::infinity(),
                                              std::numeric_limits<double>::infinity()};
    for (size_t i = 0; i < expected_distances.size(); ++i) {
      if (std::isinf(expected_distances[i])) {
        EXPECT_TRUE(std::isinf(output_data[i])) << "Distance to vertex " << i << " should be infinity";
      } else {
        EXPECT_NEAR(output_data[i], expected_distances[i], 1e-6) << "Incorrect distance to vertex " << i;
      }
    }
  }
}
\end{lstlisting}
Этот тест создает граф с 5 вершинами и несколькими ребрами, затем проверяет, правильно ли алгоритм находит кратчайшие расстояния от начальной вершины (0) до других вершин (некоторые вершины недостежимы, поэтому inf).
\newpage
\section{Заключение}
Реализация алгоритма Дейкстры с использованием CRS графов и параллельной обработки с помощью MPI позволяет эффективно находить кратчайшие пути в больших графах. Проведенные тесты подтвердили корректность и производительность алгоритма.


\end{document}
