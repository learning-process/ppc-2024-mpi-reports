\documentclass[a4paper,12pt]{article}
\usepackage[T2A]{fontenc}
\usepackage[utf8]{inputenc}
\usepackage[russian]{babel}
\usepackage{amsmath}
\usepackage{amssymb}
\usepackage{booktabs}
\usepackage{listings}
\usepackage{geometry}
\usepackage{titlesec}
\usepackage{hyperref}
\usepackage{tocloft}
\usepackage{xcolor} 
\usepackage{float}


\geometry{top=20mm, bottom=20mm, left=25mm, right=25mm}

\lstset{
  language=C++,
  basicstyle=\ttfamily\small,
  keywordstyle=\bfseries\color{blue},
  commentstyle=\itshape\color{green!50!black},
  stringstyle=\color{red!60!black},
  numbers=left,
  numberstyle=\tiny,
  stepnumber=1,
  numbersep=8pt,
  backgroundcolor=\color{gray!10},
  showspaces=false,
  showstringspaces=false,
  breaklines=true,
  frame=single,
  tabsize=2,
  captionpos=b
}
\title{Метод Симпсона для численного интегрирования с использованием MPI}
\author{Наумов Богдан ПР2\\Научный руководитель: Иванов И.И., к.ф.-м.н., доцент}
\date{Декабрь 2024}

\begin{document}

\maketitle

\section{Введение}
Численные методы интегрирования играют ключевую роль в научных вычислениях, где аналитическое решение интегралов невозможно или слишком сложное для нахождения. Одним из наиболее известных методов численного интегрирования является метод Симпсона, который является более точной версией метода трапеций.

В данной работе рассматривается реализация метода Симпсона с использованием параллельных вычислений в рамках библиотеки MPI. Это позволяет значительно ускорить вычисления при больших объёмах данных.

\section{Постановка задачи}
Цель данной работы — разработать и реализовать метод Симпсона для численного интегрирования с использованием MPI на языке программирования C++. Задача включает в себя:
\begin{itemize}
    \item Теоретическое изучение метода Симпсона.
    \item Описание алгоритма и его реализация с использованием MPI.
    \item Анализ корректности работы алгоритма.
    \item Оценка производительности программы с использованием параллельных вычислений.
\end{itemize}

\section{Описание алгоритма}
Метод Симпсона используется для приближённого вычисления определённого интеграла \( I = \int_a^b f(x) \, dx \). Интервал \([a, b]\) делится на \(n\) подынтервалов, причём \(n\) должно быть чётным. На каждом подынтервале функция \(f(x)\) аппроксимируется параболой.

Формула Симпсона для вычисления интеграла выглядит следующим образом:
\[
I \approx \frac{h}{3} \left( f(a) + 4 \sum_{i=1, i \text{ нечётное}}^{n-1} f(x_i) + 2 \sum_{i=2, i \text{ чётное}}^{n-2} f(x_i) + f(b) \right),
\]
где \(h = \frac{b-a}{n}\).

В предложенной реализации, метод интегрирования разделён на несколько этапов:
\begin{itemize}
    \item На первом шаге происходит вычисление значений функции в пределах одного подынтервала.
    \item На втором шаге используется параллельное вычисление интегралов на разных процессах с использованием MPI.
    \item Далее, каждый процесс выполняет свою часть работы, после чего результаты собираются в общий результат с помощью операции редукции.
\end{itemize}

\section{Реализация с использованием MPI}
Реализация метода Симпсона с использованием MPI позволяет распараллелить вычисления и значительно ускорить обработку данных. Для этого используется библиотека Boost.MPI.

\subsection{Основной код}
Ниже представлен код, реализующий метод Симпсона для интегрирования с использованием MPI:

\begin{lstlisting}[language=C++, frame=single, caption={Реализация метода Симпсона с использованием MPI}, breaklines=true]
#include "mpi/naumov_b_simpson_method/include/ops_mpi.hpp"
#include <boost/mpi.hpp>
#include <cmath>
#include <functional>
#include <limits>
#include <stdexcept>
#include <utility>

namespace naumov_b_simpson_method_mpi {

double parallel_integrir_1d(const func_1d_t &func, double lower_bound, double upper_bound, 
                            int num_steps, boost::mpi::communicator &world) {
  double step_size = (upper_bound - lower_bound) / num_steps;

  int rank = world.rank();
  int size = world.size();

  int steps_per_process = num_steps / size;
  int remaining_steps = num_steps % size;

  int extra_steps = (rank < remaining_steps) ? 1 : 0;

  double local_lower_bound =
      lower_bound + rank * steps_per_process * step_size + 
      std::min(rank, remaining_steps) * step_size;
  double local_upper_bound = local_lower_bound + (steps_per_process + extra_steps) * step_size;

  double local_result = 0.0;

  if (steps_per_process + extra_steps > 0) {
    local_result += func(local_lower_bound);

    for (int i = 1; i < steps_per_process + extra_steps; ++i) {
      double x = local_lower_bound + i * step_size;
      int weight = (i % 2 == 0) ? 2 : 4;
      local_result += weight * func(x);
    }

    local_result += func(local_upper_bound);
    local_result *= step_size / 3.0;
  }

  double global_result = 0.0;
  boost::mpi::all_reduce(world, local_result, global_result, std::plus<>()); 

  return global_result;
}

}  // namespace naumov_b_simpson_method_mpi
\end{lstlisting}

\subsection{Классы для параллельных вычислений}
В этой части кода определяется класс для выполнения задачи на параллельных процессах:

\begin{lstlisting}[language=C++, frame=single, caption={Класс для параллельного выполнения задачи}, breaklines=true]
class TestMPITaskParallel : public ppc::core::Task {
 public:
  TestMPITaskParallel(
    std::shared_ptr<ppc::core::TaskData> task_data, 
    func_1d_t function, 
    double lower_bound, 
    double upper_bound, 
    int num_steps)
      : Task(std::move(task_data)),
        lower_bound_(lower_bound),
        upper_bound_(upper_bound),
        num_steps_(num_steps),
        function_(std::move(function)) {}

  bool pre_processing() override;
  bool validation() override;
  bool run() override;
  bool post_processing() override;

 private:
  double local_lower_bound_;
  double local_upper_bound_;
  int local_steps_;
  double lower_bound_;
  double upper_bound_;
  int num_steps_;
  func_1d_t function_;
  double local_result_;
  boost::mpi::communicator world;
};
\end{lstlisting}

\section{Тестирование}
Для проверки корректности работы программы были проведены тесты на различных функциях, таких как экспоненциальная, кубическая и другие. Тесты включают:

\begin{itemize}
    \item Экспоненциальная функция \( f(x) = e^x \)
    \item Кубическая функция \( f(x) = x^4 \)
    \item Синусоида \( f(x) = \sin(x) \)
    \item Линейная функция \( f(x) = 2x + 1 \)
    \item Квадратичная функция \( f(x) = x^2 \)
\end{itemize}

Каждый тест проверяет точность вычислений с использованием параллельных вычислений с MPI.

\section{Заключение}
Реализация метода Симпсона с использованием MPI позволяет эффективно решать задачи численного интегрирования на параллельных вычислительных системах. Параллельный подход значительно ускоряет процесс вычислений, особенно при больших объёмах данных. Реализованная программа показала высокую точность и корректность на всех тестах.

\end{document}
