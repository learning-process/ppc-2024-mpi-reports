\documentclass[a4paper,14pt]{report}
\usepackage[T2A]{fontenc}
\usepackage[utf8]{inputenc}
\usepackage[russian]{babel}
\usepackage{listings}
\usepackage[none]{hyphenat}
\title{Линейная фильтрация изображения (вертикальное разбиение). Ядро Гаусса 3х3}
\author{Выполнил: Садиков Иван, 
Группа: 3822Б1ПР1}
\begin{document}
\section{Введение}
Фильтрация изображений является одной из фундаментальных операций компьютерного зрения, распознавания образов и обработки изображений. Фактически, с той или иной фильтрации исходных изображений начинается работа подавляющего большинства методов. В то же время, фильтрация изображения - долгая и объемная задача, требующая распараллеливания.

\section{Постановка задачи}
Целью данной работы является реализация алгоритма линейной фильтрации изображения матрицей Гаусса с ядром $3 \times 3$, с использованием технологии MPI и языка программирования С++. Для выполнения задачи необходимо следующее:
\begin{itemize}
    \item Изучить предметную область, а именно - обработку изображений.
    \item Реализовать последовательную версию алгоритма линейной фильтрации.
    \item Реализовать параллельную версию алгоритма с упором на точность и эффективность решения.
    \item Провести тестирование последовательного и параллельного решения.
    \item Сделать вывод об эффективности параллельного решения в сравнении с последовательным.
\end{itemize}
\newpage
\section{Описание алгоритма}
Под фильтрацией изображений понимают операцию, имеющую своим результатом
изображение того же размера, полученное из исходного по некоторым правилам (фильтрам). Фильтрацию также называют сверткой. Существует немало фильтров, выполняющих самые разные функции: сглаживание, выделение контраста, перевод в оттенки серого и т. д. Наиболее простыми и потому часто используемыми фильтрами являются линейные.

Пусть существует некоторое изображение $A(x, y, z)$, где $(x, y, z)$ - интенсивность его пикселей, а также линейный фильтр $F$, определяемый вещественно значимой функцией $F(i, j)$. Тогда операция линейной свертки будет выглядеть следующим образом:
\begin{center}
    $B(x, y) = \sum_{i = 1}^{N}\sum_{j = 1}^{N}F(i, j) \cdot A(x, y)$
\end{center}
$B(x, y)$ - итоговое изображение после применение фильтра.
Функцию $F$ также называют ядром фильтром или матрицей свертки. По сути, $F$ - матрица коэффициентов, определенных математическим путём. Дабы избежать ошибок сглаживания, матрицы свертки создаются нечетными, например $3\times 3$ или $5\times 5$. 

К сожалению, применение линейной фильтрации изображений сопряжено с некоторыми трудностями, в частности - обработка граничных пикселей. Так как у граничных пикселей нет всех "соседей", классическая фильтрация становится невозможной. Существует множество путей решения данной проблемы:
\begin{itemize}
    \item Игнорирование граничных значений (грозит сильной потерей точности).
    \item Отзеркаливание значений по другую сторону от граничного пикселя.
    \item Частичное суммирование с существующими соседями.
    \item Распределение веса граничного пикселя между всеми пикселями изображения (серьезное повышение трудоемкости задачи).
\end{itemize}

В своей работе я решил проблему граничных пикселей третьим способом. За счет этого точность решения намного выше, чем в первом случае и не требует таких же затрат, как в последнем.

Фильтр Гаусса - наиболее часто применимый фильтр размытия, основанный на матрице свёртки. Ядро фильтра Гаусса имеет следующий вид:
\begin{center}
    $F_{gauss}(i, j) = \frac{1}{2\pi \sigma} \cdot \exp{(-\frac{i^2 + j^2}{2\sigma^2})}$
\end{center}
, где $\sigma$ - дисперсия случайной величины. Для своей лабораторной работы я выбрал $\sigma = 1$ 

\section{Описание схемы распараллеливания}
\begin{itemize}
    \item Вектор пикселей, где пиксель - класс, хранящий значение цветов по трем диапазонам, транспонируется и резделяется между существующими процессами. Каждому процессу отводится минимум 3 столбца, что является необходимым условием для правильного применения матрицы свертки.
    \item В каждом процессе совершается проход по отведенной области и запись соответствующих результатов в вектор localRes.
    \item Полученные данные объединяются на нулевом процессе, вновь транспонируются и подаются на ответ. 
\end{itemize}
\section{Описание MPI версии}
После чтения данных в вектор, выполняется серия broadcast операций, в том числе вектора пикселей. После обработки, данные собираются на нулевом процессе при помощи функции gatherv.
\newpage
\section{Результаты экспериментов}
Для проверки правильности работы алгоритма были вручную заполнены небольшие векторы пикселей. В качестве начальных значений брались случайные числа в диапазоне от 0 до 255. Для подсчета эталона и последующего сравнения с ним была использована библиотека numpy и функция Gaussbluration языка python. 

Тесты были проведены на матрицах пикселей размерностей $400 \times 400$, на двух, четырех, пяти процессах и показали следующие результаты:
Таблица результатов pipeline\_run:\\[0,5cm]
\begin{tabular}{|c|c|}
    \hline
    Число процессов   & Время выполнения (ms)  \\ \hline
    2               & 340                   \\ \hline
    4                &370                   \\ \hline
    5                &355                      \\ \hline
\end{tabular}\\[1cm]

Таблица результатов task\_run:\\[0,5cm]
\begin{tabular}{|c|c|}
    \hline
    Число процессов   & Время выполнения (ms)  \\ \hline
    2               & 280                   \\ \hline
    4                & 300                \\ \hline
    5                & 290                     \\ \hline
\end{tabular}\\[1cm]

Среднее время выполнения task\_run последовательной версии:\\[0,5cm]
\begin{tabular}{|c|c|}
    \hline
  & Время выполнения (ms)  \\ \hline
  & 300                   \\ \hline
\end{tabular}
\section{Выводы из результатов}
Колебание времени исполнения алгоритма на небольшом количестве процессов является не более, чем случайностью. Параллельное решение в среднем лишь немного быстрее, чем последовательное, что с большой долей вероятности является следствием применения broadcast для большого вектора входных данных. В качестве улучшения решения необходимо разделение общего вектора данных при помощи функции scatterv, но это повлечет за собой понижение общей точности решения, так как прежнее вычисление значений в крайних пикселях станет невозможным. 
\section{Заключение}
Реализовано распараллеливание алгоритма линейной фильтрации изображения матрицей Гаусса. Решение справляется со своей задачей, но может быть улучшенно с применением некоторых оптимизаций процессов распределения данных.
\newpage
\section{Приложение}
\begin{lstlistings}{
#pragma once

#include <boost/mpi/collectives.hpp>
#include <boost/mpi/communicator.hpp>
#include <boost/serialization/vector.hpp>
#include <cmath>
#include <cstdlib>
#include <ctime>
#include <iostream>
#include <memory>
#include <numbers>
#include <vector>

#include "core/task/include/task.hpp"
#include "seq/Sadikov_I_Gauss_Linear_Filtration/include/Point.h"

namespace Sadikov_I_Gauss_Linear_Filtration {
class LinearFiltrationSeq : public ppc::core::Task {
  std::vector<Point<double>> m_pixelsMatrix;
  std::vector<Point<double>> m_outMatrix;
  std::vector<double> m_gaussMatrix;
  int m_columnsCount = 0;
  int m_rowsCount = 0;
  static constexpr double sigma = 1.0;

 public:
  explicit LinearFiltrationSeq(std::shared_ptr<ppc::core::TaskData> taskData);
  bool validation() override;
  bool pre_processing() override;
  bool run() override;
  bool post_processing() override;
  void CalculateGaussMatrix();
  void CalculateNewPixelValue(int iIndex, int jIndex);
  bool CheckIndex(int index) const noexcept 
  { return index >= 0 and index < static_cast<int>(m_pixelsMatrix.size()); }
  bool CheckRowIndex(int index) const noexcept 
  { return index >= 0 and index < m_rowsCount; }
  bool CheckColumnIndex(int index) const noexcept 
  { return index >= 0 and index < m_columnsCount; }
};

class LinearFiltrationMPI : public ppc::core::Task {
  std::vector<Point<double>> m_pixelsMatrix;
  std::vector<Point<double>> m_outMatrix;
  std::vector<double> m_gaussMatrix;
  int m_columnsCount = 0;
  int m_rowsCount = 0;
  static constexpr double sigma = 1.0;
  static constexpr int m_minColumnsCount = 3;
  std::vector<int> m_sizes;
  boost::mpi::communicator world;
  std::vector<Point<double>> m_intermediateRes;

 public:
  explicit LinearFiltrationMPI(std::shared_ptr<ppc::core::TaskData> taskData);
  bool validation() override;
  bool pre_processing() override;
  bool run() override;
  bool post_processing() override;
  void CalculateGaussMatrix();
  void CalculateNewPixelValue(int iIndex, int jIndex, int position);
  bool CheckIndex(int index) const noexcept 
  { return index >= 0 and index < m_rowsCount * m_columnsCount; }
  bool CheckRowIndex(int index) const noexcept 
  { return index >= 0 and index < m_rowsCount; }
  bool CheckColumnIndex(int index) const noexcept 
  { return index >= 0 and index < m_columnsCount; }
};
}  // namespace Sadikov_I_Gauss_Linear_Filtration
\end{lstlistings}
\newpage
\begin{lstlistings}
#include "mpi/Sadikov_I_Gauss_Linear_Filtration/include/ops_mpi.h"

Sadikov_I_Gauss_Linear_Filtration::LinearFiltrationSeq::LinearFiltrationSeq(
    std::shared_ptr<ppc::core::TaskData> taskData)
    : Task(std::move(taskData)) {}

bool Sadikov_I_Gauss_Linear_Filtration::LinearFiltrationSeq::
validation() {
  internal_order_test();
  if ((taskData->inputs_count[0] > 2 || taskData->inputs_count[0] == 0) &&
      (taskData->inputs_count[1] > 2 || taskData->inputs_count[1] == 0)) 
      {
    return taskData->outputs_count[0] == taskData->inputs_count[0]
    * taskData->inputs_count[1];
  }
  return false;
}

bool Sadikov_I_Gauss_Linear_Filtration::LinearFiltrationSeq::
pre_processing() {
  internal_order_test();
  m_rowsCount = static_cast<int>(taskData->inputs_count[0]);
  m_columnsCount = static_cast<int>(taskData->inputs_count[1]);
  m_pixelsMatrix.reserve(m_rowsCount * m_columnsCount);
  auto *tmpPtr = reinterpret_cast<Point<double> *>(taskData->inputs[0]);
  for (int i = 0; i < m_columnsCount * m_rowsCount; ++i) {
    m_pixelsMatrix.emplace_back(tmpPtr[i]);
  }
  m_gaussMatrix.reserve(9);
  CalculateGaussMatrix();
  m_outMatrix = std::vector<Point<double>>(m_columnsCount * m_rowsCount);
  return true;
}

bool Sadikov_I_Gauss_Linear_Filtration::LinearFiltrationSeq::run() {
  internal_order_test();
  for (auto i = 0; i < m_rowsCount; ++i) {
    for (auto j = 0; j < m_columnsCount; ++j) {
      CalculateNewPixelValue(i, j);
    }
  }
  return true;
}

bool Sadikov_I_Gauss_Linear_Filtration::LinearFiltrationSeq::
post_processing() {
  internal_order_test();
  for (auto i = 0; i < m_rowsCount * m_columnsCount; ++i) {
    reinterpret_cast<Point<double> *>(taskData->outputs[0])[i] =
    m_outMatrix[i];
  }
  return true;
}

void Sadikov_I_Gauss_Linear_Filtration::LinearFiltrationSeq::
CalculateGaussMatrix() {
  double sum = 0;
  for (auto x = -1; x < 2; ++x) {
    for (auto y = -1; y < 2; ++y) {
      m_gaussMatrix.emplace_back(1 / (2 * std::numbers::pi * sigma * sigma) *
      std::pow(std::numbers::e, -((x * x + y * y) / 2.0 * sigma * sigma)));
      sum += m_gaussMatrix.back();
    }
  }
  for (auto &&it : m_gaussMatrix) {
    it /= sum;
  }
}

void Sadikov_I_Gauss_Linear_Filtration::LinearFiltrationSeq::
CalculateNewPixelValue(int iIndex, int jIndex) {
  auto index = 0;
  for (auto g = 0; g < 9; ++g) {
    if (g < 3) {
      if (CheckRowIndex(iIndex - 1) and CheckColumnIndex(jIndex - 1 + g)) {
        index = (iIndex - 1) * m_columnsCount + jIndex - 1 + g;
        if (CheckIndex(index)) {
          m_outMatrix[iIndex * m_columnsCount + jIndex] += 
          m_pixelsMatrix[index] * m_gaussMatrix[g];
        }
      }
    } else if (g > 2 && g < 6) {
      index = iIndex * m_columnsCount + jIndex - 4 + g;
      if (CheckColumnIndex(jIndex - 4 + g)) {
        if (CheckIndex(index)) {
          m_outMatrix[iIndex * m_columnsCount + jIndex] += 
          m_pixelsMatrix[index] * m_gaussMatrix[g];
        }
      }
    } else {
      if (CheckRowIndex(iIndex + 1) and CheckColumnIndex(jIndex - 7 + g)) {
        index = (iIndex + 1) * m_columnsCount + jIndex - 7 + g;
        if (CheckIndex(index)) {
          m_outMatrix[iIndex * m_columnsCount + jIndex] += m_pixelsMatrix[index] * m_gaussMatrix[g];
        }
      }
    }
  }
}

Sadikov_I_Gauss_Linear_Filtration::LinearFiltrationMPI::LinearFiltrationMPI(
    std::shared_ptr<ppc::core::TaskData> taskData)
    : Task(std::move(taskData)) {}

bool Sadikov_I_Gauss_Linear_Filtration::LinearFiltrationMPI::validation() {
  internal_order_test();
  if (world.rank() == 0) {
    if ((taskData->inputs_count[0] > 2 || taskData->inputs_count[0] == 0) &&
        (taskData->inputs_count[1] > 2 || taskData->inputs_count[1] == 0)) {
      return taskData->outputs_count[0] == taskData->inputs_count[0] 
      * taskData->inputs_count[1];
    }
    return false;
  }
  return true;
}

bool Sadikov_I_Gauss_Linear_Filtration::LinearFiltrationMPI::pre_processing() {
  internal_order_test();
  if (world.rank() == 0) {
    m_rowsCount = static_cast<int>(taskData->inputs_count[0]);
    m_columnsCount = static_cast<int>(taskData->inputs_count[1]);
    m_pixelsMatrix.reserve(m_rowsCount * m_columnsCount);
    auto *tmpPtr = reinterpret_cast<Point<double> *>(taskData->inputs[0]);
    for (auto column = 0; column < m_columnsCount; ++column) {
      for (auto row = 0; row < m_rowsCount; ++row) {
        m_pixelsMatrix.emplace_back(tmpPtr[m_columnsCount * row + column]);
      }
    }
    const int proceses = m_columnsCount / m_minColumnsCount;
    if (proceses > world.size()) {
      int averegeCount = m_columnsCount / world.size();
      m_sizes = std::vector<int>(world.size(), averegeCount * m_rowsCount);
      m_sizes.back() += (m_columnsCount % world.size()) * m_rowsCount;
    } else {
      int m_columns = m_columnsCount;
      if (!m_pixelsMatrix.empty()) {
        m_sizes.resize(world.size());
      }
      for (auto i = 0; i < world.size(); ++i) {
        if (m_columns >= m_minColumnsCount) {
          m_sizes[i] = m_minColumnsCount * m_rowsCount;
          m_columns -= m_minColumnsCount;
          if (m_columns < m_minColumnsCount) {
            m_sizes.back() += m_columns * m_rowsCount;
            break;
          }
        }
      }
    }
    m_gaussMatrix.reserve(9);
    m_outMatrix = std::vector<Point<double>>(m_columnsCount * m_rowsCount);
    CalculateGaussMatrix();
  }
  return true;
}

bool Sadikov_I_Gauss_Linear_Filtration::LinearFiltrationMPI::run() {
  internal_order_test();
  broadcast(world, m_sizes, 0);
  broadcast(world, m_columnsCount, 0);
  broadcast(world, m_rowsCount, 0);
  broadcast(world, m_pixelsMatrix, 0);
  broadcast(world, m_gaussMatrix, 0);
  // restart tests comment
  if (world.rank() < static_cast<int>(m_sizes.size())) {
    m_intermediateRes.resize(m_sizes[world.rank()]);
    int position = 0;
    for (auto row = std::accumulate(m_sizes.begin(), m_sizes.begin() + world.rank(), 0) / m_rowsCount;
         row < std::accumulate(m_sizes.begin(), m_sizes.begin() + world.rank() + 1, 0) / m_rowsCount; ++row) {
      for (auto column = 0; column < m_rowsCount; ++column) {
        CalculateNewPixelValue(row, column, position);
        position++;
      }
    }
  }
  if (world.rank() == 0 && !m_pixelsMatrix.empty()) {
    std::vector<Point<double>> localRes(m_rowsCount * m_columnsCount);
    boost::mpi::gatherv(world, m_intermediateRes, localRes.data(), m_sizes, 0);
    m_outMatrix = std::move(localRes);
  } else if (world.rank() != 0 && !m_pixelsMatrix.empty()) {
    boost::mpi::gatherv(world, m_intermediateRes, 0);
  }
  return true;
}

bool Sadikov_I_Gauss_Linear_Filtration::LinearFiltrationMPI::post_processing() {
  internal_order_test();
  if (world.rank() == 0) {
    for (auto i = 0; i < m_rowsCount; ++i) {
      for (auto j = 0; j < m_columnsCount; ++j) {
        reinterpret_cast<Point<double> *>(taskData->outputs[0])[j + i * m_columnsCount] =
            m_outMatrix[m_rowsCount * j + i];
      }
    }
  }
  return true;
}

void Sadikov_I_Gauss_Linear_Filtration::LinearFiltrationMPI::
CalculateGaussMatrix() {
  double sum = 0;
  for (auto x = -1; x < 2; ++x) {
    for (auto y = -1; y < 2; ++y) {
      m_gaussMatrix.emplace_back(1 / (2 * std::numbers::pi * sigma * sigma) *
      std::pow(std::numbers::e, -((x * x + y * y) / 2.0 * sigma * sigma)));
      sum += m_gaussMatrix.back();
    }
  }
  for (auto &&it : m_gaussMatrix) {
    it /= sum;
  }
}

void Sadikov_I_Gauss_Linear_Filtration::LinearFiltrationMPI::
CalculateNewPixelValue(int iIndex, int jIndex, int position) {
  auto index = 0;

  for (auto g = 0; g < 9; ++g) {
    if (g < 3) {
      if (CheckRowIndex(jIndex - 1 + g) and CheckColumnIndex(iIndex - 1)) {
        index = (iIndex - 1) * m_rowsCount + jIndex - 1 + g;
        if (CheckIndex(index)) {
          m_intermediateRes[position] += m_pixelsMatrix[index] 
          * m_gaussMatrix[g];
        }
      }
    } else if (g > 2 && g < 6) {
      index = iIndex * m_rowsCount + jIndex - 4 + g;
      if (CheckRowIndex(jIndex - 4 + g)) {
        if (CheckIndex(index)) {
          m_intermediateRes[position] += m_pixelsMatrix[index] 
          * m_gaussMatrix[g];
        }
      }
    } else {
      if (CheckRowIndex(jIndex - 7 + g) and CheckColumnIndex(iIndex + 1)) {
        index = (iIndex + 1) * m_rowsCount + jIndex - 7 + g;
        if (CheckIndex(index)) {
          m_intermediateRes[position] += m_pixelsMatrix[index]
          * m_gaussMatrix[g];
        }
      }
    }
  }
}
\end{lstlistings}
\end{document}

