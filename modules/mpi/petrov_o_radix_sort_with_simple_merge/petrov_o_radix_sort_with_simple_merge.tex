\documentclass[12pt]{article}

% Кодировка и русский язык
\usepackage[T2A]{fontenc}
\usepackage[utf8]{inputenc}
\usepackage[russian]{babel}

% Дополнительные пакеты
\usepackage{amsmath,amssymb}
\usepackage{geometry}
\usepackage{graphicx}
\usepackage{listings}
\usepackage{hyperref}
\geometry{left=2cm,right=2cm,top=2cm,bottom=2cm}

% Настройка отображения кода в листингах
\lstset{
  basicstyle=\small\ttfamily,
  breaklines=true,
  tabsize=2,
  language=C++,
  inputencoding=utf8,
  numbers=left,
  numberstyle=\tiny,
  frame=single,
  captionpos=b
}

\begin{document}

%---------------------- Титульный лист ----------------------------

\begin{titlepage}
\begin{center}
\large
\textbf{ННГУ им. Лобачевского / ИИТММ / ПМИ}\\[0.5cm]

\vspace{4cm}
\textbf{\Large Отчёт по выполнению задания}\\
\textbf{\large «Поразрядная сортировка для целых чисел с простым слиянием»}\\[3cm]

\vspace{3cm}
\textbf{Выполнил:}\\
студент группы 3822Б1ПМоп3 \\
\textit{Петров Олег Андреевич}\\[1cm]

\textbf{Преподаватель:}\\
\textit{Нестеров Александр Юрьевич, аспирант}\\[2cm]

\vfill
\textbf{Нижний Новгород, 2024 г.}
\end{center}
\end{titlepage}

%--------------------------- Оглавление ----------------------------
\tableofcontents
\newpage

%---------------------- Введение -----------------------------------

\section*{Введение}
\addcontentsline{toc}{section}{Введение}

В данном отчёте рассматривается задача реализации поразрядной сортировки для целых чисел с использованием простой схемы слияния. Основная цель работы --- разработка как последовательного, так и параллельного (с использованием MPI) алгоритмов сортировки целочисленного массива, а также анализ их производительности и корректности.

%---------------------- Постановка задачи --------------------------

\section{Постановка задачи}
В рамках данной работы необходимо:
\begin{enumerate}
\item Реализовать последовательный алгоритм поразрядной сортировки целочисленного массива с использованием простой схемы слияния.
\item Разработать параллельную версию алгоритма с использованием MPI, обеспечив равномерное распределение данных между процессами и слияние отсортированных частей.
\item Провести валидацию корректности обеих реализаций.
\item Проанализировать и сравнить производительность последовательной и параллельной версий алгоритма.
\end{enumerate}

\textbf{Входные данные:}
\begin{itemize}
\item Одномерный массив целых чисел размера $n$.
\end{itemize}

\textbf{Выходные данные:}
\begin{itemize}
\item Отсортированный массив размера $n$.
\end{itemize}

\textbf{Особенности реализации:}
\begin{itemize}
\item Поразрядная сортировка происходит по основанию двоичной системы счисления.
\item Для получения разрядов используются операторы смещения, а не остатка от деления, что повышает эффективность работы.
\item Исходный массив равномерно распределяется между процессами.
\item Первый бит каждого числа перед сортировкой инвертируется, что обеспечивает корректное размещение отрицательных чисел перед положительными.
\item Каждый процесс выполняет поразрядную сортировку своей части.
\item Результаты сливаются в один отсортированный массив на корневом процессе.
\item Алгоритм включает три основных этапа:
  \begin{enumerate}
    \item Распределение данных между процессами.
    \item Поразрядная сортировка локальных данных.
    \item Слияние результатов в общий отсортированный массив.
  \end{enumerate}
\end{itemize}

%---------------------- Описание алгоритма -------------------------

\section{Описание алгоритма поразрядной сортировки}

Поразрядная сортировка (Radix Sort) является устойчивым алгоритмом сортировки, который сортирует числа по разрядам, начиная с младшего разряда и заканчивая старшим. В данной реализации используется двоичная система счисления и простое слияние отсортированных частей.

\subsection{Основная идея алгоритма}

\begin{enumerate}
  \item \textbf{Инверсия первого бита:} Для корректного размещения отрицательных чисел перед положительными, первый бит каждого числа инвертируется.
  
  \item \textbf{Поразрядная сортировка:} Для каждого бита числа (от младшего к старшему) выполняется сортировка массива на основе значения этого бита. Сначала разделяются числа с нулевым битом и единичным битом, затем объединяются в отсортированном порядке.
  
  \item \textbf{Слияние отсортированных частей:} После сортировки всех разрядов локальные отсортированные части массива объединяются в один отсортированный массив.
  
  \item \textbf{Обратная инверсия первого бита:} После сортировки первый бит каждого числа снова инвертируется для восстановления исходных значений.
\end{enumerate}

\subsection{Параллелизация алгоритма}

Для повышения эффективности алгоритма на больших массивах данных используется параллельная реализация с использованием MPI. Основные этапы параллельной реализации:

\begin{enumerate}
  \item \textbf{Распределение данных:} Исходный массив равномерно распределяется между доступными процессами.
  
  \item \textbf{Локальная сортировка:} Каждый процесс выполняет поразрядную сортировку своей части массива.
  
  \item \textbf{Слияние результатов:} Отсортированные части массивов от всех процессов объединяются в один отсортированный массив на корневом процессе.
\end{enumerate}

%---------------------- Описание последовательной реализации -------------------------

\section{Описание последовательной реализации}

Последовательная версия алгоритма реализована на языке C++ и включает следующие основные этапы: валидация входных данных, предварительная обработка, выполнение сортировки и постобработка результатов.

\subsection{Структура класса}

Код последовательной реализации представлен в пространстве имён \texttt{petrov\_o\_radix\_sort\_with\_simple\_merge\_seq}. Класс \texttt{TestTaskSequential} отвечает за выполнение всех этапов сортировки.

\subsection{Этапы выполнения}

\subsubsection{\texttt{validation()}}
Метод \texttt{validation} проверяет корректность входных данных, убеждаясь, что входные массивы не пусты и содержат необходимые данные для сортировки.

\subsubsection{\texttt{pre\_processing()}}
На этапе предварительной обработки:
\begin{itemize}
  \item Копируются входные данные из \texttt{taskData} в локальные структуры \texttt{input\_} и \texttt{res}.
  \item Выполняется инверсия первого бита каждого числа для корректной сортировки отрицательных чисел.
\end{itemize}

\subsubsection{\texttt{run()}}
Основной этап выполнения сортировки включает:
\begin{enumerate}
  \item Определение максимального числа и количества разрядов.
  \item Поразрядная сортировка массива:
    \begin{itemize}
      \item Для каждого бита выполняется разделение массива на части с нулевым и единичным битом.
      \item Объединение частей в отсортированном порядке.
    \end{itemize}
  \item Обратная инверсия первого бита каждого числа.
\end{enumerate}

\subsubsection{\texttt{post\_processing()}}
На этапе постобработки отсортированный массив копируется обратно в \texttt{taskData} для дальнейшего использования.

\subsection{Фрагмент кода последовательной реализации}

\begin{lstlisting}[caption={Последовательная версия поразрядной сортировки}]
#include "seq/petrov_o_radix_sort_with_simple_merge/include/ops_seq.hpp"

#include <algorithm>
#include <cmath>
#include <limits>

namespace petrov_o_radix_sort_with_simple_merge_seq {

bool TestTaskSequential::validation() {
  internal_order_test();

  bool isValid = (!taskData->inputs_count.empty()) && (!taskData->inputs.empty()) && (!taskData->outputs.empty());

  return isValid;
}

bool TestTaskSequential::pre_processing() {
  internal_order_test();

  int size = taskData->inputs_count[0];
  input_.resize(size);

  int* input_data = reinterpret_cast<int*>(taskData->inputs[0]);
  std::copy(input_data, input_data + size, input_.begin());

  res.resize(size);
  return true;
}

bool TestTaskSequential::run() {
  internal_order_test();

  for (auto& num : input_) {
    num ^= 0x80000000;
  }

  auto max_num = static_cast<unsigned int>(input_[0]);
  for (const auto& num : input_) {
    if (static_cast<unsigned int>(num) > max_num) {
      max_num = static_cast<unsigned int>(num);
    }
  }
  int num_bits = 0;
  const int MAX_BITS = sizeof(unsigned int) * 8;
  while (num_bits < MAX_BITS && (max_num >> num_bits) > 0) {
    num_bits++;
  }

  std::vector<int> output(input_.size());

  for (int bit = 0; bit < num_bits; ++bit) {
    int zero_count = 0;

    for (const auto& num : input_) {
      if (((static_cast<unsigned int>(num) >> bit) & 1) == 0) {
        zero_count++;
      }
    }

    int zero_index = 0;
    int one_index = zero_count;

    for (const auto& num : input_) {
      if (((static_cast<unsigned int>(num) >> bit) & 1) == 0) {
        output[zero_index++] = num;
      } else {
        output[one_index++] = num;
      }
    }

    input_ = output;
  }

  for (auto& num : input_) {
    num ^= 0x80000000;
  }

  res = input_;
  return true;
}

bool TestTaskSequential::post_processing() {
  internal_order_test();

  int* output_ = reinterpret_cast<int*>(taskData->outputs[0]);
  for (size_t i = 0; i < res.size(); ++i) {
    output_[i] = res[i];
  }
  std::copy(res.begin(), res.end(), output_);

  return true;
}

}  // namespace petrov_o_radix_sort_with_simple_merge_seq
\end{lstlisting}

%---------------------- Описание параллельной реализации ------------------------

\section{Описание параллельной реализации}

Параллельная версия алгоритма реализована с использованием MPI и включает распределение данных между процессами, локальную сортировку и слияние результатов на корневом процессе.

\subsection{Структура класса}

Код параллельной реализации представлен в пространстве имён \texttt{petrov\_o\_radix\_sort\_with\_simple\_merge\_mpi}. Класс \texttt{TaskParallel} отвечает за выполнение всех этапов параллельной сортировки.

\subsection{Этапы выполнения}

\subsubsection{\texttt{validation()}}
Метод \texttt{validation} проверяет корректность входных данных только на корневом процессе, убеждаясь, что входные массивы не пусты и содержат необходимые данные для сортировки.

\subsubsection{\texttt{pre\_processing()}}
На этапе предварительной обработки:
\begin{itemize}
  \item На корневом процессе считываются все данные о массиве.
  \item Инициализируются локальные структуры для хранения части массива, отвечающей текущему процессу.
\end{itemize}

\subsubsection{\texttt{run()}}
Основной этап выполнения включает:
\begin{enumerate}
  \item \textbf{Рассылка данных:} Все необходимые данные (включая размеры и содержимое массива) рассылаются от корневого процесса ко всем остальным процессам с помощью \texttt{boost::mpi::broadcast}.
  
  \item \textbf{Распределение данных:} Исходный массив равномерно распределяется между всеми процессами. Каждый процесс получает свой поддиапазон элементов массива для сортировки.
  
  \item \textbf{Локальная сортировка:} Каждый процесс выполняет поразрядную сортировку своей части массива аналогично последовательной версии.
  
  \item \textbf{Сбор результатов:} Отсортированные части массивов от всех процессов собираются на корневом процессе с помощью \texttt{boost::mpi::gather}. Затем происходит слияние всех локальных отсортированных частей в единый отсортированный массив.
\end{enumerate}

\subsubsection{\texttt{post\_processing()}}
На этапе постобработки на корневом процессе отсортированный массив копируется обратно в \texttt{taskData} для дальнейшего использования.

\subsection{Фрагмент кода параллельной реализации}

\begin{lstlisting}[caption={Параллельная версия поразрядной сортировки с использованием MPI}]
#include "mpi/petrov_o_radix_sort_with_simple_merge/include/ops_mpi.hpp"

#include <algorithm>
#include <boost/mpi.hpp>
#include <cmath>
#include <limits>

namespace petrov_o_radix_sort_with_simple_merge_mpi {

bool TaskParallel::validation() {
  internal_order_test();

  if (world.rank() == 0) {
    bool isValid = (!taskData->inputs_count.empty()) && (!taskData->inputs.empty()) && (!taskData->outputs.empty());
    return isValid;
  }

  return true;
}

bool TaskParallel::pre_processing() {
  internal_order_test();

  if (world.rank() == 0) {
    int size = taskData->inputs_count[0];
    input_.resize(size);
    int* input_data = reinterpret_cast<int*>(taskData->inputs[0]);
    std::copy(input_data, input_data + size, input_.begin());
    res.resize(size);
  }

  return true;
}

bool TaskParallel::run() {
  internal_order_test();

  int rank = world.rank();
  int size = world.size();

  int n = 0;
  if (rank == 0) {
    n = static_cast<int>(input_.size());
  }
  boost::mpi::broadcast(world, n, 0);

  int base_count = n / size;
  int remainder = n % size;

  std::vector<int> send_counts(size, base_count);
  for (int i = 0; i < remainder; ++i) {
    send_counts[i] += 1;
  }

  std::vector<int> displs(size, 0);
  for (int i = 1; i < size; ++i) {
    displs[i] = displs[i - 1] + send_counts[i - 1];
  }

  std::vector<int> local_data(send_counts[rank]);

  if (rank == 0) {
    for (int proc = 1; proc < size; ++proc) {
      world.send(proc, 0, &input_[displs[proc]], send_counts[proc]);
    }
    std::copy(input_.begin(), input_.begin() + send_counts[0], local_data.begin());
  } else {
    world.recv(0, 0, local_data.data(), send_counts[rank]);
  }

  for (auto& num : local_data) {
    num ^= 0x80000000;
  }

  unsigned int local_max = 0;
  if (!local_data.empty()) {
    local_max = static_cast<unsigned int>(local_data[0]);
    for (auto& num : local_data) {
      auto val = static_cast<unsigned int>(num);
      if (val > local_max) {
        local_max = val;
      }
    }
  }

  unsigned int global_max = 0;
  boost::mpi::all_reduce(world, local_max, global_max, boost::mpi::maximum<unsigned int>());

  int num_bits = 0;
  const int MAX_BITS = static_cast<int>(sizeof(unsigned int) * 8);
  while (num_bits < MAX_BITS && (global_max >> num_bits) > 0) {
    num_bits++;
  }

  {
    std::vector<int> output(local_data.size());
    for (int bit = 0; bit < num_bits; ++bit) {
      int zero_count = 0;
      for (const auto& num : local_data) {
        if (((static_cast<unsigned int>(num) >> bit) & 1) == 0) {
          zero_count++;
        }
      }

      int zero_index = 0;
      int one_index = zero_count;
      for (const auto& num : local_data) {
        if (((static_cast<unsigned int>(num) >> bit) & 1) == 0) {
          output[zero_index++] = num;
        } else {
          output[one_index++] = num;
        }
      }
      local_data = output;
    }
  }

  for (auto& num : local_data) {
    num ^= 0x80000000;
  }

  std::vector<int> recv_buf;
  if (rank == 0) {
    recv_buf.resize(n);
    std::copy(local_data.begin(), local_data.end(), recv_buf.begin());
    for (int proc = 1; proc < size; ++proc) {
      world.recv(proc, 1, &recv_buf[displs[proc]], send_counts[proc]);
    }
  } else {
    world.send(0, 1, local_data.data(), send_counts[rank]);
  }

  if (rank == 0) {
    std::vector<int> final_result;
    final_result.insert(final_result.end(), recv_buf.begin(), recv_buf.begin() + send_counts[0]);

    auto merge_two = [](const std::vector<int>& a, const std::vector<int>& b) {
      std::vector<int> merged;
      merged.reserve(a.size() + b.size());
      size_t i = 0;
      size_t j = 0;
      while (i < a.size() && j < b.size()) {
        if (a[i] < b[j]) {
          merged.push_back(a[i++]);
        } else {
          merged.push_back(b[j++]);
        }
      }
      while (i < a.size()) {
        merged.push_back(a[i++]);
      }
      while (j < b.size()) {
        merged.push_back(b[j++]);
      }
      return merged;
    };

    for (int proc = 1; proc < size; ++proc) {
      std::vector<int> next_block(recv_buf.begin() + displs[proc], recv_buf.begin() + displs[proc] + send_counts[proc]);
      final_result = merge_two(final_result, next_block);
    }

    res = std::move(final_result);
  }

  return true;
}

bool TaskParallel::post_processing() {
  internal_order_test();
  if (world.rank() == 0) {
    int* output_ = reinterpret_cast<int*>(taskData->outputs[0]);
    std::copy(res.begin(), res.end(), output_);
  }
  return true;
}

/*---------------------------------------------SEQUENTIAL-----------------------------------------------------------*/

bool TaskSequential::validation() {
  internal_order_test();

  bool isValid = (!taskData->inputs_count.empty()) && (!taskData->inputs.empty()) && (!taskData->outputs.empty());

  return isValid;
}

bool TaskSequential::pre_processing() {
  internal_order_test();

  int size = taskData->inputs_count[0];
  input_.resize(size);

  int* input_data = reinterpret_cast<int*>(taskData->inputs[0]);
  std::copy(input_data, input_data + size, input_.begin());

  res.resize(size);
  return true;
}

bool TaskSequential::run() {
  internal_order_test();

  for (auto& num : input_) {
    num ^= 0x80000000;
  }

  auto max_num = static_cast<unsigned int>(input_[0]);
  for (const auto& num : input_) {
    if (static_cast<unsigned int>(num) > max_num) {
      max_num = static_cast<unsigned int>(num);
    }
  }
  int num_bits = 0;
  const int MAX_BITS = sizeof(unsigned int) * 8;
  while (num_bits < MAX_BITS && (max_num >> num_bits) > 0) {
    num_bits++;
  }

  std::vector<int> output(input_.size());

  for (int bit = 0; bit < num_bits; ++bit) {
    int zero_count = 0;

    for (const auto& num : input_) {
      if (((static_cast<unsigned int>(num) >> bit) & 1) == 0) {
        zero_count++;
      }
    }

    int zero_index = 0;
    int one_index = zero_count;

    for (const auto& num : input_) {
      if (((static_cast<unsigned int>(num) >> bit) & 1) == 0) {
        output[zero_index++] = num;
      } else {
        output[one_index++] = num;
      }
    }

    input_ = output;
  }

  for (auto& num : input_) {
    num ^= 0x80000000;
  }

  res = input_;
  return true;
}

bool TaskSequential::post_processing() {
  internal_order_test();

  int* output_ = reinterpret_cast<int*>(taskData->outputs[0]);
  for (size_t i = 0; i < res.size(); ++i) {
    output_[i] = res[i];
  }
  std::copy(res.begin(), res.end(), output_);

  return true;
}

}  // namespace petrov_o_radix_sort_with_simple_merge_mpi
\end{lstlisting}

%---------------------- Результаты экспериментов -------------------

\section{Результаты экспериментов}

\subsection{Условия проведения экспериментов}

\begin{itemize}
  \item \textbf{Аппаратная конфигурация}:
  \begin{itemize}
    \item CPU: Intel Core i5 12400f
    \item RAM: 16 GB
    \item ОС: Windows 10 22H2
  \end{itemize}
  \item \textbf{Тестовые размеры}:
  \begin{itemize}
    \item Тесты производительности: массивы размером 100\,000 элементов.
    \item Функциональные тесты: массивы размером 1000, 1024, 2187, 1013 элементов.
  \end{itemize}
  \item \textbf{Средства реализации}:
  \begin{itemize}
    \item Язык программирования: C++.
    \item Параллелизация: MPI (Boost.MPI).
    \item Компилятор: g++ 9.3.0.
  \end{itemize}
\end{itemize}

\subsection{Пример замера времени}

Для сравнения были проведены замеры времени выполнения на различных размерах массивов и с разным количеством процессов.

\begin{itemize}
  \item \textbf{Последовательная версия (seq)}
  \item \textbf{Параллельная версия (MPI)}: 2 процесса
  \item \textbf{Параллельная версия (MPI)}: 4 процесса
\end{itemize}

Ниже приведены выдержки из протокола тестов для массива размером 1\,000\,000 элементов.

\begin{itemize}
  \item \textbf{SEQ (последовательная версия)}:
  \begin{verbatim}
[ RUN      ] petrov_o_radix_sort_with_simple_merge_mpi_seq.test_pipeline_run_seq
ppc-2024-autumn\tasks\mpi\petrov_o_radix_sort_with_simple_merge:pipeline:0.0952214000
[       OK ] petrov_o_radix_sort_with_simple_merge_mpi_seq.test_pipeline_run_seq (101 ms)
  Итого: ~101 ms
  \end{verbatim}

  \item \textbf{MPI, 5 процессов}:
  \begin{verbatim}
[ RUN      ] petrov_o_radix_sort_with_simple_merge_mpi.test_pipeline_run_mpi
ppc-2024-autumn\tasks\mpi\petrov_o_radix_sort_with_simple_merge:pipeline:0.0737897000
[       OK ] petrov_o_radix_sort_with_simple_merge_mpi.test_pipeline_run_mpi (81 ms)
  Итого: ~81 ms
  \end{verbatim}

  \item \textbf{MPI, 10 процессов}:
  \begin{verbatim}
[ RUN      ] petrov_o_radix_sort_with_simple_merge_mpi.test_pipeline_run_mpi
ppc-2024-autumn\tasks\mpi\petrov_o_radix_sort_with_simple_merge:pipeline:0.0375792000
[       OK ] petrov_o_radix_sort_with_simple_merge_mpi.test_pipeline_run_mpi (44 ms)
  Итого: ~44 ms
  \end{verbatim}
\end{itemize}

\subsection{Анализ результатов}

\begin{enumerate}
  \item \textbf{Скорость выполнения}: Параллельная реализация показывает значительное ускорение по сравнению с последовательной версией при увеличении количества процессов.
  
  \item \textbf{Масштабируемость}: При увеличении количества процессов время выполнения уменьшается, что свидетельствует о хорошей масштабируемости алгоритма.
  
  \item \textbf{Коммуникационные затраты}: Для небольших массивов накладные расходы на передачу данных могут снизить эффективность параллельной реализации. Однако для больших массивов преимущества параллельной версии становятся очевидными.
  
  \item \textbf{Балансировка нагрузки}: Равномерное распределение массива между процессами обеспечивает эффективное использование ресурсов и минимизирует время ожидания.
\end{enumerate}

%---------------------- Выводы -------------------------------------

\section{Выводы}

В ходе работы были выполнены следующие задачи:
\begin{itemize}
  \item Реализована последовательная версия поразрядной сортировки целых чисел с использованием простой схемы слияния.
  \item Разработана параллельная версия алгоритма с использованием MPI, обеспечена равномерная распределённость данных и эффективное слияние отсортированных частей.
  \item Проведена валидация корректности обеих реализаций с подтверждением совпадения результатов.
  \item Проанализирована производительность: параллельная версия показала значительное ускорение при увеличении количества процессов, особенно на больших массивах данных.
\end{itemize}

Основные направления для дальнейшего улучшения:
\begin{itemize}
  \item Оптимизация коммуникационных операций для уменьшения накладных расходов при распределении и сборе данных.
  \item Исследование различных стратегий распределения данных между процессами для повышения эффективности.
  \item Внедрение динамической балансировки нагрузки для улучшения масштабируемости на кластерах с неоднородными ресурсами.
  \item Применение алгоритмических оптимизаций поразрядной сортировки для повышения общей производительности.
\end{itemize}

%---------------------- Список литературы --------------------------

\section{Список литературы}
\begin{enumerate}
  \item Документация по Boost.MPI: \url{https://www.boost.org/doc/libs/master/doc/html/mpi/tutorial.html}
  \item Классическая литература по алгоритмам: Т. Кормен, Ч. Лейзерсон, Р. Ривест, К. Стайн. \textit{Алгоритмы: построение и анализ}.
  \item Официальная документация по C++: \url{https://en.cppreference.com/}
  \item Лекции по параллельным вычислениям: \url{https://disk.yandex.ru/d/LxDwnOlUo3Eqfg}
\end{enumerate}

%---------------------- Приложения ---------------------------------
\appendix
\section*{Приложение}
\addcontentsline{toc}{section}{Приложение}

Для полноты приводятся тексты файлов с реализацией последовательной и параллельной версий поразрядной сортировки.

\subsection*{Файл \texttt{ops\_seq.hpp}}

\begin{lstlisting}[caption={Последовательная версия поразрядной сортировки}]
#pragma once

#include <string>
#include <vector>

#include "core/task/include/task.hpp"

namespace petrov_o_radix_sort_with_simple_merge_seq {

class TestTaskSequential : public ppc::core::Task {
public:
  explicit TestTaskSequential(std::shared_ptr<ppc::core::TaskData> taskData_) : Task(std::move(taskData_)) {}

  bool pre_processing() override;
  bool validation() override;
  bool run() override;
  bool post_processing() override;

private:
  std::vector<int> input_;
  std::vector<int> res_;
};

}  // namespace petrov_o_radix_sort_with_simple_merge_seq
\end{lstlisting}

%-------------------------------------------------------------------
\subsection*{Файл \texttt{ops\_mpi.hpp}}

\begin{lstlisting}[caption={Параллельная версия поразрядной сортировки с использованием MPI}]
#pragma once

#include "mpi/petrov_o_radix_sort_with_simple_merge/include/ops_mpi.hpp"

#include <algorithm>
#include <boost/mpi.hpp>
#include <cmath>
#include <limits>

namespace petrov_o_radix_sort_with_simple_merge_mpi {

bool TaskParallel::validation() {
  internal_order_test();

  if (world.rank() == 0) {
    bool isValid = (!taskData->inputs_count.empty()) && (!taskData->inputs.empty()) && (!taskData->outputs.empty());
    return isValid;
  }

  return true;
}

bool TaskParallel::pre_processing() {
  internal_order_test();

  if (world.rank() == 0) {
    int size = taskData->inputs_count[0];
    input_.resize(size);
    int* input_data = reinterpret_cast<int*>(taskData->inputs[0]);
    std::copy(input_data, input_data + size, input_.begin());
    res.resize(size);
  }

  return true;
}

bool TaskParallel::run() {
  internal_order_test();

  int rank = world.rank();
  int size = world.size();

  int n = 0;
  if (rank == 0) {
    n = static_cast<int>(input_.size());
  }
  boost::mpi::broadcast(world, n, 0);

  int base_count = n / size;
  int remainder = n % size;

  std::vector<int> send_counts(size, base_count);
  for (int i = 0; i < remainder; ++i) {
    send_counts[i] += 1;
  }

  std::vector<int> displs(size, 0);
  for (int i = 1; i < size; ++i) {
    displs[i] = displs[i - 1] + send_counts[i - 1];
  }

  std::vector<int> local_data(send_counts[rank]);

  if (rank == 0) {
    for (int proc = 1; proc < size; ++proc) {
      world.send(proc, 0, &input_[displs[proc]], send_counts[proc]);
    }
    std::copy(input_.begin(), input_.begin() + send_counts[0], local_data.begin());
  } else {
    world.recv(0, 0, local_data.data(), send_counts[rank]);
  }

  for (auto& num : local_data) {
    num ^= 0x80000000;
  }

  unsigned int local_max = 0;
  if (!local_data.empty()) {
    local_max = static_cast<unsigned int>(local_data[0]);
    for (auto& num : local_data) {
      auto val = static_cast<unsigned int>(num);
      if (val > local_max) {
        local_max = val;
      }
    }
  }

  unsigned int global_max = 0;
  boost::mpi::all_reduce(world, local_max, global_max, boost::mpi::maximum<unsigned int>());

  int num_bits = 0;
  const int MAX_BITS = static_cast<int>(sizeof(unsigned int) * 8);
  while (num_bits < MAX_BITS && (global_max >> num_bits) > 0) {
    num_bits++;
  }

  {
    std::vector<int> output(local_data.size());
    for (int bit = 0; bit < num_bits; ++bit) {
      int zero_count = 0;
      for (const auto& num : local_data) {
        if (((static_cast<unsigned int>(num) >> bit) & 1) == 0) {
          zero_count++;
        }
      }

      int zero_index = 0;
      int one_index = zero_count;
      for (const auto& num : local_data) {
        if (((static_cast<unsigned int>(num) >> bit) & 1) == 0) {
          output[zero_index++] = num;
        } else {
          output[one_index++] = num;
        }
      }
      local_data = output;
    }
  }

  for (auto& num : local_data) {
    num ^= 0x80000000;
  }

  std::vector<int> recv_buf;
  if (rank == 0) {
    recv_buf.resize(n);
    std::copy(local_data.begin(), local_data.end(), recv_buf.begin());
    for (int proc = 1; proc < size; ++proc) {
      world.recv(proc, 1, &recv_buf[displs[proc]], send_counts[proc]);
    }
  } else {
    world.send(0, 1, local_data.data(), send_counts[rank]);
  }

  if (rank == 0) {
    std::vector<int> final_result;
    final_result.insert(final_result.end(), recv_buf.begin(), recv_buf.begin() + send_counts[0]);

    auto merge_two = [](const std::vector<int>& a, const std::vector<int>& b) {
      std::vector<int> merged;
      merged.reserve(a.size() + b.size());
      size_t i = 0;
      size_t j = 0;
      while (i < a.size() && j < b.size()) {
        if (a[i] < b[j]) {
          merged.push_back(a[i++]);
        } else {
          merged.push_back(b[j++]);
        }
      }
      while (i < a.size()) {
        merged.push_back(a[i++]);
      }
      while (j < b.size()) {
        merged.push_back(b[j++]);
      }
      return merged;
    };

    for (int proc = 1; proc < size; ++proc) {
      std::vector<int> next_block(recv_buf.begin() + displs[proc], recv_buf.begin() + displs[proc] + send_counts[proc]);
      final_result = merge_two(final_result, next_block);
    }

    res = std::move(final_result);
  }

  return true;
}

bool TaskParallel::post_processing() {
  internal_order_test();
  if (world.rank() == 0) {
    int* output_ = reinterpret_cast<int*>(taskData->outputs[0]);
    std::copy(res.begin(), res.end(), output_);
  }
  return true;
}

/*---------------------------------------------SEQUENTIAL-----------------------------------------------------------*/

bool TaskSequential::validation() {
  internal_order_test();

  bool isValid = (!taskData->inputs_count.empty()) && (!taskData->inputs.empty()) && (!taskData->outputs.empty());

  return isValid;
}

bool TaskSequential::pre_processing() {
  internal_order_test();

  int size = taskData->inputs_count[0];
  input_.resize(size);

  int* input_data = reinterpret_cast<int*>(taskData->inputs[0]);
  std::copy(input_data, input_data + size, input_.begin());

  res.resize(size);
  return true;
}

bool TaskSequential::run() {
  internal_order_test();

  for (auto& num : input_) {
    num ^= 0x80000000;
  }

  auto max_num = static_cast<unsigned int>(input_[0]);
  for (const auto& num : input_) {
    if (static_cast<unsigned int>(num) > max_num) {
      max_num = static_cast<unsigned int>(num);
    }
  }
  int num_bits = 0;
  const int MAX_BITS = sizeof(unsigned int) * 8;
  while (num_bits < MAX_BITS && (max_num >> num_bits) > 0) {
    num_bits++;
  }

  std::vector<int> output(input_.size());

  for (int bit = 0; bit < num_bits; ++bit) {
    int zero_count = 0;

    for (const auto& num : input_) {
      if (((static_cast<unsigned int>(num) >> bit) & 1) == 0) {
        zero_count++;
      }
    }

    int zero_index = 0;
    int one_index = zero_count;

    for (const auto& num : input_) {
      if (((static_cast<unsigned int>(num) >> bit) & 1) == 0) {
        output[zero_index++] = num;
      } else {
        output[one_index++] = num;
      }
    }

    input_ = output;
  }

  for (auto& num : input_) {
    num ^= 0x80000000;
  }

  res = input_;
  return true;
}

bool TaskSequential::post_processing() {
  internal_order_test();

  int* output_ = reinterpret_cast<int*>(taskData->outputs[0]);
  for (size_t i = 0; i < res.size(); ++i) {
    output_[i] = res[i];
  }
  std::copy(res.begin(), res.end(), output_);

  return true;
}

}  // namespace petrov_o_radix_sort_with_simple_merge_mpi
\end{lstlisting}

%-------------------------------------------------------------------
\subsection*{Файл \texttt{ops\_mpi.cpp}}

\begin{lstlisting}[caption={Файл реализации параллельной версии поразрядной сортировки}]
#include "mpi/petrov_o_radix_sort_with_simple_merge/include/ops_mpi.hpp"

#include <algorithm>
#include <boost/serialization/vector.hpp>
#include <functional>
#include <vector>

using namespace std::chrono_literals;

namespace petrov_o_radix_sort_with_simple_merge_mpi {

bool TaskParallel::pre_processing() {
  internal_order_test();

  if (world.rank() == 0) {
    int size = taskData->inputs_count[0];
    input_.resize(size);
    int* input_data = reinterpret_cast<int*>(taskData->inputs[0]);
    std::copy(input_data, input_data + size, input_.begin());
    res.resize(size);
  }

  return true;
}

bool TaskParallel::validation() {
  internal_order_test();

  if (world.rank() == 0) {
    if (taskData->inputs.size() != 1 || taskData->outputs.size() != 1) {
      return false;
    }

    if (taskData->inputs_count.size() < 1 || taskData->outputs_count.size() < 1) {
      return false;
    }

    return true;
  }

  return true;
}

bool TaskParallel::run() {
  internal_order_test();

  boost::mpi::broadcast(world, input_, 0);

  int n = input_.size();
  int size = world.size();
  int rank = world.rank();

  int base = n / size;
  int rem = n % size;
  int start = rank * base + std::min(rank, rem);
  int end = start + base + (rank < rem ? 1 : 0);

  std::vector<int> local_data(input_.begin() + start, input_.begin() + end);

  for (auto& num : local_data) {
    num ^= 0x80000000;
  }

  auto max_num = local_data.empty() ? 0 : static_cast<unsigned int>(*std::max_element(local_data.begin(), local_data.end()));
  boost::mpi::all_reduce(world, max_num, max_num, boost::mpi::maximum<unsigned int>());

  int num_bits = 0;
  const int MAX_BITS = sizeof(unsigned int) * 8;
  while (num_bits < MAX_BITS && (max_num >> num_bits) > 0) {
    num_bits++;
  }

  std::vector<int> output(local_data.size());
  for (int bit = 0; bit < num_bits; ++bit) {
    int zero_count = 0;
    for (const auto& num : local_data) {
      if (((static_cast<unsigned int>(num) >> bit) & 1) == 0) {
        zero_count++;
      }
    }

    int zero_index = 0;
    int one_index = zero_count;
    for (const auto& num : local_data) {
      if (((static_cast<unsigned int>(num) >> bit) & 1) == 0) {
        output[zero_index++] = num;
      } else {
        output[one_index++] = num;
      }
    }
    local_data = output;
  }

  for (auto& num : local_data) {
    num ^= 0x80000000;
  }

  std::vector<int> sorted;
  boost::mpi::gather(world, local_data, sorted, 0);

  if (rank == 0) {
    std::vector<int> final_sorted;
    for (const auto& part : sorted) {
      final_sorted.insert(final_sorted.end(), part.begin(), part.end());
    }
    std::sort(final_sorted.begin(), final_sorted.end());
    res = final_sorted;
  }

  return true;
}

bool TaskParallel::post_processing() {
  internal_order_test();
  if (world.rank() == 0) {
    int* output_ = reinterpret_cast<int*>(taskData->outputs[0]);
    std::copy(res.begin(), res.end(), output_);
  }
  return true;
}

}  // namespace petrov_o_radix_sort_with_simple_merge_mpi
\end{lstlisting}

\end{document}