\documentclass{report}

\usepackage[T2A]{fontenc}
\usepackage[utf8]{luainputenc}
\usepackage[english, russian]{babel}
\usepackage[pdftex]{hyperref}
\usepackage[14pt]{extsizes}
\usepackage{listings}
\usepackage{color}
\usepackage{geometry}
\usepackage{enumitem}
\usepackage{amsmath}
\usepackage{titlesec}
\usepackage{multirow}
\usepackage{graphicx}

\geometry{a4paper,top=2cm,bottom=3cm,left=2cm,right=1.5cm}
\setlength{\parskip}{0.5cm}

\usepackage{xcolor}
\lstset{
    language=C++, 
    basicstyle=\ttfamily\small, 
    keywordstyle=\color{blue}\bfseries, 
    stringstyle=\color{orange}, 
    commentstyle=\color{green!50!black}\itshape, 
    morecomment=[l][\color{purple}]{\#}, 
    tabsize=2, 
    breaklines=true, 
    breakatwhitespace=false, 
    frame=single, 
    title=\lstname,
    showstringspaces=false, 
}

\makeatletter
\renewcommand\@biblabel[1]{#1.\hfil}

\begin{document}

\begin{titlepage}

\begin{center}
Министерство науки и высшего образования Российской Федерации
\end{center}

\begin{center}
Федеральное государственное автономное образовательное учреждение высшего образования \\
Национальный исследовательский Нижегородский государственный университет им. Н.И. Лобачевского
\end{center}

\begin{center}
Институт информационных технологий, математики и механики
\end{center}

\vspace{4em}

\begin{center}
\textbf{\Large Отчет по лабораторной работе} \\
\end{center}
\begin{center}
\textbf{\Large «Быстрая сортировка с простым слиянием»} \\
\end{center}

\vspace{4em}

\newbox{\lbox}
\savebox{\lbox}{\hbox{text}}
\newlength{\maxl}
\setlength{\maxl}{\wd\lbox}
\hfill\parbox{7cm}{
\hspace*{5cm}\hspace*{-5cm}\textbf{Выполнил:} \\ студент группы 3822Б1ФИ3 \\ Колодкин Григорий Владимирович\\
\\
\hspace*{5cm}\hspace*{-5cm}\textbf{Проверил:}\\ доцент кафедры МОСТ, \\ канд. техн. наук \\ Сысоев А.В.\\}
\vspace{\fill}

\begin{center} Нижний Новгород \\ 2024 \end{center}

\end{titlepage}

\setcounter{page}{2}
\tableofcontents

\newpage

\newpage

\section*{Введение}
\addcontentsline{toc}{section}{Введение}

Одной из известных задач в программировании является задача сортировки. Сортировка применяется во многих приложениях и потому задача выбора эффективного алгоритма сортировки является очень важной. Чем эффективнее алгоритм, тем лучше будет работать приложение. С ростом объема данных эта задача стала еще более актуальнее.

На данный момент существует множество алгоритмов сортировки. Например, сортировка пузырьком.Она довольна проста в понимании и реализации, но при этом очень неэффективна. Её сложность равна O($n^2$), что делает работу алгоритма очень долгой на больших данных. Также есть алгоритм сортировки вставками, слиянием и др. Но самым известным алгоритмом стал алгоритм, придуманный Чарльзом Хоаром- быстрая сортировка. Её сложность равна O(n*log(n)). 

С появлением таких технологий как многопоточность и параллельное программирование возникает вопрос о возможности распараллеливания алгоритма быстрой сортировки, чтобы сделать его еще эффективней.

\newpage

\section*{Постановка задачи}  
\addcontentsline{toc}{section}{Постановка задачи}  

Дан массив из $n$ элементов:

\[
ar = \{ ar[0], ar[1], ar[2], \dots, ar[n-1] \}
\]

Необходимо реализовать параллельную версию алгоритма быстрой сортировки, используя библиотеку \texttt{boost.mpi}. Сортировка должна проводиться параллельно на каждом из процессов, после чего результаты должны сливаться в один отсортированный массив, который и будет представлять из себя выходные данные.

Формат входных данных:

Массив из $n$ целых чисел

\[
ar = \{ ar[0], ar[1], ar[2], \dots, ar[n-1] \}
\]

Формат выходных данных:

На выходе должен быть предоставлен отсортированный по возрастанию массив ar

\newpage

\section*{Описание алгоритма}  
\addcontentsline{toc}{section}{Описание алгоритма}  

Дан массив из $n$ элементов:

\[
ar = \{ ar[0], ar[1], ar[2], \dots, ar[n-1] \}
\]

Пусть l=0, r=n-1(это левая и правая границы массива). Массив ar[l…r] разбивается на два (возможно пустых) подмассива ar[l…q] и ar[q+1…r], таких, что каждый элемент a[l…q] меньше или равен a[q], который в свою очередь, не превышает любой элемент подмассива ar[q+1…r]. Индекс вычисляется в ходе процедуры разбиения. Подмассивы ar[l…q] и ar[q+1…r] сортируются с помощью рекурсивного вызова процедуры быстрой сортировки. Поскольку подмассивы сортируются на месте, для их объединения не требуются никакие действия: весь массив a[l…r] оказывается отсортированным.

\section*{Описание схемы распараллеливания}  
\addcontentsline{toc}{section}{Описание схемы распараллеливания}  

В ходе параллельной реализации каждый процесс получает часть массива, над которой выполняет алгоритм быстрой сортировки, после чего результат отправляется на корневой процесс. Тот, в свою очередь, при помощи слияния выполняет соединение отсортированных подмассивов в один отсортированный массив.

\section*{MPI: Программная реализация параллельного алгоритма}  
\addcontentsline{toc}{section}{MPI: Программная реализация параллельного алгоритма} 

Реализация параллельного алгоритма состоит из 4 этапов:

На первом этапе корневой процесс получает входные данные и инициализирует выходные данные.

На втором этапе корневой процесс проводит валидацию. В ходе нее он проверяет, что в массиве больше одного элемента(массив из одного элемента нет смысла сортировать).

На третьем этапе корневой процесс делит массив между остальными процессами и рассылает подмассивы каждому процессу с помощью MPI функции \textbf{scatterv}. Затем каждый процесс применяет алгоритм быстрой сортировки к своему подмассиву. После этого отсортированный подмассив отправляется корневому процессу с помощью MPI функции \textbf{gatherv}. 

На последнем этапе корневой процесс при помощи слияния собирает отсортированные подмассивы в один отсортированный массив.

\section*{Проведение экспериментов и анализ их резульатов}  
\addcontentsline{toc}{section}{Проведение экспериментов и анализ их результатов}

В ходе работы были проведены два эксперимента: на массиве из 1 миллиона элементов, заполненного случайным образом целыми числами из диапазона [-100000,100000]. Второй эксперимент был проведен на целочисленном массиве из 10000 элеметов, отсортированных в обратном порядке.

В ходе экспериментов сравнивались работы: последовательного алгоритма(быстрая сортировка), параллельного алгоритма на 2-х, 3-х, 4-х процессах.

Результаты первого эксперимента:

Время работы последовательного алгоритма: 1.1169570999 С

Время работы параллельного алгоритма на 2-х процессах: 0.6265171000 С

Время работы параллельного алгоритма на 3-х процессах: 0.7606768999 С

Время работы параллельного алгоритма на 4-х процессах: 0.7019676000 С

Результаты второго эксперимента:

Время работы последовательного алгоритма: 0.4287133999 С

Время работы параллельного алгоритма на 2-х процессах: 0.1553251999 С

Время работы параллельного алгоритма на 3-х процессах: 0.0760848999 С

Время работы параллельного алгоритма на 4-х процессах: 0.0564148000 С

Результаты первого эксперимента оказались интересены с той стороны, что параллельный алгоритм на 3-х и 4-х процессах работал медленнее, чем на 2-х процессах. В остальном же эксперимент показал очевидное: распараллеливание значительно ускоряет время сортировки и делает её эффективней. Результаты второго эксперимента оказались ожидаемыми. Увеличение числа процессов приводит к сокращению времени работы алгоритма.

Подводя итоги, можно сказать, что параллельный алгоритм быстрой сортировки однозначно выигрывает у обычного. Однако, первый эксперимент показал, что увеличение числа процессов не всегда ведет к ускорению работы. Впрочем, можно предположить, что это было связано со случайностью заполнения вводных данных. Результаты второго эксперимента ясно дали понять, что увеличение числа процессов делает параллельный алгоритм эффективней.

\newpage

\section*{Заключение}  
\addcontentsline{toc}{section}{Заключение}  

В ходе данной работы был изучен алгоритм быстрой сортировки, его суть и программная реализация, а также схема распараллеливания данного алгоритма, используя слияние.

Помимо этого, были проведены эксперименты на разных массивах для сравнения эффективности последовательной и параллельной версии быстрой сортировки. Их результаты подтвердили напрашивающуюся гипотезу о том, что параллельная схема работает быстрее, что, в свою очередь, показывает важность изучения технологий параллельного программирования, в т.ч. MPI для создания эффективных алгоритмов.

\newpage

\section*{Список литературы}  
\addcontentsline{toc}{section}{Список литературы}

1. Антонов А.С. Параллельное программирование с импользованием технологии MPI / А.С Антонов-учебное пособие-М.: Изд-во МГУ, 2004.-71 с.

2. Кормен Т.Х., Лейзерсон Ч.E., Ривест Р.L., Штайн К. Алгоритмы: Построение и анализ / Т.Х.
Кормен, Ч.E. Лейзерсон, Р.L. Ривест, К. Штайн. — 3-е изд. — М.: Вильямс, 2011.

3. Малявко, А. А.  Параллельное программирование на основе технологий OpenMP, MPI, CUDA : учебное пособие для вузов / А. А. Малявко. — 2-е изд., испр. и доп. — Москва : Издательство Юрайт, 2020. — 129 с.

4. Быстрая сортировка // ИТМО Викиконспекты. URL: \url{https://neerc.ifmo.ru/wiki/index.php?title=%D0%91%D1%8B%D1%81%D1%82%D1%80%D0%B0%D1%8F_%D1%81%D0%BE%D1%80%D1%82%D0%B8%D1%80%D0%BE%D0%B2%D0%BA%D0%B0}

5. Параллельная обработка данных // Parallel.ru. URL: \url{https://parallel.ru/vvv/mpi.html#p1}

\newpage
\section*{Приложения}  
\addcontentsline{toc}{section}{Приложения}

\subsection*{Приложение 1. Алгоритм быстрой сортировки}
\begin{lstlisting}[language=C++,caption={Приложение 1.}]

int partition(std::vector<int>& arr, int low, int high) {
  int pivot = arr[high];
  int i = (low - 1);

  for (int j = low; j < high; j++) {
    if (arr[j] <= pivot) {
      i++;
      std::swap(arr[i], arr[j]);
    }
  }
  std::swap(arr[i + 1], arr[high]);
  return (i + 1);
}

void quickSort(std::vector<int>& arr, int low, int high) {
  if (low < high) {
    int pi = partition(arr, low, high);
    quickSort(arr, low, pi - 1);
    quickSort(arr, pi + 1, high);
  }
}
\end{lstlisting}

\subsection*{Приложение 2. Параллельная схема быстрой сортировки с простым слиянием}
\begin{lstlisting}[language=C++,caption={Приложение 2.}]

bool kolodkin_g_hoar_merge_sort_mpi::TestMPITaskParallel::pre_processing() {
  internal_order_test();
  if (world.rank() == 0) {
    // Init vectors
    input_ = std::vector<int>(taskData->inputs_count[0]);
    auto* tmp_ptr = reinterpret_cast<int*>(taskData->inputs[0]);
    for (unsigned i = 0; i < taskData->inputs_count[0]; i++) {
      input_[i] = tmp_ptr[i];
    }
    output_ = std::vector<int>(taskData->inputs_count[0]);
    output_ = input_;
  }
  return true;
}

bool kolodkin_g_hoar_merge_sort_mpi::TestMPITaskParallel::validation() {
  internal_order_test();
  if (world.rank() == 0) {
    return taskData->inputs_count[0] > 1;
  }
  return true;
}

bool kolodkin_g_hoar_merge_sort_mpi::TestMPITaskParallel::run() {
  internal_order_test();
  unsigned int num_processes = world.size();
  std::vector<int> send_counts(num_processes, 0);
  std::vector<int> displacements(num_processes, 0);

  if (world.rank() == 0) {
    send_counts.resize(num_processes);
    displacements.resize(num_processes);

    for (size_t i = 0; i < num_processes; i++) {
      send_counts[i] = input_.size() / num_processes;
      if (i == num_processes - 1) {
        send_counts[i] += input_.size() % num_processes;
      }
    }

    for (size_t i = 1; i < num_processes; i++) {
      displacements[i] = displacements[i - 1] + send_counts[i - 1];
    }
  }
  world.barrier();
  for (size_t i = 0; i < num_processes; i++) {
    boost::mpi::broadcast(world, send_counts[i], 0);
    boost::mpi::broadcast(world, displacements[i], 0);
  }
  std::vector<int> local_input_(send_counts[world.rank()]);
  boost::mpi::scatterv(world, input_.data(), send_counts, displacements, local_input_.data(), local_input_.size(), 0);


  quickSort(local_input_, 0, local_input_.size() - 1);

  std::vector<int> sorted_data;
  if (world.rank() == 0) {
    sorted_data.resize(input_.size());
  }

  boost::mpi::gatherv(world, local_input_.data(), local_input_.size(), sorted_data.data(), send_counts, displacements,
                      0);

  if (world.rank() == 0) {
    std::vector<int> final_result;
    std::vector<int> current_indices(num_processes, 0);
    std::vector<bool> done(num_processes, false);
    while (final_result.size() < input_.size()) {
      int min_index = -1;
      int min_value = std::numeric_limits<int>::max();

      for (size_t i = 0; i < num_processes; i++) {
        if (!done[i] && current_indices[i] < send_counts[i]) {
          int value = sorted_data[displacements[i] + current_indices[i]];
          if (value < min_value) {
            min_value = value;
            min_index = i;
          }
        }
      }
      if (min_index == -1) {
        break;
      }

      final_result.push_back(min_value);
      current_indices[min_index]++;
    }
    output_ = final_result;
  }
  return true;
}

bool kolodkin_g_hoar_merge_sort_mpi::TestMPITaskParallel::post_processing() {
  internal_order_test();
  if (world.rank() == 0) {
    *reinterpret_cast<std::vector<int>*>(taskData->outputs[0]) = output_;
  }
  return true;
}
\end{lstlisting}
\end{document}