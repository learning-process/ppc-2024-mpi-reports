\documentclass[12pt]{article}
\usepackage[T2A]{fontenc}
\usepackage[utf8]{inputenc}
\usepackage[russian]{babel}
\usepackage{hyperref}
\usepackage{amsmath, amssymb}
\usepackage{titlesec}
\usepackage{tocloft}
\usepackage{multirow}
\usepackage{indentfirst} 
\usepackage{tocloft}
\usepackage{geometry}
\geometry{left=2.5cm, right=1.5cm, top=2cm, bottom=2cm}

\renewcommand{\cftsecleader}{\cftdotfill{\cftdotsep}}
\renewcommand{\cftsubsecleader}{\cftdotfill{\cftdotsep}} 
\renewcommand{\cftsecfont}{\normalfont} 
\renewcommand{\cftsecpagefont}{\normalfont}

\renewcommand{\cfttoctitlefont}{\hfill\Large\bfseries}
\renewcommand{\cftaftertoctitle}{\hfill\null}

\setlength{\cftsecindent}{0pt} % отступ раздела
\setlength{\cftsubsecindent}{20pt} % отступ подраздела
\setlength{\parindent}{1.25cm} 

\titleformat{\section}
    {\Large\bfseries\center}
    {\thesection}
    {.3em}
    {}

\titleformat{\subsection}
    {\large\bfseries\center}
    {\thesubsection}
    {0.3em}
    {}

\titleformat{\subsubsection}
    {\normalsize\bfseries\center}
    {\thesubsubsection}
    {0.3em}
    {}

\renewcommand{\thesection}{\arabic{section}.}          
\renewcommand{\thesubsection}{\thesection\arabic{subsection}}  

\begin{document}
\thispagestyle{empty}
\begin{center}
    {\bfseries
    {\large «Национальный исследовательский Нижегородский государственный университет им.
Н.И. Лобачевского» (ННГУ) }\\[2em]
    {\large Институт информационных технологий математики и механики }\\[1em]
    }
    \vspace*{\fill} 
    {\LARGE\bfseries Отчет}\\[1em]
    {\large Вычисление многомерных интегралов с использованием многошаговой схемы 
(метод прямоугольников).}\\[5em]
    \vspace{\fill} 
     \hfill\parbox{0.4\textwidth}{
        Выполнил:\\
        студент группы 3822Б1ПР2\\
        Ермолаев Владислав Александрович\\[2em]
        Преподаватель:\\
        доцент, кандидат технических наук\\
        Сысоев Александр Владимирович\\[2em]
    }
    
    Нижний Новгород, 2024 г.
\end{center}
\newpage

\tableofcontents
\newpage

\section{Введение}
В данной работе рассматривается задача вычисления многомерных интегралов методом прямоугольников с использованием параллельных вычислений на основе технологии MPI (Message Passing Interface).

Метод прямоугольников является одним из простейших численных методов интегрирования. Его основная идея заключается в разбиении области интегрирования на малые прямоугольники и вычислении суммы их площадей. Несмотря на простоту, при достаточно мелком разбиении метод обеспечивает приемлемую точность.

Однако при увеличении размерности интеграла и/или уменьшении шага разбиения вычислительная сложность задачи значительно возрастает. Это делает актуальным применение параллельных вычислений для ускорения процесса интегрирования.

\newpage
\section{Постановка задачи}
\subsection{Математическая постановка}

Дано:
\begin{itemize}
    \item Функция $f(\vec{x})$, где $\vec{x} = (x_1, x_2, ..., x_n)$ - вектор переменных
    \item Область интегрирования $D = [a_1, b_1] \times [a_2, b_2] \times ... \times [a_n, b_n]$
    \item Требуемая точность вычисления $\varepsilon > 0$
\end{itemize}

Требуется вычислить:
\[
I = \int\limits_D f(\vec{x})d\vec{x} = \int\limits_{a_1}^{b_1}\int\limits_{a_2}^{b_2}...\int\limits_{a_n}^{b_n} f(x_1,x_2,...,x_n)dx_1dx_2...dx_n
\]
\subsection{Программная постановка задачи}

Требуется реализовать последовательную и параллельную версии алгоритма вычисления многомерного интеграла методом прямоугольников.

\subsubsection{Входные данные}
\begin{itemize}
    \item $limits$ --- вектор пар $(a_i, b_i)$, задающих пределы интегрирования по каждой переменной, где:
    \begin{itemize}
        \item $a_i$ --- нижний предел интегрирования по $i$-й переменной
        \item $b_i$ --- верхний предел интегрирования по $i$-й переменной
    \end{itemize}
    \item $eps$ --- требуемая точность вычисления интеграла
    \item $f(\vec{x})$ --- интегрируемая функция нескольких переменных
\end{itemize}

\subsubsection{Выходные данные}
Значение определенного интеграла функции $f(\vec{x})$ по области, заданной пределами интегрирования, с точностью $eps$

\newpage
\section{Описание алгоритма}

Алгоритм вычисляет многомерный определённый интеграл методом прямоугольников с адаптивным шагом. Реализованы последовательная и параллельная (MPI) версии.

Интегрирование выполняется рекурсивно по каждому измерению:

\begin{enumerate}
    \item На каждом уровне рекурсии обрабатывается одно измерение $[a_i,b_i]$
    \item Область разбивается на $n$ равных частей с шагом $h = \frac{b_i-a_i}{n}$
    \item В каждой точке разбиения $x_k = a_i + h(k+\frac{1}{2})$ вычисляется:
    \begin{itemize}
        \item Значение функции $f(x_k)$, если это последнее измерение
        \item Рекурсивный вызов для следующего измерения, если измерения остались
    \end{itemize}
    \item Результат вычисляется как сумма: $I = h\sum_{k=0}^{n-1} f(x_k)$
    \item Число разбиений $n$ удваивается, пока не достигнута заданная точность:
     \[ |I_{n} - I_{n/2}| \cdot \frac{1}{3} \leq \varepsilon \] 
\end{enumerate}

\newpage
\section{Схема распараллеливания}
\subsection{Описание схемы распараллеливания}
\begin{enumerate}
    \item Разбиение по первой (внешней) переменной интегрирования $x_1$:
    \begin{itemize}
        \item Интервал $[a_1, b_1]$ делится на $p$ равных частей, где $p$ - количество процессов
        \item Каждый процесс получает подинтервал $[a_1 + \frac{i(b_1-a_1)}{p}, a_1 + \frac{(i+1)(b_1-a_1)}{p}]$
        \item Начальное количество точек разбиения $n = p$ 
    \end{itemize}

    \item Для каждого процесса:
    \begin{itemize}
        \item Начальное количество точек разбиения --- $\frac{n}{p}$ 
        \item Вычисляется локальная сумма $I_{local}$ методом прямоугольников на своем подинтервале
        \item Для внутренних интегралов используется последовательный алгоритм
    \end{itemize}

    \item Итерационный процесс:
    \begin{itemize}
        \item На каждой итерации $n$ удваивается
        \item Вычисляются новые локальные суммы
        \item Все локальные суммы собираются на процессе 0 через \texttt{boost::mpi::reduce}
        \item Проверяется условие сходимости $|I_{new} - I_{old}| \cdot \frac{1}{3} \leq \varepsilon$
    \end{itemize}
\end{enumerate}

\subsection{Особенности реализации}
\begin{itemize}
    \item Используется рекурсивный алгоритм для вычисления внутренних интегралов
    \item Границы интегрирования и команда остановки вычислений, при достижении заданной точности $\varepsilon$, рассылаются через \texttt{boost::mpi::broadcast}
    \item Результат возвращается только процессом 0
\end{itemize}

\newpage
\section{Эксперименты}
\subsection{Условия проведения}
Для достижения корректности результатов была использована версия программы с последними изменениями (\href{https://github.com/learning-process/ppc-2024-autumn/pull/729}{PR \#729}).

\subsubsection{Технические характеристики тестовой системы:} 
\begin{itemize}
    \item Операционная система: Windows 11 24H2
    \item Система виртуализации: VMware Workstation 17   
    \item Операционная система виртуальной машины: Ubuntu 23.10
    \item Процессор: 13th Gen Intel(R) Core(TM) i5-13500H 2.60 GHz
    \item Оперативная память виртуальной машины: 8 ГБ.
\end{itemize}

\subsubsection{Синтетическая задача для теста производительности:}
\[
I = \underbrace{\int\limits_{-7}^{7} \int\limits_{-7}^{7} \cdots \int\limits_{-7}^{7}}_{10\text{ раз}} x_1 \, dx_1 \, dx_2 \cdots dx_{10}
\]

где:
\begin{itemize}
    \item Количество измерений $n = 10$
    \item Все пределы интегрирования одинаковы: $[-7, 7]$
    \item Подынтегральная функция $f(x_1, x_2, \ldots, x_{10}) = x_1$
    \item Точность вычисления $\varepsilon = 10^{-4}$
\end{itemize}

\subsection{Результаты}
Результаты локального выполнения тестов для последовательной и параллельной версии на четырех процессах:

\begin{center}
\begin{tabular}{|c|c|c|}
\hline
     Версия & Тип & Время выполнения\\
\hline
\multirow{2}{*}{seq} & pipeline & 1.5886989940 \\
\cline{2-3}
& task & 1.6847531720 \\
    \hline
\multirow{2}{*}{mpi} & pipeline & 0.6661800230 \\
\cline{2-3}
& task & 0.6551218480 \\
\hline
\end{tabular}
\end{center}

\subsection{Анализ результатов}
Анализируя результаты выполнения тестов на четырех процессах, можно сделать следующие наблюдения:
\begin{enumerate}
    \item Сравнение последовательной и параллельной версий:
    \begin{itemize}
        \item В режиме pipeline: ускорение составило 
            \[ S = \frac{1.5856989940}{0.6661800230} \approx 2.38 \]
        \item В режиме task: ускорение составило
            \[ S = \frac{1.6847531720}{0.6551218480} \approx 2.57 \]
    \end{itemize}

    \item Эффективность распараллеливания:
        \[ E_{pipeline} = \frac{2.38}{4} \approx 0.595 \text{ или } 59.5\% \]
        \[ E_{task} = \frac{2.57}{4} \approx 0.643 \text{ или } 64.3\% \]
\end{enumerate}

\subsection{Выводы}

\begin{enumerate}
    \item Параллельная реализация демонстрирует существенное ускорение по сравнению с последовательной версией, достигая ускорения в 2.57 раза на 4 процессах.
    
    \item Эффективность распараллеливания находится на достаточно хорошем уровне (59.5-64.3\%). Неполное достижение теоретического ускорения (в 4 раза) объясняется:
    \begin{itemize}
        \item Накладными расходами на межпроцессное взаимодействие
        \item Рекурсивной природой алгоритма
        \item Необходимостью синхронизации процессов
    \end{itemize}
\end{enumerate}

\newpage
\section{Заключение}
В рамках данной работы была успешно решена задача вычисления многомерных интегралов методом прямоугольников с применением параллельных вычислений на базе технологии MPI. 
Основные результаты работы:

\begin{enumerate}
    \item Разработаны и реализованы:
    \begin{itemize}
        \item Последовательный алгоритм вычисления многомерных интегралов методом прямоугольников
        \item Параллельная версия алгоритма с использованием MPI для распределения вычислительной нагрузки
    \end{itemize}
    
    \item Реализованные алгоритмы позволяют:
    \begin{itemize}
        \item Вычислять интегралы произвольной размерности
        \item Задавать произвольные пределы интегрирования для каждой переменной
        \item Контролировать точность вычислений через параметр eps
    \end{itemize}
\end{enumerate}

Разработанное решение позволяет эффективно решать задачу численного интегрирования в многомерном пространстве, используя преимущества параллельных вычислений для уменьшения времени расчетов. 

Практическая значимость работы заключается в возможности применения разработанного программного обеспечения для решения широкого класса задач, требующих вычисления многомерных интегралов, в различных областях науки и техники.

\newpage
\section{Литература}

\begin{thebibliography}{2}

\bibitem{runge} Правило Рунге [Электронный ресурс] // Википедия. Свободная энциклопедия. — URL: \url{https://ru.wikipedia.org/wiki/Правило_Рунге#Оценка_точности_вычисления_определённого_интеграла} (дата обращения: 24.12.2024).

\bibitem{numerical} Уткин П.С. Численное интегрирование [Электронный ресурс] — URL: \url{https://enlib.ru/media/uploads/2022/04/02/kibkyo.pdf} (дата обращения: 24.12.2024).

\end{thebibliography}

\newpage
\section{Приложение}

В данном разделе представлены основные функции

\subsection{integrateImpl}
\begin{verbatim}
double integrateImpl(std::deque<std::pair<double, double>>& limits,
                                         std::vector<double>& args,
                                 const function& func, double eps) {
  double I = 0;
  double I0;
  int n = 2;

  auto [a, b] = limits.front();
  limits.pop_front();
  args.push_back(double{});

  do {
    I0 = I;
    I = 0;

    double h = (b - a) / n;
    args.back() = a + h / 2;
    for (int i = 0; i < n; i++) {
      if (limits.empty())
        I += func(args) * h;
      else
        I += integrateImpl(limits, args, func, eps) * h;

      args.back() += h;
    }

    n *= 2;
  } while (std::abs(I - I0) * 1 / 3 > eps);

  args.pop_back();
  limits.emplace_front(a, b);

  return I;
}
\end{verbatim}
\newpage

\subsection{integrateMPI}
\begin{verbatim}
double integrateMPI(boost::mpi::communicator& world,
                    std::deque<std::pair<double, double>> limits,
                                double eps, const function& func) {
  std::vector<double> args;
  broadcast(world, limits, 0);
  broadcast(world, eps, 0);

  double I = 0;
  double I0;
  double localI;
  int n = world.size();

  auto [a, b] = limits.front();
  double step = (b - a) / world.size();

  double localA = a + step * world.rank();
  double localB = localA + step;

  limits.pop_front();
  args.push_back(double{});

  bool flag = true;
  while (flag) {
    I0 = I;
    I = 0;
    localI = 0;

    int count_of_points = n / world.size();
    double h = (localB - localA) / count_of_points;
    args.back() = localA + h / 2;
    for (int i = 0; i < count_of_points; i++) {
      if (limits.empty())
        localI += func(args) * h;
      else
        localI += integrateImpl(limits, args, func, eps) * h;

      args.back() += h;
    }

    reduce(world, localI, I, std::plus<>(), 0);

    if (world.rank() == 0 && std::abs(I - I0) * 1 / 3 <= eps) flag = false;
    broadcast(world, flag, 0);

    n *= 2;
  }

  if (world.rank() == 0) return I;
  return 0;
}
\end{verbatim}

\end{document}