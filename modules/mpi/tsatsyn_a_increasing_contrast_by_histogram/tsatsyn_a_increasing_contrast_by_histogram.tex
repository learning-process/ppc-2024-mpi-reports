\documentclass[a4paper,12pt]{article}
\usepackage[utf8]{inputenc}
\usepackage[T2A]{fontenc}
\usepackage[russian]{babel}
\usepackage{titlesec}
\titleformat{\section}{\normalfont\Large\bfseries}{\thesection.}{1em}{}
\titleformat{\subsection}{\normalfont\large\bfseries}{\thesubsection.}{1em}{}
\usepackage{tocloft}
\renewcommand{\cftsecleader}{\cftdotfill{\cftdotsep}}
\usepackage{hyperref}
\usepackage{geometry}
\geometry{top=2cm, bottom=2cm, left=2cm, right=2cm}
\setlength{\parindent}{1.25cm}
\setlength{\parskip}{0pt}
\usepackage{amsmath,amssymb}
\usepackage{listings}
\hypersetup{hidelinks}

\begin{document}
\begin{titlepage}
\begin{center}
\textbf{МИНИСТЕРСТВО НАУКИ И ВЫСШЕГО ОБРАЗОВАНИЯ РОССИЙСКОЙ ФЕДЕРАЦИИ} \\
\textbf{Федеральное государственное автономное образовательное учреждение высшего образования} \\
\textbf{«Национальный исследовательский Нижегородский государственный университет им. Н.И. Лобачевского»} \\[1cm]
\textbf{Институт информационных технологий, математики и механики }\\[0.5cm]
\textbf{Кафедра: Математического обеспечения и суперкомпьютерных технологий}\\[0.5cm]
Направление подготовки: «Программная инженерия»\\
Профиль подготовки: «Общий»\\

\vfill

Отчёт по лабораторной работе на тему:\\
{\Large
\textbf{«Повышение контраста полутонового изображения посредством линейной растяжки гистограммы»} \\
}
\vfill
\begin{flushright}
\textbf{Выполнил}:\\
студент группы 3822Б1ПР2 \\
Цацын Александр \\
\vspace{1cm}
\textbf{Преподаватель}: \\
Сысоев Александр Владимирович \\
\textbf{Руководитель}: \\
Нестеров Александр Юрьевич\\
Оболенский Арсений Андреевич\\
\end{flushright}
\vfill
Нижний Новгород \\
2024
\end{center}
\end{titlepage}



\tableofcontents
\newpage




\section{Введение}
В данном отчёте рассматривается задача повышения контраста полутонового изображения путём линейной растяжки гистограммы, пиксели которого имеют тип int и находятся в пределах от 0 до 255. Основная цель работы - реализовать эффективный схему распараллеливания данных и исследовать производительность этого способа по сравнению с последовательным вариантом.
\newpage


\section{Постановка задачи}
Дана матрица элементов n*m, имитирующая реальное полутоновое изображение с интенсивностью пикселей 
\( x \in [0, 255] \), где
\begin{itemize}
  \item 0 — черный цвет (минимальная яркость),
  \item 255 — белый цвет (максимальная яркость),
  \item \( x \in (0, 255) \). промежуточные значения — оттенки серого.
\end{itemize}
и нужно посредством линейной растяжки гистограммы повысить контраст изображения.\\[0.1cm]
Контраст - разница между значениями пикселей с самой низкой и самой высокой яркостью.\\ [0.1cm]
Линейная растяжка гистограммы — это метод обработки изображения, который повышает контраст, распределяя значения интенсивности по всему диапазону.
\newpage


\section{Описание алгоритма}
Мы получаем матрицу в виде одномерного \textit{std::vector<int>} и каждый его элемент преобразуется в новый по формуле:
\begin{center} 
{\Large 
$
\text{\normalsize newPixel} = \frac{(\text{oldPixel} - \text{min\_value}) \cdot 
(\text{max\_range} - \text{min\_range})}{\text{max\_value} - \text{min\_value}} + \text{\normalsize min\_range}
$, 
}
\end{center}
где \begin{itemize}
  \item newPixel - новое значение пикселя,
  \item oldPixel - исходное значение пикселя,
  \item max\_value - пиксель с самой высоким значением интенсивности в изображении
  \item min\_value - пиксель с самой низким значением интенсивности в изображении
  \item max\_range - верхняя граница интенсивности, в данном случае 255,
  \item min\_range - нижняя граница интенсивности, в данном случае 0.
\end{itemize}
\newpage

\section{Описание схемы распараллеливания}
Схема распараллеливания состоит в одном - разделить изображение по частям на нулевом процессе и передавать каждому другому процессору свою часть, чтобы он её обработал и вернул обратно на нулевой процесс. Но подходов к этой схеме несколько - один из них не делить количество элементов для обработки на количество процессов, а выдать каждому процессу своеобразную локальную часть в цикле с шагом world.size(). Тогда не будет такой ситуации, когда количество элементов не делится нацело на количество процессов,  и поэтому какому-то процессу нужно обрабатывать что-то сверх нормы. При подходе, реализованном в этой работе, у каждого процесса будет одинаковое количество обрабатываемых элементов с максимальной разницей в 1 элемент.
\newpage

\section{Результаты экспериментов}
Для оценки производительности последовательной и параллельной версий алгоритма использовалась следующая тестовая система:
\begin{itemize}
    \item Операционная система: Windows 11
    \item Процессор: AMD Ryzen 5 5600H with Radeon Graphics
    \item Оперативная память: 16 ГБ.
\end{itemize}
На основе выполнения тестов для последовательной и параллельной (на 2,3,4 процессах) версиях была создана следующая таблица:\\[0,5cm]
\begin{tabular}{|c|c|c|c|}
    \hline
    Число процессов & Вид теста      & Время выполнения (с) & Ускорение \\ \hline
    1               & pipeline\_run & 0.697                & 1.00      \\ \hline
    1               & task\_run     & 0.635                & 1.00      \\ \hline
    2               & pipeline\_run & 1.086                & 0.64      \\ \hline
    2               & task\_run     & 1.213                & 0.52      \\ \hline
    3               & pipeline\_run & 1.106                & 0.63      \\ \hline
    3               & task\_run     & 1.178                & 0.54      \\ \hline
    4               & pipeline\_run & 1.078                & 0.65      \\ \hline
    4               & task\_run     & 1.183                & 0.54      \\ \hline
\end{tabular}
\newpage

\section{Заключение}
На основе результатов экспериментов можно сделать вывод, что данная задача не подходит для распараллеливания, так как очень сильно повышаются накладные расходы и очень слабо эффективность. В разы выгоднее эту задачу решать через seq-версию программы.
\newpage

\section{Приложение}



\subsection{ops\_seq.hpp}
\begin{lstlisting}[language=C++]
namespace tsatsyn_a_increasing_contrast_by_histogram_seq {

class TestTaskSequential : public ppc::core::Task {
 public:
  explicit TestTaskSequential(std::shared_ptr<ppc::core::TaskData> taskData_) : Task(std::move(taskData_)) {}
  bool pre_processing() override;
  bool validation() override;
  bool run() override;
  bool post_processing() override;

 private:
  std::vector<int> input_data;
  std::vector<int> res;
};

}  // namespace tsatsyn_a_increasing_contrast_by_histogram_seq
\end{lstlisting}


\subsection{ops\_seq.cpp}
\begin{lstlisting}[language=C++]
bool tsatsyn_a_increasing_contrast_by_histogram_seq::TestTaskSequential::validation() {
  internal_order_test();
  return (taskData->outputs_count[0] > 0 && taskData->inputs_count[0] > 0);
}
bool tsatsyn_a_increasing_contrast_by_histogram_seq::TestTaskSequential::pre_processing() {
  internal_order_test();
  input_data.resize(taskData->inputs_count[0]);
  auto* tempPtr = reinterpret_cast<int*>(taskData->inputs[0]);
  std::copy(tempPtr, tempPtr + taskData->inputs_count[0], input_data.begin());

  return true;
}
bool tsatsyn_a_increasing_contrast_by_histogram_seq::TestTaskSequential::run() {
  internal_order_test();
  int min_val = *std::min_element(input_data.begin(), input_data.end());
  int max_val = *std::max_element(input_data.begin(), input_data.end());
  if (max_val - min_val == 0) {
    max_val++;
  }
  res.resize(input_data.size());
  int input_sz = static_cast<int>(input_data.size());
  for (int i = 0; i < input_sz; i++) {
    res[i] = (input_data[i] - min_val) * (255 - 0) / (max_val - min_val) + 0;
  }
  return true;
}
bool tsatsyn_a_increasing_contrast_by_histogram_seq::TestTaskSequential::post_processing() {
  internal_order_test();

  auto* outputPtr = reinterpret_cast<int*>(taskData->outputs[0]);
  std::copy(res.begin(), res.end(), outputPtr);

  return true;
}
\end{lstlisting}


\subsection{ops\_mpi.hpp}
\begin{lstlisting}[language=C++]
namespace tsatsyn_a_increasing_contrast_by_histogram_mpi {

class TestMPITaskSequential : public ppc::core::Task {
 public:
  explicit TestMPITaskSequential(std::shared_ptr<ppc::core::TaskData> taskData_) : Task(std::move(taskData_)) {}
  bool pre_processing() override;
  bool validation() override;
  bool run() override;
  bool post_processing() override;

 private:
  std::vector<int> input_data;
  std::vector<int> res;
};

class TestMPITaskParallel : public ppc::core::Task {
 public:
  explicit TestMPITaskParallel(std::shared_ptr<ppc::core::TaskData> taskData_) : Task(std::move(taskData_)) {}
  bool pre_processing() override;
  bool validation() override;
  bool run() override;
  bool post_processing() override;

 private:
  std::vector<int> input_data;
  std::vector<int> res;
  boost::mpi::communicator world;
};

}  // namespace tsatsyn_a_increasing_contrast_by_histogram_mpi
\end{lstlisting}
\subsection{ops\_mpi.cpp}
\begin{lstlisting}[language=C++]
bool tsatsyn_a_increasing_contrast_by_histogram_mpi::TestMPITaskSequential::validation() {
  internal_order_test();
  return (taskData->outputs_count[0] > 0 && taskData->inputs_count[0] > 0);
}
bool tsatsyn_a_increasing_contrast_by_histogram_mpi::TestMPITaskSequential::pre_processing() {
  internal_order_test();
  input_data.resize(taskData->inputs_count[0]);
  auto* tempPtr = reinterpret_cast<int*>(taskData->inputs[0]);
  std::copy(tempPtr, tempPtr + taskData->inputs_count[0], input_data.begin());

  return true;
}
bool tsatsyn_a_increasing_contrast_by_histogram_mpi::TestMPITaskSequential::run() {
  internal_order_test();
  int min_val = *std::min_element(input_data.begin(), input_data.end());
  int max_val = *std::max_element(input_data.begin(), input_data.end());
  if (max_val - min_val == 0) {
    max_val++;
  }
  res.resize(input_data.size());
  int input_sz = static_cast<int>(input_data.size());
  for (int i = 0; i < input_sz; i++) {
    res[i] = (input_data[i] - min_val) * (255 - 0) / (max_val - min_val) + 0;
  }
  return true;
}
bool tsatsyn_a_increasing_contrast_by_histogram_mpi::TestMPITaskSequential::post_processing() {
  internal_order_test();
  auto* outputPtr = reinterpret_cast<int*>(taskData->outputs[0]);
  std::copy(res.begin(), res.end(), outputPtr);
  return true;
}
bool tsatsyn_a_increasing_contrast_by_histogram_mpi::TestMPITaskParallel::validation() {
  internal_order_test();
  if (world.rank() == 0) {
    return (taskData->outputs_count[0] > 0 && taskData->inputs_count[0] > 0);
  }
  return true;
}
bool tsatsyn_a_increasing_contrast_by_histogram_mpi::TestMPITaskParallel::pre_processing() {
  internal_order_test();
  if (world.rank() == 0) {
    input_data.resize(taskData->inputs_count[0]);
    auto* tempPtr = reinterpret_cast<int*>(taskData->inputs[0]);
    std::copy(tempPtr, tempPtr + taskData->inputs_count[0], input_data.begin());
  }
  return true;
}
bool tsatsyn_a_increasing_contrast_by_histogram_mpi::TestMPITaskParallel::run() {
  internal_order_test();
  std::vector<int> local_data;
  int min_val;
  int max_val;
  int input_sz;
  input_sz = static_cast<int>(input_data.size());
  if (world.rank() == 0) {
    min_val = *std::min_element(input_data.begin(), input_data.end());
    max_val = *std::max_element(input_data.begin(), input_data.end());
    if (max_val - min_val == 0) {
      max_val++;
    }
    for (int proc = 1; proc < world.size(); proc++) {
      for (int i = proc; i < input_sz; i += world.size()) {
        local_data.emplace_back(input_data[i]);
      }
      world.send(proc, 0, local_data);
      local_data.clear();
    }
    for (int i = 0; i < input_sz; i += world.size()) {
      local_data.emplace_back(input_data[i]);
    }
  } else {
    world.recv(0, 0, local_data);
  }
  boost::mpi::broadcast(world, max_val, 0);
  boost::mpi::broadcast(world, min_val, 0);

  for (int i = 0; i < static_cast<int>(local_data.size()); i++) {
    local_data[i] = ((local_data[i] - min_val) * (255 - 0) / (max_val - min_val)) + 0;
  }
  if (world.rank() == 0) {
    std::vector<int> expected(1, 0);
    expected.resize(taskData->inputs_count[0]);
    for (int i = 0; i < static_cast<int>(local_data.size()); i++) {
      expected[i * world.size()] = local_data[i];
    }
    for (int proc = 1; proc < world.size(); proc++) {
      world.recv(proc, 0, local_data);
      for (int i = 0; i < static_cast<int>(local_data.size()); i++) {
        expected[i * world.size() + proc] = local_data[i];
      }
    }
    res = expected;
  } else {
    world.send(0, 0, local_data);
  }

  return true;
}
bool tsatsyn_a_increasing_contrast_by_histogram_mpi::TestMPITaskParallel::post_processing() {
  internal_order_test();
  if (world.rank() == 0) {
    auto* outputPtr = reinterpret_cast<int*>(taskData->outputs[0]);
    std::copy(res.begin(), res.end(), outputPtr);
  }
  return true;
}
\end{lstlisting}





\end{document}
