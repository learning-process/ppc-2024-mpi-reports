\documentclass[a4paper,12pt]{article}
\usepackage[utf8]{inputenc}
\usepackage[russian]{babel}
\usepackage{amsmath}
\usepackage{graphicx}
\usepackage{hyperref}
\usepackage{listings}
\usepackage{xcolor}

\definecolor{codegray}{rgb}{0.5,0.5,0.5}
\definecolor{codepurple}{rgb}{0.58,0,0.82}

\lstdefinestyle{mystyle}{
    backgroundcolor=\color{white},
    commentstyle=\color{codegray},
    keywordstyle=\color{blue},
    numberstyle=\tiny\color{codegray},
    stringstyle=\color{codepurple},
    basicstyle=\ttfamily\footnotesize,
    breakatwhitespace=false,
    breaklines=true,
    captionpos=b,
    keepspaces=true,
    numbers=left,
    numbersep=5pt,
    showspaces=false,
    showstringspaces=false,
    showtabs=false,
    tabsize=2
}

\lstset{style=mystyle}

\begin{document}

\begin{titlepage}
    \begin{center}
        \large 
        \textbf{ННГУ им. Лобачевского / ИИТММ / ПМИ}\\[0.5cm]

        \vspace{4cm}
        \textbf{\Large Отчёт по лаболаторной работе}\\
        \textbf{\large «Маркировка компонент на бинарном изображении»}\\[3cm]

        \vspace{3cm}
        \textbf{Выполнил:}\\
        студент группы 3822Б1ПМоп3 \\
        \textit{Смирнов Илья Кириллович}\\[1cm]

        \textbf{Преподаватель:}\\
        \textit{кандидат технических наук, доцент кафедры ВВСП}\\
        \textit{Сысоев Александр Владимирович}\\[2cm]

        \vfill
        \textbf{Нижний Новгород, 2024 г.}
    \end{center}
\end{titlepage}

\tableofcontents
\newpage

\section{Введение}
Распознавание и маркировка компонент бинарных изображений является одной из ключевых задач в области обработки изображений. В данном отчете рассматривается реализация последовательной и параллельной версии алгоритма маркировки компонент бинарного изображения, включая её распараллеливание с использованием MPI.

\section{Постановка задачи}
Целью работы является разработка алгоритма для маркировки всех связанных компонент бинарного изображения. Алгоритм должен быть реализован как в последовательной, так и в параллельной версии. Требуется провести оценку корректности и производительности, сравнив обе версии.

\section{Описание алгоритма}
Алгоритм предполагает движение по изображению горизонтально и присваивает каждой точке уникальный идентификатор на основе значений соседних точек.

Алгоритм работает следующим образом:
\begin{enumerate}
    

    \item Инициализация: каждый пиксель A инициализируется значением 1, если все его соседи также имеют значение 1, точка считается фоновой. \\
    Соседями считаются левый, верхний и диагональный левый верхний пиксели. 

    \item Маркировка: если A имеет значение 0 и все её соседи имеют значение 0, точке присваивается новый уникальный идентификатор.

    \item Конфликты: если A имеет значение 1, но её соседи имеют разные значения, алгоритм использует таблицу эквивалентности для разрешения конфликта. В таблице эквивалентности записываются пары точек с одинаковыми значениями.

    \item Повторение: алгоритм повторяет шаги 2 и 3 для всех точек изображения, пока все точки не будут помечены уникальными идентификаторами.

    \item Разрешение конфликтов: после завершения первого прохода алгоритм проходит по изображению ещё раз, используя таблицу эквивалентности, чтобы присвоить правильные идентификаторы точкам, которые имели конфликты на первом проходе.
\end{enumerate}

\section{Описание схемы распараллеливания}
Для распараллеливания алгоритма используются подходы разделения данных:
\begin{enumerate}
    \item Инициализация:
    \begin{enumerate}
        \item Главный процесс (ранг 0) делит изображение на N частей (где N — количество процессов).
        \item Каждой части так же дорисовывается граница.
        \item Части изображения распределяются между процессами с помощью MPI\_Scatterv.
    \end{enumerate}
    \item Локальная сегментация.\\
    Каждый процесс независимо применяет последовательный алгоритм сегментации к своей части изображения. Для каждой части создаётся локальная таблица эквивалентности.

    \item Разрешение глобальной эквивалентности. \\
    Локальные таблицы эквивалентности передаются в главный процесс с помощью MPI\_Gatherv. Главный процесс объединяет их в единую глобальную таблицу эквивалентности.
    \item Далее главный процесс использует таблицу для объединения эквивалентных меток, принадлежащих одной компоненте. 
    Все метки заменяются на их каноническую (минимальную) эквивалентную метку.

    Итоговые метки перенумеровываются последовательно, чтобы устранить пропуски в нумерации и обеспечить согласованность.

\end{enumerate}

\section{Описание MPI-версии}
Параллельная версия использует библиотеку Boost.MPI. Реализация включает:
\begin{itemize}
    \item Разбиение изображения на блоки.
    \item Распределение блоков между процессами с использованием функций \texttt{MPI\_Scatterv}.
    \item Локальная обработка пикселей и сборка таблицы эквивалентности.
    \item Объединение таблиц эквивалентности с помощью \texttt{MPI\_Reduce}.
\end{itemize}

Код параллельной версии представлен в приложении.


\section{Результаты экспериментов}
Эксперименты проводились на изображениях размером $2048 \times 2048$. Сравнение времени работы алгоритма показало:
\begin{itemize}
    \item Последовательная версия: $t_{seq} = 1.6$ сек (на изображении $1024 \times 1024$).
    \item Параллельная версия на 2 процессах: $t_{par2} = 1.5$ сек.
    \item Параллельная версия на 3 процессах: $t_{par3} = 1.5$ сек.
    \item Параллельная версия на 4 процессах: $t_{par4} = 1.4$ сек.
\end{itemize}
Корректность подтверждена сравнением выходных данных последовательной и параллельной версии, которые совпали для всех тестовых и случайно сгенерированных изображений.
\section{Выводы из результатов}
Полученные результаты показывают, что использование параллельной версии позволяет несколько ускорить выполнение задачи за счёт распределения нагрузки между процессами. Параллельная версия масштабируется на изображениях больших размеров.

\section{Заключение}
В рамках работы был разработан алгоритм маркировки компонент бинарного изображения и его реализация для параллельных систем.
\newpage
\section{Литература}
\begin{enumerate}
    \item Boost.MPI Documentation: \url{https://www.boost.org/doc/libs/release/doc/html/mpi.html}
    \item Gonzalez R. C., Woods R. E. Digital Image Processing. 4th Edition.
    \item  Учебное пособие ННГУ по MPI \url{http://www.hpcc.unn.ru/mskurs/RUS/DOC/ppr04.pdf}

\end{enumerate}
\newpage
\section{Приложение}
\begin{lstlisting}[language=C++,caption={Фрагмент MPI-версии}]
std::vector<int> smirnov_i_binary_segmentation::TestMPITaskParallel::make_border(const std::vector<int>& img_,int cols_, int rows_) {
  std::vector<int> new_img((cols_ + 2) * (rows_ + 2), 1);
  for (int row = 0; row < rows_; row++) {
    int src_start = row * cols_;
    int dst_start = (row + 1) * (cols_ + 2) + 1;
    std::copy(&img_[src_start], &img_[src_start + cols_], &new_img[dst_start]);
  }
  return new_img;
}
std::vector<int> smirnov_i_binary_segmentation::TestMPITaskParallel::del_border(const std::vector<int>& img_, int cols_, int rows_) {
  int bordered_cols = cols_ + 2;
  std::vector<int> result(cols_ * rows_);
  for (int row = 0; row < rows_; row++) {
    int src_start = (row + 1) * bordered_cols + 1;
    int dst_start = row * cols_;

    std::copy(&img_[src_start], &img_[src_start + cols_], &result[dst_start]);
  }

  return result;
}

void smirnov_i_binary_segmentation::TestMPITaskParallel::merge_equivalence(std::map<int, std::set<int>>& eq_table, int a, int b) {
  if (a == b) return;

  auto find_a = eq_table.find(a);
  auto find_b = eq_table.find(b);

  if (find_a == eq_table.end() && find_b == eq_table.end()) {
    eq_table[a] = {b};
    eq_table[b] = {a};
  } else if (find_a != eq_table.end() && find_b == eq_table.end()) {
    eq_table[a].insert(b);
    eq_table[b] = eq_table[a];
  } else if (find_a == eq_table.end() && find_b != eq_table.end()) {
    eq_table[b].insert(a);
    eq_table[a] = eq_table[b];
  } else {
    eq_table[a].insert(eq_table[b].begin(), eq_table[b].end());
    eq_table[b] = eq_table[a];
  }
}
std::string smirnov_i_binary_segmentation::TestMPITaskParallel::serialize_eq_table(
    const std::map<int, std::set<int>>& eq_table) {
  std::ostringstream oss;
  for (const auto& pair : eq_table) {
    int key = pair.first;
    auto values = pair.second;
    oss << key << ":";
    for (int value : values) {
      oss << value << ",";
    }
    oss << ";";
  }
  return oss.str();
}
std::map<int, std::set<int>> smirnov_i_binary_segmentation::TestMPITaskParallel::deserialize_eq_table(
    const std::string& str) {
  std::map<int, std::set<int>> eq_table;
  std::istringstream iss(str);
  std::string entry;

  while (std::getline(iss, entry, ';')) {
    if (entry.empty()) continue;

    size_t colon_pos = entry.find(':');
    int key = std::stoi(entry.substr(0, colon_pos));
    std::set<int> values;

    std::string values_str = entry.substr(colon_pos + 1);
    std::istringstream value_stream(values_str);
    std::string value;

    while (std::getline(value_stream, value, ',')) {
      if (!value.empty()) {
        values.insert(std::stoi(value));
      }
    }

    eq_table[key] = values;
  }
  return eq_table;
}
void smirnov_i_binary_segmentation::TestMPITaskParallel::merge_global_eq_table(
    std::map<int, std::set<int>>& global_eq_table, const std::map<int, std::set<int>>& local_eq_table) {
  for (const auto& pair : local_eq_table) {
    int key = pair.first;
    auto values = pair.second;
    if (global_eq_table.find(key) == global_eq_table.end()) {
      global_eq_table[key] = values;
    } else {
      global_eq_table[key].insert(values.begin(), values.end());
    }

    for (int value : values) {
      global_eq_table[value].insert(key);
    }
  }
}
bool smirnov_i_binary_segmentation::TestMPITaskParallel::run() {
  internal_order_test();
  int rank;
  int size;
  MPI_Comm_rank(MPI_COMM_WORLD, &rank);
  MPI_Comm_size(MPI_COMM_WORLD, &size);

  int dims[2];
  if (rank == 0) {
    dims[0] = rows;
    dims[1] = cols;
  }

  MPI_Bcast(dims, 2, MPI_INT, 0, MPI_COMM_WORLD);

  rows = dims[0];
  cols = dims[1];

  if (rank != 0) {
    img.resize(cols * rows);
  }

  MPI_Bcast(img.data(), cols * rows, MPI_INT, 0, MPI_COMM_WORLD);
  int* sendcounts = new int[size];
  std::fill(sendcounts, sendcounts + size, 0);
  int* displs = new int[size]();
  int rows_per_proc = rows / size;
  int extra_rows = rows % size;
  int offset = 0;
  for (int i = 0; i < size; ++i) {
    if (i < extra_rows) {
      sendcounts[i] = (rows_per_proc + 1) * cols;
    } else {
      sendcounts[i] = rows_per_proc * cols;
    }
    displs[i] = offset;
    offset += sendcounts[i];
  }
  std::vector<int> local_img(sendcounts[rank], 1);
  MPI_Scatterv(img.data(), sendcounts, displs, MPI_INT, local_img.data(), sendcounts[rank], MPI_INT, 0, MPI_COMM_WORLD);

  std::vector<int> bord_img = make_border(local_img, cols, sendcounts[rank] / cols);
  std::map<int, std::set<int>> local_eq_table;
  int bord_cols = cols + 2;
  int px_A;
  int px_B;
  int px_C;
  int px_D;
  int mark = 2 + rank * (sendcounts[0] + 5);
  // D B
  // C A
  // direct local pass
  for (size_t i = bord_cols + 1; i < bord_img.size() - bord_cols; i++) {
    if (i % bord_cols == 0 || static_cast<int>(i) % bord_cols == bord_cols - 1) {
      continue;
    }
    px_A = i;
    px_B = i - bord_cols;
    px_C = i - 1;
    px_D = i - 1 - bord_cols;
    if (bord_img[px_A] == 0 && bord_img[px_C] == 1 && bord_img[px_B] == 1 && bord_img[px_D] == 1) {
      bord_img[px_A] = mark;
      mark++;
    } else if (bord_img[px_A] == 0 && bord_img[px_C] == 1 && bord_img[px_B] == 1 && bord_img[px_D] != 1) {
      bord_img[px_A] = bord_img[px_D];
    } else if ((bord_img[px_A] == 0 && bord_img[px_C] != 1 && bord_img[px_B] == 1) ||
               (bord_img[px_A] == 0 && bord_img[px_C] == 1 && bord_img[px_B] != 1)) {
      bord_img[px_A] = std::max(bord_img[px_C], bord_img[px_B]);
    } else if (bord_img[px_A] == 0 && bord_img[px_C] != 1 && bord_img[px_B] != 1) {
      if (bord_img[px_C] == bord_img[px_B]) {
        bord_img[px_A] = bord_img[px_B];
      } else {
        bord_img[px_A] = std::min(bord_img[px_C], bord_img[px_B]);
        merge_equivalence(local_eq_table, bord_img[px_B], bord_img[px_C]);
      }
    }
  }

  for (auto& pair : local_eq_table) {
    std::set<int>& values = pair.second;
    for (int v : values) {
      local_eq_table[v].insert(values.begin(), values.end());
    }
  }

  for (size_t i = bord_cols + 1; i < (bord_img.size() - bord_cols); i++) {
    if (i % bord_cols == 0 || static_cast<int>(i) % bord_cols == bord_cols - 1) {
      continue;
    }

    int label = bord_img[i];
    auto find_label = local_eq_table.find(label);
    if (find_label != local_eq_table.end()) {
      bord_img[i] = *std::min_element(find_label->second.begin(), find_label->second.end());
    }
  }

  std::vector<int> local_mask;
  local_mask = del_border(bord_img, cols, sendcounts[rank] / cols);
  if (rank != 0) {
    mask = std::vector<int>();
  }
  int* recvcounts = new int[size]();
  int* recvdispls = new int[size]();
  offset = 0;

  for (int i = 0; i < size; ++i) {
    if (i < extra_rows) {
      recvcounts[i] = (rows_per_proc + 1) * cols;
    } else {
      recvcounts[i] = rows_per_proc * cols;
    }
    recvdispls[i] = offset;
    offset += recvcounts[i];
  };
  MPI_Gatherv(local_mask.data(), sendcounts[rank], MPI_INT, mask.data(), recvcounts, recvdispls, MPI_INT, 0,
              MPI_COMM_WORLD);

  std::map<int, std::set<int>> global_eq_table;

  std::string local_eq_table_str = serialize_eq_table(local_eq_table);
  int str_len = local_eq_table_str.size();
  std::vector<int> lengths(size);
  MPI_Gather(&str_len, 1, MPI_INT, lengths.data(), 1, MPI_INT, 0, MPI_COMM_WORLD);

  std::vector<char> recv_buf(std::accumulate(lengths.begin(), lengths.end(), 0));
  std::vector<int> offsets(size, 0);
  std::partial_sum(lengths.begin(), lengths.end() - 1, offsets.begin() + 1);
  MPI_Gatherv(local_eq_table_str.data(), str_len, MPI_CHAR, recv_buf.data(), lengths.data(), offsets.data(), MPI_CHAR,
              0, MPI_COMM_WORLD);
  // resolving border conflicts and renaming marks
  if (rank == 0) {
    for (int i = 0; i < size; ++i) {
      std::map<int, std::set<int>> temp_eq_table = deserialize_eq_table(std::string(&recv_buf[offsets[i]], lengths[i]));
      merge_global_eq_table(global_eq_table, temp_eq_table);
    }

    for (int i = 1; i < cols * rows; i++) {
      if (mask[i - 1] != mask[i] && mask[i - 1] != 1 && mask[i] != 1 && i % cols != 0) {
        merge_equivalence(global_eq_table, mask[i - 1], mask[i]);
      }
    }
    for (int i = cols; i < cols * rows; i++) {
      if (mask[i - cols] != mask[i] && mask[i - cols] != 1 && mask[i] != 1) {
        merge_equivalence(global_eq_table, mask[i - cols], mask[i]);
      }
    }
    for (int i = cols + 1; i < cols * rows; i++) {
      if (mask[i - cols - 1] != mask[i] && mask[i - cols - 1] != 1 && mask[i] != 1 && i % cols != 0) {
        merge_equivalence(global_eq_table, mask[i - cols - 1], mask[i]);
      }
    }
    for (auto& pair : global_eq_table) {
      std::set<int>& values = pair.second;

      for (int v : values) {
        global_eq_table[v].insert(values.begin(), values.end());
      }
    }

    for (size_t i = 0; i < (mask.size()); i++) {
      int label = mask[i];
      auto find_label = global_eq_table.find(label);
      if (find_label != global_eq_table.end()) {
        mask[i] = *std::min_element(find_label->second.begin(), find_label->second.end());
      }
    }
    std::set<int> unique_labels(mask.begin(), mask.end());
    unique_labels.erase(0);
    unique_labels.erase(1);

    std::map<int, int> label_mapping;
    int new_label = 2;

    for (int label : unique_labels) {
      auto it = global_eq_table.find(label);
      if (it != global_eq_table.end()) {
        int canonical_label = *std::min_element(it->second.begin(), it->second.end());
        if (label_mapping.find(canonical_label) == label_mapping.end()) {
          label_mapping[canonical_label] = new_label++;
        }
        label_mapping[label] = label_mapping[canonical_label];
      } else {
        label_mapping[label] = new_label++;
      }
    }
    for (size_t i = 0; i < mask.size(); ++i) {
      if (mask[i] > 1) {
        mask[i] = label_mapping[mask[i]];
      }
    }
  }

  return true;
}

\end{lstlisting}

\end{document}
`
