\documentclass[a4paper,12pt]{article}
\usepackage[T2A]{fontenc}
\usepackage[utf8]{inputenc}
\usepackage[russian]{babel}
\usepackage{amsmath}
\usepackage{amssymb}
\usepackage{booktabs}
\usepackage{listings}
\usepackage{geometry}
\usepackage{titlesec}
\usepackage{hyperref}
\usepackage{tocloft}
\usepackage{xcolor} 
\usepackage{float}


\geometry{top=20mm, bottom=20mm, left=25mm, right=25mm}

\lstset{
  language=C++,
  basicstyle=\ttfamily\small,
  keywordstyle=\bfseries\color{blue},
  commentstyle=\itshape\color{green!50!black},
  stringstyle=\color{red!60!black},
  numbers=left,
  numberstyle=\tiny,
  stepnumber=1,
  numbersep=8pt,
  backgroundcolor=\color{gray!10},
  showspaces=false,
  showstringspaces=false,
  breaklines=true,
  frame=single,
  tabsize=2,
  captionpos=b
}

\begin{document}

\begin{titlepage}
    \centering
    \large
    Министерство науки и высшего образования Российской Федерации\\[0.5cm]
    Федеральное государственное автономное образовательное учреждение высшего образования\\[0.5cm]
    \textbf{«Национальный исследовательский Нижегородский государственный университет им. Н.И. Лобачевского»}\\
    (ННГУ)\\[1cm]
    Институт информационных технологий, математики и механики\\[0.5cm]
    Направление подготовки: \textbf{«Программная инженерия»}\\[1.5cm]

    \vfill
    {\LARGE \textbf{ОТЧЕТ}}\\[0.5cm]
    {\Large по задаче}\\[0.5cm]
    {\LARGE \textbf{«Построение выпуклой оболочки для компонент бинарного изображения»}}\\[0.5cm]
    {\Large \textbf{Вариант №32}}\\[2.5cm]

    \hfill\parbox{0.5\textwidth}{
        \textbf{Выполнил:} \\
        студент группы 3822Б1ПР3 \\
        Громов А.С.
    }\\[0.5cm]

    \hfill\parbox{0.5\textwidth}{
        \textbf{Преподаватели:} \\
        Нестеров А.Ю.\\
        Оболенский А.А.

    }\\[2cm]

    Нижний Новгород\\
    2024
\end{titlepage}
\tableofcontents
\newpage

\section{Введение}

\hspace*{1.25em}Рассмотрение задачи вычисления выпуклой оболочки множества точек представляет собой важный аспект вычислительной геометрии. Выпуклая оболочка находит широкое применение в таких областях, как анализ данных, обработка изображений и задачи оптимизации. Основной целью данной работы является создание параллельного алгоритма вычисления выпуклой оболочки, реализованного с использованием библиотеки Boost.MPI.

\section{Постановка задачи}

\hspace*{1.25em}Необходимо реализовать алгоритм вычисления выпуклой оболочки на основе алгоритма Джарвиса. Реализация включает:
\begin{itemize}
    \item последовательную версию алгоритма,
    \item параллельную версию с использованием Boost.MPI для ускорения вычислений,
    \item тестирование корректности и оценку производительности.
\end{itemize}

\section{Описание алгоритма}

\subsection{Алгоритм Джарвиса}
\hspace*{1.25em}Алгоритм Джарвиса строит выпуклую оболочку путем обхода точек множества с выбором локально оптимальных вершин:
\begin{enumerate}
    \item Найти самую левую точку как начальную вершину.
    \item Итеративно добавлять точки, образующие минимальный угол с предыдущей стороной оболочки.
    \item Завершить, когда первая точка будет выбрана второй раз.
\end{enumerate}

\subsection{Маркировка связных областей}
\hspace*{1.25em}Для предварительной обработки изображения используется алгоритм двухэтапной маркировки (первый и второй проходы), позволяющий идентифицировать все связные компоненты, что упрощает задачу построения оболочки.

\section{Распараллеливание}

\hspace*{1.25em}Схема распараллеливания строится на распределении компонентов между процессами. Основной процесс (ранг $0$) выполняет:
\begin{itemize}
    \item Инициализацию изображения и разбиение компонентов.
    \item Отправку данных другим процессам.
    \item Сбор локальных выпуклых оболочек и их объединение.
\end{itemize}
Остальные процессы вычисляют локальные оболочки для своих компонентов и отправляют результаты основному процессу.

\section{Тестирование}

\hspace*{1.25em}Для проверки корректности работы программы были написаны тесты с использованием Google Test. Примеры входных данных:
\begin{itemize}
    \item Полностью заполненные изображения.
    \item Пустые изображения.
    \item Изображения с чередующимися пикселями.
\end{itemize}

\subsection{Результаты тестов}
\begin{table}[H]
\centering
\begin{tabular}{cccc}
\toprule
\textbf{Кол-во процессов} & \textbf{Общее время, мс} & \textbf{Тест Big Image 2} & \textbf{Тест Big Image 3} \\
\midrule
1 & 713 & 178 & 424 \\
2 & 706 & 176 & 417 \\
3 & 706 & 178 & 414 \\
4 & 704 & 175 & 415 \\
\bottomrule
\end{tabular}
\caption{Результаты тестирования производительности для параллельного алгоритма}
\end{table}

\section{Заключение}

\hspace*{1.25em}В ходе работы разработаны и протестированы последовательная и параллельная версии алгоритма вычисления выпуклой оболочки. Использование Boost.MPI позволило ускорить вычисления на больших данных за счет распределения компонентов между процессами. 

Наилучшее ускорение наблюдается при увеличении числа процессов до 4. При дальнейшем росте числа процессов эффективность уменьшается из-за накладных расходов на коммуникацию.

\section{Литература}
\begin{enumerate}
    \item \href{https://en.wikipedia.org/wiki/Convex_hull}{Wikipedia. Convex hull}.
    \item \href{https://algorithmica.org/ru/convex-hulls}{Алгоритмика. Выпуклые оболочки}.
    \item \href{https://ru.wikipedia.org/wiki/%D0%90%D0%BB%D0%B3%D0%BE%D1%80%D0%B8%D1%82%D0%BC_%D0%94%D0%B6%D0%B0%D1%80%D0%B2%D0%B8%D1%81%D0%B0}{Wikipedia. Алгоритм Джарвиса}.
\end{enumerate}

\section{Приложения}
\addcontentsline{toc}{section}{Приложения}
\subsection*{Алгоритм Джарвиса}
\begin{lstlisting}[language=C++]
std::vector<Point> jarvis(std::vector<Point> points) {
  std::vector<Point> hull;
  if (points.size() < 3) {
    return points;
  }
  int leftmost = 0;
  for (size_t i = 1; i < points.size(); ++i) {
    if (points[i].x < points[leftmost].x || (points[i].x == points[leftmost].x && points[i].y < points[leftmost].y)) {
      leftmost = i;
    }
  }
  int p = leftmost;
  int q;
  do {
    hull.push_back(points[p]);
    q = (p + 1) % points.size();
    for (size_t i = 0; i < points.size(); ++i) {
      if (calculateProduct(points[p], points[q], points[i]) < 0) {
        q = i;
      }
    }
    p = q;
  } while (p != leftmost);
  return hull;
}
\end{lstlisting}
\end{document}