% Настройки для pdfLaTeX и поддержки русского языка
\documentclass[14pt]{extarticle}
\usepackage[T2A]{fontenc}  % Должен быть загружен перед babel
\usepackage[utf8]{inputenc}
\usepackage[russian]{babel}

\usepackage{bookmark}
\usepackage{mathptmx} % Times New Roman для текста и математики

\usepackage{cite}
\makeatletter
\renewcommand{\@biblabel}[1]{#1.}
\makeatother

\usepackage{hyperref}
\usepackage[dvips]{graphicx}
\usepackage{float}
\graphicspath{{picture/}}
\usepackage{amssymb,amsfonts,amsmath,amsthm}
\usepackage{indentfirst}
\usepackage[usenames,dvipsnames]{color}
\usepackage{makecell}
\usepackage{multirow}
\usepackage{ulem}
\renewcommand{\arraystretch}{1.5}
\linespread{1.3}
\frenchspacing

\usepackage[table,xcdraw]{xcolor}
\usepackage{wrapfig}
\usepackage{pgfplots}
\pgfplotsset{compat=newest}

\usepackage{geometry}
\geometry{left = 2.5cm}
\geometry{right = 1.5cm}
\geometry{top = 2cm}
\geometry{bottom = 2cm}

\renewcommand\textbullet{\ensuremath{\bullet}}
\usepackage{cancel}

\usepackage{algorithm}
\usepackage{algpseudocode}
\floatname{algorithm}{Алгоритм}

\usepackage{listings}
\usepackage{color}
\lstdefinestyle{mystyle}{
    basicstyle=\footnotesize,
    breakatwhitespace=false, 
    breaklines=true,
    captionpos=b,
    keepspaces=true, 
    numbers=left,
    numbersep=5pt,
    showspaces=false,
    showstringspaces=false,
    showtabs=false,                  
    tabsize=2
}
\lstset{style=mystyle}
\lstset{basicstyle=\footnotesize\ttfamily,breaklines=true}

\newtheorem{theorem}{Теорема}[section]
\newtheorem{corollary}{Следствие}[section]
\newtheorem{lemma}{Лемма}[section]
\newtheorem{proposition}{Утверждение}[section]
\theoremstyle{definition}
\newtheorem{definition}{Определение}[section]
\newtheorem{example}{Пример}[section]
\theoremstyle{remark}
\newtheorem*{remark}{Замечание}

\newcommand{\argmin}{\operatornamewithlimits{argmin}}
\newcommand{\Mod}[1]{\ (\mathrm{mod}\ #1)}

\usepackage[inkscapeformat=pdf]{svg}

% Для использования fontspec необходимо использовать XeLaTeX или LuaLaTeX
\usepackage{fontspec}
\setmainfont{Times New Roman}

% Настройка заголовков с использованием titlesec
\usepackage{titlesec}
\titleformat{\section}{\bfseries\Large}{\thesection.\hspace{0.5em}}{0pt}{}
\titleformat{\subsection}{\bfseries\large}{\thesubsection.\hspace{0.5em}}{0pt}{}
\titleformat{name=\section,numberless}{\bfseries\Large\filcenter}{}{0pt}{}

% Отступы заголовков
\titlespacing*{\section}{0pt}{3.5ex plus 1ex minus .2ex}{2.3ex plus .2ex}
\titlespacing*{\subsection}{0pt}{3.25ex plus 1ex minus .2ex}{1.5ex plus .2ex}

% Настройка содержания с помощью tocloft для добавления линии точек
\usepackage{tocloft}
\renewcommand{\cftsecleader}{\cftdotfill{\cftdotsep}}
\renewcommand{\cftsecfont}{}
\renewcommand{\cftsecpagefont}{}
\setlength{\cftbeforesecskip}{0pt}

\begin{document}
\thispagestyle{empty}
\begin{center}
{\normalsize\textbf{МИНИСТЕРСТВО НАУКИ И ВЫСШЕГО ОБРАЗОВАНИЯ\\
РОССИЙСКОЙ ФЕДЕРАЦИИ}}

\vspace{0.3cm}
{\normalsize Федеральное государственное автономное образовательное учреждение\\
высшего образования\\
«Национальный исследовательский
Нижегородский государственный университет им. Н.И. Лобачевского»
(ННГУ)}

\vspace{0.5cm}
{\normalsize\textbf{Институт информационных технологий, математики и механики}}

\vspace{1cm}
{\normalsize Направление подготовки:\\
02.03.02 Фундаментальная информатика и информационные технологии}

\vspace{1cm}
{\normalsize\textbf{ОТЧЕТ}\\
\textbf{по лабораторной работе}
}

\vspace{1cm}
{\normalsize на тему:\\
«Линейная фильтрация изображений.\\ Ядро Гаусса 3x3»}

\vfill

\begin{flushright}
\begin{tabular}{l}
Выполнил:
студент группы 3822Б1ФИ1\\
\underline{\hspace{3cm}} Шурыгин С.Е.\\
Проверил:\\
кандидат технических наук\\
\underline{\hspace{3cm}}  Сысоев А.В.
\end{tabular}
\end{flushright}

\vfill

{\normalsize\text{Нижний Новгород}\\
\text{2024}}

\end{center}

\newpage
\tableofcontents
\newpage

\addcontentsline{toc}{section}{Введение}
\section*{Введение}
В современной обработке изображений линейная фильтрация является одним из фундаментальных инструментов для решения широкого спектра задач, включая подавление шума, сглаживание и выделение особенностей изображений. Особое место среди методов линейной фильтрации занимает фильтр Гаусса, который обеспечивает равномерное размытие изображения с сохранением важных структурных особенностей.

\newpage

\section{Постановка задачи}
Основной целью данной работы является разработка и реализация эффективного параллельного алгоритма линейной фильтрации изображений с использованием ядра Гаусса размером 3×3 на основе блочного разбиения данных.

Для достижения поставленной цели необходимо решить следующие задачи:

\begin{enumerate}
    \item Реализовать последовательный алгоритм фильтрации изображения с использованием ядра Гаусса 3×3.

    \item Разработать схему блочного разбиения исходного изображения.

    \item Реализовать параллельный алгоритм фильтрации с использованием технологии MPI.

    \item Провести исследование эффективности разработанного параллельного алгоритма.
\end{enumerate}
Входными данными алгоритма является полутоновое изображение размером $M \times N$ пикселей, где каждый пиксель представлен целым числом в диапазоне [0, 255]. Выходными данными является отфильтрованное изображение того же размера.
\clearpage
\section{Описание алгоритма}

Последовательный алгоритм линейной фильтрации изображения с использованием ядра Гаусса 3×3 реализован в виде четырех основных этапов:

\subsection{Валидация входных данных}
На этапе валидации производится проверка корректности входных данных:
проверка наличия всех необходимых входных параметров (матрица изображения, количество строк и столбцов), проверка размерности изображения (минимум 3×3 пикселя), проверка соответствия размеров входных и выходных буферов, а также валидация значений пикселей (должны находиться в диапазоне [0, 255]).

\subsection{Предварительная обработка}
На этапе предварительной обработки выполняется извлечение размеров изображения из входных параметров, копирование исходных данных изображения во внутренний буфер и инициализация выходного буфера для результата фильтрации.

\subsection{Основной этап фильтрации}

Для каждого пикселя изображения (кроме граничных) вычисляется взвешенная сумма значений в окне 3×3 вокруг текущего пикселя. Весовые коэффициенты соответствуют значениям ядра Гаусса, а результат сохраняется в выходной буфер. Обработка выполняется построчно, начиная с позиции (1,1) до (rows-2, cols-2), при этом для каждого пикселя учитываются значения 8 соседних пикселей и самого пикселя.

\subsection{Пост-обработка}
На заключительном этапе результаты фильтрации копируются из внутреннего буфера в выходной массив, после чего освобождаются временные буферы.

Особенности реализации: алгоритм использует двойную точность для промежуточных вычислений, граничные пиксели не обрабатываются в текущей реализации, все операции выполняются с учетом линейной структуры хранения двумерного изображения в памяти.
\clearpage
\section{Описание схемы распараллеливания}

Параллельная реализация алгоритма фильтрации изображения основана на блочном разбиении данных с использованием технологии MPI.

Изображение разбивается на блоки с учетом необходимости обработки граничных областей. Для каждого блока вычисляются:
\begin{itemize}
    \item Размер блока данных с учетом перекрытия
    \item Смещение относительно начала изображения
    \item Координаты обрабатываемых пикселей
\end{itemize}

Процесс параллельной обработки организован следующим образом: корневой процесс (ранг 0) выполняет начальное распределение данных, используя функции MPI для рассылки частей изображения. Каждый процесс получает свой блок данных и список координат пикселей для обработки. После обработки своей части данных процессы отправляют результаты обратно корневому процессу, который формирует итоговое изображение.

\clearpage
\section{Описание MPI реализации}
Процесс параллельной обработки организован следующим образом:

Корневой процесс (ранг 0) выполняет:
\begin{itemize}
    \item \textbf{\texttt{validation()}}: Валидация входных данных, проверка их корректности.
    \item \textbf{\texttt{pre\_processing()}}: Вычисление индексов обработки для матриц, распределение блоков данных.
    \item Рассылку параметров задачи остальным процессам, включая размеры и смещения данных.\\
\end{itemize}

Каждый процесс получает:
\begin{itemize}
    \item Локальный блок данных изображения.
    \item Координаты пикселей для обработки, необходимые для применения фильтра.
    \item Размеры и смещения для своего блока из распределенных данных.\\
\end{itemize}

Обработка данных:
\begin{itemize}
    \item Каждый процесс применяет фильтр Гаусса к своему блоку данных.
    \item Используется матрица ядра размером $3 \times 3$.
    \item Учитываются граничные условия при обработке, чтобы обеспечить корректность результатов.\\
\end{itemize}

Сбор результатов:
\begin{itemize}
    \item Результаты обработки собираются на корневом процессе (\texttt{rank = 0}).
    \item Выполняется финальная компоновка изображения для получения окончательного результата.\\
\end{itemize}

Благодаря использованию методов, таких как валидация входных данных и предварительная обработка, обеспечивает корректность и согласованность передаваемой информации. Ключевые функции, включая \texttt{run()} и \texttt{post\_processing()}, позволяют эффективно делить задачу на более мелкие части, обрабатывая данные локально и затем собирая результаты для финального вывода.

\clearpage
\section{Проведение экспиримента}

Для оценки эффективности разработанного параллельного алгоритма было проведено экспериментальное исследование. В качестве входных данных использовались случайно сгенерированные матрицы различных размеров: 100×100, 1000×1000 и 10000×10000 элементов. Каждый элемент матрицы представлял собой целое число в диапазоне [0, 255], что соответствует значениям пикселей в полутоновом изображении.

Тестирование проводилось в двух режимах:
\begin{itemize}
    \item Последовательное выполнение (TaskSeq)
    \item Параллельное выполнение с использованием MPI (TaskMpi)
\end{itemize}

\subsection{Результаты измерений}
\begin{table}[h]
\centering
\begin{tabular}{|c|c|c|c|}
\hline
\textbf{Размер матрицы} & \textbf{TaskSeq} & \textbf{TaskMpi} & \textbf{Ускорение} \\
\hline
100×100 & 0.00068 с & 0.00236 с & 0.29x \\
1000×1000 & 0.07154 с & 0.05276 с & 1.36x \\
10000×10000 & 7.82631 с & 4.15147 с & 1.89x \\
\hline
\end{tabular}
\caption{Время выполнения алгоритма}
\label{tab:execution_time}
\end{table}

\subsection{Анализ результатов}
Анализ полученных результатов показывает следующие закономерности:

\begin{enumerate}
    \item Для малых размеров матрицы (100×100) последовательная версия алгоритма работает быстрее параллельной. Это объясняется накладными расходами на инициализацию MPI и распределение данных между процессами, которые превышают выигрыш от параллельной обработки при малом объеме вычислений.
    
    \item При увеличении размера матрицы до 1000×1000 параллельная версия начинает демонстрировать преимущество, обеспечивая ускорение в 1.36 раза по сравнению с последовательной версией.
    
    \item Для больших матриц (10000×10000) преимущество параллельной версии становится наиболее существенным - достигается ускорение в 1.89 раза. Это подтверждает эффективность выбранной схемы распараллеливания для обработки больших объемов данных.
\end{enumerate}

\subsection{Выводы}
Экспериментальное исследование показало, что разработанный для малого количества данных будет проще использовать последовательную версию, так как она быстрее, проще в реализации и понимании. Однако, когда речь заходит о больших массивах данных следует использовать параллельную версию.
\clearpage

\addcontentsline{toc}{section}{Заключение}
\section*{Заключение}

В рамках данной лабораторной был разработан и реализован параллельный алгоритм линейной фильтрации изображений с использованием ядра Гаусса на основе блочного разбиения данных.

Проведенное экспериментальное исследование показало значительное ускорение обработки - в 1.89 раза для изображений размером 10000×10000 пикселей. Была подтверждена эффективность параллельной версии для матриц размером от 1000×1000 элементов, в то время как для малых изображений (100×100) более эффективным оказался последовательный алгоритм.

Таким образом, поставленные в работе цели были достигнуты, а разработанный параллельный алгоритм показал свою эффективность при обработке изображений большого размера.

\newpage
\addcontentsline{toc}{section}{Список литература}
\section*{Список литература}

\begin{enumerate}
\item Гергель В.П. Высокопроизводительные вычисления для многопроцессорных многоядерных систем. -- М.: Издательство Московского университета, 2019. -- 544 с.

\item Антонов А.С. Параллельное программирование с использованием технологии MPI. -- М.: Издательство МГУ, 2018. -- 272 с.

\item Гонсалес Р., Вудс Р. Цифровая обработка изображений. -- М.: Техносфера, 2020. -- 1104 с.

\item Форсайт Д., Понс Ж. Компьютерное зрение. Современный подход. -- М.: Вильямс, 2018. -- 928 с.
\end{enumerate}
\newpage
\addcontentsline{toc}{section}{Приложения}
\section*{Приложения}
\subsection*{Приложение №1}
\noindent Реализация методов классов TaskMPI и TaskSEQ:
\begin{lstlisting}[language=C++, caption=Приложение №1]
bool TaskMpi::validation() {
  internal_order_test();
  if (comm.rank() != 0) {
    return true;
  }

  if (!taskData || taskData->inputs.size() < 3 || taskData->inputs_count.size() < 3 || taskData->outputs.empty() ||
      taskData->outputs_count.empty()) {
    return false;
  }

  int num_rows = *reinterpret_cast<int*>(taskData->inputs[1]);
  int num_cols = *reinterpret_cast<int*>(taskData->inputs[2]);
  auto expected_size = static_cast<size_t>(num_rows * num_cols);

  return num_rows >= 3 && num_cols >= 3 && taskData->inputs_count[0] == expected_size &&
         taskData->outputs_count[0] == expected_size && taskData->inputs_count[0] == taskData->outputs_count[0];
}

bool TaskMpi::pre_processing() {
  internal_order_test();

  if (comm.rank() == 0) {
    auto* data_ptr = reinterpret_cast<int*>(taskData->inputs[0]);
    int data_size = taskData->inputs_count[0];

    height = *reinterpret_cast<int*>(taskData->inputs[1]);
    width = *reinterpret_cast<int*>(taskData->inputs[2]);

    input_data.assign(data_ptr, data_ptr + data_size);
    processed_data.resize((height - 2) * (width - 2), 0);

    processing_indices = computeProcessingIndices(height, width);
    calculateDistribution((height - 2) * (width - 2), 1, comm.size(), block_sizes, block_offsets);

    auto distributed_indices = distributeWorkload(processing_indices, block_sizes, block_offsets);
    computeBlockDistribution(distributed_indices, width, data_sizes, data_offsets);
  }

  return true;
}

bool TaskMpi::run() {
  internal_order_test();

  boost::mpi::broadcast(comm, width, 0);
  boost::mpi::broadcast(comm, data_sizes, 0);
  boost::mpi::broadcast(comm, data_offsets, 0);
  boost::mpi::broadcast(comm, block_sizes, 0);

  int local_data_size = data_sizes[comm.rank()];
  int local_block_size = block_sizes[comm.rank()];
  std::vector<int> local_data(local_data_size);
  std::vector<std::pair<int, int>> local_coords(local_block_size);

  if (comm.rank() == 0) {
    boost::mpi::scatterv(comm, input_data.data(), data_sizes, data_offsets, local_data.data(), data_sizes[comm.rank()],
                         0);
    boost::mpi::scatterv(comm, processing_indices.data(), block_sizes, block_offsets, local_coords.data(),
                         block_sizes[comm.rank()], 0);
  } else {
    boost::mpi::scatterv(comm, local_data.data(), data_sizes[comm.rank()], 0);
    boost::mpi::scatterv(comm, local_coords.data(), block_sizes[comm.rank()], 0);
  }

  std::vector<int> local_result(block_sizes[comm.rank()]);
  for (int i = 0; i < block_sizes[comm.rank()]; i++) {
    auto [row, col] = local_coords[i];
    int base_idx = (row - 1) * width + (col - 1) - data_offsets[comm.rank()];

    std::vector<int> neighborhood(9);
    for (int j = 0; j < 3; j++) {
      for (int k = 0; k < 3; k++) {
        neighborhood[j * 3 + k] = local_data[base_idx + width * j + k];
      }
    }

    local_result[i] = applyFilter(neighborhood, 3, 3, 1, 1);
  }

  if (comm.rank() == 0) {
    boost::mpi::gatherv(comm, local_result.data(), local_result.size(), processed_data.data(), block_sizes,
                        block_offsets, 0);
  } else {
    boost::mpi::gatherv(comm, local_result.data(), local_result.size(), 0);
  }

  return true;
}

bool TaskMpi::post_processing() {
  internal_order_test();

  if (comm.rank() == 0) {
    auto* output_ptr = reinterpret_cast<int*>(taskData->outputs[0]);
    std::vector<int> final_result = padMatrixWithZeros(processed_data, height, width);
    std::copy(final_result.begin(), final_result.end(), output_ptr);
  }

  return true;
}

bool TaskSeq::validation() {
  internal_order_test();
  if (!taskData || taskData->inputs.size() < 3 || taskData->inputs_count.size() < 3 || taskData->outputs.empty() ||
      taskData->outputs_count.empty()) {
    return false;
  }

  num_rows = *reinterpret_cast<int*>(taskData->inputs[1]);
  num_cols = *reinterpret_cast<int*>(taskData->inputs[2]);

  if (num_rows < 3 || num_cols < 3) {
    return false;
  }

  auto expected_size = static_cast<size_t>(num_rows * num_cols);
  return taskData->inputs_count[0] == expected_size && taskData->outputs_count[0] == expected_size &&
         taskData->inputs_count[0] == taskData->outputs_count[0];
}

bool TaskSeq::pre_processing() {
  internal_order_test();

  const auto* source_pixels = reinterpret_cast<const int*>(taskData->inputs[0]);
  const size_t total_pixels = taskData->inputs_count[0];

  input_data = std::vector<int>(source_pixels, source_pixels + total_pixels);
  output_data.resize(total_pixels);

  return true;
}

bool TaskSeq::run() {
  internal_order_test();

  for (int r = 1; r < num_rows - 1; ++r) {
    for (int c = 1; c < num_cols - 1; ++c) {
      std::vector<int> neighborhood(9);
      for (int i = 0; i < 3; ++i) {
        for (int j = 0; j < 3; ++j) {
          neighborhood[i * 3 + j] = input_data[(r + i - 1) * num_cols + (c + j - 1)];
        }
      }
      output_data[r * num_cols + c] = applyFilter(neighborhood, 3, 3, 1, 1);
    }
  }

  return true;
}

bool TaskSeq::post_processing() {
  internal_order_test();

  auto* result_data = reinterpret_cast<int*>(taskData->outputs[0]);
  std::copy(output_data.begin(), output_data.end(), result_data);

  return true;
}
\end{lstlisting}
\end{document}