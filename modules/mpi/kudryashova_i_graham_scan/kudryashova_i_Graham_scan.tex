\documentclass[a4paper,12pt]{article}
\usepackage[T2A]{fontenc}
\usepackage[utf8]{inputenc}
\usepackage[russian]{babel}
\usepackage{amsmath, amssymb}
\usepackage{bookmark}
\usepackage{cite}
\usepackage{geometry}
\usepackage{graphicx}
\usepackage{multirow}
\usepackage{array}
\usepackage{listings}
\usepackage{color}
\newtheorem{definition}{Определение}[section]
\geometry{left = 2.5cm}
\geometry{right = 1.5cm}
\geometry{top = 2cm}
\geometry{bottom = 2cm}
\lstdefinestyle{mystyle}{
	basicstyle=\footnotesize,
	breakatwhitespace=false, 
	breaklines=true,
	captionpos=b,
	keepspaces=true, 
	numbers=left,
	numbersep=5pt,  % номера строк
	showspaces=false, % не показывать пробелы
	showstringspaces=false,
	showtabs=false,                  
	tabsize=2
}
\lstset{style=mystyle}
\lstset{basicstyle=\footnotesize\ttfamily,breaklines=true}
\setlength{\parindent}{15pt} % Устанавливаем отступ
\setlength{\parskip}{0pt}    % Убираем вертикальный отступ между абзацами
\begin{document}
	\begin{titlepage}
	\begin{center}
		\large
		{МИНИСТЕРСТВО НАУКИ И ВЫСШЕГО ОБРАЗОВАНИЯ\\ РОССИЙСКОЙ ФЕДЕРАЦИИ}
		
		Федеральное государственное автономное образовательное учреждение высшего образования
		\vspace{0.5cm}
		
		\textbf{<<Национальный исследовательский Нижегородский государственный университет им. Н.И. Лобачевского>>}\\
		(ННГУ)\\
		\vspace{1cm}
		\textbf{Институт информационных технологий, математики и механики}\\
		\vspace{1cm}
		
		\vfill

		
		\Large
		Отчёт по лабораторной работе на тему:
		
		\textbf{<<Построение выпуклой оболочки – проход Грэхема>>}
		{\LARGE 
		}
		\bigskip
		
		
	\end{center}
	\vfill
	
	
	\vfill
	
	\hfill\begin{minipage}{0.4\textwidth}
		\textbf{Выполнил:} 

            студент группы 3822Б1ФИ3 

            Кудряшова И.\,И. \bigskip
            
	\end{minipage}%
	
	\hfill\begin{minipage}{0.4\textwidth}
		\textbf{Преподаватель:}
		
    к.т.н, доцент каф. ВВСП, 
    
    Сысоев А.\,В.
	\end{minipage}%
	\vfill
	
	\begin{center}
		Нижний Новгород\\
		2024 г.
	\end{center}
\end{titlepage}
\thispagestyle{empty}
\tableofcontents
\newpage

\section*{Введение}
\addcontentsline{toc}{section}{Введение}

\hspace*{15pt}	Алгоритмичесткое нахождения выпуклой оболочки конечного множества точек на плоскости или в других низкоразмерных евклидовых пространствах является одной из  фундаментальных задач вычислительной геометрии. Это понятие широко используется в разных сферах, таких как компьютерная графика, робототехника и анализ данных. В современных направлениях, таких как машинное обучение и компьютерное зрение, выпуклая оболочка используется для кластеризации данных и определения границ объектов.

Одним из самых известных методов для построения выпуклой оболочки является алгоритм Грэхема, предложенный американским математиком Рональдом Грэхемом в 1972 году. 
Цель данной лабораторной работы заключается в изучении алгоритма построения выпуклой оболочки с использованием метода Грэхема, а также в реализации его последовательной и MPI версий. Также в рамках лабораторной работы будет проведен анализ производительности реализации алгоритма.

\newpage

\section{Постановка задачи}

\hspace*{15pt}Дано множество точек \( P = \{ p_1, p_2, \ldots, p_n \} \) в двумерном пространстве, где каждая точка \( p_i \) представлена в виде координат \( p_i = (x_i, y_i) \). Задача алгоритма Грэхема заключается в нахождении выпуклой оболочки \( G \) данного множества точек \( P \).

\begin{definition}\label{d1}
	\textit{Выпуклая оболочка $G$} — это минимальный выпуклый многоугольник, который содержит заданное множество точек $P$ в пространстве.
\end{definition}


\begin{definition}\label{d2}
    \textit{Выпуклый многоугольник} — это многоугольник, в котором все внутренние углы меньше $\ 180^\circ$.
\end{definition}

Отметим, что выпуклая оболочка может быть построена для любого набора точек, независимо от их расположения.

Алгоритм принимает на вход множество точек \( P \) в двумерном пространстве, представленное в виде списка координат: \( P = \{ p_1, p_2, \ldots, p_n \} \), где \( p_i = ( x_i, y_i ) \) для $i$ = $\overline{1...n}$, где \( n \) — число точек в множестве \( P \).

В результате выполнения алгоритм возвращает выпуклую оболочку \( G \), представленную в виде списка точек \( G = \{ g_1, g_2, \ldots, g_m \} \), где каждая точка \( g_k \) также представлена в виде координат \( g_k = (x_k, y_k) \) для $k$ = $\overline{1...m}$, где \( m \) — число точек в выпуклой оболочке \( G \).

\section{Описание алгоритма}
\hspace*{15pt}Алгоритм сканирования Грэхема — это простой и эффективный алгоритм для вычисления выпуклой оболочки выбранного набора точек. Он работает путём итеративного добавления точек к выпуклой оболочке до тех пор, пока все точки не будут добавлены. Далее рассмотрим этапы этого процесса подробнее.


\subsection*{Выбор начальной точки}
\hspace*{15pt}Найдём точку с наименьшей координатой \(y\). Если нашлось несколько точек с такой координатой, обратим внимание на координату \(x\) и выберем ту точку, у которой она будет меньше. Обозначим вычисленную точку \( p_0 \). Она будет опорной, с неё и начнём построение оболочки.


\subsection*{Сортировка точек}
\hspace*{15pt}Отсортируем все остальные точки, не включая \( p_0 \), по углу, образуемому с осью \( x \) и вектором, направленным от \( p_0 \) к каждой из точек. Если нашлось несколько точек, имеющих одинаковый угол, оставим ту, пасстояние от которой до опорной точки  \( p_0 \) меньше.


\subsection*{Построение выпуклой оболочки}
\hspace*{15pt}Создадим пустой список и добавим в него начальную точку \( p_0 \) и первую отсортированную точку. Далее будем добавлять точки в порядке, определенном на предыдущем шаге в результате сортировки, начиная с минимальной. Для каждой новой точки будем проверять, образует ли она выпуклую оболочку с двумя предыдущими точками с помощью левого поворота.

\subsection*{Получение результата}

\hspace*{15pt}Продолжим этот процесс до тех пор, пока не будут проанализированы все точки. После обработки всех точек в списке останутся вершины, составляющие выпуклую оболочку. Эти точки можно вывести в порядке обхода.


\subsection*{Временная сложность алгоритма}

\hspace*{15pt}Поиск самой нижней точки занимает \( O(n) \) времени. Сортировка точек занимает \( O(n \log n) \) времени. Далее каждый элемент добавляется и удаляется не более одного раза, это происходит за \( O(n) \). Далее операции производимые с каждой точкой в отдельности будут выполняться за \( O(n) \). Таким образом, общая сложность составляет \( O(n) + O(n \log n) + O(n) + O(n) \), что равно \( O(n \log n) \).

\section{Описание схемы распараллеливания}

\subsection*{Распределение входных данных}

\hspace*{15pt}Корневой процесс распределяет данные между остальными процессами. Каждый процесс получает свой сегмент данных, с которым и будет работать. Главный процесс передает каждому процессу минимум 3 точки, так как иначе построить выпуклую оболочку невозможно.

\subsection*{Вычисления на каждом процессе и сбор результатов}
\hspace*{15pt}Процессы выполняют последовательный алгоритм по поиску выпуклой оболочки на предоставленном им сегменте данных. Дочерние процессы передают результат на корневой процесс.

\subsection*{Вычисления на главном процессе}
\hspace*{15pt}Корневой процесс, получив данные, также выполняет последовательный проход Грэхема. Таким образом, мы получим выпуклую оболочку, которая будет ответом.

\section{Эксперименты}

\subsection*{Функциональные тесты}
\hspace*{15pt}В лабораторной работе были написаны функциональные тесты при помощи gtest. Последовательная и параллельная реализации алгоритма корректно обрабатывают ситуации, когда входные данные представляют собой вершины выпуклых фигур, выполняет тесты с большими объёмами данных, которые были сгенерированы случайным образом, а также грамотно отслеживает некорректные входные данные.

\subsection*{Тест производительности}

\hspace*{15pt}Далее проведём тесты производительности, чтобы отследить, как изменяется время работы алгорима в зависимости от числа процессов и разного объёма входных данных.



\begin{table}[h]

\begin{tabular}{|c|c|c|c|}
	\hline
	\textbf{Объём входных данных} & \textbf{Количество процессов} & \textbf{Вид теста} & \textbf{Время (мс)} \\
	\hline
	\multirow{8}{*}{10000} & \multirow{2}{*}{1} & pipeline\_run & 151 \\
	& & task\_run & 199 \\
	\cline{2-4}
	& \multirow{2}{*}{4} & pipeline\_run & 147 \\
	& & task\_run & 93  \\
	\cline{2-4}
	& \multirow{2}{*}{5} & pipeline\_run & 145 \\
	& & task\_run & 81 \\
	\cline{2-4}
	& \multirow{2}{*}{10} & pipeline\_run & 146 \\
	& & task\_run & 70  \\
	\hline
	\multirow{8}{*}{30000} & \multirow{2}{*}{1} & pipeline\_run & 495 \\
	& & task\_run & 657 \\
	\cline{2-4}
	& \multirow{2}{*}{4} & pipeline\_run & 498 \\
	& & task\_run & 304  \\
	\cline{2-4}
	& \multirow{2}{*}{5} & pipeline\_run & 485 \\
	& & task\_run & 260 \\
	\cline{2-4}
	& \multirow{2}{*}{10} & pipeline\_run & 488 \\
	& & task\_run & 222  \\
	\hline
	\multirow{8}{*}{60000} & \multirow{2}{*}{1} & pipeline\_run & 1026 \\
	& & task\_run & 1367 \\
	\cline{2-4}
	& \multirow{2}{*}{4} & pipeline\_run & 1053 \\
	& & task\_run & 667  \\
	\cline{2-4}
	& \multirow{2}{*}{5} & pipeline\_run & 1043 \\
	& & task\_run & 561 \\
	\cline{2-4}
	& \multirow{2}{*}{10} & pipeline\_run & 1038 \\
	& & task\_run & 481  \\
	\hline
\end{tabular}

\end{table}


\subsubsection*{Выводы}
\hspace*{15pt}В результате проведённых экспериментов можно сделать вывод, что с увеличением числа процессов существенно уменьшается время выполнения программы. Также можно заметить, что на большом объёме входных данных при увеличении числа процессов наблюдаемое ускорение больше, чем при небольших объёмах данных.

\newpage

\section*{Заключение}
\addcontentsline{toc}{section}{Заключение}
\hspace*{15pt}В процессе выполнения работы был изучен алгоритм Грэхема, который считается одним из самых известных и эффективных способов построения выпуклой оболочки для множества точек в двумерном пространстве. В рамках исследования были реализованы как последовательный, так и параллельный алгоритмы. Кроме того, мы подтвердили, что применение параллельного программирования с использованием MPI способствует увеличению скорости обработки задач с большими объемами входных данных. Это открывает новые возможности для оптимизации алгоритмов в задачах, требующих высокой вычислительной мощности, и может быть полезно в различных областях, таких как компьютерная графика и обработка данных.

\newpage

\section*{Список литературы}
\addcontentsline{toc}{section}{Список литературы}

\begin{enumerate}
	\item Левитин А. В. Глава 3. Метод грубой силы: Поиск выпуклой оболочки // Алгоритмы. Введение в разработку и анализ — М.: Вильямс, 2006. — С. 157. — 576 с. — ISBN 978-5-8459-0987-9
	\item Половинкин Е. С, Балашов М. В. Элементы выпуклого и сильно выпуклого анализа. — М.: Физматлит, 2004. — 416 с. — ISBN 5-9221-0499-3.
	\item Т. Кормен, Ч. Лейзерсон, Р. Ривест
	К66 Алгоритмы: построение и анализ / Пер. с англ, под ред.
	А. Шеня. — М.: МЦНМО, 2002. —960 с.: 263 ил.
	ISBN 5-900916-37-5

\end{enumerate}

\newpage

\section*{Приложение}
\addcontentsline{toc}{section}{Приложение}

\subsection*{mpi/kudryashova\_\_i\_\_graham's\_\_scan/include/Graham'sScanMPI.hpp}
\begin{lstlisting}[language=C++]
	#pragma once
	#include <algorithm>
	#include <boost/mpi/collectives.hpp>
	#include <boost/mpi/communicator.hpp>
	#include <random>
	
	#include "core/task/include/task.hpp"
	namespace kudryashova_i_graham_scan_mpi {
		class TestMPITaskSequential : public ppc::core::Task {
			public:
			std::vector<int8_t> runGrahamScan(std::vector<int8_t>& input_data);
			explicit TestMPITaskSequential(std::shared_ptr<ppc::core::TaskData> taskData_) : Task(std::move(taskData_)) {}
			bool pre_processing() override;
			bool validation() override;
			bool run() override;
			bool post_processing() override;
			
			private:
			std::vector<int8_t> input_data;
			std::vector<int> firstHalf, secondHalf;
			std::vector<int8_t> result_vec;
		};
		class TestMPITaskParallel : public ppc::core::Task {
			public:
			std::vector<int8_t> runGrahamScan(std::vector<int8_t>& input_data);
			explicit TestMPITaskParallel(std::shared_ptr<ppc::core::TaskData> taskData_) : Task(std::move(taskData_)) {}
			bool pre_processing() override;
			bool validation() override;
			bool run() override;
			bool post_processing() override;
			
			private:
			boost::mpi::communicator world;
			std::vector<int8_t> input_data;
			std::vector<int> firstHalf, secondHalf;
			std::vector<int> local_input1_, local_input2_;
			std::vector<int> segments;
			std::vector<int8_t> local_result;
			std::vector<int8_t> result_vec;
			int delta{};
			int processes{};
		};
	}  // namespace kudryashova_i_graham_scan_mpi
\end{lstlisting}

\subsection*{mpi/kudryashova\_\_i\_\_graham's\_\_scan/include/Graham'sScanMPI.cpp}
\begin{lstlisting}[language=C++]
#include "mpi/kudryashova_i_graham's_scan/include/Graham'sScanMPI.hpp"

#include <boost/mpi.hpp>
#include <boost/serialization/access.hpp>
#include <boost/serialization/vector.hpp>

namespace kudryashova_i_graham_scan_mpi {
	
	bool kudryashova_i_graham_scan_mpi::TestMPITaskSequential::pre_processing() {
		internal_order_test();
		input_data.resize(taskData->inputs_count[0]);
		if (taskData->inputs[0] == nullptr || taskData->inputs_count[0] == 0) {
			return false;
		}
		auto* tmp_ptr = reinterpret_cast<uint8_t*>(taskData->inputs[0]);
		std::copy(tmp_ptr, tmp_ptr + taskData->inputs_count[0], input_data.begin());
		return true;
	}
	
	bool kudryashova_i_graham_scan_mpi::TestMPITaskSequential::validation() {
		internal_order_test();
		return taskData->inputs_count[0] >= 6 && taskData->inputs_count[0] % 2 == 0;
	}
	
	bool isCounterClockwise(const std::pair<int8_t, int8_t>& p1, const std::pair<int8_t, int8_t>& p2,
	const std::pair<int8_t, int8_t>& p3) {
		return (p2.first - p1.first) * (p3.second - p1.second) > (p2.second - p1.second) * (p3.first - p1.first);
	}
	
	void sortPoints(std::vector<int8_t>& points) {
		int n = points.size() / 2;
		std::pair<int8_t, int8_t> p0(points[0], points[n]);
		std::vector<std::pair<int8_t, int8_t>> pointList;
		pointList.reserve(points.size());
		for (int i = 0; i < n; ++i) {
			pointList.emplace_back(points[i], points[n + i]);
		}
		std::sort(pointList.begin(), pointList.end(),
		[&p0](const std::pair<int8_t, int8_t>& a, const std::pair<int8_t, int8_t>& b) {
			if (atan2(a.second - p0.second, a.first - p0.first) == atan2(b.second - p0.second, b.first - p0.first)) {
				return ((a.first - p0.first) * (a.first - p0.first) + (a.second - p0.second) * (a.second - p0.second)) <
				((b.first - p0.first) * (b.first - p0.first) + (b.second - p0.second) * (b.second - p0.second));
			}
			return atan2(a.second - p0.second, a.first - p0.first) < atan2(b.second - p0.second, b.first - p0.first);
		});
		for (int i = 0; i < n; ++i) {
			points[i] = pointList[i].first;
			points[n + i] = pointList[i].second;
		}
	}
	
	std::vector<int8_t> kudryashova_i_graham_scan_mpi::TestMPITaskSequential::runGrahamScan(
	std::vector<int8_t>& Graham_input_data) {
		std::vector<int8_t> hull;
		std::vector<int8_t> points = Graham_input_data;
		int n = points.size() / 2;
		int min_y_index = 0;
		for (int i = 1; i < n; ++i) {
			if (points[n + i] < points[n + min_y_index] ||
			(points[n + i] == points[n + min_y_index] && points[i] < points[min_y_index])) {
				min_y_index = i;
			}
		}
		std::swap(points[0], points[min_y_index]);
		std::swap(points[n], points[min_y_index + n]);
		std::vector<int> indices(n);
		for (int i = 1; i < n; ++i) indices[i] = i;
		sortPoints(points);
		hull.push_back(points[0]);
		hull.push_back(points[n]);
		for (int i = 1; i < n; ++i) {
			int index = indices[i];
			while (hull.size() >= 4 &&
			!isCounterClockwise({hull[hull.size() - 4], hull[hull.size() - 3]},
			{hull[hull.size() - 2], hull[hull.size() - 1]}, {points[index], points[n + index]})) {
				hull.pop_back();
				hull.pop_back();
			}
			if (!isCounterClockwise({hull[hull.size() - 2], hull[hull.size() - 1]}, {points[index], points[n + index]},
			{points[index], points[n + index]})) {
				hull.push_back(points[index]);
				hull.push_back(points[n + index]);
			}
		}
		result_vec.resize(hull.size());
		std::copy(hull.begin(), hull.end(), result_vec.data());
		return result_vec;
	}
	
	bool kudryashova_i_graham_scan_mpi::TestMPITaskSequential::run() {
		internal_order_test();
		runGrahamScan(input_data);
		return true;
	}
	
	bool kudryashova_i_graham_scan_mpi::TestMPITaskSequential::post_processing() {
		internal_order_test();
		auto* outputData = reinterpret_cast<int8_t*>(taskData->outputs[0]);
		std::copy(result_vec.begin(), result_vec.end(), outputData);
		return true;
	}
	
	bool kudryashova_i_graham_scan_mpi::TestMPITaskParallel::pre_processing() {
		internal_order_test();
		int remainder = 0;
		processes = world.size();
		int counting_proc = world.size() - 1;
		if (world.rank() == 0) {
			if (processes == 1 || (int)(taskData->inputs_count[0]) < processes) {
				delta = (taskData->inputs_count[0]) / 2;
			} else {
				delta = (taskData->inputs_count[0] / 2) / counting_proc;
				remainder = (taskData->inputs_count[0] / 2) % counting_proc;
			}
		}
		segments.resize(processes);
		if (world.rank() == 0) {
			segments[0] = delta;
			for (int i = 1; i < processes; ++i) {
				segments[i] = delta + (i <= remainder ? 1 : 0);
			}
		}
		if (world.rank() == 0) {
			input_data.resize(taskData->inputs_count[0]);
			if (taskData->inputs[0] == nullptr || taskData->inputs_count[0] == 0) {
				return false;
			}
			auto* source_ptr = reinterpret_cast<uint8_t*>(taskData->inputs[0]);
			std::copy(source_ptr, source_ptr + taskData->inputs_count[0], input_data.begin());
			sortPoints(input_data);
		}
		return true;
	}
	
	bool kudryashova_i_graham_scan_mpi::TestMPITaskParallel::validation() {
		internal_order_test();
		if (world.rank() == 0) {
			return taskData->inputs_count[0] >= 6 && taskData->inputs_count[0] % 2 == 0;
		}
		return true;
	}
	
	std::vector<int8_t> rearrangeAndSort(const std::vector<int8_t>& input) {
		std::vector<int8_t> x_values;
		std::vector<int8_t> y_values;
		for (unsigned long i = 0; i < input.size(); i += 2) {
			if (i < input.size()) {
				x_values.push_back(input[i]);
			}
			if (i + 1 < input.size()) {
				y_values.push_back(input[i + 1]);
			}
		}
		std::vector<int8_t> result;
		result.reserve(x_values.size() + y_values.size());
		result.insert(result.end(), x_values.begin(), x_values.end());
		result.insert(result.end(), y_values.begin(), y_values.end());
		return result;
	}
	
	std::vector<int8_t> kudryashova_i_graham_scan_mpi::TestMPITaskParallel::runGrahamScan(
	std::vector<int8_t>& Graham_input_data) {
		std::vector<int8_t> hull;
		std::vector<int8_t> points = Graham_input_data;
		int n = points.size() / 2;
		int min_y_index = 0;
		for (int i = 1; i < n; ++i) {
			if (points[n + i] < points[n + min_y_index] ||
			(points[n + i] == points[n + min_y_index] && points[i] < points[min_y_index])) {
				min_y_index = i;
			}
		}
		std::swap(points[0], points[min_y_index]);
		std::swap(points[n], points[min_y_index + n]);
		std::vector<int> indices(n);
		for (int i = 1; i < n; ++i) indices[i] = i;
		sortPoints(points);
		hull.push_back(points[0]);
		hull.push_back(points[n]);
		for (int i = 1; i < n; ++i) {
			int index = indices[i];
			while (hull.size() >= 4 &&
			!isCounterClockwise({hull[hull.size() - 4], hull[hull.size() - 3]},
			{hull[hull.size() - 2], hull[hull.size() - 1]}, {points[index], points[n + index]})) {
				hull.pop_back();
				hull.pop_back();
			}
			if (!isCounterClockwise({hull[hull.size() - 2], hull[hull.size() - 1]}, {points[index], points[n + index]},
			{points[index], points[n + index]})) {
				hull.push_back(points[index]);
				hull.push_back(points[n + index]);
			}
		}
		result_vec.resize(hull.size());
		std::copy(hull.begin(), hull.end(), result_vec.data());
		return result_vec;
	}
	
	bool kudryashova_i_graham_scan_mpi::TestMPITaskParallel::run() {
		internal_order_test();
		broadcast(world, segments.data(), processes, 0);
		if (world.rank() == 0) {
			size_t count = taskData->inputs_count[0];
			size_t halfSize = count / 2;
			firstHalf.resize(halfSize);
			secondHalf.resize(count - halfSize);
			std::copy(input_data.begin(), input_data.begin() + halfSize, firstHalf.begin());
			std::copy(input_data.begin() + halfSize, input_data.begin() + count, secondHalf.begin());
			int pointer = 0;
			for (int proc = 1; proc < world.size(); ++proc) {
				int proc_segment = segments[proc];
				world.send(proc, 0, firstHalf.data() + pointer, proc_segment);
				world.send(proc, 1, secondHalf.data() + pointer, proc_segment);
				pointer += proc_segment;
			}
		}
		if (world.rank() != 0) {
			local_input1_.resize(segments[world.rank()]);
			local_input2_.resize(segments[world.rank()]);
			world.recv(0, 0, local_input1_.data(), segments[world.rank()]);
			world.recv(0, 1, local_input2_.data(), segments[world.rank()]);
			std::vector<int8_t> local_GrahamScan_data;
			local_GrahamScan_data.reserve(local_input1_.size() + local_input2_.size());
			local_GrahamScan_data.insert(local_GrahamScan_data.end(), local_input1_.begin(), local_input1_.end());
			local_GrahamScan_data.insert(local_GrahamScan_data.end(), local_input2_.begin(), local_input2_.end());
			local_result = runGrahamScan(local_GrahamScan_data);
		}
		size_t local_size = local_result.size();
		std::vector<size_t> sizes(world.size());
		world.barrier();
		gather(world, local_size, sizes, 0);
		std::vector<int8_t> full_results;
		std::vector<size_t> displacements(world.size(), 0);
		if (world.rank() == 0) {
			size_t total_size = std::accumulate(sizes.begin(), sizes.end(), 0);
			full_results.resize(total_size);
			displacements[0] = 0;
			for (int i = 1; i < world.size(); ++i) {
				displacements[i] = displacements[i - 1] + sizes[i - 1];
			}
			if ((int)(taskData->inputs_count[0]) < world.size() || (world.size() == 1)) {
				runGrahamScan(input_data);
				return true;
			}
		} else {
			world.send(0, 0, local_result.data(), local_size);
		}
		if (world.rank() == 0) {
			for (int i = 1; i < world.size(); ++i) {
				world.recv(i, 0, full_results.data() + displacements[i], sizes[i]);
			}
			std::copy(local_result.begin(), local_result.end(), full_results.data() + displacements[world.rank()]);
			std::vector<int8_t> full_results_sort = rearrangeAndSort(full_results);
			runGrahamScan(full_results_sort);
		}
		return true;
	}
	
	bool kudryashova_i_graham_scan_mpi::TestMPITaskParallel::post_processing() {
		internal_order_test();
		if (world.rank() == 0) {
			if (!taskData->outputs.empty()) {
				auto* outputData = reinterpret_cast<int8_t*>(taskData->outputs[0]);
				std::copy(result_vec.begin(), result_vec.end(), outputData);
			} else {
				return false;
			}
		}
		return true;
	}
}  // namespace kudryashova_i_graham_scan_mpi
	}  // namespace kudryashova_i_graham_scan_mpi
\end{lstlisting}
\end{document}