\documentclass[12pt]{article}
\usepackage[utf8]{inputenc}
\usepackage[russian]{babel}
\usepackage[T1]{fontenc}
\usepackage{amsmath}
\usepackage{geometry}
\usepackage{titlesec}
\usepackage{hyperref}
\usepackage{tocloft}
\usepackage{xcolor} 
\usepackage{listings}
\usepackage{float}
\usepackage{multirow}

\geometry{a4paper, left=25mm, right=15mm, top=20mm, bottom=20mm}

\hypersetup{
    colorlinks=true,
    linkcolor=black,
    urlcolor=black,
    pdftitle={Отчет по проекту},
    pdfauthor={Зайцев Артём},
    pdfsubject={Построение выпуклой оболочки. Проход Джарвиса}
}

\lstset{
  language=C++,
  basicstyle=\ttfamily\small,
  keywordstyle=\bfseries\color{blue},
  commentstyle=\itshape\color{green!50!black},
  stringstyle=\color{red!60!black},
  numbers=left,
  numberstyle=\tiny,
  stepnumber=1,
  numbersep=8pt,
  backgroundcolor=\color{gray!10},
  showspaces=false,
  showstringspaces=false,
  breaklines=true,
  frame=single,
  tabsize=2,
  captionpos=b
}


\begin{document}

\begin{titlepage}
\centering
    \large
    Министерство науки и высшего образования Российской Федерации\\[0.5cm]
    Федеральное государственное автономное образовательное учреждение высшего образования\\[0.5cm]
    \textbf{<<Национальный исследовательский Нижегородский государственный университет им. Н.И. Лобачевского>>}\\
    (ННГУ)\\[1cm]
    Институт информационных технологий, математики и механики\\
\vspace*{\fill}
{\Huge \textbf{Отчет по лабораторной работе}}

\vspace{0.5cm}
{\Huge \textbf{Выделение рёбер на изображении с использованием оператора Собеля}}

\vspace*{\fill}

{\large Нижний Новгород\\ 2024}
\end{titlepage}

\newpage
\section*{Введение}
В данной работе рассматривается задача выделения рёбер на изображении с использованием оператора Собеля. Этот оператор является базовым инструментом для выделения границ на изображениях, так как позволяет определять резкие изменения интенсивности пикселей в горизонтальном и вертикальном направлениях. Для увеличения производительности обработка изображений была распараллелена с использованием MPI.

\newpage
\section*{Постановка задачи}
Необходимо реализовать последовательную и параллельную версии алгоритма выделения рёбер с использованием оператора Собеля. Цель работы — сравнение производительности и корректности обеих версий. Основные этапы работы включают:
\begin{itemize}
    \item Разработка последовательной версии оператора Собеля.
    \item Распараллеливание вычислений с помощью библиотеки MPI.
    \item Проведение экспериментов для оценки ускорения параллельной версии.
\end{itemize}

\newpage
\section*{Описание алгоритма}
Оператор Собеля использует два свёрточных ядра для вычисления градиентов:
\begin{equation}
\text{Sobel Kernel X} = \begin{bmatrix}
-1 & 0 & 1 \\
-2 & 0 & 2 \\
-1 & 0 & 1
\end{bmatrix}, \quad
\text{Sobel Kernel Y} = \begin{bmatrix}
-1 & -2 & -1 \\
 0 &  0 &  0 \\
 1 &  2 &  1
\end{bmatrix}.
\end{equation}

Для каждого пикселя изображения вычисляется градиент:
\begin{equation}
G = \sqrt{G_x^2 + G_y^2},
\end{equation}
где $G_x$ и $G_y$ — результаты свёртки с ядрами X и Y соответственно.

\newpage
\section*{Описание схемы распараллеливания}
Для повышения производительности была реализована параллельная версия алгоритма с использованием MPI. Основная идея распараллеливания:
\begin{itemize}
    \item Изображение делится на части (по строкам) между процессами.
    \item Каждый процесс обрабатывает свою часть изображения.
    \item Результаты объединяются в финальное изображение с помощью MPI.
\end{itemize}
Синхронизация между процессами выполняется с помощью функций \texttt{scatterv} и \texttt{gatherv}.

\newpage
\section*{MPI-версия - описание программной реализации}
Параллельная версия алгоритма состоит из следующих этапов:
\begin{enumerate}
    \item \textbf{Инициализация:} Главный процесс читает изображение, делит его на части и отправляет другим процессам.
    \item \textbf{Обработка:} Каждый процесс выполняет свёртку с ядрами Собеля для своей части изображения.
    \item \textbf{Сбор результатов:} Части обработанного изображения собираются главным процессом.
\end{enumerate}
Пример фрагмента кода распределения данных:
\begin{lstlisting}[language=C++,caption={Распределение данных между процессами}]
boost::mpi::scatterv(world, image, sendcounts, senddispls, locimg.data(), locimg.size(), 0);
\end{lstlisting}

\newpage
\section*{Результаты экспериментов}
Эксперименты проводились на изображении размером $1024 \times 1024$ пикселя. Были измерены следующие параметры:
\begin{itemize}
    \item Время выполнения последовательной версии.
    \item Время выполнения параллельной версии на разном количестве процессов.
    \item Ускорение параллельной версии.
\end{itemize}

Результаты представлены в таблице:
\begin{table}[H]
\begin{tabular}{|l|l|ll|l|}
\hline
\multirow{2}{*}{N}      & \multirow{2}{*}{Seq}       & \multicolumn{2}{l|}{MPI}          & \multirow{2}{*}{$\frac{\text{MPI}}{\text{Seq}}$} \\ \cline{3-4}
                        &                            & \multicolumn{1}{l|}{n} & time     &                           \\ \hline
\multirow{2}{*}{$10^4$} & \multirow{2}{*}{0.005021}  & \multicolumn{1}{l|}{2} & 0.004227 & 0.841864                   \\ \cline{3-5} 
                        &                            & \multicolumn{1}{l|}{4} & 0.003992 & 0.795060                 \\ \hline
\multirow{2}{*}{$10^5$} & \multirow{2}{*}{0.055782}  & \multicolumn{1}{l|}{2} & 0.047529 & 0.852049                 \\ \cline{3-5} 
                        &                            & \multicolumn{1}{l|}{4} & 0.046559 & 0.834660                  \\ \hline
\multirow{2}{*}{$10^6$} & \multirow{2}{*}{0.599623}  & \multicolumn{1}{l|}{2} & 0.498934 & 0,832079                  \\ \cline{3-5} 
                        &                            & \multicolumn{1}{l|}{4} & 0.485892 & 0,810329                  \\ \hline
\end{tabular}
\end{table}


\newpage
\section*{Выводы из результатов}
Параллельная версия алгоритма оператора Собеля показала значительное ускорение при увеличении числа процессов. Однако эффективность параллелизации снижается из-за накладных расходов на передачу данных между процессами. Для больших изображений параллельная версия становится более выгодной.

\newpage
\section*{Заключение}
В работе был реализован алгоритм выделения рёбер на изображении с использованием оператора Собеля. Реализация включала последовательную и параллельную версии. Эксперименты подтвердили корректность работы обеих версий и показали преимущество параллельной реализации для больших изображений.

\newpage
\section*{Литература}
\begin{enumerate}
    \item Gonzalez, R. C., Woods, R. E. \textit{Digital Image Processing}. Pearson, 2018.
    \item Boost MPI Documentation: \texttt{https://www.boost.org/doc/libs/release/doc/html/mpi.html}
\end{enumerate}

\newpage
\section*{Приложение}
\subsection{ops\_mpi.hpp}
\begin{lstlisting}
#pragma once

#include <array>
#include <boost/mpi/collectives.hpp>
#include <boost/mpi/communicator.hpp>
#include <vector>

#include "core/task/include/task.hpp"

namespace koshkin_m_sobel_mpi {

// clang-format off
static inline const std::array<std::array<int8_t, 3>, 3> SOBEL_KERNEL_X = {{
  {{-1, 0, 1}},
  {{-2, 0, 2}},
  {{-1, 0, 1}}
}};
static inline const std::array<std::array<int8_t, 3>, 3> SOBEL_KERNEL_Y = {{
  {{-1, -2, -1}},
  {{ 0,  0,  0}},
  {{ 1,  2,  1}}
}};
// clang-format on

class TestTaskSequential : public ppc::core::Task {
 public:
  explicit TestTaskSequential(std::shared_ptr<ppc::core::TaskData> taskData_) : Task(std::move(taskData_)) {}
  bool pre_processing() override;
  bool validation() override;
  bool run() override;
  bool post_processing() override;

 private:
  std::pair<size_t, size_t> imgsize;
  std::vector<uint8_t> image;
  std::vector<uint8_t> resimg;
};

class TestTaskParallel : public ppc::core::Task {
 public:
  explicit TestTaskParallel(std::shared_ptr<ppc::core::TaskData> taskData_) : Task(std::move(taskData_)) {}
  bool pre_processing() override;
  bool validation() override;
  bool run() override;
  bool post_processing() override;

 private:
  std::pair<size_t, size_t> imgsize;
  std::vector<uint8_t> image;
  std::vector<uint8_t> resimg;

  boost::mpi::communicator world;
};

}  // namespace koshkin_m_sobel_mpi
\end{lstlisting}
\subsection{ops\_mpi.cpp}
\begin{lstlisting}
#include "../include/ops_mpi.hpp"

#include <algorithm>
#include <boost/serialization/utility.hpp>
#include <climits>
#include <cmath>
#include <cstdint>
#include <cstdio>
#include <utility>

int16_t conv_kernel3(const std::array<std::array<int8_t, 3>, 3>& kernel, const std::vector<uint8_t>& img, size_t i,
                     size_t j, size_t height) {
  // clang-format off
  return kernel[0][0] * img[(i - 1) * height + (j - 1)]
       + kernel[0][1] * img[(i - 1) * height + j]
       + kernel[0][2] * img[(i - 1) * height + (j + 1)]
       + kernel[1][0] * img[i * height + (j - 1)]
       + kernel[1][1] * img[i * height + j]
       + kernel[1][2] * img[i * height + (j + 1)]
       + kernel[2][0] * img[(i + 1) * height + (j - 1)]
       + kernel[2][1] * img[(i + 1) * height + j]
       + kernel[2][2] * img[(i + 1) * height + (j + 1)];
  // clang-format on
}

bool koshkin_m_sobel_mpi::TestTaskSequential::pre_processing() {
  internal_order_test();

  imgsize = {taskData->inputs_count[0], taskData->inputs_count[1]};
  auto& [width, height] = imgsize;
  const int padding = 2;
  image.resize((width + padding) * (height + padding));
  resimg.resize(width * height, 0);

  const auto* in = reinterpret_cast<uint8_t*>(taskData->inputs[0]);
  for (size_t row = 0; row < height; row++) {
    std::copy(in + (row * width), in + ((row + 1) * width), image.begin() + ((row + 1) * (width + padding) + 1));
  }

  return true;
}

bool koshkin_m_sobel_mpi::TestTaskSequential::validation() {
  internal_order_test();
  return taskData->inputs_count[0] > 0 && taskData->inputs_count[1] > 0 &&
         taskData->outputs_count[0] == taskData->inputs_count[0] &&
         taskData->outputs_count[1] == taskData->inputs_count[1];
}

bool koshkin_m_sobel_mpi::TestTaskSequential::run() {
  internal_order_test();

  const auto [width, height] = imgsize;

  for (size_t i = 1; i < (width + 2) - 1; i++) {
    for (size_t j = 1; j < (height + 2) - 1; j++) {
      const auto accX = conv_kernel3(SOBEL_KERNEL_X, image, i, j, (height + 2));
      const auto accY = conv_kernel3(SOBEL_KERNEL_Y, image, i, j, (height + 2));
      resimg[(i - 1) * height + (j - 1)] = std::clamp(std::sqrt(std::pow(accX, 2) + std::pow(accY, 2)), 0., 255.);
    }
  }

  return true;
}

bool koshkin_m_sobel_mpi::TestTaskSequential::post_processing() {
  internal_order_test();
  std::copy(resimg.begin(), resimg.end(), reinterpret_cast<uint8_t*>(taskData->outputs[0]));
  return true;
}

bool koshkin_m_sobel_mpi::TestTaskParallel::pre_processing() {
  internal_order_test();

  if (world.rank() == 0) {
    imgsize = {taskData->inputs_count[0], taskData->inputs_count[1]};
    auto& [width, height] = imgsize;

    image.resize((width + 2) * (height + 2));
    resimg.resize(width * height, 0);

    const auto* in = reinterpret_cast<uint8_t*>(taskData->inputs[0]);
    for (size_t row = 0; row < height; row++) {
      std::copy(in + (row * width), in + ((row + 1) * width), image.begin() + ((row + 1) * (width + 2) + 1));
    }
  }

  return true;
}

bool koshkin_m_sobel_mpi::TestTaskParallel::validation() {
  internal_order_test();
  return world.rank() != 0 || (taskData->inputs_count[0] > 0 && taskData->inputs_count[1] > 0 &&
                               taskData->outputs_count[0] == taskData->inputs_count[0] &&
                               taskData->outputs_count[1] == taskData->inputs_count[1]);
}

bool koshkin_m_sobel_mpi::TestTaskParallel::run() {
  internal_order_test();

  const int pad = 2;

  boost::mpi::broadcast(world, imgsize, 0);
  const auto [width, height] = imgsize;
  const auto [pwidth, pheight] = std::make_pair(width + pad, height + pad);

  const int perproc = width / world.size();
  const int leftover = width % world.size();

  std::vector<int> sendcounts(world.size(), 0);
  std::vector<int> senddispls(world.size(), 0);
  std::vector<int> recvcounts(world.size(), 0);
  std::vector<int> recvdispls(world.size(), 0);

  for (int i = 0; i < world.size(); i++) {
    const int extra = i < leftover ? 1 : 0;
    sendcounts[i] = (perproc + extra + pad) * pheight;
    recvcounts[i] = (perproc + extra) * height;
  }
  for (int i = 1; i < world.size(); i++) {
    senddispls[i] = senddispls[i - 1] + sendcounts[i - 1] - pheight * pad;
    recvdispls[i] = recvdispls[i - 1] + recvcounts[i - 1];
  }

  std::vector<uint8_t> locimg(sendcounts[world.rank()]);
  boost::mpi::scatterv(world, image, sendcounts, senddispls, locimg.data(), locimg.size(), 0);

  const int actw = sendcounts[world.rank()] / pheight;
  std::vector<uint8_t> locres(recvcounts[world.rank()]);
  for (int i = 1; i < actw - 1; i++) {
    for (size_t j = 1; j < (height + 2) - 1; j++) {
      const auto accX = conv_kernel3(SOBEL_KERNEL_X, locimg, i, j, (height + 2));
      const auto accY = conv_kernel3(SOBEL_KERNEL_Y, locimg, i, j, (height + 2));
      locres[(i - 1) * height + (j - 1)] = std::clamp(std::sqrt(std::pow(accX, 2) + std::pow(accY, 2)), 0., 255.);
    }
  }

  boost::mpi::gatherv(world, locres, resimg.data(), recvcounts, recvdispls, 0);

  return true;
}

bool koshkin_m_sobel_mpi::TestTaskParallel::post_processing() {
  internal_order_test();
  if (world.rank() == 0) {
    std::copy(resimg.begin(), resimg.end(), reinterpret_cast<uint8_t*>(taskData->outputs[0]));
  }
  return true;
}
\end{lstlisting}

\end{document}
