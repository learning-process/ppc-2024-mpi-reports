\documentclass[a4paper, 14pt]{extarticle}

% Поддержка языков
\usepackage[english, russian]{babel} 

% Настройка кодировок
%\usepackage[utf8]{inputenc}
%\usepackage[T2A]{fontenc}

% Настройка шрифтов
\usepackage{fontspec}
\setmainfont[Ligatures=TeX]{Times New Roman} % Шрифт для основного текста документа
\setsansfont[Ligatures=TeX]{Arial}
%\setmonofont{Consolas} % Шрифт для кода

% Настройка отступов от краев страницы
\usepackage[left=3cm, right=1.5cm, top=2cm, bottom=2cm]{geometry}

\usepackage{titleps} % Колонтитулы
\usepackage{subfig} % Для подписей к рисункам и таблицам
\usepackage{graphicx} % для вставки картинок
\graphicspath{{./img/}} % Путь до папки с изображениями
% Пакет для отрисовки графиков
\usepackage{tikz}
\usetikzlibrary{arrows,positioning,shadows}
\usepackage{stmaryrd} % Стрелки в формулах
\usepackage{indentfirst} % Красная строка после заголовка
\usepackage{hhline} % Улучшенные горизонтальные линии в таблицах
\usepackage{multirow} % Ячейки в несколько строчек в таблицах
\usepackage{longtable} % Многостраничные таблицы
\usepackage{paralist,array} % Список внутри таблицы
\usepackage[normalem]{ulem}  % Зачеркнутый текст
\usepackage{upgreek, tipa} % Красивые греческие буквы
\usepackage{amsmath, amsfonts, amssymb, amsthm, mathtools} % ams пакеты для математики, табуляции
\usepackage{nicematrix} % Особые матрицы pNiceArray

\linespread{1.5} % Межстрочный интервал
\setlength{\parindent}{1.25cm} % Табуляция
\setlength{\parskip}{0cm}

% Пакет для красивого выделения кода
\usepackage{minted}
\setminted{fontsize=\footnotesize}

% Добавляем гипертекстовое оглавление в PDF
\usepackage[
bookmarks=true, colorlinks=true, unicode=true,
urlcolor=black,linkcolor=black, anchorcolor=black,
citecolor=black, menucolor=black, filecolor=black,
]{hyperref}

% Убрать переносы слов
\tolerance=1
\emergencystretch=\maxdimen
\hyphenpenalty=10000
\hbadness=10000

\newpagestyle{main}{
	% Верхний колонтитул
	\setheadrule{0cm} % Размер линии отделяющей колонтитул от страницы
	\sethead{}{}{} % Содержание {слева}{по центру}{справа}
	% Нижний колонтитул
	\setfootrule{0cm} % Размер линии отделяющей колонтитул от страницы
	\setfoot{}{}{\thepage} % Содержание {слева}{по центру}{справа}
}
\pagestyle{main}

% Выделение кода
\usepackage{listings}
\usepackage{xcolor} % для цветовой подсветки синтаксиса (не обязательно)
\lstset{
    language=C++, % язык программирования
    basicstyle=\ttfamily\small, % шрифт и размер текста
    keywordstyle=\color{blue}, % стиль для ключевых слов
    commentstyle=\color{green}, % стиль для комментариев
    stringstyle=\color{red}, % стиль для строк
    breaklines=true, % перенос строк
    numbers=left, % нумерация строк
    numberstyle=\tiny\color{gray}, % стиль для номеров строк
    frame=single, % рамка вокруг кода
    captionpos=b, % подпись снизу
    inputencoding=utf8, % кодировка ввода
}

% Гипперссылки
\usepackage{hyperref} % для автоматических ссылок
\addto\captionsrussian{
    \def\tablename{Таблица}
    \def\figurename{Рисунок}
    \def\refname{Список литературы}
}

\input{preamble/customCommand.tex}