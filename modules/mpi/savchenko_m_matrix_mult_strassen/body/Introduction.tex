Цель данной лабораторной работы заключается в изучении и реализации алгоритма Штрассена для умножения плотных матриц, а также исследовании его производительности в двух версиях: последовательной и параллельной. 

Алгоритм Штрассена представляет собой улучшенный подход к умножению матриц, который снижает вычислительную сложность по сравнению с классическим методом. Это достигается за счет уменьшения количества скалярных умножений за счет рекурсивного разложения матриц. 

Важной частью работы является создание параллельной версии алгоритма с использованием технологии MPI (Message Passing Interface), которая позволяет эффективно распределять вычисления между несколькими процессами. Это особенно актуально при работе с большими матрицами, где последовательные алгоритмы становятся узким местом.

В ходе лабораторной работы:
\begin{itemize}
    \item Реализованы последовательная и параллельная версии алгоритма.
    \item Проведено тестирование обеих реализаций для оценки их производительности.
    \item Проанализирована эффективность параллельной реализации.
\end{itemize}

Далее будет представлена теоретическая основа алгоритма Штрассена, особенности его реализации и результаты экспериментов.
