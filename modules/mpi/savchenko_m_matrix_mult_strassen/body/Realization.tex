\subsection{Последовательная реализация}
В последовательной версии алгоритма Штрассена реализовано рекурсивное разложение входных матриц, их подматриц, а также вычисление семи промежуточных матриц \(M_1, M_2, \dots, M_7\). Если размер матрицы на текущем уровне рекурсии становится меньше или равен пороговому значению (\(4 \times 4\) в данной реализации), используется классический алгоритм умножения.

Особенности реализации:
\begin{itemize}
    \item Функция проверяет, является ли размер матрицы степенью двойки. Если нет, то матрица заполняется нулями до ближайшего размера.
    \item Матрицы разбиваются на подматрицы \(A_{11}, A_{12}, \dots, B_{22}\).
    \item Рекурсивно вычисляются \(M_1, M_2, \dots, M_7\).
    \item Итоговая матрица \(C\) собирается из блоков \(C_{11}, C_{12}, C_{21}, C_{22}\).
\end{itemize}

\subsection{Параллельная реализация}
Для параллельной реализации используется библиотека MPI. Вычисления промежуточных матриц \(M_1, M_2, \dots, M_7\) распределяются между процессами.

Особенности параллельной реализации:
\begin{itemize}
    \item Главный процесс (с рангом 0) выполняет разбиение матриц и отправляет данные другим процессам.
    \item Вычисление каждой из матриц \(M_1\)–\(M_7\) выполняется параллельно.
    \item Результаты собираются обратно в главный процесс с использованием MPI-функции \texttt{reduce}.
\end{itemize}
