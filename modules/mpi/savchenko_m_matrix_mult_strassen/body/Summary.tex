\begin{enumerate}
\item \textbf{Эффективность алгоритма Штрассена:}  \\
   Реализованный алгоритм показал свою эффективность при умножении плотных матриц большого размера. Наиболее значительный прирост производительности наблюдается при увеличении числа процессов для матриц размером \(512 \times 512\) и \(1024 \times 1024\).

\item \textbf{Влияние кратности размерности матриц:} \\ 
   Алгоритм Штрассена оптимален для матриц, размерность которых кратна степени двойки. Некратные размерности приводят к необходимости дополнения матриц до ближайшего подходящего размера, что увеличивает объем вычислений. Например, умножение матриц \(600 \times 600\) фактически сводится к вычислениям для матриц \(1024 \times 1024\).

\item \textbf{Параллелизация алгоритма:}  \\
   Использование MPI для параллельной версии алгоритма позволило значительно сократить время выполнения программы, особенно для больших матриц.  
   \begin{itemize}
       \item При размере матрицы \(1024 \times 1024\) переход с \(1\) на \(7\) процессов уменьшил время вычислений почти в \(3\) раза.
       \item Однако на малых матрицах (\(128 \times 128\)) выигрыш от параллелизации незначителен из-за накладных расходов на коммуникацию между процессами.
   \end{itemize}

\item \textbf{Ограничения параллельной версии:}  \\
   При увеличении числа процессов эффективность роста производительности начинает снижаться. Это связано с:
   \begin{itemize}
       \item Ростом накладных расходов на синхронизацию данных.
       \item Ограничениями алгоритма, которые не позволяют равномерно распределить задачи между всеми процессами при малых размерах матриц.
   \end{itemize}
\end{enumerate}
