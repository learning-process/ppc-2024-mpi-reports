\subsection{Описание алгоритма Штрассена}

Алгоритм Штрассена является одной из самых известных техник ускорения умножения квадратных матриц. В отличие от классического метода с вычислительной сложностью \(O(n^3)\), алгоритм Штрассена уменьшает сложность до порядка \(O(n^{\log_2 7}) \approx O(n^{2.81})\). Это достигается за счет сокращения количества скалярных умножений через рекурсивное разложение.

Алгоритм делит исходные матрицы \(A\) и \(B\) размером \(n \times n\) на подматрицы размером \(n/2 \times n/2\):
\[
A = \begin{bmatrix}
A_{11} & A_{12} \\
A_{21} & A_{22}
\end{bmatrix}, \quad
B = \begin{bmatrix}
B_{11} & B_{12} \\
B_{21} & B_{22}
\end{bmatrix}.
\]

Для вычисления произведения \(C = A \cdot B\) требуется вычислить семь матриц \(M_1, M_2, \dots, M_7\) вместо восьми в классическом методе:
\[
\begin{aligned}
M_1 &= (A_{11} + A_{22})(B_{11} + B_{22}), \\
M_2 &= (A_{21} + A_{22})B_{11}, \\
M_3 &= A_{11}(B_{12} - B_{22}), \\
M_4 &= A_{22}(B_{21} - B_{11}), \\
M_5 &= (A_{11} + A_{12})B_{22}, \\
M_6 &= (A_{21} - A_{11})(B_{11} + B_{12}), \\
M_7 &= (A_{12} - A_{22})(B_{21} + B_{22}).
\end{aligned}
\]

Итоговая матрица \(C\) собирается следующим образом:
\[
\begin{aligned}
C_{11} &= M_1 + M_4 - M_5 + M_7, \\
C_{12} &= M_3 + M_5, \\
C_{21} &= M_2 + M_4, \\
C_{22} &= M_1 - M_2 + M_3 + M_6.
\end{aligned}
\]

\subsection{Сравнение с классическим методом}
Классический алгоритм умножения матриц требует \(n^3\) операций умножения, в то время как алгоритм Штрассена за счет уменьшения количества скалярных умножений позволяет снизить вычислительную сложность. Однако это достигается ценой увеличения числа операций сложения и вычитания, а также расхода памяти из-за рекурсивного разложения.

\subsection{Особенности параллельной реализации}
Алгоритм Штрассена хорошо подходит для параллелизации, так как вычисления \(M_1, M_2, \dots, M_7\) независимы друг от друга. В данной работе используется технология MPI, которая позволяет распределить эти вычисления между несколькими процессами.

\textbf{Преимущества параллельной реализации:}
\begin{itemize}
    \item Снижение времени выполнения за счет распределения нагрузки.
    \item Увеличение объема обрабатываемых данных за счет использования кластеров или многопроцессорных систем.
\end{itemize}

\textbf{Ограничения:}
\begin{itemize}
    \item Дополнительные затраты на передачу данных между процессами.
    \item Необходимость синхронизации при сборе результатов.
\end{itemize}