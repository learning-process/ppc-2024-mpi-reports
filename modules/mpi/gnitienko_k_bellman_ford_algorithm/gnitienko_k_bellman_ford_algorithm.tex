\documentclass[12pt]{article}
\usepackage[utf8]{inputenc}
\usepackage[russian]{babel}
\usepackage{amsmath}
\usepackage{hyperref} 
\usepackage{graphicx}
\usepackage{float}
\usepackage{listings}
\usepackage{xcolor}
\usepackage{array}

\definecolor{codegreen}{rgb}{0,0.6,0}
\definecolor{codegray}{rgb}{0.5,0.5,0.5}
\definecolor{codepurple}{rgb}{0.58,0,0.82}
\definecolor{backcolour}{rgb}{0.95,0.95,0.92}

\lstdefinestyle{cppstyle}{
    backgroundcolor=\color{backcolour},   
    commentstyle=\color{codegreen},
    keywordstyle=\color{magenta},
    numberstyle=\tiny\color{codegray},
    stringstyle=\color{codepurple},
    basicstyle=\ttfamily\footnotesize,
    breakatwhitespace=false,         
    breaklines=true,                 
    captionpos=b,                    
    keepspaces=true,                 
    numbers=left,                    
    numbersep=5pt,                  
    showspaces=false,                
    showstringspaces=false,
    showtabs=false,                  
    tabsize=2
}

\lstset{style=cppstyle}
\usepackage[left=2cm,right=2cm,top=2cm,bottom=2cm]{geometry}

\title{}
\author{}
\date{}

\begin{document}

\begin{titlepage}
\centering
    \large
    Министерство науки и высшего образования Российской Федерации\\[0.5cm]
    Федеральное государственное автономное образовательное учреждение высшего образования\\[0.5cm]
    \textbf{<<Национальный исследовательский Нижегородский государственный университет им. Н.И. Лобачевского>>}\\
    (ННГУ)\\[1cm]
    Институт информационных технологий, математики и механики\\
\vspace*{\fill}
{\Huge \textbf{Отчет по лабораторной работе № 3}}

\vspace{0.5cm}
{\Huge \textbf{Поиск кратчайших путей из одной вершины (алгоритм Беллмана-Форда). С CRS графов.}}

\vspace{0.5cm}
{\Huge \textbf{Вариант 23}}

\vspace{1.5cm}
{\Large \textbf{Выполнил:} Гнитиенко Кирилл ПР1} 

\vspace{0.5cm}
{\Large \textbf{Преподаватель:} Сысоев А.В., доцент кафедры ВВСП} 

\vspace*{\fill}

{\large Нижний Новгород\\ 2024}
\end{titlepage}

\newpage
\section*{Введение}

В данной работе рассматривается реализация алгоритма Беллмана-Форда для поиска кратчайших путей в графе. Алгоритм Беллмана-Форда является одним из классических алгоритмов в теории графов, способных находить кратчайшие пути от одной вершины (источника) до всех остальных вершин в графе, даже если в графе присутствуют рёбра с отрицательным весом. В рамках данной лабораторной работы была разработана последовательная и параллельная версии алгоритма, использующие представление графа в формате CRS (Compressed Row Storage). Целью работы является исследование эффективности параллельной реализации алгоритма, а также оценка корректности результатов.

\newpage
\section*{Постановка задачи}

Задача состоит в разработке и реализации алгоритма Беллмана-Форда для поиска кратчайших путей в графе. Исходный граф представлен в виде матрицы смежности, которая конвертируется в формат CRS. Необходимо реализовать как последовательную, так и параллельную версии алгоритма с использованием MPI. Основные требования включают корректность вычислений, возможность обработки графов с отрицательными весами ребер (при этом должна быть осуществлена проверка на наличие отрицательных циклов), а также оценка производительности параллельной версии относительно последовательной.

\newpage
\section*{Описание алгоритма}

Алгоритм Беллмана-Форда используется для нахождения кратчайших путей от одной вершины до всех остальных в графе. В отличие от алгоритма Дейкстры, он поддерживает графы с рёбрами отрицательного веса, но не работает в случае наличия отрицательных циклов.

\subsection{Шаги работы алгоритма} \begin{enumerate}

\item \textbf{Инициализация:}
   Все расстояния до вершин, кроме начальной, устанавливаются в $\infty$. Расстояние до начальной вершины равно 0.

\item \textbf{Итерация:}
   В цикле выполняется поиск кратчайших путей. На каждой итерации для каждого ребра $(u, v)$ с весом $w(u, v)$ проверяется, можно ли сократить расстояние до вершины $v$, используя вершину $u$. Если расстояние через вершину $u$ меньше, чем текущее расстояние до вершины $v$, то оно обновляется.

\item \textbf{Повторение итераций:}
   Алгоритм повторяет итерации $V - 1$ раз (где $V$ — количество вершин в графе). Это гарантирует, что все кратчайшие пути без отрицательных циклов будут найдены, так как в графе с $V$ вершинами нет путей длиннее $V - 1$ рёбер.

\item \textbf{Проверка на отрицательные циклы:}
   После завершения $V - 1$ итераций проверяется, можно ли улучшить расстояние хотя бы для одного рёбра. Если это возможно, то в графе существует отрицательный цикл.
   \end{enumerate}

\subsection{Время работы}
Время работы алгоритма — $O(V \cdot E)$, где $V$ — количество вершин, а $E$ — количество рёбер. Это связано с тем, что для каждой из $V - 1$ итераций алгоритм проверяет все рёбра графа.


\newpage
\section*{Описание схемы распараллеливания}

Для распараллеливания алгоритма Беллмана-Форда используется модель передачи сообщений (MPI). Основной подход заключается в распределении рёбер графа между процессами. Каждый процесс: \begin{itemize}
    \item Получает свою часть рёбер, которые он обрабатывает локально.
На каждой итерации обновляет значения кратчайших расстояний для вершин, связанных с этими рёбрами.
    \item Выполняет синхронизацию с другими процессами, обмениваясь информацией о минимальных расстояниях.
\end{itemize}
Процессы выполняют итерации параллельно, а синхронизация данных происходит через операцию \texttt{reduce}. Эта схема обеспечивает согласованность результатов на каждом шаге, но добавляет накладные расходы на передачу данных.

Основные аспекты реализации распараллеливания: \begin{itemize}

 \item Разделение рёбер: Все рёбра графа распределяются между процессами.
 \item Синхронизация данных: На каждой итерации массив минимальных расстояний обновляется всеми процессами через глобальный обмен.
 \item Сходимость: Алгоритм завершается, если на очередной итерации не происходит обновлений расстояний.
 \end{itemize}
Данная схема позволяет увеличить производительность на больших графах, но накладные расходы становятся значительными при увеличении числа процессов или малых размерах графа.



\newpage
\section*{MPI-версия - описание программной реализации}

В MPI-версии алгоритма используется библиотека \texttt{boost::mpi}. Код разбит на следующие части:
\begin{itemize}
    \item \texttt{BellmanFordAlgMPI::pre\_processing()} : Получает входные данные, преобразует матрицу смежности в формат CRS (только на процессе 0).
    \item \texttt{BellmanFordAlgMPI::validation()} : Проверяет корректность входных данных (только на процессе 0).
    \item \texttt{BellmanFordAlgMPI::run()} : Основная логика алгоритма. Инициализирует расстояния, проводит итерации релаксации и проверку на отрицательные циклы.
    \item \texttt{BellmanFordAlgMPI::post\_processing()} : Процесс 0 передает результаты.
    \item \texttt{BellmanFordAlgMPI::Iteration()} : Логика итерации поиска кратчайших путей. Распределяет вершины между процессами, проводит поиск кратчайщих путей и выполняет редукцию расстояний.
\end{itemize}

\newpage
\section*{Результаты экспериментов}

Эксперименты проводились на графе размером в 10 000 вершин, с различным количеством процессоров. Результаты времени работы представлены в таблице ниже.

\begin{table}[H]
    \centering
    \begin{tabular}{|c|c|c|c|c|}
    \hline
    Размер графа & Последовательное время (с) & \multicolumn{3}{c|}{Параллельное время (с)} \\
    \cline{3-5}
     &  & 2 процесса & 4 процесса & 8 процессов \\
    \hline
    10000 & 1.52 & 2.4 & 4.0 & 7.7 \\
    \hline
    \end{tabular}
    \caption{Сравнение времени работы последовательной и параллельной версий алгоритма Беллмана-Форда}
    \label{tab:time_results}
\end{table}


Корректность работы алгоритма проверялась путем сравнения результатов последовательной и параллельной версий на одних и тех же графах. Ошибок обнаружено не было.

\newpage
\section*{Выводы из результатов}

Результаты экспериментов, представленные в таблице, показывают, что параллельная версия алгоритма Беллмана-Форда не даёт ожидаемого улучшения производительности при увеличении числа процессов. Наоборот, время выполнения параллельных версий увеличивается по мере добавления процессов:

Это поведение может указывать на следующие причины:
\begin{itemize}
    \item \textbf{Накладные расходы на синхронизацию:} При увеличении числа процессов возрастает накладка на синхронизацию данных между процессами. В случае малого размера графа эти накладные расходы могут существенно превышать выигрыш от параллельной обработки.
    \item \textbf{Недостаточно эффективный код:} В текущей реализации алгоритма могут быть недостаточно эффективно использованы ресурсы параллельных вычислений. Например, операции обновления минимальных расстояний и синхронизации данных могут быть выполнены не в оптимальном порядке, что приводит к дополнительным задержкам. Возможным улучшением является более эффективное распределение данных и минимизация количества синхронизаций между процессами.
\end{itemize}


Таким образом, экспериментальные данные показывают, что параллельная версия алгоритма в данном случае не даёт преимуществ по времени.

\newpage
\section*{Заключение}

На основе полученных результатов можно сделать вывод, что параллельная реализация алгоритма Беллмана-Форда не приводит к значительному ускорению вычислений для графов размером порядка 10000 вершин. Эксперименты показали, что при увеличении числа процессов время выполнения параллельных версий увеличивается, что может быть связано с накладными расходами на синхронизацию данных между процессами, которые становятся более заметными на малых графах. Кроме того, недостаточная оптимизация текущей реализации алгоритма, возможно, приводит к неэффективному использованию параллельных вычислений, что также влияет на производительность.

Для более крупных графов, где количество рёбер и вершин значительно увеличивается, параллельная версия может продемонстрировать значительное улучшение производительности. Важно отметить, что эффективность параллельных вычислений зависит от размера задачи и архитектуры вычислительных систем. При решении более масштабных задач с большим числом рёбер и вершин параллельная версия алгоритма, вероятно, будет работать быстрее, так как накладные расходы на синхронизацию станут менее значимыми.

В ходе работы была проверена корректность обеих версий алгоритма, и ошибки не были обнаружены. Это подтверждает, что как последовательная, так и параллельная версии алгоритма работают корректно. Однако для повышения производительности параллельной реализации требуется дальнейшая оптимизация.

\newpage
\section*{Литература}

\begin{enumerate}
    \item \href{https://ru.wikipedia.org/wiki/%D0%90%D0%BB%D0%B3%D0%BE%D1%80%D0%B8%D1%82%D0%BC_%D0%91%D0%B5%D0%BB%D0%BB%D0%BC%D0%B0%D0%BD%D0%B0_%E2%80%94_%D0%A4%D0%BE%D1%80%D0%B4%D0%B0}{Wiki - Алгоритм Белллмана-Форда}
    
    \item \href{https://habr.com/ru/articles/548266/}{Habr - MPI}
    
    \item Антонов А.С. Параллельное программирование с использованием технологии MPI: Учебное пособие
\end{enumerate}

\newpage
\section*{Приложение}

\begin{lstlisting}[language=C++, caption={Код алгоритма Беллмана-Форда}]
#pragma once

#include <gtest/gtest.h>

#include <boost/mpi/collectives.hpp>
#include <boost/mpi/communicator.hpp>
#include <memory>
#include <numeric>
#include <string>
#include <utility>
#include <vector>

#include "core/task/include/task.hpp"

namespace gnitienko_k_bellman_ford_algorithm_mpi {

class BellmanFordAlgSeq : public ppc::core::Task {
 public:
  explicit BellmanFordAlgSeq(std::shared_ptr<ppc::core::TaskData> taskData_) : Task(std::move(taskData_)) {}
  bool pre_processing() override;
  bool validation() override;
  bool run() override;
  bool post_processing() override;

 private:
  size_t V{};
  std::vector<int> values;
  std::vector<int> columns;
  std::vector<int> row_ptr;
  std::vector<int> shortest_paths;
  const int INF = std::numeric_limits<int>::max();

  bool Iteration(std::vector<int>& paths);
  bool check_negative_cycle();
  void toCRS(const int* input_matrix);
};

class BellmanFordAlgMPI : public ppc::core::Task {
 public:
  explicit BellmanFordAlgMPI(std::shared_ptr<ppc::core::TaskData> taskData_) : Task(std::move(taskData_)) {}
  bool pre_processing() override;
  bool validation() override;
  bool run() override;
  bool post_processing() override;

 private:
  size_t V{};
  size_t values_size{};
  size_t columns_size{};
  size_t row_ptr_size{};
  std::vector<int> values;
  std::vector<int> columns;
  std::vector<int> row_ptr;
  std::vector<int> shortest_paths;
  boost::mpi::communicator world;
  const int INF = std::numeric_limits<int>::max();

  bool Iteration(std::vector<int>& paths);
  bool check_negative_cycle();
  void toCRS(const int* input_matrix);
};

}  // namespace gnitienko_k_bellman_ford_algorithm_mpi

#include "mpi/gnitienko_k_bellman-ford-algorithm/include/ops_mpi.hpp"

#include <algorithm>
#include <boost/serialization/vector.hpp>
#include <functional>
#include <random>
#include <string>
#include <thread>
#include <vector>

void gnitienko_k_bellman_ford_algorithm_mpi::BellmanFordAlgSeq::toCRS(const int* input_matrix) {
  row_ptr.push_back(0);
  for (size_t i = 0; i < V; ++i) {
    for (size_t j = 0; j < V; ++j) {
      if (input_matrix[i * V + j] != 0) {
        values.push_back(input_matrix[i * V + j]);
        columns.push_back(j);
      }
    }
    row_ptr.push_back(values.size());
  }
}

bool gnitienko_k_bellman_ford_algorithm_mpi::BellmanFordAlgSeq::pre_processing() {
  internal_order_test();
  auto* input_matrix = reinterpret_cast<int*>(taskData->inputs[0]);

  toCRS(input_matrix);

  shortest_paths.resize(V, INF);
  shortest_paths[0] = 0;
  return true;
}

bool gnitienko_k_bellman_ford_algorithm_mpi::BellmanFordAlgSeq::validation() {
  internal_order_test();
  auto* input_matrix = reinterpret_cast<int*>(taskData->inputs[0]);
  V = taskData->inputs_count[0];
  size_t j = 0;
  for (size_t i = 0; i < V; i++) {
    if (input_matrix[i * V + j] != 0) return false;
    j++;
  }
  return taskData->inputs_count[0] == taskData->outputs_count[0] && taskData->inputs_count[0] != 0;
}

bool gnitienko_k_bellman_ford_algorithm_mpi::BellmanFordAlgSeq::Iteration(std::vector<int>& paths) {
  bool changed = false;
  for (size_t i = 0; i < V; i++) {
    for (int j = row_ptr[i]; j < row_ptr[i + 1]; j++) {
      int v = columns[j];
      int weight = values[j];
      if (paths[i] != INF && paths[i] + weight < paths[v]) {
        paths[v] = paths[i] + weight;
        changed = true;
      }
    }
  }

  return changed;
}

bool gnitienko_k_bellman_ford_algorithm_mpi::BellmanFordAlgSeq::check_negative_cycle() {
  for (size_t i = 0; i < V; ++i) {
    for (int j = row_ptr[i]; j < row_ptr[i + 1]; ++j) {
      int v = columns[j];
      int weight = values[j];
      if (shortest_paths[i] != INF && shortest_paths[i] + weight < shortest_paths[v]) {
        std::cerr << "Negative cycle detected in seq!" << std::endl;
        return true;
      }
    }
  }
  return false;
}

bool gnitienko_k_bellman_ford_algorithm_mpi::BellmanFordAlgSeq::run() {
  internal_order_test();

  bool changed = false;
  for (size_t i = 0; i < V - 1; i++) {
    changed = Iteration(shortest_paths);
    if (!changed) break;
  }

  for (size_t i = 0; i < V; i++) {
    if (shortest_paths[i] == INF) shortest_paths[i] = 0;
  }

  return !check_negative_cycle();
}

bool gnitienko_k_bellman_ford_algorithm_mpi::BellmanFordAlgSeq::post_processing() {
  internal_order_test();
  for (size_t i = 0; i < V; i++) {
    reinterpret_cast<int*>(taskData->outputs[0])[i] = shortest_paths[i];
  }
  return true;
}

// PARALLEL
//----------------------------------------------------------------------------

void gnitienko_k_bellman_ford_algorithm_mpi::BellmanFordAlgMPI::toCRS(const int* input_matrix) {
  row_ptr.push_back(0);
  for (size_t i = 0; i < V; ++i) {
    for (size_t j = 0; j < V; ++j) {
      if (input_matrix[i * V + j] != 0) {
        values.push_back(input_matrix[i * V + j]);
        columns.push_back(j);
      }
    }
    row_ptr.push_back(values.size());
  }

  values_size = values.size();
  columns_size = columns.size();
  row_ptr_size = row_ptr.size();
}

bool gnitienko_k_bellman_ford_algorithm_mpi::BellmanFordAlgMPI::check_negative_cycle() {
  for (size_t i = 0; i < V; ++i) {
    for (int j = row_ptr[i]; j < row_ptr[i + 1]; ++j) {
      int v = columns[j];
      int weight = values[j];
      if (shortest_paths[i] != INF && shortest_paths[i] + weight < shortest_paths[v]) {
        std::cerr << "Negative cycle detected in mpi!" << std::endl;
        return true;
      }
    }
  }
  return false;
}

bool gnitienko_k_bellman_ford_algorithm_mpi::BellmanFordAlgMPI::Iteration(std::vector<int>& paths) {
  bool changed = false;
  std::vector<int> start_paths = paths;
  int rank = world.rank();
  int size = world.size();

  int local_size = V / size;
  int remainder = V % size;
  int start = rank * local_size + std::min(rank, remainder);
  int end = start + local_size + (rank < remainder ? 1 : 0);

  for (int i = start; i < end; i++) {
    for (int j = row_ptr[i]; j < row_ptr[i + 1]; j++) {
      int v = columns[j];
      int weight = values[j];
      if (start_paths[i] != INF && start_paths[i] + weight < start_paths[v]) {
        start_paths[v] = start_paths[i] + weight;
      }
    }
  }

  std::vector<int> reduced_paths(V, INF);
  boost::mpi::reduce(world, start_paths.data(), V, reduced_paths.data(), boost::mpi::minimum<int>(), 0);

  if (world.rank() == 0) {
    for (size_t i = 0; i < V; i++) {
      if (paths[i] != reduced_paths[i]) {
        changed = true;
        break;
      }
    }
    paths = reduced_paths;
  }

  boost::mpi::broadcast(world, paths.data(), paths.size(), 0);
  boost::mpi::broadcast(world, changed, 0);

  return changed;
}

bool gnitienko_k_bellman_ford_algorithm_mpi::BellmanFordAlgMPI::pre_processing() {
  internal_order_test();
  if (world.rank() == 0) {
    auto* input_matrix = reinterpret_cast<int*>(taskData->inputs[0]);

    toCRS(input_matrix);
  }
  return true;
}

bool gnitienko_k_bellman_ford_algorithm_mpi::BellmanFordAlgMPI::validation() {
  internal_order_test();
  if (world.rank() == 0) {
    auto* input_matrix = reinterpret_cast<int*>(taskData->inputs[0]);
    V = taskData->inputs_count[0];
    size_t j = 0;
    for (size_t i = 0; i < V; i++) {
      if (input_matrix[i * V + j] != 0) return false;
      j++;
    }
    return taskData->inputs_count[0] == taskData->outputs_count[0] && taskData->inputs_count[0] != 0;
  }
  return true;
}

bool gnitienko_k_bellman_ford_algorithm_mpi::BellmanFordAlgMPI::run() {
  internal_order_test();
  boost::mpi::broadcast(world, V, 0);
  boost::mpi::broadcast(world, values_size, 0);
  boost::mpi::broadcast(world, columns_size, 0);
  boost::mpi::broadcast(world, row_ptr_size, 0);
  values.resize(values_size);
  columns.resize(columns_size);
  row_ptr.resize(row_ptr_size);
  boost::mpi::broadcast(world, values.data(), values.size(), 0);
  boost::mpi::broadcast(world, columns.data(), columns.size(), 0);
  boost::mpi::broadcast(world, row_ptr.data(), row_ptr.size(), 0);

  shortest_paths.resize(V, INF);
  shortest_paths[0] = 0;

  bool changed = false;
  for (size_t i = 0; i < V - 1; ++i) {
    changed = Iteration(shortest_paths);
    if (!changed) break;
  }

  if (world.rank() == 0) {
    for (size_t i = 0; i < V; ++i) {
      if (shortest_paths[i] == INF) shortest_paths[i] = 0;
    }
  }

  if (world.rank() == 0) {
    return !check_negative_cycle();
  }

  return true;
}

bool gnitienko_k_bellman_ford_algorithm_mpi::BellmanFordAlgMPI::post_processing() {
  internal_order_test();
  if (world.rank() == 0) {
    for (size_t i = 0; i < V; ++i) {
      reinterpret_cast<int*>(taskData->outputs[0])[i] = shortest_paths[i];
    }
  }
  return true;
}

\end{lstlisting}

\end{document}