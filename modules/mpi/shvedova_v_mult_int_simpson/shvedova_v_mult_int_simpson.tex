\documentclass[12pt]{article}
\usepackage[T2A]{fontenc}
\usepackage[utf8]{inputenc}
\usepackage[russian]{babel}
\usepackage{amsmath, amssymb}
\usepackage{geometry}
\usepackage{graphicx}
\usepackage{listings}
\usepackage{hyperref}
\usepackage{xcolor} % Для цветов в блоках кода
\geometry{left=2cm, right=2cm, top=2cm, bottom=2cm}

\lstset{
  language=C++,
  basicstyle=\ttfamily\small,
  keywordstyle=\bfseries\color{blue},
  commentstyle=\itshape\color{green!50!black},
  stringstyle=\color{red!60!black},
  numbers=left,
  numberstyle=\tiny,
  stepnumber=1,
  numbersep=8pt,
  backgroundcolor=\color{gray!10},
  showspaces=false,
  showstringspaces=false,
  breaklines=true,
  frame=single,
  tabsize=2,
  captionpos=b
}

\title{Вычисление многомерных интегралов с использованием многошаговой схемы (методом Симпсона) - вариант 9}
\author{Выполнила: Шведова Виталина }
\date{\today}

\begin{document}
\sloppy

\maketitle

\tableofcontents
\newpage

\section*{Введение}
\addcontentsline{toc}{section}{Введение}

Данный отчет содержит в себе полное описание реализации задачи вычисления многомерных интегралов с использованием многошаговой схемы методом Симпсона.

Вычисление многомерных интегралов - одна из важнейших задач математического анализа. Она примерняется для вычисления объемов, вероятностей и других характеристик сложных систем. Метод Симпсона - один из численных методов, обеспечивающий высокую точность за счет использования полиномов второй степени для аппроксимации функции.

MPI (Message Passing Interface) — это стандарт интерфейса для параллельных вычислений, который позволяет организовать эффективное взаимодействие между процессами. MPI широко применяется для распараллеливания вычислительных задач в распределенных системах.

\section*{Постановка задачи}
\addcontentsline{toc}{section}{Постановка задачи}
Целью работы является разработка и исследование алгоритма вычисления многомерных интегралов методом Симпсона, с акцентом на его последовательную и параллельную реализации. Для достижения этой цели необходимо:
\begin{itemize}
    \item Реализовать последовательный метод Симпсона для численного вычисления многомерных интегралов.
    \item Разработать параллельную версию алгоритма с использованием библиотеки MPI, разделив вычисления между процессами для увеличения производительности.
    \item Сравнить производительность последовательной и параллельной версий, проанализировав ускорение, достигаемое за счёт распараллеливания.
    \item Провести эксперименты с использованием разного числа процессов и различными входными параметрами точности для оценки эффективности параллельного подхода.
    \item Выполнить функциональное тестирование последовательной и параллельной версий алгоритма, чтобы убедиться в корректности их работы и совпадении результатов.
    \item Подготовить отчёт, отражающий этапы разработки, результаты экспериментов и выводы по эффективности алгоритма.
\end{itemize}

\section*{Описание алгоритма}
\addcontentsline{toc}{section}{Описание алгоритма}
Метод Симпсона представляет собой численный метод интегрирования, основанный на аппроксимации функции полиномом второй степени на каждом сегменте области интегрирования. Для последовательного вычисления многомерного интеграла используются следующие этапы:

\begin{itemize}
    \item \textbf{Инициализация параметров интегрирования.}
          Задаётся область интегрирования в виде списка пар нижних и верхних границ для каждого измерения. Указывается точность, которая определяет допустимую погрешность вычисления результата.

    \item \textbf{Рекурсивное разбиение области интегрирования.}
          Многомерный интеграл преобразуется в последовательность одномерных интегралов:
          \begin{itemize}
              \item Для текущего измерения извлекаются его границы и фиксируется значение аргумента на этом этапе.
              \item Одномерный интеграл вычисляется методом Симпсона.
              \item Если остаются неизученные измерения, алгоритм переходит к следующему измерению, применяя те же шаги.
          \end{itemize}

    \item \textbf{Метод Симпсона для одномерного интеграла.}
          Одномерный интеграл на каждом шаге вычисляется следующим образом:
          \begin{itemize}
              \item Интервал интегрирования делится на равные сегменты.
              \item Значения функции вычисляются в начальной и конечной точках, а также во всех промежуточных узлах.
              \item Значения в узлах суммируются с весами: 4 для нечетных индексов и 2 для четных.
              \item Полученная сумма умножается на шаг сегмента, делённый на 3, что завершает формулу Симпсона.
          \end{itemize}

    \item \textbf{Адаптивное уточнение результата.}
          Для обеспечения заданной точности интеграл пересчитывается с увеличением числа сегментов в два раза на каждом этапе. Итерации продолжаются до тех пор, пока разница между текущим и предыдущим результатами не станет меньше указанной погрешности.

    \item \textbf{Возврат к многомерному случаю.}
          После завершения интегрирования по одному измерению результат передаётся на следующий уровень рекурсии. Алгоритм продолжает работу, пока не будут обработаны все измерения, после чего возвращается итоговое значение многомерного интеграла.
\end{itemize}

\section*{Схема распараллеливания}
\addcontentsline{toc}{section}{Схема распараллеливания}

\subsection*{Описание схемы распараллеливания}
Для ускорения вычислений многомерного интеграла метод Симпсона распараллеливается по следующей схеме:
\begin{itemize}
    \item \textbf{Распределение области интегрирования.}
          Область интегрирования по первому измерению делится на равные части между всеми доступными процессами. Каждый процесс отвечает за выполнение вычислений на своей части диапазона.

    \item \textbf{Локальные вычисления.}
          Каждый процесс выполняет одномерное интегрирование на своём диапазоне. Если область имеет несколько измерений, для каждой фиксированной координаты текущего измерения рекурсивно вычисляется интеграл по следующему измерению.

    \item \textbf{Сбор результатов.}
          Частичные результаты, полученные каждым процессом, объединяются в общий результат. Это позволяет аккумулировать вклад всех процессов для получения окончательного значения многомерного интеграла.

    \item \textbf{Адаптивное уточнение.}
          После каждой итерации уточняется точность интегрирования. Если заданная точность не достигнута, число сегментов увеличивается, и вычисления продолжаются.
\end{itemize}

\subsection*{Описание реализации с использованием MPI}
В рамках реализации алгоритма параллельное выполнение организовано с использованием библиотеки MPI в обертке Boost. Основные этапы соответствуют общей схеме распараллеливания:

\begin{itemize}
    \item \textbf{Распределение данных.}
          Нулевой процесс передаёт границы интегрирования и точность всем остальным процессам с помощью функции \texttt{MPI\_Broadcast}. Эти данные необходимы для локальных вычислений на каждом процессе.

    \item \textbf{Локальные вычисления.}
          Каждый процесс рассчитывает одномерный интеграл на своём диапазоне. В реализации используется функция \texttt{integrateSimpson1D}, которая применяет метод Симпсона к текущему диапазону сегментов. Если область интегрирования имеет несколько измерений, процесс вызывает рекурсивную функцию \texttt{integrateSimpsonImpl}.

    \item \textbf{Сбор частичных результатов.}
          Локальные результаты, вычисленные каждым процессом, передаются на нулевой процесс с использованием функции \texttt{MPI\_Reduce}. Итоговое значение интеграла аккумулируется на нулевом процессе.

    \item \textbf{Адаптивное уточнение результата.}
          На нулевом процессе проверяется сходимость текущего результата. Если разница между текущим и предыдущим значениями превышает заданную точность, все процессы продолжают работу с увеличенным числом сегментов. Информация о необходимости продолжения итераций передаётся через \texttt{MPI\_Broadcast}.

    \item \textbf{Возврат результата.}
          После достижения требуемой точности результат возвращается только на нулевой процесс, а остальные процессы возвращают нулевое значение.
\end{itemize}

\subsection*{Условия экспериментов}
Для оценки производительности последовательной и параллельной версий алгоритма использовалась следующая тестовая система:
\begin{itemize}
    \item Операционная система: WSL (Windows Subsystem for Linux) Ubuntu 24.04.
    \item Процессор: 13th Gen Intel(R) Core(TM) i7-13700H, 2.40 GHz.
    \item Оперативная память: 16 ГБ.
    \item Среда выполнения: библиотека MPI, настроенная для многопроцессорного режима.
\end{itemize}

Тестовая задача заключалась в вычислении трёхмерного интеграла для функции:
\[
    f(x, y, z) = \sin(x^2 + y^2 + z^2) \cdot e^{-(x + y + z)},
\]
где $x \in [0, 10]$, $y \in [0, 9]$, $z \in [0, 8]$. Заданная точность интегрирования: $0.001$.

\section*{Результаты экспериментов}
\addcontentsline{toc}{section}{Результаты экспериментов}

\subsection*{Результаты замеров}
Результаты выполнения тестов для последовательной и параллельной версии метода Симпсона приведены в таблице. Для каждого числа процессов (\texttt{n}) выполнены тесты с использованием \texttt{pipeline\_run} и \texttt{task\_run}.

\begin{tabular}{|c|c|c|c|}
    \hline
    Число процессов & Сценарий      & Время выполнения (с) & Ускорение \\ \hline
    1               & pipeline\_run & 3.245                & 1.00      \\ \hline
    1               & task\_run     & 3.977                & 1.00      \\ \hline
    2               & pipeline\_run & 3.374                & 0.96      \\ \hline
    2               & task\_run     & 3.469                & 1.15      \\ \hline
    3               & pipeline\_run & 2.621                & 1.24      \\ \hline
    3               & task\_run     & 2.735                & 1.45      \\ \hline
\end{tabular}

\subsection*{Анализ результатов}
Результаты экспериментов демонстрируют следующее:
\begin{itemize}
    \item Время выполнения уменьшается с увеличением числа процессов, что подтверждает корректность параллельной реализации.
    \item Ускорение параллельной версии значительно лучше при увеличении числа процессов до трёх. Однако эффективность использования процессов не является линейной, что обусловлено накладными расходами на синхронизацию и передачу данных между процессами.
    \item Наибольшее ускорение наблюдается в тесте \texttt{task\_run} при использовании 3 процессов, где ускорение составило около 1.45.
    \item Результаты тестов \texttt{pipeline\_run} и \texttt{task\_run} схожи, что говорит о равномерной оптимизации алгоритма как для основного вычисления, так и для конвейерной обработки данных.
\end{itemize}

\subsection*{Вывод}
Эксперименты показали, что параллельная реализация метода Симпсона демонстрирует хорошее ускорение с увеличением числа процессов. Однако можно заметить, что реализация потенциально может быть улучшена. Одним из очевидных подходов к модернизации является отказ от рекурсивного подхода и замена его на итеративный с дальнейшим добавлением параллелизма.

\section*{Заключение}
\addcontentsline{toc}{section}{Заключение}

В данной работе был разработан алгоритм численного вычисления многомерных интегралов с использованием многошаговой схемой методом Симпсона. Реализация включает как последовательный, так и параллельный подход, что позволяет оценить эффективность распараллеливания вычислений с применением библиотеки MPI.

В ходе работы выполнены следующие задачи:
\begin{itemize}
    \item Реализован последовательный метод Симпсона для многомерных интегралов, с использованием рекурсивного подхода для работы с несколькими измерениями.
    \item Разработана параллельная версия алгоритма, которая распределяет область интегрирования между процессами, выполняя локальные вычисления на каждом из них.
    \item Проведено функциональное и производительное тестирование, что подтвердило корректность реализации и её способность к ускорению вычислений при увеличении числа процессов.
\end{itemize}

Таким образом, выполненная работа подтверждает, что метод Симпсона является эффективным инструментом для численного интегрирования, а его параллельная реализация позволяет существенно ускорить вычисления при обработке сложных многомерных задач.

\section*{Литература}
\addcontentsline{toc}{section}{Литература}

\begin{enumerate}
    \item \href{https://proglib.io/p/git-for-half-an-hour}{Статья "Гит для начинающих"}.
    \item \href{https://habr.com/ru/articles/119090/}{Habr. Статья "Google testing framework (gtest)"}.
    \item \href{https://parallel.ru/vvv/mpi.html#p1}{MPI. Терминология и обозначения}.
\end{enumerate}

\appendix
\section*{Приложение}
\addcontentsline{toc}{section}{Приложение}

В данном разделе представлены основные фрагменты реализации алгоритма, включая функции из файлов \texttt{ops\_mpi.hpp} и \texttt{ops\_mpi.cpp}, использованные для решения задачи.

\subsection*{Функция integrateSimpson1D}
\addcontentsline{toc}{subsection}{Функция integrateSimpson1D}

\begin{lstlisting}[language=C++]
double integrateSimpson1D(double lowerBound, double upperBound, int segmentCount,
                          const Simpson1DFunction& integrand) {
    double stepSize = (upperBound - lowerBound) / segmentCount;
    double result = integrand(lowerBound) + integrand(upperBound);

    for (int i = 1; i < segmentCount; ++i) {
        double x = lowerBound + i * stepSize;
        result += ((i % 2 == 0) ? 2 : 4) * integrand(x);
    }

    return result * stepSize / 3.0;
}
\end{lstlisting}

\subsection*{Функция integrateSimpsonImpl}
\addcontentsline{toc}{subsection}{Функция integrateSimpsonImpl}

\begin{lstlisting}[language=C++]
double integrateSimpsonImpl(Limits& integrationLimits,
                             std::vector<double>& currentArguments,
                             const FunctionType& integrandFunction, double precision) {
    if (integrationLimits.empty()) {
        return integrandFunction(currentArguments);
    }

    auto [lowerBound, upperBound] = integrationLimits.front();
    integrationLimits.pop_front();
    currentArguments.push_back(0.0);

    double previousResult = 0.0;
    double currentResult = 0.0;
    int segmentCount = 2;

    do {
        previousResult = currentResult;
        currentResult = integrateSimpson1D(lowerBound, upperBound, segmentCount, [&](double x) {
            currentArguments.back() = x;

            if (integrationLimits.empty()) {
                return integrandFunction(currentArguments);
            }
            return integrateSimpsonImpl(integrationLimits, currentArguments, integrandFunction, precision);
        });

        segmentCount *= 2;
    } while (std::abs(currentResult - previousResult) > precision);

    currentArguments.pop_back();
    integrationLimits.emplace_front(lowerBound, upperBound);

    return currentResult;
}
\end{lstlisting}

\subsection*{Функция integrateSimpsonParallel}
\addcontentsline{toc}{subsection}{Функция integrateSimpsonParallel}

\begin{lstlisting}[language=C++]
double shvedova_v_mult_int_simpson_mpi::integrateSimpsonParallel(boost::mpi::communicator& world,
                                                                 Limits integrationLimits, double precision,
                                                                 const FunctionType& integrandFunction) {
  std::vector<double> currentArguments;
  broadcast(world, integrationLimits, 0);
  broadcast(world, precision, 0);

  double globalResult = 0.0;
  double previousResult = 0.0;
  double localResult = 0.0;
  int segmentCount = 2 * world.size();

  auto [globalLowerBound, globalUpperBound] = integrationLimits.front();
  double globalStepSize = (globalUpperBound - globalLowerBound) / world.size();

  double localLowerBound = globalLowerBound + globalStepSize * world.rank();
  double localUpperBound = localLowerBound + globalStepSize;

  integrationLimits.pop_front();
  currentArguments.push_back(0.0);

  bool continueIterations = true;
  while (continueIterations) {
    previousResult = globalResult;
    globalResult = 0.0;

    localResult = integrateSimpson1D(localLowerBound, localUpperBound, segmentCount / world.size(), [&](double x) {
      currentArguments.back() = x;

      if (integrationLimits.empty()) {
        return integrandFunction(currentArguments);
      }
      return integrateSimpsonImpl(integrationLimits, currentArguments, integrandFunction, precision);
    });

    reduce(world, localResult, globalResult, std::plus<>(), 0);

    if (world.rank() == 0) {
      continueIterations = (std::abs(globalResult - previousResult) > precision);
    }
    broadcast(world, continueIterations, 0);

    segmentCount *= 2;
  }

  return world.rank() == 0 ? globalResult : 0.0;
}
\end{lstlisting}

\end{document}

