\documentclass[12pt]{article}
\usepackage[russian]{babel}
\usepackage[T2A]{fontenc}
\usepackage[utf8]{inputenc}
\usepackage{graphicx}
\usepackage{listings}
\usepackage{color}
\usepackage{amsmath}
\usepackage{hyperref}

\definecolor{dkgreen}{rgb}{0,0.6,0}
\definecolor{gray}{rgb}{0.5,0.5,0.5}
\definecolor{mauve}{rgb}{0.58,0,0.82}

\lstset{
  frame=tb,
  language=C++,
  aboveskip=3mm,
  belowskip=3mm,
  showstringspaces=false,
  columns=flexible,
  basicstyle={\small\ttfamily},
  numbers=none,
  numberstyle=\tiny\color{gray},
  keywordstyle=\color{blue},
  commentstyle=\color{dkgreen},
  stringstyle=\color{mauve},
  breaklines=true,
  breakatwhitespace=true,
  tabsize=3
}

\begin{document}

\title{Отчёт по задаче: Параллельная сортировка с использованием MPI}
\author{Ваше Имя}
\date{\today}
\maketitle

\section{Введение}
В данном отчёте представлена реализация параллельной сортировки с использованием библиотеки MPI. Задача заключается в сортировке массива целых чисел с использованием алгоритма Бэтчера (Batcher's Odd-Even Merge Sort) в параллельном режиме.

\section{Описание задачи}
Алгоритм Бэтчера основан на разделении массива на две части, сортировке каждой части и последующем слиянии с использованием чередующихся сравнений элементов. В параллельной реализации массив распределяется между процессами, каждый процесс сортирует свою часть, после чего происходит слияние результатов.

\section{Реализация}
Реализация задачи состоит из нескольких файлов:
\begin{itemize}
    \item \texttt{main.cpp} - содержит тесты для проверки корректности и производительности реализации.
    \item \texttt{ops\_mpi.hpp} - заголовочный файл с объявлением классов для последовательной и параллельной сортировки.
    \item \texttt{ops\_mpi.cpp} - реализация методов классов для сортировки.
\end{itemize}

\subsection{Классы и методы}
\begin{itemize}
    \item \texttt{TestMPITaskSequential} - класс для последовательной сортировки.
    \item \texttt{TestMPITaskParallel} - класс для параллельной сортировки с использованием MPI.
\end{itemize}

\subsection{Основные функции}
\begin{itemize}
    \item \texttt{batcher\_sort} - рекурсивная функция для сортировки массива.
    \item \texttt{batcher\_merge} - функция для слияния двух отсортированных массивов.
\end{itemize}

\section{Тестирование}
Для проверки корректности реализации были написаны тесты, которые проверяют сортировку массива различного размера. Также были проведены тесты производительности для оценки эффективности параллельной реализации.

\subsection{Пример теста}
\begin{lstlisting}
TEST(kovalchuk_a_odd_even, Test_Vector_10) {
  const int count_elements = 10;
  boost::mpi::communicator world;
  std::vector<int> global_vector;
  std::vector<int> global_result(count_elements, 0);

  // Create TaskData
  std::shared_ptr<ppc::core::TaskData> taskDataPar = std::make_shared<ppc::core::TaskData>();
  if (world.rank() == 0) {
    global_vector = getRandomVector(count_elements);

    taskDataPar->inputs.emplace_back(reinterpret_cast<uint8_t*>(global_vector.data()));
    taskDataPar->inputs_count.emplace_back(count_elements);
    taskDataPar->outputs.emplace_back(reinterpret_cast<uint8_t*>(global_result.data()));
    taskDataPar->outputs_count.emplace_back(global_result.size());
  }

  kovalchuk_a_odd_even::TestMPITaskParallel testMpiTaskParallel(taskDataPar);
  ASSERT_TRUE(testMpiTaskParallel.validation());
  testMpiTaskParallel.pre_processing();
  testMpiTaskParallel.run();
  testMpiTaskParallel.post_processing();

  if (world.rank() == 0) {
    std::vector<int> reference_result = global_vector;
    std::sort(reference_result.begin(), reference_result.end());

    ASSERT_EQ(reference_result, global_result);
  }
}
\end{lstlisting}

\section{Результаты}
Тесты показали, что параллельная реализация сортировки работает корректно и эффективно. Время выполнения сортировки уменьшается с увеличением количества процессов.

\section{Заключение}
Параллельная сортировка с использованием MPI позволяет значительно ускорить процесс сортировки больших массивов данных. Реализация алгоритма Бэтчера доказала свою эффективность в параллельном режиме.

\end{document}
