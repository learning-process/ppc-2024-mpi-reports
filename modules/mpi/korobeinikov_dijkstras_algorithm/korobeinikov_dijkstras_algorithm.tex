\documentclass[12pt]{article}
\usepackage[T2A]{fontenc}
\usepackage[utf8]{inputenc}
\usepackage[russian]{babel}
\usepackage{hyperref}
\usepackage{geometry}
\usepackage{amsmath}
\usepackage{amsfonts}
\usepackage{graphicx}

\usepackage{xcolor}
\geometry{top=2cm,bottom=2cm,left=2cm,right=2cm}
\linespread{1.5}

%C++ code 
\usepackage{listings}
\usepackage{color}
% параметры стиля
\lstdefinestyle{mystyle}{
	basicstyle=\footnotesize,
	breakatwhitespace=false, 
	breaklines=true,
	captionpos=b,
	keepspaces=true, 
	numbers=left,
	numbersep=5pt,  % номера строк
	showspaces=false, % не показывать пробелы
	showstringspaces=false,
	showtabs=false,                  
	tabsize=2
}

% применяем стиль
\lstset{style=mystyle}
% используем заданный нами моноширинный шрифт
\lstset{basicstyle=\footnotesize\ttfamily,breaklines=true}




\begin{document}
\begin{titlepage}
	\begin{center}
		\large
		{МИНИСТЕРСТВО НАУКИ И ВЫСШЕГО ОБРАЗОВАНИЯ\\ РОССИЙСКОЙ ФЕДЕРАЦИИ}
		
		Федеральное государственное автономное образовательное учреждение высшего образования
		\vspace{0.5cm}
		
		\textbf{<<Национальный исследовательский Нижегородский государственный университет им. Н.И. Лобачевского>>}\\
		(ННГУ)\\
		\vspace{1cm}
		\textbf{Институт информационных технологий, математики и механики}\\
		\vspace{1cm}
		
		
		\vfill
		
		\vfill
		
		\Large
		\textbf{Отчёт по лабораторной работе на тему} \\
		\textbf{<<Поиск кратчайших путей из одной вершины. Алгоритм Дейкстры. С CRS графов.>>}
		{\LARGE 
		}
		\bigskip
		
		
	\end{center}
	\vfill
	
	\hfill\begin{minipage}{0.4\textwidth}
		\textbf{Выполнил:} 

            студент группы 3822Б1ФИ1

            \underline{\hspace{3cm}} Коробейников А. П. \bigskip
            
	\end{minipage}%
	
	\hfill\begin{minipage}{0.4\textwidth}
		\textbf{Проверил:}

           доцент кафедры ВВСП, к.т.н.
           
           \underline{\hspace{3cm}} Сысоев А.В.\bigskip
	\end{minipage}%
	\vfill
	
	\begin{center}
		Нижний Новгород\\
		2024 г.
	\end{center}
\end{titlepage}
\thispagestyle{empty} % убираем колонтитулы с этой страницы
\tableofcontents

\newpage

\section*{Введение}
\addcontentsline{toc}{section}{Введение}
   Алгоритм Дейкстры является одним из наиболее известных и широко используемых алгоритмов для решения задачи нахождения кратчайшего пути в графах с неотрицательными весами рёбер. Он был предложен Эдсгером Дейкстрой в 1956 году и с тех пор нашел применение в различных областях, включая маршрутизацию в сетях, планирование и оптимизацию.\\

    В данной лабораторной работе рассматривается применение обычного алгоритма Дейкстры к графам, представленным в виде сжатых разреженных структур (CRS, или Compressed Row Storage). Такой способ хранения графов позволяет эффективно использовать память и ускоряет операции доступа к данным, что особенно важно при работе с большими графами, где количество рёбер значительно меньше, чем количество возможных пар вершин.\\

    Целью данной работы является изучение и реализация алгоритма Дейкстры для CRS графов в последовательной и паралелльной версиях, а также будет проведен анализ его производительности.


   



\newpage

\section{Постановка задачи}
Задача заключается в нахождении кратчайших путей от заданной начальной вершины до всех остальных вершин в направленном графе, представленном в формате Compressed Row Storage (CRS). Этот формат позволяет эффективно хранить разреженные графы и быстро получать доступ к их элементам.

\section*{Входные данные}
\begin{itemize}
    \item \( G = (V, E) \) — направленный граф, где:
    \begin{itemize}
        \item \( V \) — множество вершин, \( |V| = n \);
        \item \( E \) — множество ребер с положительными весами, заданными функцией \( w: E \to \mathbb{R}^+ \).
    \end{itemize}
    \item \( sv \) — начальная вершина для поиска кратчайших путей.
    \item Структура данных CRS, содержащая:
    \begin{itemize}
        \item \( \text{values} \) — массив весов ребер;
        \item \( \text{col} \) — массив индексов конечных вершин для каждого ребра;
        \item \( \text{RowIndex} \) — массив указателей на начало каждой строки (вершины) в массиве \( \text{values} \).
    \end{itemize}
\end{itemize}

\section*{Выходные данные}
Массив \( \text{D} \), где \( \text{D}[v] \) — кратчайшее расстояние от начальной вершины \( sv \) до вершины \( v \) для всех \( v \in V \).

\newpage

\section{Описание алгоритма}

\subsection*{Алгоритм}
Алгоритм Дейкстры находит кратчайшие пути в графе с положительными весами, используя жадный подход. Он последовательно выбирает вершину с минимальным известным расстоянием, обновляет расстояния до её соседей и повторяет процесс, пока не будут обработаны все вершины.

\begin{enumerate}
    \item \textbf{Инициализация}:
    \begin{itemize}
        \item Создаем массив расстояний \( \text{dist} \), инициализировав его значениями, близкими к бесконечности, кроме элемента с индексом стартовой вершины \( \text{dist}[sv] = 0 \) (расстояние до самой себя равно нулю). Также создаем массив \( \text{visited} \), в котором хранится информация о посещении верщины. 
    \end{itemize}
    
    \item \textbf{Поиск минимальной вершины}:
    \begin{itemize}
        \item На каждом шаге проходим по массиву и ищем не посещённую вершину с минимальным расстоянием.
    \end{itemize}

    \item \textbf{Обработка вершин}:
    \begin{itemize}
        \item Извлекаем вершину \( u \) с минимальным расстоянием.
         \item Отмечаем вершину \( u \) как посещённую.
        \item Для каждой соседней вершины \( v \), связанной с \( u \) через ребро \( (u, v) \):
        \begin{itemize}
            \item Вычисляем новое расстояние \( \text{alt} = \text{dist}[u] + w(u, v) \).
            \item Если \( \text{alt} < \text{dist}[v] \), обновляем \( \text{dist}[v] \).
        \end{itemize}
         
    \end{itemize}
    
    \item \textbf{Завершение}:
    \begin{itemize}
        \item Повторяем шаги 2 и 3 до тех пор, пока все вершины не будут обработаны.
    \end{itemize}
\end{enumerate}

\subsection*{Сложность алгоритма}

Сложность алгоритма Дейкстры на CRS графах составляет \( O((|E| + |V|) \log |V|) \). Это делает его эффективным для разреженных графов, поскольку количество операций, необходимых для обработки каждой вершины и каждого ребра, значительно уменьшается благодаря эффективному представлению графа в формате CRS.


\newpage

\section{Описание схемы распараллеливания}

Алгоритм распараллеливается следующим образом:

\begin{itemize}
    \item \textbf{Инициализация данных:} Все процессы получают необходимые данные о графе.
    \item \textbf{Распределение работы:} Каждому процессу назначается часть вершин для обработки.
    \item \textbf{Поиск локальной минимальной вершины:} Каждый процесс ищет минимальное расстояние среди своих вершин.
    \item \textbf{Сбор глобального минимума:} Для нахождения глобальной минимальной вершины собираем со всех процессах локальные минимумы.
    \item \textbf{Обновление расстояний:} Процессы обновляют расстояния до соседних вершин.
    \item \textbf{Завершение:} Процесс продолжается до тех пор, пока все вершины не будут обработаны.
\end{itemize}

\section{Описание MPI-версии программной реализации}

\subsection*{Подготовка и распределение данных}
В методе \textbf{pre-processing()} на корневом процессе (с рангом 0) происходит инициализация необходимых данных. Поля \texttt{values}, \texttt{col} и \texttt{RowIndex} заполняются данными, полученными из \texttt{taskData->inputs}. Для каждого массива используется \texttt{std::copy()} для копирования данных из указателей, которые ссылаются на входные данные. Переменные \texttt{size} и \texttt{sv} также инициализируются значениями из входных данных. После выполнения этой процедуры корневой процесс рассылает необходимые данные всем остальным процессам, используя функции \texttt{boost::mpi::broadcast()}.

\subsection*{Валидация входных данных}
В методе \textbf{validation()} корневой процесс выполняет проверку корректности входных данных. Он проверяет, что все значения в \texttt{taskData->inputs[2]} неотрицательны, а также что количество входных и выходных данных соответствует ожидаемым значениям. В случае, если все проверки пройдены успешно, функция возвращает \texttt{true}, что подтверждает корректность данных.

\subsection*{Локальные вычисления}
В методе \textbf{run()} происходит распределение работы между процессами. Сначала корневой процесс рассылает значения переменных \texttt{size}, \texttt{sv} и \texttt{count\_edges} с помощью \texttt{boost::mpi::broadcast()}. Остальные процессы инициализируют свои массивы \texttt{values}, \texttt{col} и \texttt{RowIndex} в соответствии с полученными данными. Далее каждый процесс определяет диапазон индексов, который он будет обрабатывать, в зависимости от своего ранга и общего количества процессов.

Каждый процесс создает массив \texttt{visited} для отслеживания посещенных вершин, а также массив \texttt{res} для хранения кратчайших расстояний. Изначально все расстояния инициализируются значением \texttt{std::numeric\_limits<int>::max()}, кроме стартовой вершины \texttt{sv}, для которой расстояние равно 0.

\subsection*{Сбор результатов с процессов}
В цикле, выполняемом для каждой вершины, каждый процесс находит локальную минимальную вершину, которая еще не была посещена. С помощью \texttt{boost::mpi::all\_reduce()} происходит обмен данными между процессами для нахождения глобального минимума. Если глобальный минимум равен бесконечности, процесс завершает свою работу. В противном случае, выбранная вершина отмечается как посещенная, и выполняется обновление расстояний до соседних вершин.

\subsection*{Вычисление итогового результата}
В результате работы алгоритма каждый процесс получает массив \texttt{res}, содержащий кратчайшие расстояния от стартовой вершины до всех остальных вершин графа. Эти результаты могут быть использованы для дальнейших вычислений 
\newpage

\section{Эксперименты}
 В рамках эксперимента мы будем использовать граф с разным количеством вершин, чтобы с увеличением их количества  оценить эффективность алгоритма.


\subsection*{Результаты экспериментов}
\begin{tabular}{|c|c|c|c|c|}
    \hline
    Число процессов & Количество вершин  &Количество рёбер   & Время выполнения (с)   \\ \hline
    1               & 2000               & 4000000           & 0.0061                 \\ \hline
    2               & 2000               & 4000000           & 0.0187                 \\ \hline
    3               & 2000               & 4000000           & 0.0216                 \\ \hline
    5               & 2000               & 4000000           & 0.0267                 \\ \hline
    1               & 5000               & 25000000          & 0.0378                 \\ \hline
    2               & 5000               & 25000000          & 0.0987                 \\ \hline
    3               & 5000               & 25000000          & 0.1059                 \\ \hline
    5               & 5000               & 25000000          & 0.1508                 \\ \hline
    1               & 10000              & 100000000         & 0.1632                 \\ \hline
    2               & 10000              & 100000000         & 0.3822                 \\ \hline
    3               & 10000              & 100000000         & 0.4458                 \\ \hline
    5               & 10000              & 100000000         & 0.6890                 \\ \hline
\end{tabular}

\subsection*{Анализ результатов}
Время выполнения тестов увеличивается с увеличением числа процессов, это связано с тем, что в алгоритме Дейсктра распараллеливание вычисление происходит только на этапе выбора новой обрабатываемой вершины, а на этапе обновления расстояний распараллелить вычисления невозможно. А операций передачи данных довольно много, то есть расспараллеливать алгоритм Дейсктра таким образом не совсем уместно.


\newpage
\section*{Заключение}
\addcontentsline{toc}{section}{Заключение}
В данной работе был рассмотрен и реализован алгоритм Дейкстры на CRS графах. Реализация включает в себя  последовательный и параллельный варианты. Были проведены эксперименты, в ходе которых было выявлено, что выбор такого метода распараллеливания вычислений не совсем корректный. Алгоритм Дейкстры нужно распараллеливать другим образом или не распараллеливать вовсе. 

\newpage
\section*{Список литературы}
\addcontentsline{toc}{section}{Список литературы}
\begin{enumerate}
    \item \url{https://neerc.ifmo.ru/wiki/index.php?title=%D0%90%D0%BB%D0%B3%D0%BE%D1%80%D0%B8%D1%82%D0%BC_%D0%94%D0%B5%D0%B9%D0%BA%D1%81%D1%82%D1%80%D1%8B}{ Алгоритм Дейкстры}.
    \item \url{https://www.youtube.com/watch?v=SvEFV1ewljg&t=805s}{ Алгоритм кратчайшего пути Дейкстры. Анализ производительности последовательного и параллельного выполнения.}.

\end{enumerate}
\newpage
\section*{Приложение}
\addcontentsline{toc}{section}{Приложение}
\subsection*{ops\_mpi\_korobeinikov.hpp}
\addcontentsline{toc}{subsection}{ops\_mpi\_korobeinikov.hpp}
\begin{lstlisting}[language=C++]
#pragma once

#include <gtest/gtest.h>

#include <boost/mpi/collectives.hpp>
#include <boost/mpi/communicator.hpp>
#include <numeric>
#include <utility>

#include "core/task/include/task.hpp"

namespace korobeinikov_a_test_task_mpi_lab_03 {

class TestMPITaskSequential : public ppc::core::Task {
 public:
  explicit TestMPITaskSequential(std::shared_ptr<ppc::core::TaskData> taskData_) : Task(std::move(taskData_)) {}
  bool pre_processing() override;
  bool validation() override;
  bool run() override;
  bool post_processing() override;

 private:
  std::vector<int> values;
  std::vector<int> col;
  std::vector<int> RowIndex;

  std::vector<int> res;

  int size;
  int sv;
};

class TestMPITaskParallel : public ppc::core::Task {
 public:
  explicit TestMPITaskParallel(std::shared_ptr<ppc::core::TaskData> taskData_) : Task(std::move(taskData_)) {}
  bool pre_processing() override;
  bool validation() override;
  bool run() override;
  bool post_processing() override;

 private:
  std::vector<int> values;
  std::vector<int> col;
  std::vector<int> RowIndex;

  std::vector<int> res;

  int size;
  int sv;

  boost::mpi::communicator world;
};

}  // namespace korobeinikov_a_test_task_mpi_lab_03

\end{lstlisting}
\subsection*{ops\_mpi\_korobeinikov.cpp}
\addcontentsline{toc}{subsection}{ops\_mpi\_korobeinikov.cpp}
\begin{lstlisting}[language=C++]
#include "mpi/korobeinikov_dijkstras_algorithm/include/ops_mpi_korobeinikov.hpp"

bool korobeinikov_a_test_task_mpi_lab_03::TestMPITaskSequential::pre_processing() {
  internal_order_test();
  // Init value for input and output
  values.reserve(taskData->inputs_count[0]);
  auto *tmp_ptr_1 = reinterpret_cast<int *>(taskData->inputs[0]);
  std::copy(tmp_ptr_1, tmp_ptr_1 + taskData->inputs_count[0], values.begin());

  col.reserve(taskData->inputs_count[1]);
  auto *tmp_ptr_2 = reinterpret_cast<int *>(taskData->inputs[1]);
  std::copy(tmp_ptr_2, tmp_ptr_2 + taskData->inputs_count[1], col.begin());

  RowIndex.reserve(taskData->inputs_count[2]);
  auto *tmp_ptr_3 = reinterpret_cast<int *>(taskData->inputs[2]);
  std::copy(tmp_ptr_3, tmp_ptr_3 + taskData->inputs_count[2], RowIndex.begin());

  size = *reinterpret_cast<int *>(taskData->inputs[3]);
  sv = *reinterpret_cast<int *>(taskData->inputs[4]);

  res = std::vector<int>(size, 0);
  return true;
}

bool korobeinikov_a_test_task_mpi_lab_03::TestMPITaskSequential::validation() {
  internal_order_test();
  // Check count elements of output

  bool flag = true;
  for (size_t i = 0; i < taskData->inputs_count[0]; i++) {
    if (taskData->inputs[2][i] < 0) {
      flag = false;
    }
  }
  return taskData->inputs.size() == 5 && taskData->inputs_count.size() == 5 && taskData->outputs.size() == 1 &&
         taskData->outputs_count.size() == 1 && flag && *(reinterpret_cast<int *>(taskData->inputs[4])) >= 0 &&
         (*reinterpret_cast<int *>(taskData->inputs[4])) < (*reinterpret_cast<int *>(taskData->inputs[3])) &&
         (int)taskData->outputs_count[0] == *reinterpret_cast<int *>(taskData->inputs[3]) &&
         (int)taskData->inputs_count[2] == *reinterpret_cast<int *>(taskData->inputs[3]) + 1;
}

bool korobeinikov_a_test_task_mpi_lab_03::TestMPITaskSequential::run() {
  internal_order_test();

  std::vector<bool> visited(size, false);
  std::vector<int> D(size, std::numeric_limits<int>::max());
  D[sv] = 0;
  for (int i = 0; i < size; i++) {
    int min = std::numeric_limits<int>::max();
    int index = -1;
    for (int j = 0; j < size; j++) {
      if (!visited[j] && D[j] < min) {
        min = D[j];
        index = j;
      }
    }
    if (index == -1) break;
    visited[index] = true;
    for (int j = RowIndex[index]; j < RowIndex[index + 1]; j++) {
      int v = col[j];
      int weight = values[j];

      if (!visited[v] && (D[index] + weight < D[v])) {
        D[v] = D[index] + weight;
      }
    }
  }
  res = D;
  return true;
}

bool korobeinikov_a_test_task_mpi_lab_03::TestMPITaskSequential::post_processing() {
  internal_order_test();
  std::copy(res.begin(), res.end(), reinterpret_cast<int *>(taskData->outputs[0]));
  return true;
}

bool korobeinikov_a_test_task_mpi_lab_03::TestMPITaskParallel::pre_processing() {
  internal_order_test();
  if (world.rank() == 0) {
    // Init value for input and output
    values.resize(taskData->inputs_count[0]);
    auto *tmp_ptr_1 = reinterpret_cast<int *>(taskData->inputs[0]);
    std::copy(tmp_ptr_1, tmp_ptr_1 + taskData->inputs_count[0], values.begin());

    col.resize(taskData->inputs_count[1]);
    auto *tmp_ptr_2 = reinterpret_cast<int *>(taskData->inputs[1]);
    std::copy(tmp_ptr_2, tmp_ptr_2 + taskData->inputs_count[1], col.begin());

    RowIndex.resize(taskData->inputs_count[2]);
    auto *tmp_ptr_3 = reinterpret_cast<int *>(taskData->inputs[2]);
    std::copy(tmp_ptr_3, tmp_ptr_3 + taskData->inputs_count[2], RowIndex.begin());

    size = *reinterpret_cast<int *>(taskData->inputs[3]);
    sv = *reinterpret_cast<int *>(taskData->inputs[4]);
  }
  return true;
}

bool korobeinikov_a_test_task_mpi_lab_03::TestMPITaskParallel::validation() {
  internal_order_test();
  if (world.rank() == 0) {
    bool flag = true;
    for (size_t i = 0; i < taskData->inputs_count[0]; i++) {
      if (taskData->inputs[2][i] < 0) {
        flag = false;
      }
    }

    return taskData->inputs.size() == 5 && taskData->inputs_count.size() == 5 && taskData->outputs.size() == 1 &&
           taskData->outputs_count.size() == 1 && flag && *(reinterpret_cast<int *>(taskData->inputs[4])) >= 0 &&
           (*reinterpret_cast<int *>(taskData->inputs[4])) < (*reinterpret_cast<int *>(taskData->inputs[3])) &&
           (int)taskData->outputs_count[0] == *reinterpret_cast<int *>(taskData->inputs[3]) &&
           (int)taskData->inputs_count[2] == *reinterpret_cast<int *>(taskData->inputs[3]) + 1;
  }
  return true;
}

struct ComparePairs {
  std::pair<int, int> operator()(const std::pair<int, int> &a, const std::pair<int, int> &b) const {
    if (a.first < b.first) return a;
    if (a.first > b.first) return b;
    return a;
  }
};

bool korobeinikov_a_test_task_mpi_lab_03::TestMPITaskParallel::run() {
  internal_order_test();

  broadcast(world, size, 0);
  broadcast(world, sv, 0);
  int count_edges = values.size();
  broadcast(world, count_edges, 0);

  if (world.rank() != 0) {
    values.resize(count_edges);
    col.resize(count_edges);
    RowIndex.resize(size + 1);
  }

  boost::mpi::broadcast(world, values.data(), values.size(), 0);
  boost::mpi::broadcast(world, col.data(), col.size(), 0);
  boost::mpi::broadcast(world, RowIndex.data(), RowIndex.size(), 0);

  int num_use_proc = std::min(world.size(), size);
  if (num_use_proc == 0) {
    return true;
  }
  int delta = size % num_use_proc == 0 ? (size / num_use_proc) : ((size / num_use_proc) + 1);

  int begin_for_proc = world.rank() * delta;
  int end_for_proc = std::min(size, delta * (world.rank() + 1));

  std::vector<bool> visited(size, false);
  res = std::vector<int>(size, std::numeric_limits<int>::max());
  res[sv] = 0;

  for (int k = 0; k < size; k++) {
    int local_min = std::numeric_limits<int>::max();
    int local_index = -1;

    for (int i = begin_for_proc; i < end_for_proc; i++) {
      if (!visited[i] && res[i] < local_min) {
        local_min = res[i];
        local_index = i;
      }
    }
    std::pair<int, int> local_pair = {local_min, local_index};
    std::pair<int, int> global_pair = {std::numeric_limits<int>::max(), -1};

    boost::mpi::all_reduce(world, local_pair, global_pair, ComparePairs());

    if (global_pair.first == std::numeric_limits<int>::max()) {
      break;
    }

    visited[global_pair.second] = true;

    for (int j = RowIndex[global_pair.second]; j < RowIndex[global_pair.second + 1]; j++) {
      int v = col[j];
      int w = values[j];

      if (!visited[v] && (res[global_pair.second] + w < res[v])) {
        res[v] = res[global_pair.second] + w;
      }
    }
  }

  return true;
}

bool korobeinikov_a_test_task_mpi_lab_03::TestMPITaskParallel::post_processing() {
  internal_order_test();
  if (world.rank() == 0) {
    std::copy(res.begin(), res.end(), reinterpret_cast<int *>(taskData->outputs[0]));
  }
  return true;
}
\end{lstlisting}

\end{document}