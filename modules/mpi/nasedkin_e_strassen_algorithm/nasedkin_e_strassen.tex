\documentclass[a4paper,12pt]{article}
\usepackage[utf8]{inputenc}
\usepackage[russian]{babel}
\usepackage{amsmath}
\usepackage{amssymb}
\usepackage{booktabs}
\usepackage{listings}
\usepackage{geometry}
\usepackage{titlesec}
\usepackage{hyperref}
\usepackage{tocloft}
\usepackage{xcolor} 
\usepackage{float}

\geometry{top=20mm, bottom=20mm, left=25mm, right=25mm}

\lstset{
  language=C++,
  basicstyle=\ttfamily\small,
  keywordstyle=\bfseries\color{blue},
  commentstyle=\itshape\color{green!50!black},
  stringstyle=\color{red!60!black},
  numbers=left,
  numberstyle=\tiny,
  stepnumber=1,
  numbersep=8pt,
  backgroundcolor=\color{gray!10},
  showspaces=false,
  showstringspaces=false,
  breaklines=true,
  frame=single,
  tabsize=2,
  captionpos=b
}

\begin{document}

% Титульный лист
\begin{titlepage}
    \centering
    \large
    Министерство науки и высшего образования Российской Федерации\\[0.5cm]
    Федеральное государственное автономное образовательное учреждение высшего образования\\[0.5cm]
    \textbf{«Национальный исследовательский Нижегородский государственный университет им. Н.И. Лобачевского»}\\
    (ННГУ)\\[1cm]
    Институт информационных технологий, математики и механики\\[0.5cm]
    Направление подготовки: \textbf{«Программная инженерия»}\\[2cm]

    \vfill % Вертикальное заполнение для центрирования текста
    {\LARGE \textbf{ОТЧЕТ}}\\[0.5cm]
    {\Large по задаче}\\[0.5cm]
    {\LARGE \textbf{«Алгоритм Штрассена. Умножение плотных матриц.»}}\\[0.5cm]
    {\Large \textbf{Вариант №3}}\\[2.5cm]

    \hfill\parbox{0.5\textwidth}{
        \textbf{Выполнил:} \\
        студент группы 3822Б1ПР3 \\
        \textbf{Наседкин Егор}
    }\\[0.5cm]

    \hfill\parbox{0.5\textwidth}{
        \textbf{Преподаватели:} \\
        Нестеров А.Ю.\\
        Оболенский А.А.

    }\\[2cm]

    Нижний Новгород\\
    2024
\end{titlepage}

\tableofcontents
\newpage

\section{Введение}

\hspace*{1.25em}Алгоритм Штрассена это метод умножения матриц, который уменьшает вычислительную сложность по сравнению с классическим методом. Его применение особенно эффективно для больших матриц благодаря рекурсивному разбиению задач на подзадачи меньшего размера.

В данной работе реализован параллельный алгоритм Штрассена с использованием библиотеки MPI. Этот подход позволяет использовать преимущества многопроцессорных систем для ускорения вычислений.

\section{Постановка задачи}

\hspace*{1.25em}Целью работы является разработка параллельной версии алгоритма Штрассена для умножения квадратных матриц.

Задачи:
\begin{itemize}
    \item Реализация последовательной версии метода Штрассена.
    \item Реализация параллельной версии для ускорения выполнения алгоритма на больших наборах данных.
    \item Проведение тестирования корректности алгоритма.
    \item Анализ эффективности последовательного и параллельного выполнения на наборах данных разного уровня сложности.
\end{itemize}

Цель исследования заключается в доказательстве преимуществ параллельного метода при решении задач глобальной оптимизации.

\section{Описание алгоритма}

\hspace*{1.25em}Алгоритм Штрассена основывается на разбиении исходных матриц на блоки и выполнении вычислений через промежуточные матрицы.

\begin{enumerate}
    \item \textbf{Основные этапы алгоритма}:
    \begin{itemize}
        \item Разбиение исходных матриц на блоки
        \item Вычисление промежуточных матриц $M_1, \dots, M_7$.
        \item Объединение результатов для формирования итоговой матрицы.
    \end{itemize}
\end{enumerate}
Параллельная версия алгоритма разделяет вычисления между процессами с использованием MPI, что позволяет выполнять отдельные этапы независимо.

\section{Описание схемы распараллеливания}

Схема распараллеливания алгоритма Штрассена для умножения матриц включает следующие этапы:

\begin{enumerate}
\item Предобработка данных: Процесс с рангом 0 инициализирует входные матрицы A и B, а также определяет их размерность n. Затем данные распределяются между всеми процессами.

\item Разделение матриц: Матрицы A и B делятся на подматрицы размером n/2 x n/2. Эти подматрицы распределяются между процессами для дальнейших вычислений.

\item Вычисление промежуточных матриц Mi: Каждый процесс вычисляет одну или несколько промежуточных матриц Mi по формулам алгоритма Штрассена. Для этого используются операции сложения и вычитания подматриц, а также рекурсивное применение алгоритма Штрассена для умножения подматриц.

\item Сбор результатов: Процесс с рангом 0 собирает вычисленные промежуточные матрицы Mi от всех процессов. На основе этих матриц вычисляются итоговые подматрицы C11, C12, C21 и C22 результирующей матрицы C.

\item Формирование результирующей матрицы: Подматрицы C11, C12, C21 и C22 объединяются в итоговую матрицу C. Если матрицы были дополнены нулями на этапе предобработки, лишние строки и столбцы удаляются.

\item Завершение работы: Процессы завершают вычисления, а процесс с рангом 0 записывает результат умножения матриц в выходные данные.
\end{enumerate}

Таким образом, схема распараллеливания алгоритма Штрассена позволяет эффективно использовать вычислительные ресурсы для умножения больших матриц.

\section{Заключение}

В ходе выполнения работы была реализована параллельная версия алгоритма Штрассена для умножения квадратных матриц с использованием библиотеки MPI. Последовательная версия алгоритма была разработана и протестирована на корректность. Параллельная реализация позволила распределить вычисления между процессами, что особенно полезно при работе с большими объемами данных. Тестирование проводилось на наборах данных различного размера, что подтвердило работоспособность и корректность алгоритма.
\section{Литература}
\begin{enumerate}
    \item \href{https://en.wikipedia.org/wiki/Strassen_algorithm}{Strassen algorithm. Wikipedia.}.
    \item \href{https://www.youtube.com/watch?v=OSelhO6Qnlc&ab_channel=Insidecode}{YouTube. Strassen algorithm for matrix multiplication (divide and conquer) - Inside code}.
    \item \href{https://www.open-mpi.org/doc/}{Open MPI Documentation.}.
    \item \href{https://www.boost.org/doc/libs/release/doc/html/mpi.html}{Boost.MPI Documentation}.
\end{enumerate}

\section{Приложения}
\addcontentsline{toc}{section}{Приложения}
\subsection*{Функция StrassenAlgorithmSEQ::strassen\_multiply\_seq()}
\begin{lstlisting}[language=C++]
std::vector<double> StrassenAlgorithmSEQ::strassen_multiply_seq(const std::vector<double>& matrixA,
                                                const std::vector<double>& matrixB, size_t size) {
  if (matrixA.empty() || matrixB.empty() || size == 0) {
    return {};
  }

  if (size == 1) {
    return {matrixA[0] * matrixB[0]};
  }

  size_t new_size = 1;
  while (new_size < size) {
    new_size *= 2;
  }

  size_t half_size = new_size / 2;

  std::vector<double> A11(half_size * half_size);
  std::vector<double> A12(half_size * half_size);
  std::vector<double> A21(half_size * half_size);
  std::vector<double> A22(half_size * half_size);

  std::vector<double> B11(half_size * half_size);
  std::vector<double> B12(half_size * half_size);
  std::vector<double> B21(half_size * half_size);
  std::vector<double> B22(half_size * half_size);

  for (size_t i = 0; i < half_size; ++i) {
    for (size_t j = 0; j < half_size; ++j) {
      A11[i * half_size + j] = matrixA[i * new_size + j];
      A12[i * half_size + j] = matrixA[i * new_size + j + half_size];
      A21[i * half_size + j] = matrixA[(i + half_size) * new_size + j];
      A22[i * half_size + j] = matrixA[(i + half_size) * new_size + j + half_size];

      B11[i * half_size + j] = matrixB[i * new_size + j];
      B12[i * half_size + j] = matrixB[i * new_size + j + half_size];
      B21[i * half_size + j] = matrixB[(i + half_size) * new_size + j];
      B22[i * half_size + j] = matrixB[(i + half_size) * new_size + j + half_size];
    }
  }

  std::vector<std::vector<double>> M(7);
  std::vector<std::vector<double>> tasks = {matrix_add(A11, A22, half_size),
    matrix_add(A21, A22, half_size),
     A11,
    A22,
    matrix_add(A11, A12, half_size),
    matrix_subtract(A21, A11, half_size),
    matrix_subtract(A12, A22, half_size)};

  std::vector<std::vector<double>> tasksB = {
      matrix_add(B11, B22, half_size),      B11, matrix_subtract(B12, B22, half_size),
      matrix_subtract(B21, B11, half_size), B22, matrix_add(B11, B12, half_size),
      matrix_add(B21, B22, half_size)};

  for (int i = 0; i < 7; ++i) {
    M[i] = strassen_recursive(tasks[i], tasksB[i], half_size);
  }

  for (int i = 0; i < 7; ++i) {
    std::vector<double> result;
    M[i] = result;
  }

  std::vector<double> C11 =
      matrix_add(matrix_subtract(matrix_add(M[0], M[3], half_size), M[4], half_size), M[6], half_size);
  std::vector<double> C12 = matrix_add(M[2], M[4], half_size);
  std::vector<double> C21 = matrix_add(M[1], M[3], half_size);
  std::vector<double> C22 =
      matrix_add(matrix_subtract(matrix_add(M[0], M[2], half_size), M[1], half_size), M[5], half_size);

  std::vector<double> result(size * size);
  for (size_t i = 0; i < half_size; ++i) {
    for (size_t j = 0; j < half_size; ++j) {
      result[i * size + j] = C11[i * half_size + j];
      result[i * size + j + half_size] = C12[i * half_size + j];
      result[(i + half_size) * size + j] = C21[i * half_size + j];
      result[(i + half_size) * size + j + half_size] = C22[i * half_size + j];
    }
  }
  return result;
  return {};
}
\end{lstlisting}
\subsection*{Функция StrassenAlgorithmMPI::strassen\_multiply()}
\begin{lstlisting}[language=C++]
std::vector<double> StrassenAlgorithmMPI::strassen_multiply(const std::vector<double>& matrixA,
                                const std::vector<double>& matrixB, size_t size) {
      boost::mpi::environment env;
      boost::mpi::communicator world;

      int rank = world.rank();
      int num_procs = world.size();

      if (size == 1) {
        return {matrixA[0] * matrixB[0]};
      }

      size_t new_size = 1;
      while (new_size < size) {
        new_size *= 2;
      }

      size_t half_size = new_size / 2;

        std::vector<double> A11(half_size * half_size);
        std::vector<double> A12(half_size * half_size);
        std::vector<double> A21(half_size * half_size);
        std::vector<double> A22(half_size * half_size);

        std::vector<double> B11(half_size * half_size);
        std::vector<double> B12(half_size * half_size);
        std::vector<double> B21(half_size * half_size);
        std::vector<double> B22(half_size * half_size);

      for (size_t i = 0; i < half_size; ++i) {
        for (size_t j = 0; j < half_size; ++j) {
          A11[i * half_size + j] = matrixA[i * new_size + j];
          A12[i * half_size + j] = matrixA[i * new_size + j + half_size];
          A21[i * half_size + j] = matrixA[(i + half_size) * new_size + j];
          A22[i * half_size + j] = matrixA[(i + half_size) * new_size + j + half_size];

          B11[i * half_size + j] = matrixB[i * new_size + j];
          B12[i * half_size + j] = matrixB[i * new_size + j + half_size];
          B21[i * half_size + j] = matrixB[(i + half_size) * new_size + j];
          B22[i * half_size + j] = matrixB[(i + half_size) * new_size + j + half_size];
        }
      }

        std::vector<std::vector<double>> M(7);
        if (rank == 0) {
          std::vector<std::vector<double>> tasks = {
              matrix_add(A11, A22, half_size),
              matrix_add(A21, A22, half_size),
              A11,
              A22,
              matrix_add(A11, A12, half_size),
              matrix_subtract(A21, A11, half_size),
              matrix_subtract(A12, A22, half_size)
          };

          std::vector<std::vector<double>> tasksB = {
              matrix_add(B11, B22, half_size),
              B11,
              matrix_subtract(B12, B22, half_size),
              matrix_subtract(B21, B11, half_size),
              B22,
              matrix_add(B11, B12, half_size),
              matrix_add(B21, B22, half_size)
          };

          for (int i = 0; i < 7; ++i) {
            if (i % num_procs == 0) {
              M[i] = strassen_recursive(tasks[i], tasksB[i], half_size);
            } else {
              world.send(i % num_procs, i, tasks[i]);
              world.send(i % num_procs, i, tasksB[i]);
            }
          }
        }

        for (int i = 0; i < 7; ++i) {
          if (i % num_procs == rank && i % num_procs != 0) {
            std::vector<double> taskA;
            std::vector<double> taskB;

            world.recv(0, i, taskA);
            world.recv(0, i, taskB);

            M[i] = strassen_recursive(taskA, taskB, half_size);

            world.send(0, i, M[i]);
          }
        }

        if (rank == 0) {
          for (int i = 0; i < 7; ++i) {
            if (i % num_procs != 0) {
              std::vector<double> result;
              world.recv(i % num_procs, i, result);
              M[i] = result;
            }
          }
        }

        if (rank == 0) {

          std::vector<double> C11 =
              matrix_add(matrix_subtract(matrix_add(M[0], M[3], half_size), M[4], half_size), M[6], half_size);
          std::vector<double> C12 = matrix_add(M[2], M[4], half_size);
          std::vector<double> C21 = matrix_add(M[1], M[3], half_size);
          std::vector<double> C22 =
              matrix_add(matrix_subtract(matrix_add(M[0], M[2], half_size), M[1], half_size), M[5], half_size);


          std::vector<double> result(size * size);
          for (size_t i = 0; i < half_size; ++i) {
            for (size_t j = 0; j < half_size; ++j) {
              result[i * size + j] = C11[i * half_size + j];
              result[i * size + j + half_size] = C12[i * half_size + j];
              result[(i + half_size) * size + j] = C21[i * half_size + j];
              result[(i + half_size) * size + j + half_size] = C22[i * half_size + j];
            }
          }
          return result;
        }
        return {};
    }
\end{lstlisting}
\end{document}