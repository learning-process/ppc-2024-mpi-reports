\documentclass[12pt]{article}
\usepackage[utf8]{inputenc}
\usepackage[T2A]{fontenc}
\usepackage[russian]{babel}
\usepackage{graphicx}
\usepackage{amsmath}
\begin{document}

\title{Линейное растяжение гистограммы с использованием MPI}
\author{Коньков И.}
\date{\today}
\maketitle

\section*{Введение}
Линейное растяжение гистограммы — это метод улучшения контраста изображения путем перераспределения значений интенсивности пикселей. В данной работе реализован параллельный алгоритм с использованием MPI для обработки больших изображений.

\section*{Метод}
Алгоритм линейного растяжения гистограммы выполняется по формуле:
\[
I_{\text{new}} = \frac{I - I_{\text{min}}}{I_{\text{max}} - I_{\text{min}}} \times 255
\]
где:
\begin{itemize}
    \item \( I \) — исходное значение интенсивности пикселя,
    \item \( I_{\text{min}} \) — минимальное значение интенсивности на изображении,
    \item \( I_{\text{max}} \) — максимальное значение интенсивности на изображении,
    \item \( I_{\text{new}} \) — новое значение интенсивности пикселя.
\end{itemize}

\section*{Реализация}
Алгоритм реализован с использованием MPI для параллельной обработки данных. Основные этапы работы:
\begin{itemize}
    \item Распределение данных между процессами.
    \item Поиск локальных минимумов и максимумов.
    \item Сборка глобальных минимумов и максимумов с использованием \texttt{MPI\_Allreduce}.
    \item Применение формулы линейного растяжения к каждому пикселю.
    \item Сборка обработанных данных в корневом процессе.
\end{itemize}

\section*{Тестирование}
Проведены следующие тесты:
\begin{itemize}
    \item Обработка изображения со случайными значениями.
    \item Обработка изображения с одинаковыми значениями пикселей.
    \item Обработка изображения с отрицательными значениями.
    \item Производительность на большом изображении (1 000 000 пикселей).
\end{itemize}

\section*{Результаты}
\begin{itemize}
    \item Алгоритм корректно обрабатывает изображения различных размеров.
    \item Время выполнения для изображения из 1 000 000 пикселей составило 0.0010972 секунд (зависит от конфигурации системы).
    \item Обработка изображений с одинаковыми значениями пикселей корректно сохраняет их значения.
\end{itemize}

\section*{Заключение}
Реализованный алгоритм эффективно выполняет линейное растяжение гистограммы для больших изображений с использованием MPI. Параллельная обработка позволяет значительно ускорить выполнение задачи.

\end{document}
