\documentclass[a4paper,12pt]{article}
\usepackage[utf8]{inputenc}
\usepackage[russian]{babel}
\usepackage{amsmath}
\usepackage{amsfonts}
\usepackage{amssymb}
\usepackage{hyperref}
\usepackage{listings}
\usepackage{color}

\definecolor{codegray}{rgb}{0.5,0.5,0.5}
\definecolor{codepurple}{rgb}{0.58,0,0.82}
\definecolor{backcolour}{rgb}{0.95,0.95,0.92}

\lstdefinestyle{mystyle}{
backgroundcolor=\color{backcolour},
commentstyle=\color{codegray},
keywordstyle=\color{blue},
numberstyle=\tiny\color{codegray},
stringstyle=\color{codepurple},
basicstyle=\ttfamily\footnotesize,
breakatwhitespace=false,
breaklines=true,
captionpos=b,
keepspaces=true,
numbers=left,
numbersep=5pt,
showspaces=false,
showstringspaces=false,
showtabs=false,
tabsize=2
}

\lstset{style=mystyle}

\begin{document}

\title{Анализ и реализация Быстрой сортировки с параллельным и последовательным подходами}
\author{}
\date{}
\maketitle

\section*{Введение}
Данная работа посвящена анализу и реализации алгоритма Быстрой сортировки (Quicksort) в двух версиях: последовательной и параллельной. Изучены особенности алгоритма, его эффективность при обработке больших объемов данных, а также сравнение времени выполнения и сложности этих двух подходов.

\section*{Постановка задачи}
Целью данной работы является исследование производительности алгоритма Быстрой сортировки при применении последовательной и параллельной реализаций. Оценить, как распараллеливание алгоритма влияет на эффективность обработки больших массивов данных, а также рассмотреть сложности реализации параллельного подхода.

\section*{Описание алгоритма}
\subsection*{Последовательная версия}
Последовательная версия алгоритма Быстрой сортировки использует традиционный рекурсивный подход с выбором опорного элемента, который делит массив на две части: меньшие и большие элементы относительно опорного. Затем для каждой из частей выполняется рекурсивная сортировка.

\begin{itemize}
\item \textbf{Используемая структура данных}: \texttt{std::vector} для хранения данных.
\item \textbf{Алгоритм}: Выбор опорного элемента, разбиение массива, рекурсивная сортировка.
\item \textbf{Ключевая функция}: \texttt{iterative\_quicksort(std::vector\& data)} реализует сортировку без рекурсии.
\end{itemize}

\begin{lstlisting}[language=C++]
void quicksort(std::vector<int>& arr, int low, int high) {
    if (low < high) {
        int pi = partition(arr, low, high);
        quicksort(arr, low, pi - 1);
        quicksort(arr, pi + 1, high);
    }
}

int partition(std::vector<int>& arr, int low, int high) {
    int pivot = arr[high];
    int i = (low - 1);
    for (int j = low; j <= high - 1; j++) {
        if (arr[j] < pivot) {
            i++;
            std::swap(arr[i], arr[j]);
        }
    }
    std::swap(arr[i + 1], arr[high]);
    return (i + 1);
}
\end{lstlisting}

\subsection*{Параллельная версия}
Для распараллеливания алгоритма используются процессы, распределяющие данные между собой, с последующим слиянием отсортированных частей массива. Каждый процесс выполняет локальную сортировку и передает данные другим процессам для объединения.

\begin{itemize}
\item \textbf{Используемая библиотека}: \texttt{boost::mpi} для распределения данных и их слияния.
\item \textbf{Алгоритм}: Распределение данных между процессами, локальная сортировка, слияние данных с помощью очереди с приоритетом.
\item \textbf{Ключевые функции}:
    \begin{itemize}
    \item \texttt{worker\_function}: выполняет локальную сортировку.
    \item \texttt{master\_function}: осуществляет слияние данных с помощью приоритетной очереди.
    \end{itemize}
\end{itemize}

\begin{lstlisting}[language=C++]
#include <boost/mpi.hpp>
#include <vector>
#include <iostream>
#include <algorithm>

namespace mpi = boost::mpi;

void worker_function(std::vector<int>& data) {
    std::sort(data.begin(), data.end());
}

void master_function(mpi::communicator& world, std::vector<int>& data) {
    if (world.rank() == 0) {
        // Master process will merge sorted data
        // Code for merging data from workers
    }
}
\end{lstlisting}

\section*{Распараллеливание}
Для распараллеливания алгоритма используется библиотека \texttt{boost::mpi}, которая позволяет распределить данные по множеству процессов. Каждый процесс выполняет сортировку своей части массива, а затем данные сливаются с помощью очереди с приоритетом. При этом каждый процесс может обрабатывать различные подмассивы параллельно.

Распараллеливание существенно ускоряет выполнение алгоритма при увеличении объема данных. Однако необходимо учитывать дополнительные затраты на коммуникацию между процессами.

\section*{Тестирование}
Тестирование проводилось как для последовательной, так и для параллельной версий алгоритма. Для каждой версии были проведены следующие виды тестов:
\begin{itemize}
\item \textbf{Юнит-тесты}: для проверки корректности работы алгоритма на различных типах данных (случайные данные, отсортированные массивы и массивы с повторяющимися значениями).
\item \textbf{Перфоманс-тесты}: для оценки времени выполнения при увеличении объема данных, включая тесты для параллельной версии на различных числах процессов.
\end{itemize}

\section*{Заключение}
В ходе работы была реализована и протестирована последовательная и параллельная версия алгоритма Быстрой сортировки. Параллельная версия показала значительное улучшение времени выполнения при обработке больших объемов данных, особенно на многопроцессорных системах. Тем не менее, распараллеливание требует дополнительных усилий для управления процессами и передачи данных между ними. Для небольших объемов данных последовательная версия является более эффективной и проще в реализации, в то время как для больших данных преимущество параллельной версии очевидно.

\section*{Литература}
\begin{enumerate}
    \item Седжвик, Р. (2011). \textit{Алгоритмы на C++}. Издательство "Питер".
    \item Гамма, Э., Хелм, Р., Джонсон, Р., Влиссидес, Дж. (1995). \textit{Приёмы объектно-ориентированного проектирования: Паттерны проектирования}. Издательство "Вильямс".
    \item Вуст, Б. (2001). \textit{Программирование на C++ с использованием библиотеки Boost}. Издательство "Сентябрь".
    \item Программирование с использованием MPI: http://www.mpi-forum.org/
\end{enumerate}

\section*{Приложение}
\begin{itemize}
    \item Код последовательной версии алгоритма.
    \item Код параллельной версии алгоритма.
    \item Результаты тестирования.
\end{itemize}

\end{document}
