\documentclass[12pt]{article}
\usepackage[T2A]{fontenc}
\usepackage[utf8]{inputenc}
\usepackage[russian]{babel}

\usepackage{hyperref}
\usepackage{amsmath, amssymb}
\usepackage{indentfirst} 
\usepackage{geometry}

\geometry{
    a4paper,
    top=20mm,
    right=15mm,
    bottom=20mm,
    left=25mm,
}

\setlength{\parindent}{12.5mm} 

\begin{document}
\begin{centering}
\thispagestyle{empty}
МИНИСТЕРСТВО НАУКИ И ВЫСШЕГО ОБРАЗОВАНИЯ РОССИЙСКОЙ ФЕДЕРАЦИИ\\
Федеральное государственное автономное образовательное учреждение высшего образования\\
    {\bfseries\large «Национальный исследовательский Нижегородский государственный университет им.
Н.И. Лобачевского»}\\
    {\bfseries\large Институт информационных технологий математики и механики }\\
    Направление подготовки: 09.03.04 Программная инженерия\\
Направленность (профиль) программы бакалавриата: Разработка программно-информационных систем\\


    \vspace{6cm} 
    {\LARGE\bfseries Отчет по задаче}\\
    {\large Многошаговая схема решения двумерных задач глобальной оптимизации. 
Распараллеливание по характеристикам.}\\
    \vspace{6cm} 

    {\hfill\parbox{0.5\textwidth}{%
        Выполнил:\\
        Малышев Антон Александрович\\
        группа 3822Б1ПР3\\
        
        \vspace{0.1cm}
        
        Проверил:\\
        доцент, кандидат технических наук\\
        Сысоев Александр Владимирович}
    }
    \vspace{2cm} 
    
    Нижний Новгород, 2024 г.
\newpage
\end{centering}

\tableofcontents
\newpage

\section{Введение}
В современной науке и инженерии часто возникают задачи оптимизации многомерных функций с ограничениями. Особую сложность представляет оптимизация функций с ограничениями, когда область поиска решения определяется системой неравенств. В таких случаях необходимо не только найти оптимальное значение целевой функции, но и обеспечить выполнение всех ограничений. Это существенно усложняет задачу и требует применения специальных методов обработки ограничений.

Практическая значимость работы заключается в создании универсального инструмента для решения задач глобальной оптимизации, который может быть применен в различных прикладных областях. Разработанный алгоритм позволяет эффективно находить глобальные минимумы функций двух переменных с учетом произвольных ограничений, что делает его полезным для широкого круга практических задач.
\newpage

\section{Постановка задачи}
Рассматривается задача глобальной оптимизации функции двух переменных с ограничениями:

\begin{equation}
\begin{aligned}
& \text{найти } \min_{(x,y) \in D} f(x,y) \\
& D = \{(x,y) \in \mathbb{R}^2: x \in [x_{min}, x_{max}], y \in [y_{min}, y_{max}], g_i(x,y), i=1,\ldots,m\}
\end{aligned}
\end{equation}

где:
\begin{itemize}
    \item $f(x,y)$ -- целевая функция
    \item $[x_{min}, x_{max}]$, $[y_{min}, y_{max}]$ -- границы области поиска
    \item $g_i(x,y)$ -- функции ограничений
    \item $\varepsilon > 0$ -- заданная точность
\end{itemize}

Требуется разработать последовательный и параллельный алгоритмы для решения данной задачи, используя:
\begin{enumerate}
    \item Метод градиентного спуска для локального поиска
    \item Метод туннелирования для выхода из локальных минимумов
    \item Начальное сканирование области на равномерной сетке
\end{enumerate}

Необходимо провести сравнительный анализ эффективности последовательной и параллельной реализаций на различных тестовых функциях.
\newpage

\section{Описание алгоритма}
Алгоритм глобальной оптимизации основан на комбинации методов градиентного спуска и туннелирования для поиска глобального минимума целевой функции с учетом ограничений. Алгоритм состоит из следующих основных этапов:

\subsection{Инициализация}
На начальном этапе происходит разбиение области поиска равномерной сеткой размером $20 \times 20$ точек. Для каждой точки сетки $(x_i, y_i)$ запускается процедура локального поиска.

\subsection{Локальный поиск}
Процедура локального поиска реализует модифицированный метод градиентного спуска:

\begin{enumerate}
    \item Вычисление градиента в текущей точке $(x, y)$ с использованием разностной схемы:
    \[
        \nabla f(x, y) = \begin{pmatrix}
            \frac{f(x + h, y) - f(x - h, y)}{2h} \\
            \frac{f(x, y + h) - f(x, y - h)}{2h}
        \end{pmatrix}
    \]
    где $h = 10^{-7}$ -- шаг дифференцирования.

    \item Итеративное обновление координат с адаптивным шагом обучения:
    \[
        \begin{pmatrix} x_{k+1} \\ y_{k+1} \end{pmatrix} = 
        \begin{pmatrix} x_k \\ y_k \end{pmatrix} - 
        \alpha_k \nabla f(x_k, y_k)
    \]
    где $\alpha_k$ -- текущий шаг обучения, который уменьшается при отсутствии улучшения:
    \[
        \alpha_{k+1} = 0.95 \alpha_k
    \]

    \item Проверка ограничений для новой точки:
    \begin{itemize}
        \item Границы области поиска: $x_{min} \leq x \leq x_{max}$, $y_{min} \leq y \leq y_{max}$
        \item Дополнительные ограничения в виде функций $g_i(x, y)$
    \end{itemize}

    \item Критерии остановки:
    \begin{itemize}
        \item Достижение максимального числа итераций (100)
        \item Норма градиента меньше заданной точности: $\|\nabla f(x, y)\| < \varepsilon$
    \end{itemize}
\end{enumerate}

\subsection{Метод туннелирования}
После нахождения локального минимума применяется метод туннелирования для поиска других потенциальных минимумов:

\begin{enumerate}
    \item Генерация случайных точек в окрестности текущего минимума:
    \[
        r = R\sqrt{|\xi|}, \quad \theta = 2\pi\xi
    \]
    \[
        x_{new} = x_{best} + r\cos(\theta), \quad y_{new} = y_{best} + r\sin(\theta)
    \]
    где $\xi \in [-1, 1]$ -- случайное число, $R$ -- радиус поиска, определяемый как $0.1\min(x_{max}-x_{min}, y_{max}-y_{min})$.

    \item Для каждой сгенерированной точки запускается процедура локального поиска.
    
    \item Процесс туннелирования продолжается до тех пор, пока:
    \begin{itemize}
        \item Не будет достигнуто максимальное число итераций
        \item Не произойдет 10 последовательных попыток без улучшения результата
    \end{itemize}
\end{enumerate}

\subsubsection{Проверка ограничений}
Проверка ограничений осуществляется с помощью функции \texttt{check\_constraints}, которая проверяет выполнение всех заданных ограничений для текущей точки:

\[
    check\_constraints(x, y) = \bigwedge_{i=1}^n g_i(x, y)
\]

где $g_i(x, y)$ -- $i$-е ограничение из набора ограничений.

\subsubsection{Особенности реализации}
\begin{itemize}
    \item Начальный шаг обучения установлен как $\alpha_0 = 0.01$
    \item Для туннелирования генерируется случайная точка в каждой итерации, максимальное количество попыток туннелирования - 20
    \item Используется адаптивное уменьшение шага обучения при отсутствии улучшения
    \item При нарушении ограничений точке присваивается значение $+\infty$
    \item Алгоритм гарантирует нахождение точки, удовлетворяющей всем ограничениям
\end{itemize}
\newpage

\section{Схема распараллеливания}
\subsection{Распределение ограничений}

По условию задачи было принято следующее распределение:
\begin{itemize}
    \item Все ограничения равномерно распределяются между доступными процессами
    \item Каждый процесс отвечает за проверку своего подмножества ограничений
    \item Результаты проверок собираются
\end{itemize}

Распределение реализовано следующим образом:
\begin{equation}
    \text{local\_constraints\_count} = \left\lfloor\frac{\text{total\_constraints}}{\text{num\_processes}}\right\rfloor + 
    \begin{cases}
        1, & \text{если } \text{rank} < \text{remaining\_constraints} \\
        0, & \text{иначе}
    \end{cases}
\end{equation}

\subsection{Оптимизация коммуникаций}
Из-за необходимости синхронизации промежуточных вычислений в каждой точке для повышения производительности было принято решение отказаться от использования all\_reduce.

Для минимизации накладных расходов на коммуникации:
\begin{itemize}
    \item Используется разделяемая память для массива результатов проверки ограничений
    \item Обмен данными происходит только на этапе туннелирования
    \item Локальный поиск выполняется независимо каждым процессом
\end{itemize}

\subsection{Особенности туннелирования в параллельной реализации}

Так как в реализации туннелирования используется рандом, то для одинаковой работы процессов:
\begin{itemize}
    \item Процесс с рангом 0 генерирует все случайные точки для туннелирования
    \item Точки рассылаются всем процессам с помощью broadcast
\end{itemize}
\newpage

\section{Эксперименты}

Тест производительности специально разработан для оценки эффективности параллельной реализации по сравнению с последовательной. В параллельной версии наиболее эффективно распараллеливается проверка ограничений, так как они могут выполняться независимо друг от друга на разных процессах, поэтому в тестовой задаче был сделан акцент на вычислительной сложности ограничений

\subsection{Тестовая функция}

В качестве тестовой функции для оценки производительности была выбрана следующая целевая функция:

\begin{equation}
f(x,y) = x^2 + y^2 + 10\sin(x)\cos(y)
\end{equation}

Данная функция обладает следующими характеристиками:
\begin{itemize}
    \item Непрерывность и гладкость
    \item Наличие множества локальных минимумов
    \item Глобальный минимум находится в точке (0,0)
    \item Достаточная сложность для демонстрации эффективности параллельного алгоритма
\end{itemize}

\subsection{Ограничения}
\subsubsection{Ограничение 1: Множество Мандельброта}
\begin{equation}
\begin{split}
z_{n+1} &= z_n^2 + c \\
c &= x + yi \\
|z_n| &\leq 2 \text{ для } n \leq 500
\end{split}
\end{equation}

Точка $(x,y)$ должна принадлежать множеству Мандельброта. Это обеспечивает сложную нелинейную границу области допустимых значений.

\subsubsection{Ограничение 2: Интегральное условие}
\begin{equation}
\int\limits_{0}^{|x|}\int\limits_{0}^{|y|} \sin(u^2 + v^2)e^{-0.1(u^2 + v^2)}\cos(uv)dudv \leq 10
\end{equation}

Данное ограничение требует численного интегрирования, что существенно увеличивает вычислительную сложность.

\subsubsection{Ограничение 3: Решение системы ОДУ}
Система уравнений маятника:
\begin{equation}
\begin{cases}
\dot{u} = v \\
\dot{v} = -\sin(u)
\end{cases}
\end{equation}
с начальными условиями $u(0)=x$, $v(0)=y$.

\subsubsection{Ограничение 4: Ряд Фурье}
\begin{equation}
\left|\sum_{n=1}^{300} \frac{\sin(nx)\cos(ny)}{n}\right| \leq 2
\end{equation}

\subsection{Параметры оптимизации}

Поиск проводился в области:
\begin{equation}
(x,y) \in [-0.5, 0.5] \times [-0.5, 0.5]
\end{equation}

с точностью $\varepsilon = 10^{-4}$. 

Дополнительные параметры алгоритма:
\begin{itemize}
    \item Начальная сетка: $20 \times 20$ точек
    \item Максимальное число итераций: 100
    \item Начальный шаг обучения: 0.01
    \item Коэффициент туннелирования: 0.1
    \item Число попыток туннелирования: 20
\end{itemize}

\subsection{Результаты}
Результаты тестов для последовательной и параллельной версии (-n 4):

\begin{center}
\begin{tabular}{|c|c|c|}
    \hline
        & Запуск & Время\\
    \hline
    seq & pipeline\_run & 5.3154056910 \\
    \hline    
    seq & task\_run & 5.8681763400 \\
    \hline
    mpi & pipeline\_run & 3.5960323120 \\
    \hline
    mpi & task\_run & 3.5297631610 \\
    \hline
\end{tabular}
\end{center}
\begin{enumerate}
    \item \textbf{Ускорение параллельной версии:}
    \begin{itemize}
        \item Для pipeline\_run: $S_{pipeline} = \frac{5.3154}{3.5960} \approx 1.48$
        \item Для task\_run: $S_{task} = \frac{5.8682}{3.5298} \approx 1.66$
    \end{itemize}

    \item \textbf{Эффективность распараллеливания:}
    \begin{itemize}
        \item Для pipeline\_run: $E_{pipeline} = \frac{1.48}{4} \approx 0.37$ или 37\%
        \item Для task\_run: $E_{task} = \frac{1.66}{4} \approx 0.42$ или 42\%
    \end{itemize}
\end{enumerate}

\subsection{Выводы}
Эффективность распараллеливания:
\begin{itemize}
    \item Достигнуто ускорение в 1.48-1.66 раз при использовании 4 процессов
    \item Эффективность распараллеливания составила 37-42\%
    \item Параллельная версия показала стабильное улучшение производительности для обоих методов запуска (pipeline\_run и task\_run)
\end{itemize}
\newpage

\section{Заключение}
В данной работе была рассмотрена задача глобальной оптимизации функции двух переменных с ограничениями. Основные результаты работы можно сформулировать следующим образом:

\begin{enumerate}
    \item Разработан комбинированный алгоритм глобальной оптимизации, сочетающий:
    \begin{itemize}
        \item начальное сканирование области поиска
        \item метод градиентного спуска для локальной оптимизации
        \item метод туннелирования для преодоления локальных минимумов
    \end{itemize}
    
    \item Предложена параллельная реализация алгоритма, позволяющая существенно сократить время поиска глобального минимума за счет одновременного исследования различных областей пространства поиска.
    
    \item Проведено теоретическое обоснование и экспериментальное исследование эффективности разработанных алгоритмов на наборе тестовых функций.
\end{enumerate}

Практическая значимость полученных результатов подтверждается возможностью применения разработанного алгоритма для широкого класса прикладных задач оптимизации. Предложенный подход обеспечивает:

\begin{itemize}
    \item надежный поиск глобального минимума
    \item корректную обработку произвольных ограничений
    \item возможность распараллеливания вычислений
\end{itemize}

Таким образом, поставленные в работе цели достигнуты, а разработанные алгоритмы могут служить основой для создания программного обеспечения, предназначенного для решения практических задач глобальной оптимизации.
\newpage

\section{Литература}

\begin{thebibliography}{2}

\bibitem{mpi_ref}
Microsoft Corporation.
\emph{Message Passing Interface Reference}.
Microsoft Learn. (Веб-ресурс). URL: https://learn.microsoft.com/ru-ru/message-passing-interface/mpi-reference

\bibitem{grad}
Градиентный спуск. Википедия. (Веб-ресурс). URL: https://ru.wikipedia.org/wiki/Градиентный\_спуск

\end{thebibliography}
\newpage

\section{Приложение}
В приложении предоставлены основные функции алгоритма. 
\subsection{optimize}
\begin{verbatim}
void TestTaskSequential::optimize() {
  double x_step = (data_.x_max - data_.x_min) / Constants::grid_initial_size;
  double y_step = (data_.y_max - data_.y_min) / Constants::grid_initial_size;
  Point best_point(0, 0, std::numeric_limits<double>::max());

  for (int i = 0; i <= Constants::grid_initial_size; i++) {
    for (int j = 0; j <= Constants::grid_initial_size; j++) {
      double x = data_.x_min + i * x_step;
      double y = data_.y_min + j * y_step;

      Point local_min = local_search(x, y);

      if (local_min.value < best_point.value) {
        best_point = local_min;
      }
    }
  }

  int no_improvement_count = 0;
  for (int i = 0; i < Constants::max_iterations && no_improvement_count < 10; i++) {
    Point tunneled_point = tunnel_search(best_point);

    if (tunneled_point.value < best_point.value - data_.eps) {
      best_point = tunneled_point;
      no_improvement_count = 0;
    } else {
      no_improvement_count++;
    }
  }

  res_ = local_search(best_point.x, best_point.y);
}
\end{verbatim}
\newpage

\subsection{local\_search}
\begin{verbatim}
Point TestTaskSequential::local_search(double x0, double y0) {
  Point current(x0, y0);
  current.value = traget_function_(x0, y0);

  double learning_rate = Constants::start_learning_rate;
  const double learning_rate_decay = 0.95;

  double dx = (traget_function_(x0 + Constants::h, y0) - 
                traget_function_(x0 - Constants::h, y0)) / (2 * Constants::h);
  double dy = (traget_function_(x0, y0 + Constants::h) - 
                traget_function_(x0, y0 - Constants::h)) / (2 * Constants::h);

  for (int i = 0; i < Constants::max_iterations; i++) {
    double new_x = current.x - learning_rate * dx;
    double new_y = current.y - learning_rate * dy;

    if (new_x < data_.x_min || new_x > data_.x_max || new_y < data_.y_min 
        || new_y > data_.y_max) {
      learning_rate *= learning_rate_decay;
      continue;
    }

    if (!check_constraints(new_x, new_y)) {
      learning_rate *= learning_rate_decay;
      continue;
    }

    double new_value = traget_function_(new_x, new_y);

    if (new_value < current.value) {
      current.x = new_x;
      current.y = new_y;
      current.value = new_value;
    } else {
      learning_rate *= learning_rate_decay;
    }

    double grad_norm = std::sqrt(dx * dx + dy * dy);
    if (grad_norm < data_.eps) {
      break;
    }
  }

  if (!check_constraints(current.x, current.y)) {
    return Point(current.x, current.y, std::numeric_limits<double>::max());
  }

  return current;
}
\end{verbatim}
\newpage

\subsection{tunnel\_search}
\begin{verbatim}
Point TestTaskSequential::tunnel_search(
    const Point& current_min) {
  std::random_device rd;
  std::mt19937 gen(rd());
  std::uniform_real_distribution<> dis(-1.0, 1.0);

  double radius = std::min(data_.x_max - data_.x_min, 
                        data_.y_max - data_.y_min) * Constants::tunnel_rate;
  Point best_point = current_min;

  double angle;
  double r;
  double new_x;
  double new_y;
  for (int i = 0; i < Constants::num_tunnels; i++) {
    angle = 2 * M_PI * dis(gen);
    r = radius * std::sqrt(std::abs(dis(gen)));

    new_x = current_min.x + r * std::cos(angle);
    new_y = current_min.y + r * std::sin(angle);

    if (new_x >= data_.x_min && new_x <= data_.x_max && new_y >= data_.y_min 
            && new_y <= data_.y_max) {
      Point new_point = local_search(new_x, new_y);
      if (new_point.value < best_point.value) {
        best_point = new_point;
      }
    }
  }
  return best_point;
}
\end{verbatim}
\newpage

\subsection{check\_constraints}
\begin{verbatim}

bool TestTaskSequential::check_constraints(double x, double y) {
  return std::all_of(constraints_.begin(), constraints_.end(),
                     [x, y](const constraint_t& constraint) { 
                        return constraint(x, y); 
    });
}
\end{verbatim}

\end{document}